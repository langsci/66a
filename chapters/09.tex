\chapter{Grammatical relations}

Pichi verbs exhibit a large degree of flexibility in the number and type of nominal participants they may co-occur with. The language has no deeply entrenched lexical contrast between transitive and intransitive verbs – there are only very few verbs that cannot be employed in transitive and intransitive clauses alike (cf. \sectref{sec:9.2.1}). The vast majority of verbs can act freely as intransitive or transitive verbs. However, with the class of labile verbs, either option has consequences for the semantic role attributed to the subject, the causative reading of the verb, and with most verbs, lexical aspect (cf. \sectref{sec:9.2.3}). In addition, any transitive verb may also occur in a double-object construction (cf. \sectref{sec:9.3.4}). Moreover, most verbs may appear with deverbal copies of themselves, so-called cognate objects (\sectref{sec:9.3.3}). In this way, even verbs unlikely to occur with objects in other contexts can be used transitively. 


Pichi has numerous more or less lexicalised verb-noun combinations featuring verbs with general meanings (cf. \sectref{sec:9.3.1}). Next to these, we also find combinations of verbs and associative objects. These objects may fulfil various non-core semantic roles (cf. \sectref{sec:9.3.2}). Clauses featuring referentially empty, expletive subjects reflect a need for the subject position to be filled in Pichi clauses (cf. \sectref{sec:9.2.4}). Reflexivity and reciprocity are largely expressed by the same formal means (cf. \sectref{sec:9.3.5}–\sectref{sec:9.3.6}). Verb valency may be adjusted through a rich variety of causative and impersonal constructions involving 3\textsuperscript{rd} person pronouns or human-denoting generic nouns (cf. \sectref{sec:9.4}). Finally, the expression of weather phenomena (cf. \sectref{sec:9.3.7}) and body states (cf. \sectref{sec:9.3.8}) provides good examples for the configuration of semantic roles and grammatical relations in two specific semantic fields.


\section{Expression of participants}\label{sec:9.1}

Pichi expresses the relation that holds between a verb and the core participants \is{core participants}subject and object(s) by word order\is{word order} with full nouns and a combination of word order and morphological case-marking with personal pronouns. Non-core participants\is{core participants} are expressed as prepositional phrases\is{prepositional phrases}, or in specific cases, as adverbial phrases without prepositions. Besides that, SVCs are recruited to mark participants, even if they are less frequent in terms of general frequency. 

\subsection{Subjects}

Verbs usually co-occur with at least one participant, namely a subject. Nonetheless, in certain discourse contexts, subject ellipsis occurs (cf. \sectref{sec:9.4.1}) and some SVCs allow for subjects to remain unexpressed (e.g. in certain types of motion-direction SVCs, cf. \sectref{sec:11.2.1}). Subjects subsume the actor roles of agent \REF{ex:key:1039} and experiencer\is{experiencer} \REF{ex:key:1040}:


\ea%1039
    \label{ex:key:1039}
    \gll \textbf{Dɛn}  kéch=an,    \textbf{dɛn}  bít=an.\\
\textsc{3pl}  catch=\textsc{3sg.obj}  \textsc{3pl}  beat=\textsc{3sg.obj}\\

\glt ‘They caught him (and) they beat him.’ [ye05ce 095]
\z


\ea%1040
    \label{ex:key:1040}
    \gll \textbf{E}    lɛ́k  dáns,  \textbf{e}    lɛ́k  ambiente.\\
\textsc{3sg.sbj}  like  dance  \textsc{3sg.sbj}  like  live.it.up\\

\glt ‘She likes dancing, she likes to live it up.’ [ra07fn 098]
\z

Next to that, subjects may instantiate the undergoer semantic roles of stimulus\is{stimulus}/body state in certain idiomatic expressions \REF{ex:key:1041}, theme\is{theme} in the intransitive alternation of locative verbs\is{locative verbs} \REF{ex:key:1042} and property items \REF{ex:key:1043}, and patient\is{patient} in the intransitive alternation of change-of-state verbs \REF{ex:key:1044}:


\ea%1041
    \label{ex:key:1041}
    \gll \textbf{Tɔ́sti}  kéch  mí.\\
thirst  catch  \textsc{1sg.indp}\\

\glt ‘I’m thirsty.’ [dj07ae 327]
\z


\ea%1042
    \label{ex:key:1042}
    \gll \textbf{\'{I}n}    sidɔ́n  na  Ela Nguema.\\
\textsc{3sg.indp}  stay    \textsc{loc}  \textsc{place}\\

\glt ‘She [\textsc{emp}] stays in Ela Nguema.’ [ye07fn 017]
\z


\ea%1043
    \label{ex:key:1043}
    \gll \textbf{Di}  \textbf{gɛ́l}  strét.\\
\textsc{def}  girl  be.straight\\

\glt ‘The girl is sincere.’ [ye07je 109]
\z


\ea%1044
    \label{ex:key:1044}
    \gll \textbf{A}    kɔ́t.\\
\textsc{1sg.sbj}  cut\\

\glt ‘I’m cut [I have a gash].’ [dj07ae 399]
\z

Beyond that, Pichi also uses semantically empty expletive subjects with certain verbs. These are covered in detail in \sectref{sec:9.2.4}.\is{subjects}

\subsection{Objects}

Objecthood is marked by word order\is{word order} alone with full nouns \REF{ex:key:1045} and by morphological case and word order with pronominal objects \REF{ex:key:1046}. Full nouns and pronominal objects may both appear in double-object constructions (cf. \sectref{sec:9.3.4}). The overt expression of objects is, in principle, optional although in practice prototypically transitive verbs are very likely to occur with an object. In \REF{ex:key:1047}, the object of \textit{nák} ‘hit’ remains unexpressed, but it is coreferential with the suject \textit{e} ‘\textsc{3sg.sbj}’ of the main clause: 


\ea%1045
    \label{ex:key:1045}
    \gll Wé  dɛn  \textbf{sút}    \textbf{di}  \textbf{pɔ́sin}  di  pɔ́sin  kin  sék.\\
\textsc{sub}  \textsc{3pl}  shoot  \textsc{def}  person  \textsc{def}  person  \textstylePichiglossZchn{\textsc{hab}} \textstylePichiglossZchn{shake}\\

\glt ‘When they’ve shot the person, the person shakes.’ [ed03sb 112]
\z


\ea%1046
    \label{ex:key:1046}
    \gll Gó  \textbf{púl}=\textbf{an}      dé.\\
go  remove=\textsc{3sg.obj}  \textsc{be.loc}\\

\glt ‘Go remove it there.’ [ro05ee 093]
\z


\ea%1047
    \label{ex:key:1047}
    \gll Ɛf  yu  nó  \textbf{nák},    e    nó  fít  brók.\\
if  \textsc{2sg}  \textsc{neg}  hit    \textsc{3sg.sbj}  \textsc{neg}  can  break\\

\glt ‘If you don’t hit (it), it can’t break.’ [au07se 036]
\z

Objects instantiate undergoer semantic roles such as patient\is{patient}\is{patient}, theme\is{theme}, stimulus\is{stimulus}, recipient\is{recipient} and beneficiary, as well as the actor role of experiencer\is{experiencer}. Hence the only role that is never instantiated by an object is the agent, which is reserved for subjects. 


The goal\is{} and source of motion verbs\is{motion verbs} like \textit{gó} ‘go’ and \textit{kɔmɔ́t} ‘go/come out’ may also be expressed as objects, although prepositional phrases\is{prepositional phrases} are more common. Compare the goal object \textit{colegio} ‘college’ in \REF{ex:key:1048}:



\ea%1048
    \label{ex:key:1048}
    \gll So  wé  yu  \textbf{kɔmɔ́t}    \textbf{colegio}  \op...\cp{}\\
so  \textsc{sub}  \textsc{2sg}  come.out  college\\

\glt ‘So when you came out of college (...)’ [ab03ay 029]
\z

Transitive clauses involving movement verbs and their objects may also represent cases of idiomatic transitivity as in \REF{ex:key:1049}: 


\ea%1049
    \label{ex:key:1049}
    \gll Di  tín    de  \textbf{go}  \textbf{mí}    bad.\\
\textsc{def}  thing  \textsc{ipfv}  \textsc{pot}  \textsc{1sg.indp}  bad\\

\glt ‘The matter is going bad for me.’ [dj07ae 161]
\z

Cognate objects \is{cognate objects}are referentially empty syntactic objects. They serve the pragmatic function of expressing emphasis: 


\ea%1050
    \label{ex:key:1050}
    \gll Dán    torí    bin  de  \textbf{swít}    mí    \textbf{wán}    \textbf{swít}.\\
that    story  \textsc{pst}  \textsc{ipfv}  be.tasty  \textsc{1sg.indp}  one    be.tasty\\

\glt ‘I really enjoyed that story.’ [ye07ga.006]
\z

Beyond that, a variety of other, non-core semantic roles may be expressed by objects in lexicalised verb-noun collocations involving associative objects (cf. \sectref{sec:9.3.2}.). 

\subsection{Prepositional phrases}\label{sec:9.1.3}

Participants with non-core semantic roles are most commonly expressed through prepositional phrases\is{prepositional phrases} and in specific cases through SVCs (cf. \sectref{sec:9.1.4}). \tabref{tab:key:9.1} lists the prepositions employed for the expression of non-core semantic roles. Refer to \tabref{tab:key:8.1} and \tabref{tab:key:8.7} for locative and temporal uses of prepositions and locative nouns.\is{non-locative prepositions}

%%please move \begin{table} just above \begin{tabular
\begin{table}
\caption{Non-locative uses of prepositions}
\label{tab:key:9.1}

\begin{tabularx}{\textwidth}{XXX}
\lsptoprule

Prepositions & Gloss & Other uses/comments\\
\midrule
\itshape fɔ & ‘\textsc{prep}’ & General location\\
\itshape wet & ‘with’ & NP coordination\\
\itshape bikɔs & ‘due to’ & Clause linker ‘because’\\
\itshape fɔséka/fɔséko & ‘due to’ & —\\
\itshape lɛk & ‘like’ & \textsc{if-}clause linker \\
\itshape bay & ‘by’ & Only idiomatic use\\
\itshape bitáwt & ‘without’ & Rare\\
\lspbottomrule
\end{tabularx}
\end{table}


The semantic roles expressed by the prepositions listed in \tabref{tab:key:9.1} are provided in \tabref{tab:key:9.2} below. The table reveals a bipartite structure in the marking of semantic roles. The prepositions \textit{fɔ} ‘\textsc{prep}’ and \textit{wet} ‘with’ may express virtually all roles listed. In contrast, all other prepositions express a single semantic role. In addition, the prepositions \textit{bay} ‘by’ and \textit{bitáwt} ‘without’ are marginal, and in the case of \textit{bay}, only encountered in idiomatic expressions. Given the large range of functions covered by \textit{fɔ} and \textit{wet}, the expression of semantic roles therefore relies just as much on the meaning of the verb as it does on that of the preposition it co-occurs with.

%%please move \begin{table} just above \begin{tabular
\begin{table}
\caption{Expression of non-locative semantic roles by prepositions}
\label{tab:key:9.2}

\begin{tabularx}{\textwidth}{lccccccc}
\lsptoprule
 & \itshape fɔ & \itshape wet & \itshape bikɔs & \itshape fɔséka & \itshape lɛk & \itshape bay & \itshape bitáwt\\
\midrule
Beneficiary\is{beneficiary} & x &  &  &  &  &  & \\
Stimulus\is{stimulus} & x & x &  &  &  &  & \\
Comitative &  & x &  &  &  &  & \\
\textsc{neg} comitative &  &  &  &  &  &  & x\\
Instrument & x & x &  &  &  & x & \\
Circumstance & x & x &  &  &  &  & \\
Cause & x & x & x & x &  &  & \\
Purpose & x &  &  &  &  &  & \\
Manner  &  & x &  &  & x &  & \\
\lspbottomrule
\end{tabularx}
\end{table}
The preposition \textit{fɔ} ‘\textsc{prep}’ may introduce the stimulus NP of a small number of experiential verbs with affected agents. The corpus features five such verbs: \textit{bísin} ‘bother, be busy (with)’, \textit{gládin} ‘be glad (about)’, \textit{kɔ́stɔn} ‘be used to’, \textit{lúkɔt} ‘watch out (for)’, \textit{sém} ‘be ashamed (about)’. Of these verbs, only \textit{lúkɔt} is intransitive; the only non-subject participant this verb may appear with is a stimulus PP \REF{ex:key:1051}:


\ea%1051
    \label{ex:key:1051}
    \gll \textbf{Lúkɔt}  \textbf{fɔ}  tif-mán      dɛn!\\
look.out  \textsc{prep}  steal.\textsc{cpd}{}-man    \textsc{pl}\\

\glt ‘Watch out for thieves!’ [dj07ae 096]
\z

In contrast, the stimulus of the verbs \textit{bísin} and \textit{kɔ́stɔn} may either be expressed as a PP in an intransitive clause or an object in a transitive clause. There is no difference in meaning between the two options: 


\ea%1052
    \label{ex:key:1052}
    \gll A    \textbf{bísin}  dán    gál.\\
\textsc{1sg.sbj}  be.busy  that    girl\\

\glt ‘I checked out that girl.’ [dj07ae 025]
\z


\ea%1053
    \label{ex:key:1053}
    \gll Sí  fɔ́s  tɛ́n    a    bin  dé    hía,    a    nó
bin  de  \textbf{bísin}  \textbf{fɔ}  Pagalú    gɛ́l  dɛn.\\
see  first  time    \textsc{1sg.sbj}  \textsc{pst}  \textsc{be.loc}  here    \textsc{1sg.sbj}  \textsc{neg}
\textsc{pst}  \textsc{ipfv}  be.busy  \textsc{prep}  Annobón    girl  \textsc{pl}\\

\glt ‘See formerly (when) I was here, I wasn’t checking out 
Annobonese girls.’ [ed03sp 005]
\z


\ea%1054
    \label{ex:key:1054}
    \gll Láyf    hád    pero  a    dɔ́n  \textbf{kɔ́stɔn=an}    só.\\
life    be.hard  but    \textsc{1sg.sbj}  \textsc{prf}  be.used.to=\textsc{3sg.obj}  like.that\\

\glt ‘Life is hard, but I’ve just got used to it.’ [dj07ae 101]
\z


\ea%1055
    \label{ex:key:1055}
    \gll Wi  dɔ́n \textbf{  kɔ́stɔn}    \textbf{fɔr}=an\textbf{.}\\
\textsc{1pl}  \textsc{prf}  be.used.to  \textsc{prep}=\textsc{3sg.obj}\\

\glt ‘We’ve got used to it.’ [ur07fn 218]
\z

The verb \textit{kɔ́stɔn} ‘be used to’ is also attested with a third option: it may take a stimulus PP marked by the preposition \textit{wet} ‘with’.


\ea%1056
    \label{ex:key:1056}
    \gll A    dɔ́n  \textbf{kɔ́stɔn}    \textbf{wet}    di  trɔ́n    láyf.\\
\textsc{1sg.sbj}  \textsc{prf}  be.used.to  with    \textsc{def}  strong  life\\

\glt ‘I’ve got used to a difficult life.’ [dj07ae 102]
\z

The preposition \textit{fɔ} ‘\textsc{prep}’ may also mark the stimulus of motion of some agent-induced motion verbs\is{motion verbs} like \textit{háyd} ‘hide (from)’ or \textit{rɔ́n} ‘run away (from)’ as in the following example (cf. \sectref{sec:8.1.5} for the use of \textit{fɔ} in marking locative source roles): 


\ea%1057
    \label{ex:key:1057}
    \gll E    \textbf{háyd}  \textbf{fɔ} in    kɔ́mpin.\\
\textsc{3sg.sbj}  hide    \textsc{prep}  \textsc{3sg.poss}  friend\\

\glt ‘He hid from his friend.’ [dj07re 040]
\z

Verbs other than the ones covered above invariably appear with stimulus objects rather than PPs. Compare \textit{lúk} ‘look (at)’ in \REF{ex:key:1058}. Other verbs in this group are \textit{sí} ‘see’, \textit{hía/yɛ́r} ‘hear’ and \textit{listin} ‘listen’:


\ea%1058
    \label{ex:key:1058}
    \gll A    \textbf{lúk}=\textbf{an}.\\
\textsc{1sg.sbj}  look=\textsc{3sg.obj}\\

\glt ‘I looked at him.’ [ab03ab 069]\is{stimulus}
\z

Prepositional phrases introduced by \textit{fɔ} ‘\textsc{prep}’ also denote the semantic roles of purpose \REF{ex:key:1059} and cause \REF{ex:key:1060}, the latter in combination with a body state\is{body states}: 


\ea%1059
    \label{ex:key:1059}
    \gll Mí    gí  dɛ́n    diez  mil      \textbf{fɔ}  \textbf{transporte}.\\
\textsc{1sg.indp}  give  \textsc{3pl.indp}  ten  thousand  \textsc{prep}  transport\\

\glt ‘I \textsc{[emp]} gave them ten thousand (Francs) for transport.’ [fr03cd 005]
\z


\ea
	\label{ex:key:1060}
	\gll  E    dáy  \textbf{fɔ}  \textbf{tɔ́sti}.\\
\textsc{3sg.sbj}  die  \textsc{prep}  thirsty\\

\glt ‘He died of thirst.’ [dj05be 123]
\z

Nevertheless, in the vast majority of cases, a cause of death due to a body state like \textit{hángri} ‘hunger’, \textit{tɔ́sti} ‘thirst’ \REF{ex:key:1061} or \textit{sɔfút} ‘wound, injury’ \REF{ex:key:1062} is marked by \textit{wet} ‘with’. Note however that the cause of a sickness is usually expressed as an associative object (cf. \sectref{sec:9.3.2}):\is{associative preposition}


\ea%1061
    \label{ex:key:1061}
    \gll E    dáy  \textbf{wet}    \textbf{tɔ́sti}.\\
\textsc{3sg.sbj}  die  with    thirsty\\

\glt ‘He died of thirst.’ [ro05ee 064]
\z


\ea%1062
    \label{ex:key:1062}
    \gll E    dáy  \textbf{wet}    \textbf{sɔfút}.\\
\textsc{3sg.sbj}  die  with    wound\\

\glt ‘She died of her injury.’ [ro05ee 066]
\z

The prepositions \textit{fɔséka} ‘due to’ (and its less frequent variant \textit{foséko}) \REF{ex:key:1063} and \textit{bikɔs} ‘because, due to’ \REF{ex:key:1064} introduce prepositional phrases\is{prepositional phrases} with the semantic role of cause. However, \textit{bikɔs} is seldom used as a preposition and far more common as a linker of cause clauses (cf. \sectref{sec:10.7.7}). 


Take note of the verb-object phrase \textit{bɔ́n pikín} ‘give.birth child’ = ‘childbirth’, which is nominalised in its entirety and involves \textit{bɔ́n} employed as a deverbal noun: 



\ea%1063
    \label{ex:key:1063}
    \gll Na  \textbf{fɔséka}  \textbf{bɔ́n}     \textbf{pikín},  e    dáy.\\
\textsc{foc}  due.to  give.birth  child  \textsc{3sg.sbj}  die\\

\glt ‘It’s due to childbirth (that) she died.’ [dj05be 052]
\z


\ea%1064
    \label{ex:key:1064}
    \gll Náw  só    pó    gál  dɛn  dɔ́n  bɔ́s    in    héd
\textbf{bikɔs}  \textbf{nátin}.\\
now  like.that  poor  girl  \textsc{3pl}  \textsc{prf}  burst  \textsc{3sg.poss}  head
due.to  nothing\\

\glt ‘Now the poor girl, her head has been burst open 
because of nothing.’ [ye05rr 004]\is{cause}
\z

The role of instrument is expressed through \textit{wet} ‘with’ if instruments \REF{ex:key:1065}, materials \REF{ex:key:1066}, and functions \REF{ex:key:1067} are involved. Instruments and materials can also be expressed by argument-introducing SVCs involving \textit{ték} ‘take’ (cf. \sectref{sec:9.1.4}):


\ea%1065
    \label{ex:key:1065}
    \gll Dɛn  sút=an    \textbf{wet}    \textbf{gɔ́n}  na  in    héd.\\
\textsc{3pl}  shoot=\textsc{3sg.obj}  with    gun  \textsc{loc}  \textsc{3sg.poss}  head\\

\glt ‘He was shot in the head with a gun.’ [ro05ee 054]
\z


\ea%1066
    \label{ex:key:1066}
    \gll Di  hós    bíl    \textbf{wet}    \textbf{plɛ́nk}.\\
\textsc{def}  house  build  with    board\\

\glt ‘The house is built from boards.’ [dj07ae 459]
\z


\ea%1067
    \label{ex:key:1067}
    \gll A    wáka  \textbf{wet}    \textbf{fút}.\\
\textsc{1sg.sbj}  walk  with    foot\\

\glt ‘I walked by foot.’ [dj07ae 357]
\z

Besides that, the preposition \textit{fɔ} \is{associative preposition}is used for an instrument role in a more general sense of ‘by means of’ \REF{ex:key:1068}. Still, the functional overlap of \textit{wet} and \textit{fɔ} may lead to variation in the marking of certain expressions. Compare ‘walk by foot’ in \REF{ex:key:1067} above with \REF{ex:key:1069} below: 


\ea%1068
    \label{ex:key:1068}
    \gll E    de  kwɛ́nch  \textbf{fɔ}  in    sɛ́f.\\
\textsc{3sg.sbj}  \textsc{ipfv}  die    \textsc{prep}  \textsc{3sg.poss}  self\\

\glt ‘It goes off by itself.’ [ma03ni 017]
\z


\ea%1069
    \label{ex:key:1069}
    \gll A    wáka  \textbf{fɔ}  \textbf{fút}  wet    mi    maleta.\\
\textsc{1sg.sbj}  walk  \textsc{prep}  foot  with    \textsc{1sg.poss}  suitcase\\

\glt ‘I walked by foot with my suitcase.’ [ab03ay 075]
\z

The preposition \textit{bay} ‘by (means) of’ is only attested in an idiom in the corpus where it marks an instrument NP in a way similar to the general instrument sense denoted by \textit{fɔ} ‘\textsc{prep}’ in the two preceding examples:


\ea%1070
    \label{ex:key:1070}
    \gll El  diez  de  agosto,  \textbf{bay}  \textbf{gɔ́d}  \textbf{in}    \textbf{páwa},  a    go  pás  na  yá.\\
the  ten  of  August  by  God  \textsc{3sg.poss}  power  \textsc{1sg.sbj}  \textsc{pot}  pass  \textsc{loc}  here\\

\glt ‘(On) the tenth of August, by the grace of God, I’ll pass by this place\textstyleannotationreference{.}’ [ab07fn 113]
\z

The preposition \textit{wet} ‘with’ introduces participants with a comitative role \REF{ex:key:1071}. A comitative role may also be expressed through an SVC involving \textit{fála} ‘follow’ if the accompanee is human (cf. e.g. \ref{ex:key:1573}). Comitative \textit{wet} ‘with’ may shade off into general circumstance \REF{ex:key:1072}: 


\ea%1071
    \label{ex:key:1071}
    \gll Yu  de  ɛ́nta    \textbf{wet}    \textbf{sús}?\\
\textsc{2sg}  \textsc{ipfv}  enter  with    shoe\\

\glt ‘You’re coming in with shoes?’ [ge07fn 092]
\z


\ea%1072
    \label{ex:key:1072}
    \gll Yu  nó    dán  tín    \textbf{wet}    \textbf{yun-bɔ́y}      nɔ́?\\
\textsc{2sg}  know  that  thing  with    young\textsc{.cpd}{}-boy  \textsc{intj}\\

\glt ‘You know that thing about young guys right?’ [au07se 061]
\z

Negative comitative is occasionally expressed through a PP introduced by \textit{bitáwt} ‘without’ (with the alternative pronunciation w\textit{itáwt}) \REF{ex:key:1073}. However, clausal alternatives are preferred to this rare preposition. One means of rendering ‘without’ is by employing a relative/adverbial clause construction introduced by \textit{wé} ‘\textsc{sub}’ as in \REF{ex:key:1074}: 


\ea%1073
    \label{ex:key:1073}
    \gll Dán  mán    de  wáka  \textbf{bitáwt}  \textbf{sús}\\
that  man    \textsc{ipfv}  walk  without  shoe\\

\glt ‘That man is walking without shoes.’ [ge07fn 133]
\z


\ea%1074
    \label{ex:key:1074}
    \gll A    pás    bɔkú  tɛ́n    \textbf{wé}  a    nó  chɔ́p.\\
\textsc{1sg.sbj}  pass    much  time    \textsc{sub}  \textsc{1sg.sbj}  \textsc{neg}  eat\\

\glt ‘I spent a long time without eating.’ [au07ec 080]
\z

The use of a PP is only one of numerous means of expressing manner in Pichi \is{}(cf. e.g. \sectref{sec:7.7.2}), in which case the preposition \textit{wet} ‘with’ usually serves this purpose \REF{ex:key:1075}. An equative or similative\is{similatives} participant is introduced by \textit{lɛk} ‘like’ \REF{ex:key:1076}:


\ea%1075
    \label{ex:key:1075}
    \gll Yu  nó  de  tɔ́k=an    \textbf{wet}    \textbf{páwa}.\\
\textsc{2sg}  \textsc{neg}  \textsc{ipfv}  talk=\textsc{3sg.obj}  with    power\\

\glt ‘You’re not saying it forcefully.’ [lo07he 065]
\z


\ea%1076
    \label{ex:key:1076}
    \gll \op...\cp{}  wi  fít  dé    \textbf{lɛk}  \textbf{kɔ́mpin}.\\
    {} \textsc{1pl}  can  \textsc{be.loc}  like  friend\\

\glt ‘(..) we can be (like) friends.’ [ru03wt 029]\is{prepositional phrases}
\z

The expression of a beneficiary role by means of \textit{fɔ} ‘\textsc{prep}’ is covered in detail in \sectref{sec:9.3.4} on double-object constructions.\is{adverbial phrases} 

\subsection{Serial verb constructions}\label{sec:9.1.4}

Serial verb constructions (SVC) are utilised to introduce syntactic objects\is{objects} denoting theme, the standard in comparative constructions, instrument\is{}s and materials, as well as the accompanee in comitative (cf. \textsc{\sectref{sec:11.2}).} The areally widespread SVC employing a verb meaning ‘give’ to mark a beneficiary or recipient\is{recipient} role does not exist in Pichi. Compare the following SVC, in which the (fronted) object of \textit{ték} ‘take’ denotes a material: 


\ea%1077
    \label{ex:key:1077}
    \gll Na  ús=káyn  tín    dɛn  \textbf{ték}    \textbf{mék}    dís,  digamos  dí  bɔ́tul?\\
\textsc{foc}  \textsc{q}=kind  thing  \textsc{3pl}  take    make  this  let’s.say  this  bottle\\

\glt ‘What is, let’s say this bottle, made of?’ [ye05ce 113]
\z

On the whole, SVCs are not as frequent as other means of marking participants in Pichi – to the exception of the standard in comparison. The use of a comparative construction \is{comparative SVC}featuring the verb \textit{pás} ‘surpass’ is the ordinary way of introducing the standard object:


\ea%1078
    \label{ex:key:1078}
    \gll Na  dɛ́n  bin  de  transfiere  mɔní  mɔ́    na  Western  Union
\textbf{pás}  \textbf{guineano}  \textbf{dɛn}.\\
\textsc{foc}  \textsc{3pl}  \textsc{pst}  \textsc{ipfv}  transfiere  money  more  \textsc{loc}  \textsc{name}  \textsc{name}
pass  Guinean    \textsc{pl}\\
\glt ‘It’s them who were transferring more money by Western Union
than Equatoguineans.’ [ye07je 185]
\z

Motion-direction SVCs involving motion verbs\is{motion verbs} like \textit{wáka} ‘walk’ and \textit{gó} ‘go’ express locative roles, often in combination with a prepositional phrase, as in the following example: 


\ea%1079
    \label{ex:key:1079}
    \gll Di  gɛ́l    \textbf{wáka}  \textbf{gó}  na  tɔ́n.\\
\textsc{def}  girl    walk  go  \textsc{loc}  town\\

\glt ‘The girl walked to town.’ [ne05fn 243]\is{serial verb constructions}
\z

\section{Verb classes}\label{sec:9.2}

Four lexical classes of verbs may be identified in terms of the grammatical relations they specify and with respect to the semantic roles expressed by their subject and object(s). Intransitive verbs occur with no participant other than an actor subject; transitive verbs occur with a subject and may optionally appear with one or two objects\is{objects}; labile verbs take part in a transitivity alternation: in the intransitive clause, labile verbs appear with an undergoer subject. In the transitive clause, they appear with an actor subject and an undergoer object. In addition, most labile verbs exhibit changes in their lexical aspect class and their causative reading in either alternation. Finally, expletive verbs take referentially empty subjects and may be used transitively or intransitively.

\subsection{Intransitive verbs}\label{sec:9.2.1}

Pichi features a small number of intransitive verbs which do not occur with objects\is{objects}. Elicitation with 360 verbs in the corpus revealed the intransitive verbs listed in \tabref{tab:key:9.3} below. The group of dynamic intransitive verbs is made up of locomotion verbs as well as verbs denoting other body experiences, weather verbs, verbs of existence in time and space, and an inherently reciprocal\is{reciprocity} “verb of social interaction” \citep[201]{Levin1993}. All these verbs have in common that they involve experiencer\is{experiencer} and theme\is{theme} subject\is{subjects}s, hence actors that are affected by the situation denoted by the verb.

%%please move \begin{table} just above \begin{tabular
\begin{table}
\caption{Intransitive dynamic verbs}
\label{tab:key:9.3}

\begin{tabularx}{\textwidth}{lXX}
\lsptoprule
Semantic class & Verb & Gloss\\
\midrule 
Locomotion verbs & \itshape fláy & ‘fly’\\
& \itshape gráp & ‘get up’\\
& \itshape kán & ‘come’\\
& \itshape rɔ́n & ‘run’\\
& \itshape swín & ‘swim’\\
& \itshape wáka & ‘walk’\\

\tablevspace
Body state, physical & \itshape hángri & ‘be hungry’\\
activity \& experiential verbs & \itshape tɔ́sti & ‘be thirsty’\\
& \itshape bɛ́lch & ‘belch’\\
& \itshape dáy & ‘die’\\
& \itshape lúkɔt & ‘look out’\\
& \itshape mekés & ‘hurry’\\
& \itshape ambiente & ‘live it up’\\
& \itshape pachá & ‘live it up’\\

\tablevspace
Existence verbs & \itshape sté & ‘stay’\\
& \itshape líf & ‘live’\\

\tablevspace
Weather verbs & \itshape fɔ́l & ‘rain’\\
& \itshape brék & ‘(to) dawn’\\

\tablevspace
Verb of social interaction & \itshape fɛ́t & ‘fight’\\
\lspbottomrule
\end{tabularx}
\end{table}
Inchoative-stative and stative intransitive verbs fall into three classes: modal and aspectual verbs (e.g. \textit{fít} ‘can’ and \textit{dɔ́n} ‘be finished’), verbs denoting existence in place or time (e.g. \textit{blánt} ‘reside’), and property items, most of which are human propensities (e.g. \textit{badhát} ‘be mean’, \textit{fúlis} ‘be foolish’, \textit{ráyt} ‘be right’) and physical properties (e.g. \textit{hád} ‘be hard’, \textit{sáful} ‘be slow’). One explanation for the intransitivity of verbs from these three classes is the high time-stability of the situations they denote.

%%please move \begin{table} just above \begin{tabular
\begin{table}
\caption{Intransitive (inchoative-)stative verbs}
\label{tab:key:9.4}

\begin{tabularx}{\textwidth}{lXX}
\lsptoprule
Semantic class & Verb & Gloss\\
\midrule 
Modal \& aspectual verbs & \itshape fít & ‘can’\\
& \itshape hébul & ‘be capable’\\
& \itshape tínk & ‘think’\\
& \itshape dɔ́n & ‘be done, finished’\\
Existence verbs & \itshape dé & \textsc{‘be.loc’}\\
& \itshape bí & \textsc{‘be’}\\
& \itshape blánt & ‘reside’\\
Property items & \itshape badhát & ‘be mean’\\
& \itshape bɛ́ta & ‘be very good’\\
& \itshape difrɛn & ‘be different’\\
& \itshape fúlis & ‘be foolish\\
& \itshape hád & ‘be hard’\\
& \itshape lás & ‘be last, end up’\\
& \itshape ráyt & ‘be right’\\
& \itshape sáful & ‘be slow’\\
& \itshape síryɔs & ‘be serious’\\
& \itshape smát & ‘be quick’\\
& \itshape trú & ‘be true’\\
& \itshape wɛ́l & ‘be well’\\
& \itshape wíkɛd & ‘be wicked’\\
\lspbottomrule
\end{tabularx}
\end{table}
Intransitive verbs may only appear with a subject\is{subjects} and may not take objects\is{objects}. Participants other than subjects appear in the guise of prepositional phrases\is{prepositional phrases}. For instance, the locomotion verbs \textit{fláy} ‘fly’ and \textit{wáka} ‘walk’ are intransitive. The use of theme\is{theme} \REF{ex:key:1080} or goal\is{goal} objects \REF{ex:key:1081} is rejected as ungrammatical:


\ea[*]{%1080
    \label{ex:key:1080}
    \gll Di  piloto  de  \textbf{fláy}  di  avión.\\
  \textsc{def}  pilot  \textsc{ipfv}  fly  \textsc{def}  plane\\
\glt Intended: ‘The pilot is flying the plane.’ [dj07ae 006]
}\z


\ea[*]{%1081
    \label{ex:key:1081}
    \gll \textbf{Wáka}  hós!\\
 walk  house\\
\glt Intended: ‘Walk home!’ [dj07ae 131]
}\z

SVCs and PPs\is{prepositional phrases} may be employed if the goal is to be made explicit. Compare the following two sentences:


\ea%1082
    \label{ex:key:1082}
    \gll Di  gɛ́l    \textbf{wáka}  \textbf{gó}  na  tɔ́n.\\
\textsc{def}  girl    walk  go  \textsc{loc}  town\\

\glt ‘The girl walked to town.’ [ne05fn 243]
\z


\ea%1083
    \label{ex:key:1083}
    \gll A    wánt  \textbf{fláy}  \textbf{rích}    na  tɔ́n   náw    náw.\\
\textsc{1sg.sbj}  want  fly  arrive  \textsc{loc}  town  now    \textsc{rep}\\

\glt ‘I want to fly [hurry] to town right now.’ [dj07ae 362]
\z

In contrast to \textit{wáka} ‘walk’ and \textit{fláy} ‘fly’, other motion verbs\is{motion verbs} like \textit{gó} ‘go’ can appear in transitive clauses, in which the goal is expressed as an object. This is particularly so when the goal object is a named place. Compare the object \textit{Lubá} ‘(the town of) Luba’ in \REF{ex:key:1084}:


\ea%1084
    \label{ex:key:1084}
    \gll Dí  miércoles  a      de  \textbf{gó}  \textbf{Lubá}.\\
this  wednesday  \textsc{1sg.sbj}    \textsc{ipfv}  go  \textsc{place}\\

\glt ‘This Wednesday, I’m going to Luba.’ [ro05ee 119]
\z

The transitive motion verb \textit{gó} ‘go’ and the intransitive motion verb \textit{rɔ́n} ‘run’ are also found with a meaning other than physical motion through space. Three such cases of idiomatic transitivity follow with \textit{gó} ‘go’ in (\ref{ex:key:1085}–\ref{ex:key:1086}) and \textit{rɔ́n} ‘run’ in \REF{ex:key:1087}:


\ea%1085
    \label{ex:key:1085}
    \gll Di  tín    de  \textbf{gó}  \textbf{mí}    bád.\\
\textsc{def}  thing  \textsc{ipfv}  go  \textsc{1sg.indp}  bad\\

\glt ‘The matter is going bad for me.’ [dj07ae 161]
\z


\ea%1086
    \label{ex:key:1086}
    \gll Dí  fáyn    klós    de  \textbf{gó}  \textbf{yú}.\\
this  fine  clothing    \textsc{ipfv}  go  \textsc{2sg.indp}\\

\glt \textit{Lit.} ‘These fine clothes go [fít] you.’ [nn05fn 391]
\z


\ea%1087
    \label{ex:key:1087}
    \gll \op...\cp{}  e    de  \textbf{rɔ́n}  \textbf{mí}    kɔ́ntri  tín    dɛn.\\
  {} \textsc{3sg.sbj}  \textsc{ipfv}  run  \textsc{1sg.indp}  country  thing  \textsc{pl}\\

\glt ‘(...) she was giving me a traditional treatment.’ 


\glt [\textit{Lit.} ‘She was running the village thing for me.’] [ab03ay 101]
\z

The intransitive and dynamic body state\is{body states}, body process, and experiential verbs listed in \tabref{tab:key:9.3} above require the use of a PP if a participant other than the subject is to be expressed. The stimulus\is{stimulus} of \textit{lúkɔt} ‘look out’ needs to be expressed as a \textit{fɔ}{}-prepositional phrase: 


\ea[*]{%1088
    \label{ex:key:1088}
    \gll \textbf{Lúkɔt}    tif-mán      dɛn!\\
 look.out  steal.\textsc{cpd}bɛc    \textsc{pl}\\
\glt Intended: ‘Watch out for thieves!’ [dj07ae 095]
}\z


\ea%1089
    \label{ex:key:1089}
    \gll \textbf{Lúkɔt}  \textbf{fɔ} tif-mán      dɛn!\\
look.out  \textsc{prep}  steal\textsc{.cpd}{}-man    \textsc{pl}\\

\glt ‘Watch out for thieves!’ [dj07ae 096]
\z

The verb \textit{fɛ́t} ‘fight’ cannot take an object either \REF{ex:key:1090}. A comitative participant needs to be expressed as a prepositional phrase \REF{ex:key:1091} or within a coordinate structure \REF{ex:key:1092}:


\ea[*]{%1090
    \label{ex:key:1090}
    \gll Djunais  de  \textbf{fɛ́t}    Boyé.\\
 \textsc{name}  \textsc{ipfv}  fight  \textsc{name}\\
\glt Intended: ‘Djunais is fighting Boyé.’ [dj07ae 395]
}\z


\ea%1091
    \label{ex:key:1091}
    \gll Djunais  de  \textbf{fɛ́t}    \textbf{wet}    Boyé.\\
\textsc{name}  \textsc{ipfv}  fight  with    \textsc{name}\\

\glt ‘Djunais is fighting with Boyé.’ [dj07ae 396]
\z


\ea%1092
    \label{ex:key:1092}
    \gll Djunais  \textbf{wet}    Boyé  dɛn  de  \textbf{fɛ́t}.\\
\textsc{name}  with    \textsc{name}  \textsc{3pl}  \textsc{ipfv}  fight\\

\glt ‘Djunais and Boyé are fighting.’ [dj07ae 394]
\z

The ground associated with the intransitive stative verb \textit{blánt} ‘reside’ may only be expressed as a prepositional phrase (\ref{ex:key:1093}–\ref{ex:key:1094}): 


\ea[*]{%1093
    \label{ex:key:1093}
    \gll A    \textbf{blánt}  Malábo.\\
 \textsc{1sg.sbj}  reside  Malabo\\
\glt Intended: ‘I reside in Malabo.’ [dj07ae 027]
}\z


\ea%1094
    \label{ex:key:1094}
    \gll A    \textbf{blánt}  \textbf{na} Malábo.\\
\textsc{1sg.sbj}  reside  \textsc{loc}  Malabo\\

\glt ‘I reside in Malabo.’ [dj07ae 026]
\z

Intransitive property items include \textit{gúd} ‘be good’ \REF{ex:key:1095} and \textit{bɛ́ta/bɛ́tɛ} ‘be very good, better’ \REF{ex:key:1096}. With both property items, a valency-increasing causative construction is required in order to add a participant in addition to the subject \REF{ex:key:1097}{\fff}: 


\ea[*]{%1095
    \label{ex:key:1095}
    \gll Gɔ́d  go  \textbf{gúd}=\textbf{an}.\\
  God  \textsc{pot}  good=\textsc{3sg.obj}\\
\glt Intended: ‘God will make it good.’ [dj07ae 155]
}\z


\ea[*]{%1096
    \label{ex:key:1096}
    \gll Gɔ́d  go  \textbf{bɛ́tar}=\textbf{an}.\\
  God  \textsc{pot}  very.good=\textsc{3sg.obj}\\
\glt Intended: ‘God will better it [things].’  [dj07ae 154]
}\z


\ea%1097
    \label{ex:key:1097}
    \gll Gɔ́d    go  mék    \textbf{e}    \textbf{bɛ́tɛ}.\\
God    \textsc{pot}  make  \textsc{3sg.sbj}  be.very.good\\

\glt ‘God will make it [things] good.’ [dj07ae 159]
\z

Compare the intransitive verb \textit{gúd} in the examples above to the transitive, causative \is{causative constructions}use of the labile verb \textit{fáyn} ‘be fine’, which may be used transitively and intransitively with the corresponding changes in the semantic role of the subject. The undergoer (theme\is{theme}) subject of \textit{fáyn} in \REF{ex:key:1098} becomes an actor (agent) subject\is{subjects} in \REF{ex:key:1099}. Even if this transitive, causativising use of \textit{fáyn} is rather unusual, it is not ungrammatical: 


\ea%1098
    \label{ex:key:1098}
    \gll Di  húman  \textbf{fáyn}.\\
\textsc{def}  woman  be.fine\\

\glt ‘The woman is beautiful.’ [dj05ae 149]
\z


\ea%1099
    \label{ex:key:1099}
    \gll Gɔ́d    go  \textbf{fáyn}=\textbf{an}.\\
God    \textsc{pot}  fine=\textsc{3sg.obj}\\

\glt ‘God will make it [things] fine.’ [dj07ae 156]
\z

Nevertheless, most if not all Pichi verbs may take cognate objects\is{cognate objects}, i.e. deverbal copies of themselves. In this way, even verbs unlikely to occur with objects in other contexts can be used transitively. Example \REF{ex:key:1100} involves the intransitive dynamic verb \textit{dáy} ‘die’ followed by a cognate object: 


\ea%1100
    \label{ex:key:1100}
    \gll Ey,  dán  káyn  spɛ́tikul,  a    \textbf{dáy}  \textbf{dáy}.\\
\textsc{intj}  that  kind    glasses  \textsc{1sg.sbj}  die  die\\

\glt ‘Hey, that kind of glasses, (if I had it) I would die.’ [ne07ga 015]
\z

\subsection{Transitive verbs}

Verbs other than the ones listed in \tabref{tab:key:9.3} and \tabref{tab:key:9.4} may appear in transitive clauses followed by an object. Syntactic transitivity is therefore not only a feature of highly transitive verbs with prototypical agent subjects like \textit{bít} ‘beat’, \textit{nák} ‘hit’, \textit{kíl} ‘kill’, or \textit{híb} ‘throw’. The labile verbs covered in \sectref{sec:9.2.3} as well as other (inchoative-)stative and dynamic verbs characterised by a low degree of inherent transitivity may also be followed by objects. For instance, we find verbs denoting body state\is{body states}s and body functions amongst this group.


In \REF{ex:key:1101}, the verb \textit{swɛla} ‘swallow’ is followed by the patient\is{patient} object \textit{ín} ‘\textsc{3sg.indp}’, and \REF{ex:key:1102} features the stimulus\is{stimulus} object \textit{mí} ‘\textsc{1sg.indp}’, object to the body process verb \textit{láf} ‘laugh’: 



\ea%1101
    \label{ex:key:1101}
    \gll A    \textbf{swɛla}  \textbf{ín},    dɛn  bin  tɛ́l  mí    sé
di  tín    go  gró  na  mi    bɛlɛ́.\\
\textsc{1sg.sbj}  swallow  \textsc{3sg.indp}  \textsc{3pl}  \textsc{pst}  tell  \textsc{1sg.indp}  \textsc{quot}
\textsc{def}  thing  \textsc{pot} grow  \textsc{loc}  \textsc{1sg.poss}  belly\\

\glt ‘I swallowed it, (and) I was told that the thing would grow in my stomach.’ [dj07ae 079]
\z


\ea%1102
    \label{ex:key:1102}
    \gll Dásɔl  e    \textbf{láf}    \textbf{mí}.\\
only    \textsc{3sg.sbj}  laugh  \textsc{1sg.indp}\\

\glt ‘He just laughed at me.’ [dj07ae 108]
\z

There are no restrictions on the transitive use of body function verbs involving “effected objects{\fff}” \citep{Hopper1985}, objects that come into existence by the situation denoted by the verb. Compare \textit{swɛ́t} ‘sweat’ featuring the effected object{\fff} \textit{wɔtá} ‘water’ = ‘sweat’ \REF{ex:key:1103}. The same holds for experiential (or human propensity) verbs like \textit{jɛ́lɔs} ‘envy, be jealous’, which may appear with a stimulus object \REF{ex:key:1104}: 


\ea%1103
    \label{ex:key:1103}
    \gll A    de    \textbf{swɛ́t}  \textbf{wɔtá}.\\
\textsc{1sg.sbj}  \textsc{ipfv}    sweat  water\\

\glt ‘I’m sweating.’ [dj07ae 124]
\z


\ea%1104
    \label{ex:key:1104}
    \gll A    de  \textbf{jɛ́lɔs}  \textbf{dán}    \textbf{mán}  só.\\
\textsc{1sg.sbj}  \textsc{ipfv}  envy  that    man    like.that\\

\glt ‘I just envy that man.’ [ye07je 121]
\z

Other verbs low on the transitivity scale behave no differently. For instance, typically stative situations denoted by colour-denoting property items may appear in transitive clauses with a patient object. Compare the labile verb \textit{blú} ‘be blue, make blue’ in \REF{ex:key:1105}:


\ea%1105
    \label{ex:key:1105}
    \gll A    wánt  \textbf{blú}      \textbf{dí}  \textbf{motó}  mék    e    chénch  kɔ́la.\\
\textsc{1sg.sbj}  want  make.blue  this  car    \textsc{sbjv}    \textsc{3sg.sbj}  change  colour\\

\glt ‘I want to (paint) this car blue for it to change (its) colour.’ [dj07ae 150]
\z

In the same vein, neither a physical property like \textit{hɔ́t} ‘be hot’ \REF{ex:key:1106}, nor a value concept like \textit{día} ‘be expensive’ \REF{ex:key:1107} is barred from appearing in a transitive clause. Note the causativising effect of the transitive use of these inchoative-stative labile verbs in \REF{ex:key:1106} and \REF{ex:key:1107} below as well as in \REF{ex:key:1105} above: 

\ea%1106
    \label{ex:key:1106}
    \gll \textbf{Hɔ́t}    \textbf{di}  \textbf{chɔ́p}  bifó    yu  sɛ́n=an    bifó.\\
heat    \textsc{def}  food    before  \textsc{2sg}  send=\textsc{3sg.obj}  before\\

\glt ‘Heat the food before you send it to the front [of the restaurant].’ [dj07ae 152]
\z


\ea%1107
    \label{ex:key:1107}
    \gll Di  gɔ́bna    \textbf{dɔ́n}  \textbf{día}        \textbf{di}  \textbf{petrol}.\\
\textsc{def}  government  \textsc{prf}  make.expensive  \textsc{def}  petrol\\

\glt ‘The government has made petrol more expensive.’ [dj07ae 167]
\z

Likewise, motion verbs\is{motion verbs} other than the intransitive locomotion verbs listed in \sectref{sec:9.2.1} freely alternate between transitive and intransitive uses. The following sentence presents the non-literal use of the manner-of-motion/caused-motion verb\is{motion verbs} \textit{sube} ‘go/bring up, rise/raise’, the body state\is{body states} \textit{fíba} ‘fever’, and the animate experiencer\is{experiencer} \textit{ín} ‘\textsc{3sg.indp}’:\is{patient} 


\ea%1108
    \label{ex:key:1108}
    \gll Fíba    nó  \textbf{sube}  \textbf{ín}.\\
fever  \textsc{neg}  go.up  \textsc{3sg.indp}\\

\glt ‘His fever hasn’t risen.’ [\textit{Lit.} ‘The fever hasn’t risen on him.’] [eb07fn 171]\is{transitivity}
\z

\subsection{Labile verbs} \label{sec:9.2.3}

A large number of Pichi verbs are labile; they alternate in their meaning depending on whether they occur with an object in a transitive\is{transitivity} clause or without an object in an intransitive clause. Labile verbs participate in a transitivity alternation that causes a co-variation of the semantic macro-role of the subject\is{subjects} (undergoer vs. actor), the causation reading of the verb (non-causative vs. causative)\is{causative constructions}, and with most verbs, the stativity value (inchoative-stative vs. dynamic).


Five subclasses of labile verbs can be identified in semantic terms: change-of-state verbs, locative verbs\is{locative verbs}, property items, experiential verbs, and aspectual verbs. In formal terms, only two broad classes, however, need to be distinguished. With the first three subclasses, the intransitive-transitive alternation is accompanied by an inchoative-stative/dynamic alternation. With experiential and aspectual verbs, however, both the intransitive and the transitive alternants are dynamic. 


\tabref{tab:key:9.5} lists the relevant features of labile verbs in accordance with the two formal and five semantic classes. An additional co-variation feature not included in the table is the tense\is{tense} interpretation of the inchoative-stative and dynamic variants of class (a). The unmarked inchoative-stative variants of class (a) verbs receive a present tense\is{present tense} interpretation when they are used as stative verbs. Alternatively, they receive a past tense\is{past tense} interpretation if they are used as inchoative verbs.

In turn, the unmarked dynamic variants of class (b) behave like other dynamic verbs and receive a past tense interpretation (cf. \sectref{sec:6.3.1} for an extensive treatment). The abbreviation \textsc{ista} stands for inchoative-stative. \is{aspect}


%%please move \begin{table} just above \begin{tabular
\begin{table}
\caption{Characteristics of labile verbs}
\label{tab:key:9.5}
\small
\begin{tabularx}{\textwidth}{r>{\raggedright}p{24mm}p{24mm}p{29mm}Q}
\lsptoprule

\multicolumn{2}{c}{Semantic class of verb} & Role of subject in \textsc{intr/tr} clause & Causation reading in \textsc{intr/tr} clause & Lexical aspect \mbox{in \textsc{intr/tr} clause}\\
\midrule
a. & Change of state; Locative; Property item& Actor/undergoer & Non-causative/causative & \textsc{ista}/dynamic\\


\tablevspace
b. & Experiential; Aspectual & Actor/undergoer & Non-causative/causative & Dynamic/dynamic\\
\lspbottomrule
\end{tabularx}
\end{table}
Class (a) labile verbs are employed as inchoative-stative verbs in intransitive clauses and as dynamic verbs in transitive clauses. Either use co-varies with the “role of the subject\is{subjects}”: Intransitive clauses have an undergoer subject, while transitive clauses feature an actor subject and an undergoer object. 


In the corpus, change-of-state verbs constitute the largest subclass of labile verbs. Some representative change-of-state verbs are provided in \REF{ex:key:1109}. With some verbs, the change of state of the subject is more likely to have been caused by (a) an external (usually animate and unmentioned) agent{\fff}, with others (b) by a cause internal to the subject (cf. \citealt{Croft1990}; \citealt{Haspelmath1993}; \citealt{LevinHovav1995}). This difference is reflected in the glosses given. Group (a) verbs are rendered with their dynamic meanings, group (b) with their stative meanings. The verbs are also loosely grouped along semantic criteria, such as ‘destruction’ (e.g. \textit{brók} ‘break’, \textit{chakrá} ‘destroy’), ‘material transformation’ (e.g. \textit{bɛ́n} ‘bend’, \textit{bwɛ́l} ‘boil’), ‘body state{\fff}s’ (e.g. \textit{bɛlfúl} ‘be satiated’, \textit{táya} ‘be tired’), and ‘natural states’ (e.g. \textit{rɔ́tin} ‘be rotten’, \textit{sók} ‘be wet’), ‘other human states’ (e.g. \textit{wɛ́r} ‘wear’, \textit{máred} ‘marry’):


\eabox{\label{ex:key:1109}
\begin{tabularx}{.9\textwidth}{rlX rlX}
         & \multicolumn{2}{l}{Change-of-state verbs} & \multicolumn{3}{c}{}\\
\raggedleft a. & \itshape bɛ́n & ‘bend, fold’ &  & \itshape spwɛ́l & ‘spoil’\\
& \itshape bíl & ‘build’ &  & \itshape wás & ‘wash’\\
& \itshape brók & ‘break’ &  & \itshape wék(ɔp) & ‘wake up’\\
& \itshape bwɛ́l & ‘boil’ &  & \itshape wɛ́r & ‘wear’\\
& \itshape chénch & ‘change’ &  &  & \\
& \itshape chɛ́r & ‘tear’ & \raggedleft b. & \itshape bɛlfúl & ‘be satiated’\\
& \itshape fíks & ‘fix, repair’ &  & \itshape chák & ‘be drunk’\\
& \itshape fráy & ‘fry’ &  & \itshape dráy & ‘be dry’\\
& \itshape hát & ‘hurt’ &  & \itshape drɔ́ngo & ‘be dead drunk’\\
& \itshape kɔ́ba & ‘cover’ &  & \itshape fúlɔp & ‘be full’\\
& \itshape kɔ́t & ‘cut’ &  & \itshape láyt & ‘be lit, be tipsy’\\
& \itshape krás & ‘crash’ &  & \itshape rɛdí & ‘be ready’\\
& \itshape kúk & ‘cook’ &  & \itshape rɔ́tin & ‘be rotten’\\
& \itshape lɔ́k & ‘close’ &  & \itshape sók & ‘be wet’\\
& \itshape máred & ‘marry’ &  & \itshape táya & ‘be tired’\\
\end{tabularx}
}
In the intransitive clause in \REF{ex:key:1110} below, the change-of-state verb \textit{chák} ‘be drunk, get drunk’ takes an undergoer subject\is{subjects} (with the specific role of patient\is{patient}). In the transitive clause in \REF{ex:key:1111}, \textit{chák} now takes an actor subject (with the specific role of agent) and an undergoer (patient) object. In the intransitive clause, the verb has a non-causative meaning, while the verb in the transitive clause has a causative meaning\is{causative constructions}. 


At the same time, the aspectual reading of the bare factative change of state verb is adjusted. When the verb is employed as a bare inchoative-stative verb in a basic intransitive clause, as in \REF{ex:key:1110} below, it normally receives a present tense\is{present tense} interpretation – the situation holds at reference time. In turn, the dynamic variant of \textit{chák} receives a default past tense\is{past tense} interpretation in \REF{ex:key:1111}. 



\ea%1110
    \label{ex:key:1110}
    \gll Di  wach-mán    \textbf{chák}.\\
\textsc{def}  watch.\textsc{cpd}{}-man  be.drunk\\

\glt ‘The guard is drunk.’ [dj07ae 048]
\z


\ea%1111
    \label{ex:key:1111}
    \gll Dɛn    \textbf{chák}    \textbf{di}  \textbf{wach-mán}    fɔ́s  fɔ  mék
dɛn  fít  gó  tíf.\\
\textsc{3pl}    get.drunk  \textsc{def}  watch.\textsc{cpd}{}-man  first  \textsc{prep}  \textsc{sbjv}  
\textsc{3pl}  can  go  steal\\

\glt ‘They got the guard drunk first in order for them to be able to steal.’ [dj07ae 052]
\z

When used intransitively with factative TMA, there is generally a stronger tendency for change-of-state verbs from group (b) to receive a stative interpretation, as in \REF{ex:key:1110} above. In contrast, many group (a) verbs are more likely to receive an inchoative interpretation focussing on the change-of-state, since most of these verbs feature an implicit agent or (natural) force\is{force}. When verbs with implicit agents appear in intransitive clauses, there is therefore a higher tendency for speakers to employ the perfect tense-aspect rather than factative TMA in order to indicate a change-of-state. The use of perfect marking via \textit{dɔ́n} ‘\textsc{prf}’ focuses the end-state of the change of state. 


Compare \textit{fráy} ‘fry’, an “agentive” group (a) verb in the intransitive and transitive clause, respectively. The combination of perfect marking and “agentive” verb renders a resultative meaning \is{resultative constructions}very close to passive\is{passive} voice in \REF{ex:key:1112}: 



\ea%1112
    \label{ex:key:1112}
    \gll Di  plantí  \textbf{dɔ́n}  \textbf{fráy}.\\
\textsc{def}  plantain  \textsc{prf}  fry\\

\glt ‘The plantain has been fried.’ [dj07ae 418]
\z


\ea%1113
    \label{ex:key:1113}
    \gll A    bigín  de  pica-píca,    wi  \textbf{fráy}  \textbf{patata},
wi  \textbf{fráy}  \textbf{plantí}.\\
\textsc{1sg.sbj}  begin  \textsc{ipfv}  \textsc{red.cpd}{}-cut.up  \textsc{1pl}  fry  potato
\textsc{1pl}  fry  plantain\\

\glt ‘I began to cut up (the trimmings), we fried potatoes, we fried plantain.’ [ye03cd.172]\is{factative TMA}
\z

Change-of-state verbs also differ with respect to their likelihood to occur in intransitive or transitive clauses. The higher “agentivity” of group (a) verbs like \textit{fráy} ‘fry’ makes it less likely for these verbs to appear in agentless, intransitive clauses than group (b) verbs like \textit{bɛlfúl} ‘be satiated’ or \textit{táya} ‘be tired’.\is{agent}


Two further semantic classes of labile verbs are locative verbs\is{locative verbs} and property items. These two subclasses alternate between inchoative-stative and dynamic uses. The two following examples involve the intransitive \REF{ex:key:1114} and transitive \REF{ex:key:1115} use of the locative verb \textit{lé} ‘lie, lay’. The latter example also features the transitively used locative verb \textit{slíp} ‘sleep, lie, lay’. A more extensive listing of locative verbs and a detailed treatment of their distribution is given in \sectref{sec:8.1.3}:



\ea%1114
    \label{ex:key:1114}
    \gll Di  \textbf{kasára}  \textbf{lé}  míndul  tú  stík.\\
\textsc{def}  cassava  lie  middle  two  tree\\

\glt ‘The cassava is lying between two branches.’ [li07pe 080]
\z


\ea%1115
    \label{ex:key:1115}
    \gll E    \textbf{lé}  di  \textbf{bɔ́tul}  pantáp  di  tébul,  e    \textbf{slíp}  
\textbf{di}  \textbf{bɔ́tul}  pantáp  di  tébul.\\
\textsc{3sg.sbj}  lay  \textsc{def}  bottle  top    \textsc{def}  table  \textsc{3sg.sbj}  lay
\textsc{def}  bottle  top    \textsc{def}  table\\

\glt ‘He laid the bottle on the table, he brought the bottle into a 
horizontal position on the table.’ [li07pe 074]
\z

Property items behave no differently from change-of-state and locative verbs. Consider the intransitive/transitive and stative/dynamic uses of the physical property denoting verb \textit{lɔ́n} ‘be long, lengthen’ in the two following examples: 


\ea%1116
    \label{ex:key:1116}
    \gll \textbf{Dán}    \textbf{húman}  \textbf{lɔ́n}    bád.\\
that    woman  be.long  extremely\\

\glt ‘That woman is/was extremely tall.’ [li07pe 064]
\z


\ea%1117
    \label{ex:key:1117}
    \gll \MakeUppercase{A}   wánt  \textbf{lɔ́n}      \textbf{di}  \textbf{klós}.\\
\textsc{1sg.sbj}  want  lengthen    \textsc{def}  clothing\\

\glt ‘I want to lengthen this piece of clothing.’ [dj07ae 223]
\z

Property items of all other semantic types may be used in the same way as \textit{lɔ́n} ‘(be) long’ (cf. \sectref{sec:4.1.2} for a listing of relevant semantic types). Compare the intransitive meaning of ‘be small’ of the dimension concept \textit{smɔ́l} ‘(be) small’ in the intransitive clause in \REF{ex:key:1118} with the causative meaning ‘make small, shrink’ in the transitive clause in \REF{ex:key:1119}\is{causative constructions}. The imperfective marker \textit{de} ‘\textsc{ipfv}’ specifies \textit{smɔ́l} in \REF{ex:key:1119} just like any dynamic verb with simultaneous taxis:


\ea%1118
    \label{ex:key:1118}
    \gll Dí  klós    \textbf{smɔ́l}.\\
\textsc{def}  clothing  be.small\\

\glt ‘This (piece of) clothing is small.’
\z


\ea%1119
    \label{ex:key:1119}
    \gll Sɔn    klós    dɛn  dé,    hɔt-wɔtá      \textbf{de}  \textbf{smɔ́l}=\textbf{an}.\\
some  clothing  \textsc{pl}  \textsc{be.loc}  hot.\textsc{cpd}{}-water    \textsc{ipfv}  make.small=\textsc{3sg.obj}\\

\glt ‘There are some clothes, hot water shrinks them.’ [dj07ae 211]
\z

A value concept like \textit{fáyn} ‘(be) fine, beautiful’ may also be subjected to the intransitive/transitive alternation characteristic of labile verbs. Compare the intransitive, stative use of this property item: 


\ea%1120
    \label{ex:key:1120}
    \gll Libreville  \textbf{fáyn}.\\
\textsc{place}    be.fine\\

\glt ‘Libreville is (a) nice (place).’ [ma03sh 009]
\z

Now consider the transitive use of \textit{fáyn} in the following two sentences. Note that a transitive use may also lead to an idiosyncratic meaning of \textit{fáyn}. Sentence \REF{ex:key:1121} presents the regular, derived transitive meaning of ‘make beautiful’, while \REF{ex:key:1122} represents a case of idiomatic transitivity with a “dative of interest” reading of the experiencer\is{experiencer} object pronoun of \textit{fáyn}. Such a meaning is also recorded for cases of idiomatic transitivity with other verbs low on the transitivity scale, e.g. the motion verbs\is{motion verbs} \textit{gó} ‘go’ \REF{ex:key:1085}, \textit{rɔ́n} ‘run’ \REF{ex:key:1087}, and \textit{sube} ‘rise, raise’ \REF{ex:key:1108}:

\ea%1121
    \label{ex:key:1121}
    \gll Nóto  klós    go \textbf{fáyn}  \textbf{yú} sino    que
na  yú    gɛ́fɔ    fáyn    yu  sɛ́f.\\
\textsc{neg}.\textsc{foc}  clothing  \textsc{pot}  make.fine  \textsc{2sg.indp}  but    that  
\textsc{foc}  \textsc{2sg.indp}  have.to  make.fine  \textsc{2sg}  self\\

\glt ‘It’s not clothes that would make you beautiful, it’s rather you 
that has to make yourself beautiful.’ [dj07ae 176]
\z


\ea%1122
    \label{ex:key:1122}
    \gll Dán  bɛ́lps  de  \textbf{fáyn}  \textbf{mí}.\\
that  babe  \textsc{ipfv}  be.fine  \textsc{1sg.indp}\\

\glt ‘I find that babe gorgeous.’ [\textit{Lit}. ‘That babe is fine to me.’] [dj07ae 174]
\z

Although there are no restrictions on the transitive use of property items, such usage is rare in non-elicited language data. There is a pronounced preference by speakers to employ other means to render causative meaning with property items\is{causative constructions}. 


For instance, in the following two examples, the property items \textit{fáyn} ‘(be) fine’ and \textit{blák} ‘(be) black’ are employed as secondary predicates\is{secondary predicates}. Sentence \REF{ex:key:1123} features a resultative causative construction, \is{resultative constructions}and \REF{ex:key:1124} involves a resultant state resultative construction: 



\ea%1123
    \label{ex:key:1123}
    \gll Dɛn  de  \textbf{lɛ́f}=an    \textbf{fáyn}?\\
\textsc{3pl}  \textsc{ipfv}  leave=\textsc{3sg.obj}  fine\\

\glt ‘Are they making it [the house] beautiful?’ [hi03cb 041]
\z


\ea%1124
    \label{ex:key:1124}
    \gll E    \textbf{pént}  di  hós    \textbf{blák}.\\
\textsc{3sg.sbj}  paint  \textsc{def}  house  be.black\\

\glt ‘He painted the house black.’ [pa07me 037]
\z

Labile experiential and aspectual verbs in class (b) of \tabref{tab:key:9.5} differ from class (a) verbs in that they remain dynamic in both the intransitive and transitive alternation. However, the features of “role of subject\is{subjects}” and “causation reading” provided in \tabref{tab:key:9.5} co-vary in the same way with class (b) verbs as they do with class (a) verbs.


Labile experiential verbs constitute a smaller group than change-of-state verbs. I give a complete listing of experiential verbs in the corpus with glosses of intransitive meanings in \REF{ex:key:1125}. Experiential verbs comprise (a) body movements and processes, as well as (b) mental states denoting various types of affective conditions:


\eabox{\label{ex:key:1125}
\begin{tabularx}{.9\textwidth}{rlX rlX}
         & \multicolumn{2}{l}{Experiential verbs} & \multicolumn{3}{c}{}\\
\raggedleft a. & \itshape bló & ‘relax’ & \raggedleft b. & \itshape gládin & ‘be glad’\\
& \itshape bɔ́n & ‘be born’ &  & \itshape krés & ‘be crazy’\\
& \itshape gró & ‘grow’ &  & \itshape lán & ‘learn’\\
& \itshape háyd & ‘hide’ &  & \itshape sém & ‘be ashamed’\\
& \itshape hɔ́ri & ‘hurry’ &  & \itshape skía & ‘be scared’\\
& \itshape lɛ́f & ‘leave’ &  & \itshape sɔ́fa & ‘suffer’\\
& \itshape mɛ́n & ‘get better’ &  & \itshape vɛ́ks & ‘be angry’\\
& \itshape múf & ‘move’ &  & \itshape wánda & ‘wonder’\\
& \itshape rɛ́s & ‘rest’ &  & \itshape wɔ́ri & ‘worry’\\
& \itshape sék & ‘shake’ &  &  & \\
& \itshape tɔ́n & ‘turn’ &  &  & \\
& \itshape trímbul & ‘tremble’ &  &  & \\
\end{tabularx}
}
Consider the use of the group (b) dynamic experiential verb \textit{krés} ‘be crazy, drive crazy’ in the following intransitive \REF{ex:key:1126} and transitive \REF{ex:key:1127} clauses respectively: 


\ea%1126
    \label{ex:key:1126}
    \gll \MakeUppercase{A}   \textbf{krés}.\\
\textsc{1sg.sbj}  be.crazy\\

\glt ‘I went mad.’ [ro05rt 022]
\z


\ea%1127
    \label{ex:key:1127}
    \gll Wé  di  mamá  dáy,    na  ín    \textbf{krés}    \textbf{di}  \textbf{pikín}.\\
\textsc{sub}  \textsc{def}  mother  die    \textsc{foc}  \textsc{3sg.indp}  drive.mad  \textsc{def}  child\\

\glt ‘When the mother died, that’s what drove the child mad.’ [dj07ae 104]
\z

The following two sentences illustrate the use of the group (a) body movement verb \textit{háyd} ‘hide, conceal’. In both the intransitive \REF{ex:key:1128} and transitive \REF{ex:key:1129} clauses the imperfective marker \textit{de} ‘\textsc{ipfv}’ is present, so experiential verbs do not exhibit the stativity alternation that characterises the other semantic classes covered so far: 


\ea%1128
    \label{ex:key:1128}
    \gll E    \textbf{de}  \textbf{háyd}  \textbf{fɔr}=an.\\
\textsc{3sg.sbj}  \textsc{ipfv}  hide    \textsc{prep}=\textsc{3sg.obj}\\

\glt ‘She’s hiding from him.’ [dj07re 042]
\z


\ea%1129
    \label{ex:key:1129}
    \gll E    \textbf{de}  \textbf{háyd}=an.\\
\textsc{3sg.sbj}  \textsc{ipfv}  hide=\textsc{3sg.sbj}\\

\glt ‘It [the bag] is concealing it [the telephone].’ [ur07fn 078]
\z

The final class of labile verbs are aspectual verbs (also known as phasal verbs), i.e. verbs with largely temporal semantics, which usually occur in constructions with lexically fuller verbs. These verbs remain dynamic in transitive and intransitive clauses as well. Hence they do not alternate in their stativity value either. 


Aspectual verbs serve to highlight the crossing of the left boundary (inception), the middle (continuation) or the right boundary (completion) of the situation denoted by the verb they specify. The four labile aspectual verbs of inception (a) and completion (b) found in the corpus are listed in \REF{ex:key:1130}:

\eabox{\label{ex:key:1130}
\begin{tabularx}{.9\textwidth}{rlX rlX}
         & \multicolumn{2}{l}{Aspectual verbs} & \multicolumn{3}{c}{}\\
\raggedleft a. & \itshape bigín & ‘begin’ & \raggedleft b. & \itshape fínis & ‘finish’\\
& \itshape stát & ‘start’ &  & \itshape para \textup{(< Sp. ‘parar’)} & ‘stop’\\
\end{tabularx}
}
I give an example for the intransitive and transitive uses of the verb of completion \textit{fínis} ‘finish’ in the following two examples. The verbs \textit{fínis} and \textit{bigín} ‘begin’ are also employed as aspectual auxiliary verbs in completive and ingressive\is{ingressive aspect} auxiliary constructions (cf. \sectref{sec:6.4.1} and \sectref{sec:6.4.3}, respectively):\is{auxiliaries}

\ea%1131
    \label{ex:key:1131}
    \gll Dɛn-ɔ́l      \textbf{fínis}.\\
\textsc{3pl}.\textsc{cpd}{}-all  finish\\
\glt ‘They’re all finished.’ [dj03cd 157]
\z


\ea%1132
    \label{ex:key:1132}
    \gll A    de  tɛ́l  yú,    yu  go  sí  náw  
yu  nó  go  \textbf{fínis}  \textbf{dán}    \textbf{watá}.\\
\textsc{1sg.sbj}  \textsc{ipfv}  tell  \textsc{2sg.indp}  \textsc{2sg}  \textsc{pot}  see  now  
\textsc{2sg}  \textsc{neg}  \textsc{pot}  finish  that    water\\

\glt ‘I’m telling you, you’ll see now you won’t finish that water.’ [ye03cd.133]
\z

The discussion in this section has shown that labile verbs may be classified into five semantic and two form classes. I have also mentioned that the different semantic classes appear in their intransitive and transitive variants with differing likelihood. The factor that determines to a great part the distribution of labile verbs over the two clause types is “agentivity”{\fff}. On one end we find property items, change-of-state verbs denoting body state{\fff}s (e.g. \textit{táya} ‘be tired’) and natural states (e.g \textit{dráy} ‘be dry’), experiential verbs denoting body processes and movements (e.g. \textit{rɛ́s} ‘rest’), mental state verbs (e.g. \textit{gládin} ‘be glad’), as well as aspectual verbs. In natural speech, these semantic (sub)classes share a higher likelihood of occuring in intransitive clauses rather than transitive ones. 


In contrast, “agentive” change-of-state verbs denoting “destruction” and “material transformation” (e.g. \textit{brék} ‘break’), experiential verbs denoting physical movement (e.g. \textit{múf} ‘move’), and the entire class of locative verbs\is{locative verbs} (e.g. \textit{slíp} ‘sleep, lie’) generally occur with equal likelihood in both transitive and intransitive clauses.\is{labile verbs}


\subsection{Expletive verbs}\label{sec:9.2.4}

Expletive verbs take the dependent pronoun \textit{e} ‘\textsc{3sg.sbj}’ or a generic noun\is{generic nouns} as an expletive subject\is{subjects}. However, none of the verbs covered in the following exclusively occur with expletive subjects. The expletive subject is a core participant\is{core participants} in syntactic terms, but it has no referential quality and appears in constructions which require the subject position to be filled. 


Such dummy subject (pro)nouns are found with verbs with copula functions, with evaluative verbs, with Spanish-origin verbs which take expletive subjects in Spanish, and a weather verb (cf. \sectref{sec:9.3.7} for a separate treatment of the weather verb \textit{fɔ́l} ‘(to) rain’). All elements in the corpus which may take expletive subjects are listed in \tabref{tab:key:9.6}. The copula verbs \textit{bí} \textsc{‘be’,} \textit{dé} \textsc{‘be.loc’} and the focus markers \textit{cum} copulas \textit{na/nóto} occur in copula clauses with expletive subjects.


%%please move \begin{table} just above \begin{tabular
\begin{table}
\caption{Expletive verbs}
\label{tab:key:9.6}

\begin{tabularx}{\textwidth}{XXX}
\lsptoprule

Types & Verbs & Gloss\\
\midrule
Copula elements & \itshape \textit{bí} & ‘\textsc{be’}\\
& \itshape \textit{na/nóto}  & \textsc{‘foc/neg.foc’}\\
& \itshape \textit{dé}  & \textsc{‘be.loc’}\\
& \itshape \textit{fíba} & ‘seem’\\
& \itshape gɛ́t & ‘get, have, exist’\\
& \itshape \textit{níd} & ‘need, be necessary’\\
& \itshape lɛ́f & ‘leave, remain’\\
& \itshape \textit{sté} & ‘last’\\
\tablevspace
Evaluative verbs & \itshape \textit{bád} & ‘be bad’\\
& \itshape \textit{fáyn} & ‘be fine’\\
& \itshape \textit{gúd} & ‘be good’\\
& \itshape \textit{hád} & ‘be hard’\\
& \itshape \textit{ísi} & ‘be easy’\\
& \itshape \textit{nó smɔ́l} & ‘be considerable’\\
\tablevspace
Spanish expletive verbs & \itshape \textit{falta} & ‘lack’\\
& \itshape \textit{sigue} & ‘follow’\\
\tablevspace
Weather verb & \itshape \textit{fɔ́l} & ‘rain’\\
\lspbottomrule
\end{tabularx}
\end{table}
 


Sentence \REF{ex:key:1133} illustrates the expletive use of the locative-existential copula\is{copula:locative-existential} \textit{dé} in the factive clause \textit{e dé sé} ‘it’s that’. The second occurrence of \textit{dé} also shows that when the copula \textit{dé} functions as the predicate of an existential clause, the existing entity (i.e. \textit{sɔn wích} ‘(some) witches’) must be expressed as the subject of the clause. Hence, existential clauses featuring \textit{dé} have no expletive subjects: 



\ea%1133
    \label{ex:key:1133}
    \gll \textbf{E}    \textbf{dé}    \textbf{sé},    yu  sabí    sé    yá    só
\textbf{sɔn}    \textbf{wích}    \textbf{dé}    nɔ́.\\
\textsc{3sg.sbj}  \textsc{be.loc}  \textsc{quot}    \textsc{2sg}  know  \textsc{quot}    here    like.that
some  sorcerer    \textsc{be.loc}  \textsc{intj}\\

\glt ‘It is that, you know that here there are sorcerers, right.’ [ed03sb 093]
\z

There is no difference in meaning between the use of \textit{dé} \textsc{‘be.loc’} and the identity copulas \textit{bí} and \textit{na}/\textit{nóto} in factive clauses like the following two. However, contrary to other elements with expletive subjects, \textit{na}\textit{\textup{/}}\textit{nóto} may never occur with the dummy\is{dummy nouns} subject \textit{e} ‘\textsc{3sg.sbj}’ in factive clauses, nor be preceded by an emphatic pronoun as is the case in equative clauses\is{equative clauses} (cf. \sectref{sec:7.6.1}). This is so because the focus markers/identity copulas \textit{na}\textit{\textup{/}}\textit{nóto} incorporate \textsc{3sg} reference (cf. \sectref{sec:7.6.1}). The identity copula \textit{bí} may also appear with an expletive subject in factive clauses in the place of \textit{dé} ‘\textsc{be.loc}’ \REF{ex:key:1134}:


\ea%1134
    \label{ex:key:1134}
    \gll \textbf{E}    \textbf{fít}  \textbf{bí}  sé    na  paludismo.\\
\textsc{3sg.sbj}  can  \textsc{be}  \textsc{quot}    \textsc{foc}  malaria\\

\glt ‘It could be that it’s malaria.’ [fr03wt 058]
\z

In these functions, \textit{bí} and \textit{dé} are also used as introductory formulas of narratives with the meaning ‘it came to pass that’ \REF{ex:key:1135}:


\ea%1135
    \label{ex:key:1135}
    \gll \textbf{E}    \textbf{kán}  \textbf{bí}  \textbf{sé}    mi    abuela,    wé  a
bin  smɔ́l,  e    gó  ríba    \op...\cp{}\\
\textsc{3sg.sbj}  \textsc{pfv}  \textsc{be}  \textsc{quot}    \textsc{1sg.poss}  grandmother  \textsc{sub}  \textsc{1sg.sbj}
\textsc{pst}  small  \textsc{3sg.sbj}  go  river\\

\glt ‘It came to pass that my grandmother, when I was small, 
she went to the river (...)’ [ed03sb 015]
\z

Factive clauses can alternatively be formed with the help of two semantically empty dummy nouns, the generic noun \textit{tín} ‘thing’ and the noun \textit{kés} ‘matter’:


\ea%1136
    \label{ex:key:1136}
    \gll \textbf{Di}  \textbf{tín}    \textbf{dé}    \textbf{sé},    mék    e    mék    rabia  wet    mí.\\
\textsc{def}  thing  \textsc{be.loc}  \textsc{quot}    \textsc{sbjv}    \textsc{3sg.sbj}  make  anger  with    \textsc{1sg.indp}\\

\glt ‘The thing is let her be angry with me.’ [ye05rr 001]
\z


\ea%1137
    \label{ex:key:1137}
    \gll \textbf{Di}  \textbf{kés}    \textbf{dé}    \textbf{sé},    dís  dé  dɛn  a    nó  gɛ́t  tɛ́n
fɔ  wók.\\
\textsc{def}  matter  \textsc{be.loc}  \textsc{quot}    this  day  \textsc{pl}  \textsc{1sg.sbj}  \textsc{neg}  get  time
\textsc{prep}  work\\
\glt ‘It’s that these days I do not have time to work.’ [ro05ee 036]
\z

Also compare the cleft focus construction in \REF{ex:key:1138} featuring \textit{di tín} ‘the thing’ with the functionally equivalent construction in \REF{ex:key:1139}, featuring the expletive pronoun \textit{e}:


\ea%1138
    \label{ex:key:1138}
    \gll Na  só    \textbf{di}  \textbf{tín}    \textbf{dé}.\\
\textsc{foc}  like.that  \textsc{def}  thing  \textsc{be.loc}\\

\glt ‘That’s how the thing [it] is.’ [sab07fn 104]
\z


\ea%1139
    \label{ex:key:1139}
    \gll Na  só    \textbf{e}    \textbf{dé}.\\
\textsc{foc}  like.that  \textsc{3sg.sbj}  \textsc{be.loc}\\

\glt ‘That’s how it is.’ [ma03hm 077]
\z

When the verb \textit{fíba} ‘resemble’ occurs in a transitive clause, the \textsc{3sg.sbj} pronoun is not expletive \REF{ex:key:1140}: 


\ea%1140
    \label{ex:key:1140}
    \gll \textbf{E}    \textbf{fíba}    dɛ́bul.\\
\textsc{3sg.sbj}  seem  devil\\

\glt ‘He resembles a devil.’ [ra07fn 072]
\z

In contrast, when used intransitively, \textit{fíba} is best translated as ‘seem’ \REF{ex:key:1141} and may take a complement clause\is{complement clauses} \REF{ex:key:1142}. In such contexts, \textit{fíba} also takes an expletive \textsc{3sg.sbj} pronoun:


\ea%1141
    \label{ex:key:1141}
    \gll \textbf{E}    \textbf{fíba}    só.\\
\textsc{3sg.sbj}  seem  like.that\\

\glt ‘It seems so.’ [dj07ae 252]
\z


\ea%1142
    \label{ex:key:1142}
    \gll \textbf{E}    \textbf{fíba}    \textbf{sé}    nóto  yú    wán    dɛn  tíf
na  dí  kwáta.\\
\textsc{3sg.sbj}  seem  \textsc{quot}    \textsc{neg}.\textsc{foc}  \textsc{2sg.indp}  one    \textsc{3pl}  steal
\textsc{loc}  this  quarter\\

\glt ‘It seems that it’s not you alone they stole from in this neighbourhood.’ [ge07fn 165]
\z

The verb \textit{lɛ́f} ‘leave, remain’ occurs as a copula verb with an expletive subject in clauses like the following one:


\ea%1143
    \label{ex:key:1143}
    \gll \textbf{E}    \textbf{lɛ́f}    wán    pɔ́sin\\
\textsc{3sg.sbj}  leave  one    person\\

\glt ‘There is one person remaining.’
\z

The verb \textit{sté} ‘stay, last (a long time)’ also functions as a copula element in intransitive clauses \REF{ex:key:1144}. Both verbs occur with expletive \textit{e} in their copula function. Also consider \textit{níd} ‘need, be necessary’ \REF{ex:key:1145}:


\ea%1144
    \label{ex:key:1144}
    \gll \textbf{E}    \textbf{dɔ́n}  \textbf{sté},    a    tínk    sé    e    dɔ́n  sté
wé  una  bin  gɛ́t  insecticida  yá.\\
\textsc{3sg.sbj}  \textsc{prf}  be.long  \textsc{1sg.sbj}  think  \textsc{quot}    \textsc{3sg.sbj}  \textsc{prf}  last
\textsc{sub}  \textsc{2pl}  \textsc{pst}  get  insecticide  here\\

\glt ‘It’s been long, I think it’s been long since you [\textsc{pl}] have had insecticide 
[sprayed] here.’ [fr03wt 059]
\z


\ea%1145
    \label{ex:key:1145}
    \gll \textbf{E}   \textbf{níd}    \textbf{sé}    mék    a    gó  dé    tumɔ́ro.\\
\textsc{3sg.sbj}  need  \textsc{quot}    \textsc{sbjv}    \textsc{1sg.sbj}  go  there  tomorrow\\

\glt ‘It is necessary that I go there tomorrow.’ [dj07ae 512]
\z

Evaluative verbs also take expletive subjects. Examples follow with the property items \textit{hád} ‘(be) hard’ \REF{ex:key:1146} and \textit{fáyn} ‘(be) fine’ \REF{ex:key:1147}: 


\ea%1146
    \label{ex:key:1146}
    \gll \textbf{E}    \textbf{hád}    fɔ  bíl    na  yá    bikɔs  sé    di  grɔ́n 
gɛ́t  bɔkú  sansán.\\
\textsc{3sg.sbj}  hard  \textsc{prep}  build  \textsc{loc}  here    because  \textsc{quot}    \textsc{def}  ground
get  much  sand\\

\glt ‘It’s hard to build here because the ground is very sandy.’ [ro05ee 063]
\z


\ea%1147
    \label{ex:key:1147}
    \gll \textbf{E}    \textbf{fáyn} fɔ  dríng  smɔ́l-wán.\\
\textsc{3sg.sbj}  fine    \textsc{prep}  drink  small\textsc{{}-adv}\\

\glt ‘It’s good to drink moderately.’ [ma03hm 071]
\z

The two verbs \textit{falta} ‘lack’ (< Sp. \textit{faltar} ‘lack’) and \textit{sigue} ‘continue, follow’ (< Sp. \textit{seguir} ‘continue, follow’) are established loans of \ili{Spanish} origin which have been borrowed\is{borrowing} together with their selectional properties. Like their Spanish etymons these two verbs require expletive subjects. Contrary to Spanish, subject pronouns are not normally dropped in Pichi. Examples \REF{ex:key:1148} and \REF{ex:key:1149} therefore feature the expletive pronoun \textit{e} \textsc{‘3sg.sbj’}:


\ea%1148
    \label{ex:key:1148}
    \gll \textbf{E}    \textbf{falta}  mɔní  fɔ  púl    saco    dɛn  dé    fɔ
kɛ́r=an    na  hós.\\
\textsc{3sg.sbj}  lack    money  \textsc{prep}  remove  bag    \textsc{pl}  there  \textsc{prep}
carry \textsc{3sg.obj}  \textsc{loc}  house\\
\glt ‘There is money lacking to remove the bags there in order to bring 
them to the house.’ [ye03cd 004]
\z


\ea%1149
    \label{ex:key:1149}
    \gll Porque  \textbf{e}    \textbf{de}  \textbf{sigue}  wán  bád  smɛ́l.\\
because  \textsc{3sg.sbj}  \textsc{ipfv}  follow  one  bad  smell\\

\glt ‘Because there follows a bad smell.’   [dj03do 049]
\z

Yet \textit{falta} and \textit{sigue} may also take referentially full subjects in intransitive clauses. Compare \textit{falta} in \REF{ex:key:1148} and \REF{ex:key:1150}, and \textit{sigue} in \REF{ex:key:1149} and \REF{ex:key:1151}:


\ea%1150
    \label{ex:key:1150}
    \gll \textsc{E}    sé    \textbf{e}    nó  fít    \textbf{falta}.\\
\textsc{3sg.sbj}  \textsc{quot}    \textsc{3sg.sbj}  \textsc{neg}  can    lack\\

\glt ‘She said she can’t be absent.’ [ma03hm 014]
\z


\ea%1151
    \label{ex:key:1151}
    \gll Mi    rabia  dɔ́n  fínis    bɔt  wé  yu  gó
\textbf{e}    \textbf{de}  \textbf{sigue}  mɔ́.\\
\textsc{1sg.poss}  anger  \textsc{prf}  finish  but  \textsc{sub}  \textsc{2sg}  go
\textsc{3sg.sbj}  \textsc{ipfv}  follow  more\\

\glt ‘My anger is over, but when you go, it continues.’ [ro05rr 003]
\z

At this point, a word is in order on the raising properties of expletive verbs. In \REF{ex:key:1152} below, the \textsc{3sg.sbj} pronoun \textit{e} anaphorically refers to \textit{dɛn yón} ‘theirs’, so the object of the complement clause beginning with the non-finite complementiser \textit{fɔ} ‘\textsc{prep}’ has been raised into subject position in the main clause. For other speakers, however, raising is not accepted with evaluative verbs \REF{ex:key:1153}: 


\ea%1152
    \label{ex:key:1152}
    \gll \op...\cp{}  bɔt  \textbf{dɛn}  \textbf{yón}     fáyn,  \textbf{e}    fáyn    fɔ  sí.\\
  {} but  \textsc{3pl}  own    fine    \textsc{3sg.sbj}  fine    \textsc{prep}  see\\

\glt ‘(...) but theirs is beautiful, it [the wedding ceremony] is beautiful to see.’ [hi03cb 005]
\z


\ea[*]{%1153
    \label{ex:key:1153}
    \gll Dán  sáy  \textbf{fáyn}  \textbf{fɔ}  \textbf{sí}.\\
 that  side  fine    \textsc{prep}  see\\
\glt Intended: ‘That place is nice to see.’ [eb07fn]
}\z

The verb \textit{fíba} ‘resemble’ takes full complement clauses\is{complement clauses} introduced by the complementiser \textit{sé} \REF{ex:key:1154}. Neither reduced \textit{fɔ}{}-complement clauses nor SVCs are accepted in clause linkage. Hence, an SVC like \REF{ex:key:1155} is ungrammatical:


\ea%1154
    \label{ex:key:1154}
    \gll \textbf{E}    \textbf{fíba}    \textbf{sé}    Boyé  gɛ́t  mɔní.\\
\textsc{3sg.sbj}  seem  \textsc{quot}    \textsc{name}  get  money\\

\glt ‘It seems that Boyé has money.’ [dj07ae 255]
\z


\ea[*]{%1155
    \label{ex:key:1155}
    \gll Boyé  \textbf{fíba}    \textbf{gɛ́t}  mɔní.\\
  \textsc{name}  seem  get  money\\
\glt Intended: ‘Boyé seems to have money.’ [dj07ae 254]
}\z

However, the subject of the complement clause may be raised into subject position of the main clause without any structural change. The result is an idiosyncratic structure, in which the coreferential subjects of the main and subordinate clauses are both overtly expressed \REF{ex:key:1156}:


\ea%1156
    \label{ex:key:1156}
    \gll Boyé  \textbf{fíba}    \textbf{sé}    \textbf{e}    gɛ́t  mɔní.\\
\textsc{name}  seem  \textsc{quot}    \textsc{3sg.sbj}  get  money\\

\glt ‘Boyé seems to have money.’ [\textit{Lit.} ‘Boyé seems that he has money.’] [dj07ae 256]
\z

Existential constructions featuring expletive subjects and the verb \textit{gɛ́t} ‘get, exist’ are covered in detail in section \sectref{sec:7.6.3}.

\section{Valency}\label{sec:9.3}

I now turn to describing valency in select types of Pichi constructions. I cover the grammatical relations mediated by verbs in these constructions as well as the semantic roles assigned to the core participant\is{core participants}s subject and object. I also treat valency in two semantic fields, namely in the expression of weather phenomena and body state\is{body states}s. These fields are of particular interest due to the variety of valency configurations found in the clauses used to express them.\is{verb classes}

\subsection{Light verb constructions}\label{sec:9.3.1}

Pichi features numerous more or less conventionalised collocations that involve verbs with a fairly general meaning followed by undergoer objects. Many of these collocations appear to be light verb constructions, in which the bulk of semantic content is carried by the object rather than the verb. The most common of these light verb constructions are provided in \tabref{tab:key:9.7}. The most common constructions involve the verbs \textit{gɛ́t} ‘get, have’, \textit{gí} ‘give’, \textit{mék} ‘make’ and \textit{púl} ‘pull, remove’. At the lower end of the table, we find constructions involving verbs which are only found in a single collocation. 

\begin{table}
\small	
\caption{Light verb constructions}
\label{tab:key:9.7}
% TODO: This table is too high
\begin{tabularx}{\textwidth}{lll}
\lsptoprule
Verb & Object & Translation\\
\midrule
\textstyleTablePichiZchn{gɛ́t} ‘get, have’ & \textstyleTablePichiZchn{bɛlɛ́} ‘belly’ & ‘be pregnant’\\
& \textstyleTablePichiZchn{páwa} ‘strength’ & ‘be strong, be potent’\\
& \textstyleTablePichiZchn{líba} ‘liver’ & ‘have guts’\\
& \textstyleTablePichiZchn{mɔní} ‘money’ & ‘be rich’\\
& \textstyleTablePichiZchn{pikín} ‘child’ & ‘have children’\\
& \textstyleTablePichiZchn{lɔ́ki} ‘luck’ & ‘be lucky’\\
& \textstyleTablePichiZchn{bad-lɔ́k} ‘bad luck’ & ‘have bad luck’\\
& \textstyleTablePichiZchn{ráyt} ‘right’ & ‘be right’\\
& \textstyleTablePichiZchn{bád fásin} ‘bad ways’ & ‘be ill-mannered’\\
& \textstyleTablePichiZchn{trót} ‘throat’ & ‘have appetite, be lusty’\\
& \textstyleTablePichiZchn{bɔ́di} ‘body’ & ‘be chubby’\\
& \textstyleTablePichiZchn{rabia} ‘anger’ & ‘be angry’\\
& \textstyleTablePichiZchn{novio}\textit{/mán} ‘boyfriend/husband’ & ‘have a boyfriend/husband’\\
& \textstyleTablePichiZchn{novia/húman} ‘girlfriend/wife’ & ‘have a girlfriend/wife’\\
\textstyleTablePichiZchn{gí} \textstyleTableEnglishZchn{‘give’} & \textstyleTablePichiZchn{bɔbí} ‘breast’ & ‘breastfeed’\\
& \textstyleTablePichiZchn{chɔ́p} ‘food’ & ‘feed’\\
& \textit{bɛlɛ́} ‘belly’ & ‘impregnate’\\
& \textstyleTablePichiZchn{hán} ‘hand’ & ‘shake hands’\\
& \textstyleTablePichiZchn{skúl} ‘school’ & ‘give lessons’\\
& \textstyleTablePichiZchn{wán vuelta} ‘a walk’ & ‘take a walk’\\
& \textstyleTablePichiZchn{permiso} ‘permission’ & ‘give permission’\\
\textstyleTablePichiZchn{mék} \textstyleTableEnglishZchn{‘make’} & \textstyleTablePichiZchn{bigdé} ‘festivity’ & ‘have a festivity’\\
& \textstyleTablePichiZchn{chɔ́ch} ‘church’ & ‘celebrate a mass’\\
& \textstyleTablePichiZchn{fám} ‘farm’ & ‘(to) farm’\\
& \textstyleTablePichiZchn{chɔ́p} ‘food’ & ‘prepare food’\\
& \textstyleTablePichiZchn{jɔmba} ‘lover’ & ‘make love’\\
& \textstyleTablePichiZchn{wayó} ‘cunning’ & ‘be cunning, employ trickery’\\
& \textstyleTablePichiZchn{wuruwúrú} ‘confusion’ & ‘cause confusion’\\
& \textit{affaire} ‘affair’ & ‘have an affair’\\
& \textstyleTablePichiZchn{rabia} ‘anger’ & ‘be angry with somebody’\\
\textstyleTablePichiZchn{púl} \textstyleTableEnglishZchn{‘remove’} & bɛlɛ́ ‘pregnancy’ & ‘abort’\\
& \textstyleTablePichiZchn{brís} ‘air’ & ‘breathe’\\
& \textstyleTablePichiZchn{fotó} ‘photo’ & ‘take a picture’\\
& \textstyleTablePichiZchn{torí} ‘story’ & ‘narrate a story, converse’\\
\textstyleTablePichiZchn{sí}\textstyleTableEnglishZchn{ ‘see’} & \textit{tɛ́n} ‘time’ & ‘menstruate’\\
& \textstyleTablePichiZchn{tín} ‘thing’ & ‘have experience in life’\\
\textstyleTablePichiZchn{fála} \textstyleTableEnglishZchn{‘follow’} & \textstyleTablePichiZchn{húman} ‘woman’ & ‘womanise’\\
\textstyleTablePichiZchn{gó} \textstyleTableEnglishZchn{‘go’} & \textstyleTablePichiZchn{skúl} ‘school’ & ‘attend school’\\
\textstyleTablePichiZchn{kɔ́t} \textstyleTableEnglishZchn{‘cut’} & \textstyleTablePichiZchn{miná} ‘penis’ & ‘circumcise’\\
\textstyleTablePichiZchn{kráy} \textstyleTableEnglishZchn{‘cry’} & \textstyleTablePichiZchn{wɔtá} ‘water’ & ‘shed tears’\\
\textstyleTablePichiZchn{pík}\textstyleTableEnglishZchn{ ‘pick’} & \textstyleTablePichiZchn{mɔ́t} ‘mouth’ & ‘sound somebody out’\\
\textstyleTablePichiZchn{ték} \textstyleTableEnglishZchn{‘take’} & \textstyleTablePichiZchn{bɔ́di/skín} ‘body’ & ‘gain weight’\\
\lspbottomrule
\end{tabularx}
\end{table}
A good number of the constructions listed above constitute borderline cases between ordinary verb-noun collocations assembled by phrasal syntax and conventionalised or lexicalised verb-noun collocations. Two criteria may be useful in determining which of these construction are conventionalised to the point of qualifying as light verb constructions. Firstly, the object in more conventionalised collocations has a tendency to occur bare. Secondly, there is a relatively stringent restriction on pronominalising light verb objects. Some salient characteristics of light verb constructions are explored in the following by means of constructions involving the verb \textit{gɛ́t} ‘get, have’.


The verb \textit{gɛ́t} is an inchoative-stative transitive verb, which occurs with a stative \REF{ex:key:1157}, and at other times, an inchoative reading \REF{ex:key:1158}. The verb also has various functions as an existential and modal verb and expresses possession{\fff} (cf. e.g. \ref{ex:key:1159} below).



\ea%1157
    \label{ex:key:1157}
    \gll Dɛn  \textbf{gɛ́t}  wók    náw    ó.\\
\textsc{3pl}  get  work  now    \textsc{sp}\\

\glt ‘They actually have work now.’ [to03gm 008]
\z


\ea%1158
    \label{ex:key:1158}
    \gll E    \textbf{gɛ́t}  ɔ́da    húman  ‘chíp’  bɔkú  problema,  \op...\cp{}\\
\textsc{3sg.sbj}  get  other  woman  \textsc{[skt]}    much  problem\\

\glt ‘He got (himself) another woman, many problems, (...)’ [ma03ni 025]
\z

Ordinary objects of \textit{gɛ́t} may occur bare or be preceded by determiners depending on pragmatic circumstance. In \REF{ex:key:1159}, the non-specific noun \textit{bɔ́y} ‘boy’ is preceded by the indefinite determiner wán. In contrast, non-specific objects of light verbs have a strong tendency to occur devoid of any definiteness{\fff} marking. In \REF{ex:key:1159}, the noun \textit{mán} ‘man’ of the light verb construction \textit{gɛ́t mán} ‘have a man/husband’ remains bare: 


\ea%1159
    \label{ex:key:1159}
    \gll Smɔ́l  gál,  ɛf  e    nó  \textbf{gɛ́t}  \textbf{wán}  \textbf{bɔ́y},    e    nó  \textbf{gɛ́t}  \textbf{mán},
pero  di  húman,  di  bíg  wán    dɛn  sɛ́f.\\
small  girl  if  \textsc{3sg.sbj}  \textsc{neg}  get  one  boy    \textsc{3sg.sbj}  \textsc{neg}  get  man
but    \textsc{def}  woman  \textsc{def}  big  one    \textsc{pl}  \textsc{emp}\\

\glt ‘As for young girls, if they don’t have a boy-friend, if they don’t have a man
[they feel worthless], but even women, the grown ones themselves.’ [hi03cb 154]
\z

Objects of \textit{gɛ́t} may be pluralised with a post-posed \textit{dɛn} \REF{ex:key:1160} and may occur with prenominal modifiers like \textit{bɔkú} ‘much’ or the \textsc{1sg} possessive pronoun \textit{mi} \REF{ex:key:1160}. The verb \textit{gɛ́t} may also take pronominal objects \REF{ex:key:1161}, or occur with no overt object at all where reference has been established earlier on \REF{ex:key:1162}.


\ea%1160
    \label{ex:key:1160}
    \gll Bikɔs  a    \textbf{gɛ́t}  bɔkú  mi    kɔntri-mán    \textbf{dɛn}  \op...\cp{}\\
because  \textsc{1sg.sbj}  get  much  \textsc{1sg.poss}  country\textsc{.cpd}{}-man  \textsc{pl}\\

\glt ‘Because I have many of my countrymen (...)’ [ed03sb 157]
\z


\ea%1161
    \label{ex:key:1161}
    \gll \op...\cp{}  dán  mɔní  a    fít  \textbf{gɛ́t=an}    un  mes.\\
  {} that  money  \textsc{1sg.sbj}  can  get=\textsc{3sg.obj}  \textsc{def}  month\\

\glt ‘(...) as for that money, I can have it for a month.’ [ro05rt 050]
\z


\ea%1162
    \label{ex:key:1162}
    \gll \op...\cp{}  mébi  a    gɛ́t  pikín,  wé  mébi  a    nó  \textbf{gɛ́t}.\\
  {} maybe  \textsc{1sg.sbj}  get  child  \textsc{sub}  maybe  \textsc{1sg.sbj}  \textsc{neg}  get\\

\glt ‘(...) maybe I have children or maybe I don’t have [children].’ [hi03cb 158]
\z

We have seen that non-specific objects of light verbs tend to occur as bare nouns. Nonetheless, specific objects of \textit{gɛ́t} in light verb constructions may occur with determiners if so required. Compare \textit{di fɔ́s bɛlɛ́} ‘first pregnancy’ in \REF{ex:key:1163}:


\ea%1163
    \label{ex:key:1163}
    \gll Dásɔl  wé  a    dɔ́n  bíg  wé  a    fɔ́s  \textbf{gɛ́t}
\textbf{di}  \textbf{fɔ́s}  \textbf{bɛlɛ́}  \op...\cp{}\\
only    \textsc{sub}  \textsc{1sg.sbj}  \textsc{prf}  big  \textsc{sub}  \textsc{1sg.sbj}  first  get
\textsc{def}  first  belly\\

\glt ‘Then when I was grown, when I first had the first pregnancy (...)’ [ed03sb 017]
\z

The NP \textit{di fɔ́s bɛlɛ́} the first pregnancy’ in \REF{ex:key:1163} above also shows that objects of light verbs are encountered with prenominal modifiers. Likewise, object NPs in light verb constructions may be placed under focus \REF{ex:key:1164}. Although there are not many instances of pronominalised light verb objects in the data, these also occur. In \REF{ex:key:1165}, the object pronoun \textit{=an} substitutes for \textit{torí} ‘story’:


\ea%1164
    \label{ex:key:1164}
    \gll \textbf{Na}  \textbf{torí}    dɛn  de  \textbf{púl}.\\
\textsc{foc}  story  \textsc{3pl}  \textsc{ipfv}  pull\\

\glt ‘It’s a story that they’re telling.’ [au07se 009]
\z


\ea%1165
    \label{ex:key:1165}
    \gll A    go  \textbf{púl}  yú=\textbf{an}      tumɔ́ro.\\
\textsc{1sg.sbj}  \textsc{pot}  pull  \textsc{2sg.indp}=\textsc{3sg.obj}  tomorrow\\

\glt ‘I will narrate it to you tomorrow.’ [ye07de 018]
\z

\tabref{tab:key:9.8} presents a frequency analysis of \textit{gɛ́t} in verb-object collocations in a subcorpus of 30,000 words. The verb \textit{gɛ́t} enjoys a total number of 345 tokens, of which 136 tokens involve \textit{gɛ́t} as a modal verb and \textit{gɛ́t} without an overt object. In line (a) of the table, I give the remaining 209 tokens which represent uses of \textit{gɛ́t} in collocations involving full noun objects.


In line (b), I provide the total number of verb-noun collocations that do not qualify as light verb constructions according to the distributional criteria introduced above. Line (c) gives the total number of constructions that should be considered light verb constructions. I also list the four most frequent constructions with the corresponding tokens. I take care to distinguish cases in which a collocation like \textit{gɛ́t pikín} is employed with the general meaning of ‘have children’ from ones in which the collocation is used with a specific meaning like ‘have one, two, etc. children’. The corresponding percentages in relation to the total number of collocations in line (a) are given in the rightmost column.


%%please move \begin{table} just above \begin{tabular
\begin{table}
\caption{Frequency of \textit{gɛ́t} collocations}
\label{tab:key:9.8}

\begin{tabularx}{\textwidth}{lXrr}
\lsptoprule
& Construction & Total number & Percentage over (a.)\\
\midrule
a. & All \textit{gɛ́t} collocations & 209 & 100\%\\
\tablevspace 
b. & All ordinary \textit{gɛ́t} collocations & 140 & 67\%\\
\tablevspace
c. & All \textit{gɛ́t} light verb constructions & 69 & 33\%\\
   & \textit{gɛ́t pikín}   ‘have children’ & 33 & 16\%\\
   & \textit{gɛ́t mɔní}  ‘have money’ & 22 & 11\%\\
   & \textit{gɛ́t bɛlɛ́}  ‘be pregnant’ & 10 & 5\%\\
   & \textit{gɛ́t líba}   ‘have guts’ & 4 & 2\%\\
\lspbottomrule
\end{tabularx}
\end{table}
\tabref{tab:key:9.8} reveals that light verb constructions proper represent 33 per cent of the total number of occurrences of collocations involving \textit{gɛ́t} and an object. Of all \textit{gɛ́t}-constructions contained in the corpus, \textit{gɛ́t pikín} ‘have children’ is the most frequent one and accounts for 16 per cent of the total of light verb constructions. An additional information of interest is that the total number of types (different constructions) of light verb construction amounts to eleven, seven of which occur only once each. In view of these facts, I assume that the functional load of \textit{gɛ́t} as a light verb is only moderate.


Next to the borrowing\is{borrowing} of \ili{Spanish} verbs, verb-noun collocations consisting of a Pichi verb and a Spanish noun are an important means of extending the lexicon. codemixed constructions allow speakers to tap into the nominal lexicon of Spanish in order to derive new “verbal” meanings. These constructions are characterised by a high degree of structural equivalence between Pichi and Spanish. Not only is the order of constituents in verb-noun collocations the same in both languages, the meanings of the light verbs employed in the respective languages are also highly compatible with each other. There is therefore a strong tendency towards convergence in codemixed collocations. Accordingly, the verbs in these collocations may have the selectional characteristics of the Pichi verb in one instance, while in another, the Pichi verb may select its complement\is{complements} as if it were the synonymous Spanish verb (cf. \citealt[184]{Muysken2000})



For instance, none of the nouns in the collocations \textit{gɛ́t rabia} ‘be angry’ \REF{ex:key:1166}, \textit{gɛ́t novio} ‘have a boyfriend’ \REF{ex:key:1167} and \textit{gí permiso} ‘give permission’ \REF{ex:key:1168} are encountered with a determiner in the corpus. The meanings of the verbs and the distribution of nouns in these constructions are identical to those in the \ili{Spanish} equivalents \textit{tener rabia} ‘be angry’, \textit{tener novio} ‘have a boyfriend’, and \textit{dar permiso} ‘give permission’. 



\ea%1166
    \label{ex:key:1166}
    \gll If  a    \textbf{gɛ́t}  \textbf{rabia}  wet    yú,    \op...\cp{}\\
if  \textsc{1sg.sbj}  get  anger  with    \textsc{2sg.indp}\\
\glt ‘If I’m angry with you (...)’  [ro05rr 002]
\z

\ea%1167
    \label{ex:key:1167}
    \gll Mék  yu  nó    sé    yu  dɔ́n  \textbf{gɛ́t}  \textbf{novio}    na
pueblo,  na  kɔ́ntri.\\
\textsc{sbjv}  \textsc{2sg}  know  \textsc{quot}    \textsc{2sg}  \textsc{prf}  get  boyfriend  \textsc{loc}
village  \textsc{loc}  country\\

\glt ‘You should know that you already have a fiancé in the village, 
in the hometown.’ [ab03ay 010]
\z


\ea%1168
    \label{ex:key:1168}
    \gll If  di  fámbul  nó  \textbf{gí}  yú    \textbf{permiso}    \op...\cp{}\\
if  \textsc{def}  family  \textsc{neg}  give  \textsc{2sg.indp}  permission\\

\glt ‘If the family doesn’t give you permission (...)’ [ed03sb 076]
\z

On the other hand, there are established mixed collocations which feature a determiner. One of these is \textit{gí wán vuelta} ‘take a walk’. Like the Pichi verb \textit{gí} ‘give’, the Spanish verb \textit{dar} ‘give’ selects a determined object in the expression \textit{dar una vuelta} ‘take a walk’: 


\ea%1169
    \label{ex:key:1169}
    \gll E    de  \textbf{gí}  \textbf{wán}    \textbf{vuelta}  kwík\\
\textsc{3sg.sbj}  \textsc{ipfv}  give  one    round  quickly\\

\glt ‘She’s doing a round quickly.’ [dj05be 120]
\z

Other codemixed collocations are further removed from the pole of light verb constructions. The collocation \textit{gí beca} ‘give a scholarship’ (\ref{ex:key:1170}–\ref{ex:key:1171}) occurs with or without determiners in accordance with the referential properties of the \textsc{NP}: 


\ea%1170
    \label{ex:key:1170}
    \gll Dɛn  bin  \textbf{gí}  mí    \textbf{beca}.\\
\textsc{3pl}  \textsc{pst}  give  \textsc{1sg.indp}  scholarship\\

\glt ‘I was given a scholarship.’ [ed03sp 057]
\z


\ea%1171
    \label{ex:key:1171}
    \gll E    \textbf{gí}  mí    \textbf{di}  \textbf{beca}      a    gó.\\
\textsc{3sg.sbj}  give  \textsc{1sg.indp}  \textsc{def}  scholarship  \textsc{1sg.sbj}  go\\

\glt ‘He gave me the scholarship (and) I went.’ [ed03sp 065]
\z

In sum, we can observe that next to a few “proper” light verb constructions, Pichi makes use of less tightly integrated collocations featuring Pichi or \ili{Spanish} nouns by means of ordinary phrasal syntax. These constructions are flexible, allow the insertion of functional elements and modifiers, as well as object substitution by pronouns.\is{light verb constructions}

\subsection{Associative objects}\label{sec:9.3.2}

In Pichi, syntactic objects can denote various less central semantic roles which may alternatively be expressed through prepositional phrases\is{prepositional phrases}. Accordingly, associative objects appear to the right of patient\is{patient} objects in double-object constructions (cf. \sectref{sec:9.3.4}), in a position usually reserved for adverbial adjuncts\is{adjuncts}. An associative object is an instantiation of some entity typically associated with the situation denoted by the verb. Associative objects in Pichi are reminiscent of inherent object constructions as found in the Kwa languages of West Africa (see \citealt{Essegbey1999} for Ewe). Contrary to inherent objects, however, associative objects are not obligatory and may remain unexpressed at all times. Equally, associative objects usually only occur with specific verbs (cf. e.g. \ref{ex:key:1179}). The verb-object collocations described in this section therefore appear to involve specialisation or lexicalisation. The use of associative objects can therefore only serve as a productive means of increasing verb valency with the verbs listed in \tabref{tab:key:9.9}. 


Here follows an example with the verb \textit{wás} ‘wash (oneself)’ and its associative object \textit{wɔtá} ‘water’. The pragmatic context coerces a semantic role of instrument or means on the associative object: 



\ea%1172
    \label{ex:key:1172}
    \gll A    de  \textbf{wás}    \textbf{wɔtá}.\\
\textsc{1sg.sbj}  \textsc{ipfv}  wash  water\\

\glt ‘I’m washing (myself with) water.’ [dj07ae 274]
\z

All verb-noun collocations involving associative objects in the corpus are listed in \tabref{tab:key:9.9}. In most cases, the verb-noun combination given in the table is the preferred means of expressing the corresponding semantic relation between the verb and object listed.

%%please move \begin{table} just above \begin{tabular
\begin{table}
\caption{Associative objects}
\label{tab:key:9.9}

\begin{tabularx}{\textwidth}{llll}
\lsptoprule

Verb & Object & Gloss & Semantic role of object\\
\midrule
\itshape fúlɔp & \itshape pípul & ‘be full of/fill with people’ & Content\\
\itshape fúlɔp & \itshape watá & ‘be full of/fill with water’ & \\
\itshape pák & \itshape polvo & ‘be full of/fill with dust’ & \\
\itshape bít & \itshape stík & ‘beat with a stick’ & Instrument\\
\itshape cháp & \itshape kɔ́tlas & ‘chop (off) with a cutlass’ & \\
\itshape chúk & \itshape nɛ́f & ‘stab with a knife’ & \\
\itshape chúk & \itshape nidul & ‘sting with a needle’ & \\
\itshape chúk & \itshape injección & ‘give an injection’ & \\
\itshape sút & \itshape pistola & ‘shoot with a pistol’ & \\
\itshape wás & \itshape wɔtá & ‘wash oneself with water’ & \\

\tablevspace
\itshape invita & \itshape Guinness & ‘invite for a Guinness’ & Purpose\\
\itshape kapú & \itshape hós & ‘fight over a house’ & \\
\itshape wók & \itshape mɔní & ‘work for money’ & \\

\tablevspace
\itshape fɔdɔ́n & \itshape stík & ‘fall from a tree’ & Source\\
\itshape púl & \itshape wók & ‘sack from work’ & \\
\itshape smɛ́l & \itshape chɔ́p & ‘smell of food’ & \\

\tablevspace
\itshape kɔmɔ́t & \itshape pɔ́sin & ‘become a responsible person’ & Goal\\
\itshape tɔ́n & \itshape pɔ́sin & ‘turn into a person’ & \\
\itshape pré & \itshape gɔ́d & ‘pray to God’ & \\

\tablevspace
\itshape kráy & \itshape mɔní & ‘cry over (lost) money’ & Cause\\
\itshape sík & \itshape fíba & ‘be sick with fever’ & \\
\itshape sík & \itshape malérya & ‘be sick with malaria’ & \\
\itshape sík & \itshape tiphoïdea & ‘be sick with typhoid fever’ & \\

\tablevspace
\itshape báy & \itshape dos mil & ‘buy for two thousand Francs’ & Price\\
\itshape sɛ́l & \itshape dos mil & ‘sell for two thousand Francs’ & \\

\tablevspace
\itshape kɔ́l & \textsc{name} & ‘call something X’ & Reference\\
\lspbottomrule
\end{tabularx}
\end{table}
Associative objects are assigned a content role by the labile change-of-state verbs \textit{fúlɔp} ‘fill up’ \REF{ex:key:1173} and \textit{pák} ‘pack, fill up’ \REF{ex:key:1174}: 


\ea%1173
    \label{ex:key:1173}
    \gll Na  China  motó  dɛn  \textbf{fúlɔp}  \textbf{pípul}.\\
\textsc{loc}  \textsc{place}  car    \textsc{pl}  be.full  people\\

\glt ‘In China cars are full of people.’ [au07fn 107]
\z


\ea%1174
    \label{ex:key:1174}
    \gll ɔ́l  hía    \textbf{pák}    \textbf{polvo}.\\
all  here    pack  dust\\

\glt ‘Everywhere here is full of dust.’ [ge07fn 127]
\z

Content objects can be replaced by a corresponding prepositional phrase without a change in meaning. Compare the PPs introduced by \textit{wet} ‘with’ in \REF{ex:key:1175} and \REF{ex:key:1176}:\is{adjuncts}


\ea%1175
    \label{ex:key:1175}
    \gll Na  lɛk  sé    yu  \textbf{fúlɔp}  di  glás    \textbf{watá}?\\
\textsc{foc}  like  \textsc{quot}    \textsc{2sg}  fill    \textsc{def}  glass  water\\

\glt ‘That’s as if you fill up a glass with water?’ [dj07ae 066]
\z


\ea%1176
    \label{ex:key:1176}
    \gll E    \textbf{fúlɔp}  di  glás    \textbf{wet}    \textbf{watá}.\\
\textsc{3sg.sbj}  fill    \textsc{def}  glass  with    water\\

\glt ‘He filled the glass with water.’ [dj07ae 067]
\z

Instrument is among the most common semantic roles expressed by associative objects \REF{ex:key:1177}. The instrument role may also be expressed by a \textit{wet}{}-prepositional phrase \REF{ex:key:1178}:


\ea%1177
    \label{ex:key:1177}
    \gll Yu  fít  tɔ́k  sé   “dɛn    \textbf{chúk}=an    \textbf{nɛ́f}”.\\
\textsc{2sg}  can  talk  \textsc{quot}    \phantom{‘}\textsc{3pl}    {stab=\textsc{sg}.\textsc{obj}}  knife\\

\glt ‘You can say “he was stabbed with a knife”.’ [ro05ee 061]
\z


\ea%1178
    \label{ex:key:1178}
    \gll Dɛn  \textbf{chúk}  mí    \textbf{wet}    \textbf{nɛ́f}.\\
\textsc{3pl}  stab    \textsc{1sg.indp}  with    knife\\

\glt ‘I was stabbed with a knife.’ [ro05ee 060]
\z

It is noteworthy that many other verbs that assign an instrument role to a participant do not seem to take instrument associative objects; for example, \textit{kɔ́t} ‘cut’ is not attested with an associative object and requires the instrument to be expressed as a prepositional phrase: 


\ea[*]{%1179
    \label{ex:key:1179}
    \gll A    de  \textbf{kɔ́t}  di  tín    \textbf{sísɔs}.\\
 \textsc{1sg.sbj}  \textsc{ipfv}  cut  \textsc{def}  thing  scissors\\
\glt Intended: ‘I’m cutting the thing with a pair of scissors.’ [dj07ae 477]
}\z


\ea%1180
    \label{ex:key:1180}
    \gll \textbf{Kɔ́t}=an    \textbf{wet}    \textbf{sísɔs}!\\
cut=\textsc{3sg.obj}  with    scissors\\

\glt ‘Cut it with a pair of scissors!’ [dj07ae 478]
\z

Sentences \REF{ex:key:1181} and \REF{ex:key:1182} provide examples for the use of associative objects with the semantic role of purpose. These may equally be expressed through a prepositional phrase introduced by the associative preposition \textit{fɔ} \REF{ex:key:1182}: 


\ea%1181
    \label{ex:key:1181}
    \gll Yu  de  \textbf{kapú}    \textbf{hós}.\\
\textsc{2sg}  \textsc{ipfv}  fight.over  house\\

\glt ‘You’re fighting over a house.’ [to07fn 112]
\z


\ea%1182
    \label{ex:key:1182}
    \gll Dɛn  de  \textbf{kapú}    \textbf{fɔ} di  \textbf{hós}.\\
\textsc{3pl}  \textsc{ipfv}  fight.over  \textsc{prep}  \textstylePichiexamplenumberZchnZchn{\textsc{def}}  house\\

\glt ‘They’re fighting over the house.’ [ne07fn 025]
\z

The source\is{} of the motion verb\is{motion verbs} \textit{fɔdɔ́n} ‘fall’ may be realised as an associative object \REF{ex:key:1183}. Alternatively, the source may be indicated via the preposition \textit{frɔn} ‘from’ when it marks the ground \REF{ex:key:1184}. Note the possibility of additionally using the “at rest” locative noun\is{locative nouns} \textit{ɔ́p} ‘up(perside)’ to mark the ground in \REF{ex:key:1184}:


\ea%1183
    \label{ex:key:1183}
    \gll Di  pikín  \textbf{fɔdɔ́n}  \textbf{di}  \textbf{stík}.\\
\textsc{def}  child  fall    \textsc{def}  tree\\

\glt ‘The child fell from the tree.’ [ro05ee 097]
\z


\ea%1184
    \label{ex:key:1184}
    \gll Di  pikín  \textbf{fɔdɔ́n}  \textbf{frɔn}    \textbf{ɔ́p}  \textbf{di}  \textbf{stík},    frɔn    di  stík.\\
\textsc{def}  child  fall    from  up  \textsc{def}  tree    from  \textsc{def}  tree\\

\glt ‘The child fell from (up on) the tree, from the tree.’ [dj05be 201]
\z

The semantic role of the objects of \textit{smɛ́l} ‘smell’ can only be disambiguated by context. In \REF{ex:key:1185} the associative object \textit{chɔ́p} ‘food’ denotes the source of the sensation, in \REF{ex:key:1186}, \textit{chɔ́p} denotes the stimulus{\fff}: 


\ea%1185
    \label{ex:key:1185}
    \gll E    de  kúk,    áfta    e    de  \textbf{smɛ́l}  \textbf{chɔ́p}.\\
\textsc{3sg.sbj}  \textsc{ipfv}  cook  then  \textsc{3sg.sbj}  \textsc{ipfv}  smell  food\\

\glt ‘He’s cooking, afterwards he’ll smell of food.’ [dj07ae 013]
\z


\ea%1186
    \label{ex:key:1186}
    \gll Yu  de  \textbf{smɛ́l}  \textbf{chɔ́p},  dɛn  de  fráy  ɛ́ks  dé.\\
\textsc{2sg}  \textsc{ipfv}  smell  food    \textsc{3pl}  \textsc{ipfv}  fry  egg  there\\

\glt ‘You smell food, they’re frying eggs there.’ [dj07ae 016]
\z

Non-locative goal\is{goal} is the semantic role of objects associated with the verbs \textit{kɔmɔ́t} ‘come out’ and \textit{tɔ́n} ‘turn (into)’. 


\ea%1187
    \label{ex:key:1187}
    \gll E    de  trén    yú    sé    yu  go  \textbf{kɔmɔ́t}    \textbf{pɔ́sin}.\\
\textsc{3sg.sbj}  \textsc{ipfv}  train  \textsc{2sg.indp}  \textsc{quot}    \textsc{2sg}  \textsc{pot}  come.out  person\\

\glt ‘She is bringing you up to become a responsible person.’ [au07se 131]
\z


\ea%1188
    \label{ex:key:1188}
    \gll E    \textbf{tɔ́n}    \textbf{pɔ́sin}  wán    tɛ́n.\\
\textsc{3sg.sbj}  turn    person  one    time\\

\glt ‘He turned into a human-being at once.’ [ma03sh 006]
\z

The objects of \textit{sík} ‘be sick’ denote the cause of the sickness that the subject is suffering from. The verb \textit{sík} is not attested with a prepositional phrase alternative in the data; the use of an associative object appears to be the conventional way of expressing this state of affairs: 


\ea%1189
    \label{ex:key:1189}
    \gll E    de  \textbf{sík}  \textbf{fíba}.\\
\textsc{3sg.sbj}  \textsc{ipfv}  sick  fever\\

\glt ‘She’s sick with fever.’ [dj07ae 273]
\z

Another instance of an associative object with the semantic role of cause is \textit{mɔní} ‘money’, the object of \textit{kráy} ‘cry’ in \REF{ex:key:1190}:


\ea%1190
    \label{ex:key:1190}
    \gll \op...\cp{}  dán  papá  de  \textbf{kráy}  in    \textbf{mɔní}.\\
  {} that  father  \textsc{ipfv}  cry    \textsc{3sg.poss}  money.\\

\glt ‘(...) that elderly man was crying over his (lost) money.’ [ed03sb 200]
\z

An associative object may be fronted for emphasis \REF{ex:key:1191}. However, unlike patient\is{patient} or beneficiary\is{beneficiary} objects, associative objects may not be questioned with \textit{wétin} ‘what’ or \textit{údat} ‘who’. Instead, associative objects must be questioned with the corresponding adverbial question phrase or with the selective question element \textit{ús=káyn} ‘\textsc{which’}, which questions modifiers. 


Hence the clause \textit{e de sík fíba} \textsc{‘3sg.sbj} \textsc{ipfv} be sick fever’ = ‘she’s sick (with) fever’ cannot be questioned as *\textit{wétin e de sík} ‘what \textsc{3sg.sbj} \textsc{ipfv} be.sick’ = ‘what is she sick (with)?’ Rather, the question must be phrased as in \REF{ex:key:1192}:



\ea%1191
    \label{ex:key:1191}
    \gll Dán    pɔ́sin  go  ánsa    yú,      “yɛ́s,    na  malérya
e    bin  de  \textbf{sík}.”\\
that    person  \textsc{pot}  answer  \textsc{2sg.indp}    yes    \textsc{foc}  malaria
\textsc{3sg.sbj}  \textsc{pst}  \textsc{ipfv}  sick\\

\glt ‘That person would reply to you “yes, it’s malaria that 
she was sick with”.’ [dj05be 090]\is{associative objects}
\z


\ea%1192
    \label{ex:key:1192}
    \gll \'{U}s=káyn  \textbf{sík}    e    de  \textbf{sík}? \\
\textsc{q}=kind  sickness  \textsc{3sg.sbj}  \textsc{ipfv}  be.sick\\

\glt ‘What kind of sickness does she have?’ [eb07fn 244]\is{adverbial phrases}
\z

\subsection{Cognate objects}\label{sec:9.3.3}

In Pichi, “cognate objects” \citep{Baron1971} are deverbal nouns derived from themselves. Firstly, deverbal nouns occur with a few particular verbs in a non-emphatic, non-specific context and contribute little if nothing at all to the meaning denoted by the verb. 


For example, the objects of \textit{sík} ‘be sick’ and verbs of sound and speech-emission like \textit{síng} ‘sing’ and \textit{tɔ́k} ‘talk, say’ may occur with speech- or sound-denoting cognate objects in non-emphatic contexts. The cognate objects of these verbs have in common that they are not simply the corresponding action nominal of the verb. Instead, they have slightly idiosyncratic meanings: 



\ea%1193
    \label{ex:key:1193}
    \gll Wé  yu  kɔmɔ́t    \textbf{sík}    \textbf{dán}  \textbf{sík}    na  Panyá,  wé  yu  de  sík,
náw    yu  bigín  tɔ́k  Panyá.\\
\textsc{sub}  \textsc{2sg}  come.out  be.sick  that  sickness  \textsc{loc}  Spain  \textsc{sub}  \textsc{2sg}  \textsc{ipfv}  sick
now    \textsc{2sg}  begin  talk  Spanish\\

\glt ‘When you had just fallen sick with that sickness in Spain, when you were sick, 
then you began to talk Spanish.’ [ab03ab 018]
\z


\ea%1194
    \label{ex:key:1194}
    \gll A    wánt  \textbf{tɔ́k}  \textbf{dán}  \textbf{smɔ́l}  \textbf{tɔ́k}    dé.\\
\textsc{1sg.sbj}  want  talk  that  small  word  there\\

\glt ‘I want to say that small word there.’ [dj05ae 037]
\z

Aside from that, the use of cognate objects provides an important means of expressing emphasis \is{emphasis}in pragmatically marked, emphatic contexts such as (\ref{ex:key:1195}–\ref{ex:key:1196}). Emphatic cognate objects are frequently preceded by the indefinite determiner \textit{wán} ‘one, a’ which provides emphasis in other contexts as well (e.g. in the context of negative indefinite phrases, cf. \ref{ex:key:575}):


\ea%1195
    \label{ex:key:1195}
    \gll Dɛn  bin  \textbf{fáyn}  \textbf{wán}    \textbf{fáyn}.\\
\textsc{3pl}  \textsc{pst}  be.fine  one    fineness  \\

\glt ‘They were really fine.’ [mi07fn 120]
\z


\ea%1196
    \label{ex:key:1196}
    \gll Dán    torí    bin  de  \textbf{swít}    mí    \textbf{wán}    \textbf{swít}.\\
that    story  \textsc{pst}  \textsc{ipfv}  be.tasty  \textsc{1sg.indp}  one    tastiness\\

\glt ‘I really enjoyed that story.’ [ye07ga.006]
\z

The cognate object \textit{dáy} ‘death’ also appears as a cognate object to the verb \textit{dáy} ‘die’ in emphatic contexts like \REF{ex:key:1197}:


\ea%1197
    \label{ex:key:1197}
    \gll \'{E}y,  dán  káyn  spɛ́tikul,  a    \textbf{dáy}  \textbf{dáy}.\\
\textsc{intj}  that  kind    glasses  \textsc{1sg.sbj}  die  death\\

\glt ‘Hey, [if I had] that kind of glasses, I would surely die.’ [ne07ga 015]
\z

There is good reason to assume that the fronted “verb” in predicate cleft constructions like the following one is also a deverbal noun\is{predicate cleft}. One indication for this is that the verb is never fronted with predicate constituents like TMA markers. In this view, clefted verbs may also be seen as types of cognate objects: 


\ea%1198
    \label{ex:key:1198}
    \gll Na  \textbf{gó}  a    de  \textbf{gó}  ó!\\
\textsc{foc}  go  \textsc{1sg.sbj}  \textsc{ipfv}  go  \textsc{sp}\\

\glt ‘[Mind you] I’m going now!’ [ch07fn 151]
\z

\subsection{Double-object constructions}\label{sec:9.3.4}

The bulk of Pichi verbs can occur with one as well as two objects\is{objects}. The primacy of the object next to the verb – which is usually animate and has the role of recipient\is{recipient} or beneficiary – is evident in double-object constructions involving two object pronouns. The presence of two pronominal objects is ungrammatical if the clitic object pronoun \textit{=an} is preceded by the low-toned personal pronoun \textit{una/unu} ‘\textsc{2pl}’ or another \textsc{3sg.obj} pronoun \textit{=an} (for details, see \sectref{sec:3.3}, on tone-conditioned suppletive allomorphy). In such cases, it is the patient object that remains unexpressed. Compare the double-object construction in \REF{ex:key:1199} with the ungrammatical example \REF{ex:key:1200} and sentence \REF{ex:key:1201}. In the latter example, the \textsc{3sg.obj} theme\is{theme} object \textit{=an} remains unexpressed: 


\ea%1199
    \label{ex:key:1199}
    \gll Yu  gɛ́fɔ    sɛ́n    \textbf{wí=an}.\\
\textsc{2sg}  have.to  send  \textsc{1pl.indp}=\textsc{3sg.obj}\\

\glt ‘You have to send it to us.’ [ye07de 009]
\z


\ea[*]{%1200
    \label{ex:key:1200}
    \gll A    go  gí    \textbf{una=an}    tumɔ́ro.\\
 \textsc{1sg.sbj}  \textsc{pot}  give    \textsc{2pl}=\textsc{3sg.obj}  tomorrow\\
\glt Intended: ‘I’ll give you [\textsc{pl}] it tomorrow.’ [ye07de 011]
}\z


\ea%1201
    \label{ex:key:1201}
    \gll A    go  gí  \textbf{unu}  tumɔ́ro.\\
\textsc{1sg.sbj}  \textsc{pot}  give  \textsc{2pl}  tomorrow\\

\glt ‘I’ll give you (it) tomorrow.’ [ye07de 012]
\z

Double-object constructions can be divided into three types according to relevant semantic and syntactic properties. \tabref{tab:key:9.10} provides an overview of the semantic roles of objects involved in double-object constructions and their syntactic positions as primary objects immediately to the right of the verb or secondary objects following the primary objects. Some semantic roles associated with the position of primary and secondary objects may alternatively be expressed by prepositional phrases\is{prepositional phrases} or SVCs. Where such alternatives exist, they are provided in the two rightmost columns. \is{order of objects}

%%please move \begin{table} just above \begin{tabular
\begin{table}
\caption{Syntax and semantics of double-object constructions}
\label{tab:key:9.10}

\begin{tabularx}{\textwidth}{llQQQQ}
\lsptoprule

Type & Description & Primary object & Secondary object & Alternative to primary object & Alternative \mbox{to secondary} object\\
\midrule
1 & Transfer & Recipient & Theme & — & —\\
2 & Promotion & Beneficiary & Patient & \textstyleTablePichiZchn{fɔ}{}-PP & —\\
&  & Goal (\textstyleTablePichiZchn{pút}) & Theme & \textstyleTablePichiZchn{fɔ}{}-PP; \textstyleTablePichiZchn{na}{}-PP & —\\
3 & Adjunction\is{adjuncts} & Patient & Associative object & — & Diverse PPs; SVCs\\
\lspbottomrule
\end{tabularx}
\end{table}
In the type 1 double-object construction, the primary object to the right of the verb occupies the recipient role, while the secondary object that follows the recipient invariably takes on a patient \is{patient}role. This kind of construction is found with verbs expressing the transfer of an entity or an act of communication from the subject\is{subjects} to a recipient. All ditransitive communication and transfer verbs encountered in the corpus are listed in \REF{ex:key:1202}:

\eabox{\label{ex:key:1202}
\begin{tabularx}{\textwidth}{r ll ll X}
         & \multicolumn{2}{c}{Communication verbs} & \multicolumn{3}{l}{Transfer verbs}\\
& \itshape púl & ‘narrate’ & \itshape gí & ‘give’ & \\
& \itshape tɛ́l & ‘tell’ & \itshape dás & ‘give as present’ & \\
& \itshape lán & ‘teach’ & \itshape bák & ‘give back’ & \\
& \itshape tích & ‘teach’ & \itshape sɛ́n & ‘send’ & \\
& \itshape áks & ‘ask’ & \itshape báy & ‘buy’ & \\
& \itshape ríd & ‘read’ &  &  & \\
\end{tabularx}
}
Pichi has no SVCs of the \textsc{give} type in order to mark a recipient or beneficiary\is{beneficiary SVC}. In double-object constructions featuring transfer verbs, the primary object next to the verb always has the semantic role of recipient \REF{ex:key:1203}. With transfer and communication verbs, a beneficiary\is{associative preposition} is usually expressed in a PP\is{prepositional phrases} introduced by \textit{fɔ} ‘\textsc{prep’} \REF{ex:key:1204}. Hence, double-object constructions are the only means of expressing the grammatical relation between the ditransitive verb, its subject, and its recipient and theme\is{theme} objects: 


\ea%1203
    \label{ex:key:1203}
    \gll Mi    mamá  dás        \textbf{mí}    \textbf{sɔn}   \textbf{regalo}.\\
\textsc{1sg.poss}  mother  give.as.present  \textsc{1sg.indp}  some  present\\

\glt ‘My mother gave me a present.’ [ro05ee 055]
\z


\ea%1204
    \label{ex:key:1204}
    \gll Mi    mamá  dás        \textbf{sɔn}   \textbf{regalo}  \textbf{fɔ}  \textbf{mí}.\\
\textsc{1sg.poss}  mother  give.as.present  some  present  \textsc{prep}  \textsc{1sg.indp}\\

\glt ‘My mother gave (somebody) a present for me.’ [ro05ee 056]
\z

The following example features transfer verb \textit{gí} ‘give’ and a prepositional phrase introduced by \textit{fɔ} ‘\textsc{prep}’. The PP can only denote a beneficiary with the recipient remaining unexpressed. Hence the second translation is ungrammatical, since the recipient object cannot alternatively be expressed as a prepositional phrase:


\ea%1205
    \label{ex:key:1205}
    \gll Dɛn  gí  di  \textbf{mɔní}    \textbf{fɔ}  \textbf{mí}.\\
\textsc{3pl}  give  \textsc{def}  money    \textsc{prep}  \textsc{1sg.indp}\\

\glt 
 ‘They gave the money (to someone) for me.’ [lo07fn 555]
but not ‘They gave me the money.’
\z

The following two double-object constructions involve the transfer verb \textit{gí} ‘give’ \REF{ex:key:1206} and the verb of communication \textit{púl (torí)} ‘narrate (a story)’ \REF{ex:key:1207}:


\ea%1206
    \label{ex:key:1206}
    \gll Dɛn    bin  gí    \textbf{mí}      \textbf{beca}.\\
\textsc{3pl}    \textsc{pst}  give    \textsc{1sg.indp}    scholarship.\\

\glt ‘I was given a scholarship.’ [ed03sp 057]
\z


\ea%1207
    \label{ex:key:1207}
    \gll Na  ín    e    de  kán    púl  \textbf{mí}    \textbf{dán}  \textbf{torí}.\\
\textsc{foc}  \textsc{3sg.indp}  \textsc{3sg.sbj}  \textsc{ipfv}  come  pull  \textsc{1sg.indp}  that  story\\

\glt ‘That’s when she comes to tell me that story.’ [ab03ab 073]
\z

The verb \textit{sɛ́n} ‘send, throw’ denotes a situation in which both a transfer and a motion event{\fff} co-occur. When \textit{sɛ́n} is used in a double-object construction, the primary object is always a recipient \REF{ex:key:1208}. 


\ea%1208
    \label{ex:key:1208}
    \gll E    gɛ́fɔ    sɛ́n    \textbf{mí=an}.\\
\textsc{3sg.sbj}  have.to  send  \textsc{1sg.indp}=\textsc{3sg.obj}\\

\glt ‘He has to send/throw it to me.’ [ye07de 001]
\z

Like with other transfer verbs, the recipient of \textit{sɛ́n} may not be expressed as a prepositional phrase. Where we do find a prepositional phrase (usually introduced by \textit{fɔ} ‘\textsc{prep}’), it can therefore only denote a beneficiary {\fff}or a goal{\fff} but not a recipient {\fff}\REF{ex:key:1209}:


\ea%1209
    \label{ex:key:1209}
    \gll E    gɛ́fɔ    sɛ́n=an    \textbf{fɔ}  \textbf{yú}.\\
\textsc{3sg.sbj}  have.to  send=\textsc{3sg.obj}  \textsc{prep}  \textsc{2sg.indp}\\

\glt 
 ‘He has to send it to (where) you (are).’ or ‘He has to send it for you.’ [ye07de 003]
but not `He has to send it to you.'\is{recipient} 
\z

Type 2 double-object constructions are best understood in terms of syntactic promotion. A participant that is more commonly expressed as a prepositional phrase is promoted to object status. In contrast to type 1, the use of type 2 constructions is therefore optional. We find the type 2 double-object construction with two kinds of verbs. First, it is encountered with any Pichi transitive verb save transfer verbs and verbs of communication (type 1). With these verbs, which form the vast majority of Pichi verbs, the primary object has the semantic role of beneficiary. The secondary object is assigned a patient role. 


Sentence \REF{ex:key:1210} features two type 2 double-object constructions. The verb \textit{dú} ‘do’ takes the primary, beneficiary object \textit{mí} ‘\textsc{1sg.indp}’ and the patient object \textit{sɔn fébɔ} ‘a favour’.\textit{} The verb \textit{wás} ‘wash’ also takes \textit{mí} ‘\textsc{1sg.indp}’ as the beneficiary object while \textit{klós dɛn} ‘clothing’ functions as the patient object: 



\ea%1210
    \label{ex:key:1210}
    \gll A    wánt  mék    yu  dú  \textbf{mí}    \textbf{sɔn}   \textbf{fébɔ}    mék
yu  wás    \textbf{mí}    \textbf{sɔn}   \textbf{klós}    \textbf{dɛn}.\\
\textsc{1sg.sbj}  want  \textsc{sbjv}    \textsc{2sg}  do  \textsc{1sg.indp}  some  favour  \textsc{sbjv}
\textsc{2sg}  wash  \textsc{1sg.indp}  some  clothing  \textsc{pl}\\

\glt ‘I want you to do me a favour (and) wash some clothes for me.’ [ru03wt 030]
\z

The semantic role of beneficiary may subsume a maleficiary\is{maleficiary}, i.e. the affected party of a socially unacceptable action. In \REF{ex:key:1211}, a worried mother explains why she has left her teenage daughter in Spain instead of bringing her along with her to Malabo. Also, compare the first object of \textit{tíf} ‘steal’, the maleficiary \textit{mí} ‘\textsc{1sg.indp}’ in \REF{ex:key:1212}:


\ea%1211
    \label{ex:key:1211}
    \gll A    lɛ́f    mi    pikín  na  Panyá  bikɔs  a    de  fía
sé    dɛn  go  bɛlɛ́      \textbf{mí}    \textbf{mi}  \textbf{pikín}.\\
\textsc{1sg.sbj}  leave  \textsc{1sg.poss}  child  \textsc{loc}  Spain  because  \textsc{1sg.sbj}  \textsc{ipfv}  fear
\textsc{quot}    \textsc{3pl}  \textsc{pot}  impregnate  \textsc{1sg.indp}  \textsc{1sg.poss}  child\\

\glt ‘I have left my child in Spain because I fear that she would fall pregnant on me.’ [ge05fn]
\z


\ea%1212
    \label{ex:key:1212}
    \gll Dɛn    tíf    \textbf{mí}    \textbf{mi}    \textbf{sús}.\\
\textsc{3pl}    steal  \textsc{1sg.indp}  \textsc{1sg.poss}  shoe\\

\glt ‘They stole my shoes from me.’ [ge07fn 023]
\z

We have seen that a recipient\is{recipient} must be expressed as an object in type 1 double-object constructions. In contrast to type 1 constructions, type 2 constructions alternate freely with constructions in which the beneficiary is expressed as a prepositional phrase introduced by the associative preposition \textit{fɔ} ‘\textsc{prep}’. In fact, the alternative involving a prepositional phrase is more common than the corresponding double-object construction. Compare the type 2 double-object construction \REF{ex:key:1213} involving the verb \textit{bay} ‘buy’ with the PP alternative \REF{ex:key:1214}:


\ea%1213
    \label{ex:key:1213}
    \gll \'{A}fta    primera  dama  báy=an    \textbf{wán}    \textbf{motó}, \op...\cp{}\\
then  first    lady    buy=\textsc{3sg.obj}  one    car\\

\glt ‘Then the first lady bought him a car (...)’ [fr03cd 070]
\z


\ea%1214
    \label{ex:key:1214}
    \gll \MakeUppercase{A}   bin  báy  \textbf{wán}    \textbf{motó}  \textbf{fɔ}  \textbf{mi}    \textbf{mása}.\\
\textsc{sg}.sbj  \textsc{pst}  buy  one    car    \textsc{prep}  \textsc{1sg.poss}  boss\\

\glt ‘I bought a car for my boss.’ [ye0502e2 073]\is{beneficiary}
\z

The second type of type 2 construction involves the caused location verb \textit{pút} ‘put’. Here, the primary object has the semantic role of goal,\is{goal} while the secondary object fulfills a theme\is{theme} role. In \REF{ex:key:1215}, the primary object of \textit{pút} is the goal object \textit{=an} ‘\textsc{3sg.obj}’, while the secondary object \textit{saldo} ‘(mobile phone) credit’ is the theme\is{theme}. Sentence \REF{ex:key:1216} also features the goal object \textit{=an} ‘\textsc{3sg.obj}’, while the theme object is \textit{cacahuete} ‘groundnut’:


\ea%1215
    \label{ex:key:1215}
    \gll Yu  gɛ́t  móvil,  yu  dɔ́n    pút=\textbf{an}    \textbf{saldo}?\\
\textsc{2sg}  get  mobile  \textsc{2sg}  \textsc{prf}    put=\textsc{3sg.obj}  credit\\

\glt ‘Do you have a mobile-phone, have you put credit into it?’ [go0502e1 087]
\z


\ea%1216
    \label{ex:key:1216}
    \gll A    báy  dán  dís  tín,    sɔn    smɔ́l  pépa,
dɛn  de  pút=\textbf{an}    \textbf{cacahuete}.\\
\textsc{1sg.sbj}  buy  that  this  thing  some  small  paper
\textsc{3pl}  \textsc{ipfv}  put=\textsc{3sg.obj}  peanut\\

\glt ‘I bought that thing, a small paper, they put peanuts into it.’ [ed03sp 083]
\z

However, the corpus contains many more examples of \textit{pút}{}-constructions, in which the goal role is expressed through a locative construction rather than a primary object. Likewise, there is no sentence in the data in which the goal object of \textit{pút} is a full noun. The locative construction may be a PP \REF{ex:key:1217} or involve a locative noun \REF{ex:key:1218}. Unlike a few other verbs with a motion\is{motion verbs} component (cf. \sectref{sec:8.1.4}), the goal of \textit{pút} cannot be expressed as a complement of the V2 of a motion-direction SVC (e.g. \textit{*a pút=an gó na glás} \{\textsc{1sg.sbj} put=\textsc{3sg.obj} go \textsc{loc} glass\} = ‘I put it into the glass’):


\ea%1217
    \label{ex:key:1217}
    \gll Dɛn  kin  pút=an    \textbf{fɔ}  \textbf{glás}.\\
\textsc{3pl}  \textsc{hab}  put=\textsc{3sg.obj}  \textsc{prep}  glass\\

\glt ‘They put it into the glass.’ [ed03sb 096]
\z


\ea%1218
    \label{ex:key:1218}
    \gll \MakeUppercase{A}   dɔ́n    \textbf{pút}  mi    búk    \textbf{ínsay}.\\
\textsc{1sg.sbj}  \textsc{prf}    put  \textsc{1sg.poss}  book  inside\\

\glt ‘I have put my book inside.’ [dj07ae 329]\is{goal}
\z

There is a preference to interpret a PP introduced by \textit{fɔ} ‘\textsc{prep}’ as a beneficiary \is{beneficiary}in \textit{pút}{}-double-object constructions, particularly where an object pronoun theoretically allows for both interpretations as in \REF{ex:key:1219}. A sentence like \REF{ex:key:1218} above, which involves a locative noun\is{locative nouns} (i.e. \textit{ínsay} ‘inside’) is therefore preferred to avoid ambiguity. Nevertheless, an alternative with a prepositional phrase involving the general locative preposition \textit{na} may also be exploited to the same end \REF{ex:key:1220}: 


\ea%1219
    \label{ex:key:1219}
    \gll A    dɔ́n  pút  granát  \textbf{fɔr=an}.\\
\textsc{1sg.sbj}  \textsc{prf}  put  peanut  \textsc{prep}=\textsc{3sg.obj}\\

\glt ‘I have put peanuts [somewhere] for her.’ [dj07ae 331a]


\glt ?I have put peanuts into it. 
\z


\ea%1220
    \label{ex:key:1220}
    \gll A    dɔ́n  pút  granát  \textbf{na}  \textbf{ín}.\\
\textsc{1sg.sbj}  \textsc{prf}  put  peanut  \textsc{loc}  \textsc{3sg.indp}\\

\glt ‘I have put peanuts into it.’ [dj07ae 331b]
\z

Note, however, that \textit{pút} ‘put’ may also appear in a type 2 double-object construction, in which the primary object is a beneficiary – just like any other transitive verb:


\ea%1221
    \label{ex:key:1221}
    \gll Yu  pút=\textbf{an}    \textbf{wán}    \textbf{sardina}  ɔntɔ́p.\\
\textsc{2sg}  put=\textsc{3sg.obj}  one    sardine  top\\

\glt ‘(Then) you put a sardine on top for him.’ [ro05rt 064
\z

Type 3 double-object construction involve verbs that may take associative objects (cf. \tabref{tab:key:9.9} above). Type 3 constructions differ from type 1 and type 2 constructions in that the primary object occupies the semantic role of patient. The secondary object is an associative object which may alternatively be expressed without any syntactic rearrangement through the mere insertion of a preposition, serial verb, or any other element between the two objects. The associative objects in type 3 constructions may therefore be paraphrased with the same means as associative objects in single-object constructions. Compare the double-object construction in \REF{ex:key:1222} with the single-object construction involving a PP in \REF{ex:key:1223}:\is{adjuncts}


\ea%1222
    \label{ex:key:1222}
    \gll Na  lɛk  sé    yu  fúlɔp  \textbf{di}  \textbf{glás}  \textbf{watá}?\\
\textsc{foc}  like  \textsc{quot}    \textsc{2sg}  fill    \textsc{def}  glás  water\\

\glt ‘As if you filled this glass with water?’ [dj07ae 066]
\z


\ea%1223
    \label{ex:key:1223}
    \gll E    fúlɔp  \textbf{di}  \textbf{glás}    \textbf{wet}    \textbf{watá}.\\
\textsc{3sg.sbj}  fill.up  \textsc{def}  glass  with    water   \\
\glt ‘She filled the glass with water.’ [dj07ae 067]\is{double-object constructions}
\z

\subsection{Reflexivity}\label{sec:9.3.5}

In the majority of cases, reflexivity is expressed through an object NP consisting of the pronominal and reflexive anaphor \textit{sɛ́f} ‘self’ and a preceding possessive pronoun with the same person and number as the subject{\fff}. Sometimes, the body part{\fff} nouns \textit{skín} ‘body’, \textit{bɔ́di} ‘body’, and \textit{héd} ‘head’ are also employed as reflexive anaphors in the same syntactic position as \textit{sɛ́f}. A clause featuring a reflexive object NP indicates that the subject does something to her- or himself. The corpus only contains clauses in which subjects serve as antecedents to the reflexive anaphor, cf. \REF{ex:key:1224}:


\ea%1224
    \label{ex:key:1224}
    \gll Dán    gál    e    kin  fíks  \textbf{in}    \textbf{sɛ́f},    pént    \textbf{in}    \textbf{sɛ́f}.\\
that    girl    \textsc{3sg.sbj}  \textsc{hab}  fix  \textsc{3sg.poss}  self    paint  \textsc{3sg.poss}  self\\

\glt ‘That girl habitually fixes herself up, paints herself [puts on make up].’ [dj07ae 114]
\z

Aside from that, reflexive constructions also form part of idiomatic expressions with little reflexive meaning but characterised by low transitivity. I give a sentence featuring the idiom \textit{sék in sɛ́f} ‘shake \textsc{3sg.poss} self’ = ‘make an effort’: 


\ea%1225
    \label{ex:key:1225}
    \gll E    sék    \textbf{in}    \textbf{sɛ́f}  bɔkú  fɔ  tɔ́n    general.\\
\textsc{3sg.sbj}  shake  \textsc{3sg.poss}  self  much  \textsc{prep}  turn    general\\

\glt ‘He made a big effort to turn general.’ [ur07ae 498]
\z

The nouns \textit{skín} ‘body’, \textit{bɔ́di} ‘body’, and \textit{héd} ‘head’ are far less commonly used than \textit{sɛ́f} as reflexive anaphors. Equally, these three nouns usually occur as reflexive anaphors with verbs, whose meanings imply an actual physical effect on the body. The following three sentences illustrate this usage: 


\ea%1226
    \label{ex:key:1226}
    \gll \MakeUppercase{A}   de  \textbf{sí}  mi    \textbf{skín}    na  lukinglás.\\
\textsc{1sg.sbj}  \textsc{ipfv}  see  \textsc{1sg.poss}  body  \textsc{loc}  mirror\\

\glt ‘I’m seeing myself/my body in the mirror.’ [dj07ae 496]
\z


\ea%1227
    \label{ex:key:1227}
    \gll \MakeUppercase{A}   de  \textbf{kíl}  mi    \textbf{skín}    dé,    lɛk  háw    a    de  wók.\\
\textsc{1sg.sbj}  \textsc{ipfv}  kill  \textsc{1sg.poss}  body  there  like  how    \textsc{1sg.sbj}  \textsc{ipfv}  work\\

\glt ‘I’m killing myself there, the way I’m working.’ [dj07ae 494]
\z


\ea%1228
    \label{ex:key:1228}
    \gll E    dɔ́n  \textbf{chák}    in    \textbf{héd}.\\
\textsc{3sg.sbj}  \textsc{prf}  get.drunk  \textsc{3sg.poss}  head\\

\glt ‘She is dead drunk.’ [\textit{Lit.} ‘She has got her head drunk.’] [ra07fn 026]
\z

A reflexive relation within an \textsc{NP} is expressed through the use of a possessive pronoun in conjunction with the pronominal \textit{yón} ‘own’ as a modifier to a head noun: 


\ea%1229
    \label{ex:key:1229}
    \gll Bɔt  fɔ  Bata    dɛn  de  ték/    dán    wán  sí  que  \textbf{dɛn}  \textbf{yón}   máred
día,      dɛn    yón     máred    de  kári    mɔní  ɛ́n.\\
but  \textsc{prep}  \textsc{place}  \textsc{3pl}  \textsc{ipfv}  take    that    one  if  that  \textsc{3pl}  own  marriage
be.expensive  \textsc{3pl}    own    marriage    \textsc{ipfv}  carry  money  \textsc{intj}\\

\glt ‘But as for the mainlanders, they take/ as for that one, their marriage is 
expensive, their marriage costs money.’ [hi03cb 010]
\z

Besides that, Pichi has a number of inherently reflexive verbs. For most of these verbs, the use of a reflexive anaphor is optional. Such verbs denote situations involving body or mental processes and physical movements which imply volition and instigation by the actor subject rather than a spontaneous event. 


Compare \textit{wɛ́r} ‘dress (up)’ in an explicitly reflexive clause \REF{ex:key:1230} and a clause in which reflexivity remains unexpressed \REF{ex:key:1231}: 



\ea%1230
    \label{ex:key:1230}
    \gll Toichoa    \textbf{wɛ́r}    \textbf{in}    \textbf{sɛ́f}.\\
\textsc{name}    wear  \textsc{3sg.poss}  self\\

\glt ‘Toichoa has/is dressed up.’ [dj07ae 375]\is{reflexivity}
\z


\ea%1231
    \label{ex:key:1231}
    \gll A    \textbf{wɛ́r}.\\
\textsc{1sg.sbj}  wear\\

\glt ‘I’m dressed/have got dressed.’ [ye05ae 233]
\z

\subsection{Reciprocity}\label{sec:9.3.6}

Next to its use as a reflexive anaphor, the pronominal \textit{sɛ́f} ‘self’ also serves as a reciprocal pronominal with plural referents. In sentence \REF{ex:key:1232}, the reciprocal NP is an object to the verb \textit{sláp} ‘slap’, in \REF{ex:key:1233} to the locative noun \textit{bifó} ‘before’:


\ea%1232
    \label{ex:key:1232}
    \gll Dɛn  de  sláp  \textbf{dɛn}  \textbf{sɛ́f}.\\
\textsc{3pl}  \textsc{ipfv}  slap  \textsc{3pl}  self\\

\glt ‘They’re slapping each other.’ [dj07re 020]
\z


\ea%1233
    \label{ex:key:1233}
    \gll Pero    dɛn  nó  sidɔ́n  bifó    \textbf{dɛn}  \textbf{sɛ́f}.\\
but    \textsc{3pl}  \textsc{neg}  sit.down  before  \textsc{3pl}  self\\

\glt ‘But they’re not sitting in front of each other.’ [dj07re 031]
\z

Reflexive and reciprocal meaning may be disambiguated through contextual factors, i.e. verb meaning and the presence of plural referents. The occurrence of compound personal pronouns indicating dual number (i.e. \textit{dɛn-ɔl-tú} ‘the two of them’) as in \REF{ex:key:1234} or universal inclusivity (\textit{dɛn-ɔ́l} ‘all of them’) as in \REF{ex:key:1235} is also quite common in reciprocal contexts: 


\ea%1234
    \label{ex:key:1234}
    \gll Fɔ́s  na  \textbf{dɛn-ɔl-tú}      dɛn  bin  de  abraza  \textbf{dɛn}  \textbf{sɛ́f}.\\
first  \textsc{foc}  \textsc{3pl.cpd}{}-all.\textsc{cpd}{}-two  \textsc{3pl}  \textsc{pst}  \textsc{ipfv}  embrace  \textsc{3pl}  self\\

\glt ‘First it’s the two of them that were embracing each other.’ [dj07re 013]
\z


\ea%1235
    \label{ex:key:1235}
    \gll \textbf{Dɛn-ɔ́l}  dɛn  de  salút  \textbf{dɛn}  \textbf{sɛ́f}.\\
{\textsc{3pl}  all}  \textsc{3pl}  \textsc{ipfv}  greet  \textsc{3pl}  self\\

\glt ‘They’re all greeting each other.’ [dj07re 009]
\z

Reciprocal relations within the \textsc{NP} find expression through the pronominal\is{possessive pronominal} \textit{yón} as illustrated in \REF{ex:key:1236}:


\ea%1236
    \label{ex:key:1236}
    \gll Dɛn  lúk    \textbf{dɛn}  \textbf{yón}     fés.\\
\textsc{3pl} look \textsc{3pl}  own    face\\

\glt ‘They looked at thier own faces.’ [eb07fn 313\textstylePichiexamplenumberZchnZchn{]}\is{reciprocity}
\z

Pichi also has inherently reciprocal verbs, many of which preferably do not occur with the anaphor \textit{sɛ́f} (cf. \sectref{sec:9.4.3}).{\fff} 

\subsection{Weather phenomena}\label{sec:9.3.7}

Pichi has three types of constructions for expressing weather phenomena. The first type of construction consists of an intransitive clause with the weather phenomenon in the subject\is{subjects} position. The verbs used in the first type of construction have a general meaning and also occur in other contexts including transitive clauses. Three sentences follow featuring the two weather verbs \textit{bló} ‘blow’ (\ref{ex:key:1237}–\ref{ex:key:1238}) and \textit{bráyt} ‘be bright’ \REF{ex:key:1239} and the weather nouns \textit{tináda} ‘thunderstorm’, \textit{brís} ‘air’, and \textit{sán} ‘sun’: 


\ea%1237
    \label{ex:key:1237}
    \gll Tináda      de  \textbf{bló}.\\
thunderstorm  \textsc{ipfv}  blow\\

\glt ‘A thunderstorm is raging.’ [dj07ae 239]
\z


\ea%1238
    \label{ex:key:1238}
    \gll Brís  de  \textbf{bló}.\\
air  \textsc{ipfv}  blow\\

\glt ‘The wind is blowing.’ [dj07ae 242]
\z


\ea%1239
    \label{ex:key:1239}
    \gll Di  sán  \textbf{bráyt}.\\
\textsc{def}  sun  be.bright\\

\glt ‘The sun is bright/is shining.’ [dj07ae 164]
\z

Sentence \REF{ex:key:1240} exemplifies the transitive usage of \textit{bló} ‘blow’, sentence \REF{ex:key:1241} that of \textit{bráyt}, here with the meaning ‘brighten, light up’:


\ea%1240
    \label{ex:key:1240}
    \gll Di  ventilador  de  \textbf{bló}    \textbf{mí}.\\
\textsc{def}  fan      \textsc{ipfv}  blow  \textsc{1sg.indp}\\

\glt ‘The fan is blowing at me.’   [dj07ae 243]
\z


\ea%1241
    \label{ex:key:1241}
    \gll Di  sán  \textbf{bráyt}  \textbf{di}  \textbf{dé}.\\
\textsc{def}  sun  brighten  \textsc{def}  day\\

\glt ‘The sun lit up the sky.’ [dj07ae 166]
\z

In expressions where reference is made to the general atmospheric condition, the noun \textit{dé} ‘day, weather’ appears in the subject\is{subjects} position instead of a specific natural element. This usage is exemplified in the following three sentences and also in \REF{ex:key:1241} above: 


\ea%1242
    \label{ex:key:1242}
    \gll \textbf{Di}  \textbf{dé}    dák.\\
\textsc{def}  weather  be.dark\\

\glt ‘It’s dark.’ [ab07fn 115]
\z


\ea%1243
    \label{ex:key:1243}
    \gll \textbf{Di}  \textbf{dé}    \textbf{fɔ}  \textbf{tidé}    tú  hɔ́t,    tú  mɔ́ch  sán.\\
\textsc{def}  weather  \textsc{prep}  today  too  be.hot  too  much  sun\\

\glt ‘The weather of today is too hot, too much sun.’ [dj07ae 249]
\z


\ea%1244
    \label{ex:key:1244}
    \gll \textbf{Di}  \textbf{dé}    kól.\\
\textsc{def}  weather  be.cold\\

\glt ‘It’s cold.’ [dj07ae 248]
\z

The second type of construction also involves an intransitive clause but it features the expletive subject\is{expletive} pronoun \textit{e} ‘\textsc{3sg.sbj}’ rather than a weather noun. This construction is limited to a single intransitive verb, namely \textit{fɔ́l} ‘rain’, which exclusively functions as a weather verb \REF{ex:key:1245}. The verb \textit{fɔ́l} may, however, also occur in the first type of construction, together with the weather noun \textit{rén} ‘rain’ in subject position \REF{ex:key:1246}:


\ea%1245
    \label{ex:key:1245}
    \gll A    de  sí  di  dé  lɛkɛ  sé    \textbf{e}    wánt  \textbf{fɔ́l}.\\
\textsc{1sg.sbj}  \textsc{ipfv}  see  \textsc{def}  day  like  \textsc{quot}    \textsc{3sg.sbj}  want  rain\\

\glt ‘I think the weather is like it’s going to rain.’ [ye07fn 083]
\z


\ea%1246
    \label{ex:key:1246}
    \gll A    bin  chɛ́k  sé    \textbf{rén}  go  \textbf{fɔ́l}.\\
\textsc{1sg.sbj}  \textsc{pst}  check  \textsc{quot}    rain  \textsc{pot}  rain\\

\glt ‘I thought it would rain.’ [ma03hm 022]
\z

The third type of construction involves existential clauses featuring the possessive and existential verb \textit{gɛ́t} ‘get, exist’ or the locative-existential copula \textit{dé} ‘\textsc{be.loc}’ (cf. \sectref{sec:7.6.3} for details on the syntax of these clauses). This construction is only attested in codemixed utterances involving a Spanish atmospheric phenomenon: 


\ea%1247
    \label{ex:key:1247}
    \gll E    \textbf{gɛ́t}  \textbf{relámpago}.\\
\textsc{3sg.sbj}  get  lightning\\

\glt ‘There is lightning.’ [dj07ae 245]
\z


\ea%1248
    \label{ex:key:1248}
    \gll Dán  sáy,  \textbf{niebla}  \textbf{dé}    dé.\\
that  side  fog    \textsc{be.loc}  there\\

\glt ‘It’s foggy there.’ [he07fn 262]
\z

\subsection{Body states}\label{sec:9.3.8}

Body states are expressed in constructions involving transitive (cf. 1a–1c in \tabref{tab:key:9.11}) and intransitive (2a–2c) clauses. Type 1 constructions in the table involve transitive clauses. In type 1a constructions, the affected body part\is{} is found in the subject\is{subjects} position, while the experiencer\is{experiencer} is in the object position. This construction is the preferred one for expressing pain and hurt. The verb is either of the dynamic experiential verbs \textit{hát} ‘hurt’ or \textit{pén} ‘pain’ (\ref{ex:key:1249}–\ref{ex:key:1251}).

%%please move \begin{table} just above \begin{tabular
\begin{table}
\caption{Expressing body states}
\label{tab:key:9.11}

\begin{tabularx}{\textwidth}{rl CCC CCC}
\lsptoprule

& Body state verb & 1a & 1b & 1c & 2a & 2b & 2c\\
\midrule
a. & \textstyleTablePichiZchn{pén} ‘pain’ & x &  & x &  &  & \\
& \textstyleTablePichiZchn{hát} ‘hurt’ & x &  & x &  & x & \\

\tablevspace
b. & \textstyleTablePichiZchn{hángri} ‘be hungry’ &  & x & x & x &  & \\
& \textstyleTablePichiZchn{tɔ́sti} ‘be thirsty’ &  & x & x & x &  & \\
& \textstyleTablePichiZchn{slíp} ‘sleep’ &  & x & x & x &  & \\
& \textstyleTablePichiZchn{sík} ‘be sick’ &  &  &  & x &  & \\

\tablevspace
c. & \textit{kól} ‘be cold’ &  &  & x &  & x & \\
& \textit{hɔ́t} ‘be hot’ &  &  & x &  & x & \\
& \textstyleTablePichiZchn{táya} ‘be tired’ &  &  &  &  & x & \\
& \textstyleTablePichiZchn{bɛlfúl} ‘be satiated’ &  &  &  &  & x & \\
& \textit{wɛ́l} ‘be well’ &  &  &  &  & x & \\

\tablevspace
d. & \textit{gúd} ‘be well’ &  &  & x &  &  & x\\
& \textit{bád} ‘be ill’ &  &  & x &  &  & x\\
& \textit{fáyn} ‘be fine’ &  &  & x &  &  & x\\
\lspbottomrule
\end{tabularx}
\end{table}



\ea%1249
    \label{ex:key:1249}
    \gll Mi    \textbf{bɛlɛ́}    de  \textbf{hát}    mí.\\
\textsc{1sg.poss}  belly  \textsc{ipfv}  hurt    \textsc{1sg.indp}\\

\glt ‘My stomach is hurting me.’ [dj07ae 312]
\z


\ea%1250
    \label{ex:key:1250}
    \gll Mi    \textbf{bɛlɛ́}    de  \textbf{pén}    mí.\\
\textsc{1sg.poss}  belly  \textsc{ipfv}  pain    \textsc{1sg.indp}\\

\glt ‘My stomach is paining me.’ [dj07ae 314]
\z


\ea%1251
    \label{ex:key:1251}
    \gll Mi    \textbf{tít}    de  \textbf{pén}    mí.\\
\textsc{1sg.poss}  tooth  \textsc{ipfv}  pain    \textsc{1sg.indp}\\

\glt ‘My tooth is paining me.’ [dj07ae 313]
\z

In type 1b constructions, the subject\is{subjects} of the transitive clause is a deverbal noun denoting the experience, while the object instantiates the experiencer. Instead of an experiential verb, we find an idiomatically used dynamic verb \textit{kéch} ‘catch’. The body states of hunger, thirst, and sleep(iness) may be expressed in this way, usually combined with a sense of suddenness or unexpectedness. Compare the following three examples:


\ea%1252
    \label{ex:key:1252}
    \gll Smɔ́ltɛn    \textbf{slíp}    \textbf{kéch}=an.\\
shortly    sleep  catch=\textsc{3sg.obj}\\

\glt ‘Shortly afterwards, he became sleepy/fell asleep.’ [ab03ab 050]
\z


\ea%1253
    \label{ex:key:1253}
    \gll Wán    \textbf{hángri}  \textbf{kéch}  mí    dé.\\
one    hunger  catch  \textsc{1sg.indp}  there\\

\glt ‘I suddenly felt very hungry there.’ [dj07ae 324]
\z


\ea%1254
    \label{ex:key:1254}
    \gll \textbf{Tɔ́sti}  \textbf{kéch}  mí.\\
thirst  catch  \textsc{1sg.indp}\\

\glt ‘I (suddenly) felt thirsty.’ [dj07ae 327]
\z

Type 1c constructions in \tabref{tab:key:9.11} are the mirror-image of type 1b constructions. The experiencer is in the subject position, while the body state or sensation is expressed as a deverbal noun in the object position. Hunger, thirst, and sleep(iness) can be expressed by this construction with the dynamic body state verbs \textit{fíl} ‘feel’ (\ref{ex:key:1255}–\ref{ex:key:1256}). Hunger and thirst can also be expressed in combination with the verb \textit{sɔ́fa} ‘suffer, endure’ \REF{ex:key:1257}:


\ea%1255
    \label{ex:key:1255}
    \gll A    de  \textbf{fíl}  \textbf{hángri},  A    de  \textbf{fíl}  \textbf{slíp}.\\
\textsc{1sg.sbj}  \textsc{ipfv}  feel  hungry  \textsc{1sg.sbj}  \textsc{ipfv}  feel  sleep\\

\glt ‘I’m feeling hungry, I’m feeling sleepy.’ [ye07fn 132]
\z


\ea%1256
    \label{ex:key:1256}
    \gll A    \textbf{fíl}  di  pikín  in    \textbf{pén}.\\
\textsc{1sg.sbj}  feel  \textsc{def}  child  \textsc{3sg.poss}  pain\\

\glt ‘I went into labour.’ [\textit{Lit.} ‘I felt the child’s pain].’ [ab03ay 076]
\z


\ea%1257
    \label{ex:key:1257}
    \gll A    \textbf{sɔ́fa}    wán    \textbf{hángri}  na  dán  kɔ́ntri. \\
\textsc{1sg.sbj}  suffer  one    hunger  \textsc{loc}  that  country\\

\glt ‘I endured extraordinary hunger in that country.’ [dj07ae 121]
\z

Proof for the nominal status of the body state in the constructions above is provided by sentences \REF{ex:key:1256} and \REF{ex:key:1257}. In the former example, we find a dislocated possessive construction\is{possessive constructions} in the object position of \textit{fíl}. In the latter example, the indefinite determiner \textit{wán} ‘one, a’ precedes \textit{hángri} ‘hunger’, the object of \textit{sɔ́fa} ‘endure’.


The type 1c construction also serves to express the body states ‘feel hot’ and ‘feel cold’. Compare the following two examples:



\ea%1258
    \label{ex:key:1258}
    \gll A    de  \textbf{fíl}  tú  mɔ́ch  \textbf{hɔ́t}.\\
\textsc{1sg.sbj}  \textsc{ipfv}  feel  too  much  heat\\

\glt ‘I’m feeling too hot.’ [dj07ae 316]
\z


\ea%1259
    \label{ex:key:1259}
    \gll E    de  \textbf{fíl}  \textbf{gúd}    ɛf  e    dé    míndul  pípul.\\
\textsc{3sg.sbj}  \textsc{ipfv}  feel  good  if  \textsc{3sg.sbj}  \textsc{be.loc}  middle  people\\

\glt ‘She feels good if she’s amongst people.’ [ro05ee 117]
\z

Type 2 constructions involve intransitive clauses. In type 2a, the experiencer appears in the subject position. The body state is instantiated in a dynamic verb. Once more, the basic body states of hunger and thirst can be expressed in this way (\ref{ex:key:1260}–\ref{ex:key:1261}). However, other transient body states like \textit{sík} ‘be sick’ also appear in this construction \REF{ex:key:1262}: 


\ea%1260
    \label{ex:key:1260}
    \gll A    \textbf{de}  \textbf{hángri}.\\
\textsc{1sg.sbj}  \textsc{ipfv}  be.hungry\\

\glt ‘I’m hungry.’ [dj07ae 322]
\z


\ea%1261
    \label{ex:key:1261}
    \gll A    \textbf{de}  \textbf{tɔ́sti}.\\
\textsc{1sg.sbj}  \textsc{ipfv}  be.thirsty\\

\glt ‘I’m thirsty.’ [dj07ae 326]
\z


\ea%1262
    \label{ex:key:1262}
    \gll Wán    dé  wán    pikín  bin  \textbf{de}  \textbf{sík}.\\
one    day  one    child  \textsc{pst}  \textsc{ipfv}  sick\\

\glt ‘One day, a certain child was sick.’ [ye03cd 071]
\z

In type 2b constructions, the body state verb is inchoative-stative. Compare \textit{táya} ‘be tired’ \REF{ex:key:1263} \textit{wɛ́l} ‘be well’ \REF{ex:key:1264}, \textit{bɛlfúl} ‘be satiated’ \REF{ex:key:1265}, and \textit{hát} ‘be hurt’ \REF{ex:key:1266}:


\ea%1263
    \label{ex:key:1263}
    \gll A    \textbf{táya}.\\
\textsc{1sg.sbj}  be.tired\\

\glt ‘I’m tired.’ [dj07ae 318]
\z


\ea%1264
    \label{ex:key:1264}
    \gll A    \textbf{wɛ́l}.\\
\textsc{1sg.sbj}  well\\

\glt ‘I’m well.’ [li07fn 011]
\z


\ea%1265
    \label{ex:key:1265}
    \gll A    \textbf{bɛlfúl}.\\
\textsc{1sg.sbj}  be.satiated\\

\glt ‘I’m full.’ [dj07ae 524]
\z


\ea%1266
    \label{ex:key:1266}
    \gll Di  gál  \textbf{hát}.\\
\textsc{def}  girl  hurt\\

\glt ‘The girl is hurt.’ [dj05be 006]
\z

Type 2c constructions are intransitive copula clauses. The body state verb appears as an adjective complement\is{complements} to the locative-existential\is{copula:locative-existential} copula \textit{dé} \textsc{‘be.loc’} \REF{ex:key:1267}. The property items \textit{gúd} ‘be well’, \textit{bád} ‘be bad’, and \textit{fáyn} ‘be fine’ appear in such predicate adjective constructions when they express a transient body state rather than an (intrinsic) value (cf. \sectref{sec:7.6.5}):


\ea%1267
    \label{ex:key:1267}
    \gll Dán    tɛ́n    a    \textbf{dé}    \textbf{fáyn}.\\
that    time    \textsc{1sg.sbj}  \textsc{be.loc}  fine\\
\glt ‘That time, I was fine.’   [ru03wt 024]
\z

The two body state expressions \textit{sík} ‘be sick’ and \textit{bɛlfúl} ‘be satiated’ may also appear in transitive clauses involving associative objects (cf. \sectref{sec:9.3.2}):


\ea%1268
    \label{ex:key:1268}
    \gll E    de  \textbf{sík}  \textbf{fíba}.\\
\textsc{3sg.sbj}  \textsc{ipfv}  sick  fever\\

\glt ‘She’s sick with fever.’ [dj07ae 273]
\z


\ea%1269
    \label{ex:key:1269}
    \gll A    \textbf{bɛlfúl}    \textbf{plantí}.\\
\textsc{1sg.sbj}  be.satiated  plantain\\

\glt ‘I’m full with plantain.’ [dj07ae 529]\is{body states}
\z

\section{Valency adjustments}\label{sec:9.4}

Verb valency\is{valency} is adjusted in three ways. For one part, the omission of the core participant\is{core participant} subject\is{subject} (cf. \sectref{sec:9.4.1}) or object (cf. \sectref{sec:9.4.2}) reduces verb valency by one. Object omission is also at play when reflexive and reciprocal object pronouns remain unexpressed (cf. \sectref{sec:9.4.3}). Second, a notional patient\is{patient} object may be added to a clause by employing a causative construction (cf. \sectref{sec:9.4.4}). Causative constructions involve biclausal structures and secondary predication. They are therefore a means of increasing valency periphrastically.


Thirdly, an agent can be backgrounded, though not wholly removed, by employing as the subject the \textsc{3pl} dependent pronoun \textit{dɛn} or a generic human-denoting noun with impersonal reference (cf. \sectref{sec:9.4.5}). In that, agent\is{agent} backgrounding is functionally similar to passive voice in other languages. 


\subsection{Unexpressed subjects}\label{sec:9.4.1}

Subjects are normally expressed overtly but subject omission\is{subject omission} (indicated by ${\emptyset}$) occasionally occurs with verbs with impersonal reference, as with \textit{fít} ‘can’ in an excerpt from a procedural text \REF{ex:key:1270}: 


\ea%1270
    \label{ex:key:1270}
    \gll \textbf{${\emptyset}$}  \textbf{fít}  sifta    ín    sóté    tú  tɛ́n    mék    mék
dán  smɔ́l  smɔ́l  watá  dɛn  nó  lɛ́f.\\
\textsc{2sg}  can  sieve  \textsc{3sg.indp}  until  two  time    make  \textsc{sbjv}
that  small  \textsc{rep}    water  \textsc{pl}  \textsc{neg}  remain\\

\glt ‘(You) can sift it up to two times to make none of that little water remain.’ [dj03do 008]
\z

In another context, we find something similar to subject omission. The quotative marker \textit{sé} may appear at the beginning of an independent prosodic unit, rather than within a prosodically integrated sentence. In such contexts, the element \textit{sé} straddles the boundary of a verbal meaning ‘say’ and its function as a quotative marker and introducer of direct discourse. Hence, the “absence” of a subject may be seen as a form of omission (cf. also \sectref{sec:10.4}).


The following two sentences are uttered in sequence by the same speaker. Compare the ambiguous function of \textit{sé} in \REF{ex:key:1271b}, which is introduced by \textit{sé}, with (a) where \textit{sé} is firmly integrated into the sentence as a quotative marker:



\ea%1271
    \label{ex:key:1271}
\ea{
    \gll E    tɛ́l=an    \textbf{sé}  “papá  mí    nɛ́va  chɔ́p
  mi    sénwe”.\\
  \textsc{3sg.sbj}  tell=\textsc{3sg.obj}  \textsc{quot}  \phantom{“}father  \textsc{1sg.indp}  \textsc{neg}.\textsc{prf}  eat
  \textsc{1sg.indp}  \textsc{emp}\\
\glt   ‘\textstylePichitranslationZchn{He told him “please, I myself haven’t eaten yet}”.’ [ye03cd 149]
}\ex{\label{ex:key:1271b}
\gll
\textbf{Sé}    chico,  dí  tín    nó  go  dú  mí.\\
  \textsc{quot}    boy    this  thing  \textsc{neg}  \textsc{pot}  do  \textsc{1sg.indp}\\
\glt   ‘(He said) “man, this won’t do for me”.’ [ye03cd 150]
}\z\z

A final form of subject omission occurs when the particles \textit{na} ‘\textsc{foc}’ and \textit{nóto} ‘\textsc{neg}.\textsc{foc}’ incorporate \textsc{3sg} reference by default in their function as identity copulas. When pronominal reference is to be overtly established, \textit{na}\textit{\textup{/}}\textit{nóto} must be preceded by independent (emphatic) personal pronoun (cf. also \sectref{sec:7.6.1}). Dependent pronouns may not precede these two particles.

\subsection{Unexpressed objects}\label{sec:9.4.2}

In principle, objects need not be overtly expressed. In practice, highly transitive verbs are unlikely to appear without a patient object, even if the object is non-specific. The verb \textit{bló} ‘give a blow’ in \REF{ex:key:1272} denotes a situation which implies a high degree of volition and instigation by an agent. Equally, the situation involves no notion of affectedness of the agent (cf. \citealt{Naess2007}):


\ea%1272
    \label{ex:key:1272}
    \gll A    \textbf{bló}      \textbf{dí}  \textbf{pikín}.\\
\textsc{1sg.sbj}  give.blow  this  child\\

\glt ‘I gave this child [guy] a blow.’ [dj07ae 031]
\z

When \textit{bló} occurs without an object it is understood to be the homophonous \textit{bló} ‘rest, relax’ \REF{ex:key:1273}, a verb which is lower on the transitivity scale, but may also be used transitively \REF{ex:key:1274}, due to its status as a labile\is{labile verbs} experiential verb: 


\ea%1273
    \label{ex:key:1273}
    \gll A    de  \textbf{bló}    ɔ  a    de  \textbf{rɛ́s}.\\
\textsc{1sg.sbj}  \textsc{ipfv}  relax  or  \textsc{1sg.sbj}  \textsc{ipfv}  rest\\

\glt ‘I’m relaxing or I’m resting.’ [dj07ae 030]
\z


\ea%1274
    \label{ex:key:1274}
    \gll Mék    a    \textbf{bló}    \textbf{dí}  \textbf{pɔ́sin}  mék    e    fít  recupera.\\
\textsc{sbjv}    \textsc{1sg.sbj}  relax  this  person  \textsc{sbjv}    \textsc{3sg.sbj}  can  recover\\

\glt ‘Let me make this person rest for her to be able to recover.’ [dj07ae 033]
\z

When highly transitive verbs are used in a context of non-specificity, they usually occur with generic noun\is{generic nouns}s as objects. Compare the non-specific object \textit{sɔn} \textit{tín} ‘something’ of the highly transitive verb \textit{híb} ‘throw (away)’ \REF{ex:key:1275} and \textit{pɔ́sin} ‘person’, object of \textit{nák} ‘hit’ \REF{ex:key:1276}: 


\ea%1275
    \label{ex:key:1275}
    \gll \op...\cp{}  yu  \textbf{híb}    \textbf{sɔn}   \textbf{tín}    fɔ  grɔ́n    \op...\cp{}\\
  {} \textsc{2sg}  throw  some  thing  \textsc{prep}  ground\\

\glt ‘(...) (if) you throw something on the ground (...)’ [hi03cb 028]
\z


\ea%1276
    \label{ex:key:1276}
    \gll \op...\cp{}  na  ín    e    de  \textbf{nák}  \textbf{pɔ́sin}.\\
  {} \textsc{foc}  \textsc{3sg.indp}  \textsc{3sg.sbj}  \textsc{ipfv}  hit  person\\

\glt ‘(...) that’s why she’s hitting somebody.’ [au07se 191]
\z

The omission of objects is more common with verbs characterised by a lower degree of semantic transitivity, in particular where the objects are non-specific. Object omission is therefore principally found with “effected-object verbs” \citep{Hopper1985} and “affected-agent verbs” (\citealt{Tenny1994,Naess2007}).


The objects of effected-object verbs come into existence through the situation denoted by the verb. They are not affected or changed by the situation denoted by the verb like the patient objects of more prototypically transitive verbs. The non-specific effected objects of verbs of speech and sound emission often occur without a speech- and sound-denoting noun or pronoun. Consider the following use of \textit{tɔ́k} ‘say, talk’ in a transitive \REF{ex:key:1277} and in an intransitive clause \REF{ex:key:1278}: 



\ea%1277
    \label{ex:key:1277}
    \gll Bikɔs  yu  dɔ́n  \textbf{tɔ́k}  \textbf{wán}    \textbf{bád}  \textbf{tɔ́k},    e    sé
“gɔ́d    háma  yu  mɔ́t!”\\
because  \textsc{2sg}  \textsc{prf}  talk  one    bad  word  \textsc{3sg.sbj}  \textsc{quot}
\phantom{‘}God    hammer  \textsc{2sg}  mouth\\

\glt ‘Because you have said something bad, she says 
“may God hammer your mouth!”’ [au07se 030]
\z


\ea%1278
    \label{ex:key:1278}
    \gll Sé    “a    bin  sí  bɔt  a    nó  fít  \textbf{tɔ́k}.”\\
\textsc{quot}    \textsc{1sg.sbj}  \textsc{pst}  see  but  \textsc{1sg.sbj}  \textsc{neg}  can  talk.\\

\glt ‘(He) said “I saw (it) but I couldn’t talk”.’ [kw03sb 167]
\z

Another verb that may be used in this way is \textit{síng} ‘síng’ \REF{ex:key:1279}:


\ea%1279
    \label{ex:key:1279}
    \gll E    de  \textbf{síng}    na  Píchi.\\
\textsc{3sg.sbj}  \textsc{ipfv}  síng    \textsc{loc}  Pichi\\

\glt ‘He sings in Pichi.’ [au07se 233]
\z

Likewise, the effected non-specific objects of verbs denoting a process of production may remain unexpressed. Compare \textit{só} ‘sew’ (\ref{ex:key:1280}–\ref{ex:key:1281}) and \textit{kúk} ‘cook’ (\ref{ex:key:1282}–\ref{ex:key:1283}) in the transitive and intransitive sentence pairs below: 


\ea%1280
    \label{ex:key:1280}
    \gll \op...\cp{}  wé  yu  nó  nó    to  fíks  wán  klós,  to  \textbf{só}  \textbf{wán}    \textbf{klós}    \op...\cp{}\\
  {} \textsc{sub}  \textsc{2sg}  \textsc{neg}  know  to  fix  one  clothing  to  sew  one    clothing\\

\glt ‘(...) when you don’t know how to fix a dress, to sew a dress (...)’ [hi03cb 120]
\z


\ea%1281
    \label{ex:key:1281}
    \gll Di  sastre  de    \textbf{só}.\\
\textsc{def}  tailor  \textsc{ipfv}    sew\\

\glt ‘The tailor is sewing.’ [dj07ae 353]
\z


\ea%1282
    \label{ex:key:1282}
    \gll E    kin  \textbf{kúk}    \textbf{súp}.\\
\textsc{3sg.sbj}  \textsc{hab}  cook  soup\\

\glt ‘He cooks soups.’ [ye03cd 086]
\z


\ea%1283
    \label{ex:key:1283}
    \gll Di  húman  kán    na  hós    di  áwa    wé  a    de  \textbf{kúk}.\\
\textsc{def}  woman  come  \textsc{loc}  house  \textsc{def}  hour  \textsc{sub}  \textsc{1sg.sbj}  \textsc{ipfv}  cook\\

\glt ‘The woman came to the house at the time when I was cooking.’ [ro05de 022]\is{effected objects}
\z

Affected-agent verbs are also lower on the scale of semantic transitivity than prototypical transitive verbs, because the actors are themselves affected by the situation in addition to the undergoer. In this group, we find transitive motion verbs\is{motion verbs} like \textit{rích} ‘reach, arrive’ (\ref{ex:key:1284}–\ref{ex:key:1285}) and \textit{gó} ‘go (away)’ (\ref{ex:key:1286}–\ref{ex:key:1287}), whose goal\is{goal} objects may remain unexpressed: 


\ea%1284
    \label{ex:key:1284}
    \gll Yu  nɛ́a    \textbf{rích}    \textbf{Lubá}?\\
\textsc{2sg}  \textsc{neg}.\textsc{prf}  arrive  \textsc{place}\\

\glt ‘You’ve not yet been to Luba?’ [li07re 058]
\z


\ea%1285
    \label{ex:key:1285}
    \gll E    dɔ́n    \textbf{rích}.\\
\textsc{3sg.sbj}  \textsc{prf}    arrive\\

\glt ‘He has arrived.’ [dj07ae 356]
\z


\ea%1286
    \label{ex:key:1286}
    \gll Bueno,  a    de  \textbf{gó}  \textbf{mákit}  náw.\\
good  \textsc{1sg.sbj}  \textsc{ipfv}  go  market  now\\

\glt ‘Alright, I’m going to the market now.’ [ro05fe 047]
\z


\ea%1287
    \label{ex:key:1287}
    \gll A    go  \textbf{gó}.\\
\textsc{1sg.sbj}  \textsc{pot}  go\\

\glt ‘I’ll (eventually) go.’ [ra07se 097]
\z

Typical affected-agent verbs are the ingestive verbs \textit{chɔ́p} \REF{ex:key:1288} ‘eat’ and \textit{dríng} ‘drink’ \REF{ex:key:1289}. These two transitive verbs are usually encountered without a patient\is{patient} object when its reference is non-specific. Note that object omission with \textit{dríng} in combination with a habitual reading renders the idiomatic meaning ‘habitually drink alcohol’:


\ea%1288
    \label{ex:key:1288}
    \gll A    kán  \textbf{chɔ́p}.\\
\textsc{1sg.sbj}  \textsc{pfv}  eat\\

\glt ‘(Then) I ate.’ [ed03sb 016]
\z


\ea%1289
    \label{ex:key:1289}
    \gll Dí  pɔ́sin  \textbf{de}  \textbf{dríng},  na  chak-mán.\\
this  person  \textsc{ipfv}  drink  \textsc{foc}  drink\textsc{.cpd}{}-man\\

\glt ‘This person drinks, he’s a drunkard.’ [dj07ae 363]
\z

A final group of affected-agent verbs denote sensory perception, as well as mental and physical activities. Verbs belonging to this group that regularly occur without an overt non-specific object are \textit{lúk} ‘look’, \textit{hía} ‘hear, understand’, \textit{sabí}/\textit{nó} ‘know’, and \textit{sí} ‘see’. 


When \textit{sí} ‘see, perceive’ occurs without an object, its non-specific reading may translate as ‘understand’ or ‘witness’ (cf. e.g. \ref{ex:key:1278}). However, \textit{sí} is also very often encountered in a non-specific context with a \textsc{3sg} object pronoun \REF{ex:key:1290} or an object \textsc{NP} \textit{di tín} ‘the thing’ \REF{ex:key:1291}. Both of these objects are only faintly referential and therefore appear to function as dummy\is{dummy nouns} objects in very much the same way as non-referential subjects with expletive verbs (cf. \sectref{sec:9.2.4}):



\ea%1290
    \label{ex:key:1290}
    \gll Yɛ́s,  yu  de  \textbf{sí=an}?\\
yes  \textsc{2sg}  \textsc{ipfv}  see=\textsc{3sg.obj}\\

\glt ‘Yes, do you understand?’ [dj05ae 188]
\z


\ea%1291
    \label{ex:key:1291}
    \gll Yu  \textbf{sí}  \textbf{di}  \textbf{tín}?\\
\textsc{2sg}  see  \textsc{def}  thing\\

\glt ‘You see?’ [ur05fn 013]
\z

The cognition verb \textit{mɛ́mba} often appears without an explicit object with its meaning of ‘remember’ \REF{ex:key:1292}:


\ea%1292
    \label{ex:key:1292}
    \gll A    nó  de  \textbf{mɛ́mba}.\\
\textsc{1sg.sbj}  \textsc{neg}  \textsc{ipfv}  remember\\

\glt ‘I don’t remember.’ [fr03ft 047]
\z

However, when \textit{mɛ́mba} occurs in a transitive clause, it is best translated as ‘think of’, both with a specific object \REF{ex:key:1293} and a non-specific one \REF{ex:key:1294}: 


\ea%1293
    \label{ex:key:1293}
    \gll A    kin  \textbf{mɛ́mba}  \textbf{yú}    bɔkú.\\
\textsc{1sg.sbj}  \textsc{hab}  think  \textsc{2sg.indp}  much\\

\glt ‘I think of you a lot.’ [nn05fn 045 ]
\z


\ea%1294
    \label{ex:key:1294}
    \gll Nó  hambɔ́g  mí,    a    de  \textbf{mɛ́mba}  \textbf{sɔn}    \textbf{tín}!\\
\textsc{neg}  bother  \textsc{1sg.indp}  \textsc{1sg.sbj}  \textsc{ipfv}  think  some  thing\\

\glt ‘Don’t bother me, I’m thinking about something!’ [fr 05fn 111]
\z

Likewise, verbs denoting physical activities often occur with unexpressed objects. Consider \textit{plé} ‘play’ in \REF{ex:key:1295}: 


\ea%1295
    \label{ex:key:1295}
    \gll Bɔt  wi  fít  de  \textbf{plé}  a    jám        yú     yu  fɔdɔ́n.\\
but  \textsc{1pl}  can  \textsc{ipfv}  play  \textsc{1sg.sbj}  make.contact    \textsc{2sg.indp}  \textsc{2sg}  fall\\

\glt ‘But we could be playing [football], I hit you (and) you fall.’ [au07se 178]
\z

The non-specific objects of verbs denoting the characteristic property of an agent\is{agent} often remain unexpressed. A sense of non-specificity permeates the following example featuring the verb \textit{bɛt} ‘bite’. It manifests itself in the use of the bare noun \textit{dɔ́g} ‘dog’, the presence of the habitual\is{habitual aspect} aspect marker \textit{kin} and the absence of an overt object:


\ea%1296
    \label{ex:key:1296}
    \gll Dɔ́g    kin  bɛ́t.\\
dog    \textsc{hab}  bite\\

\glt ‘Dogs bite.’ [dj07ae 371]\is{objects}
\z

\subsection{Unexpressed reflexive and reciprocal nominals}\label{sec:9.4.3}

Pichi speakers may make use of the reflexive anaphor \textit{sɛ́f} or a body part{\fff} noun in order to express reflexivity and reciprocity (cf. \sectref{sec:9.3.5} and \sectref{sec:9.3.6}). There are also verbs that allow a reflexive interpretation but do not generally occur with a reflexive pronoun. Verbs whose reflexive pronouns usually remain unexpressed instantiate “middle voice” \citep{Kemmer1993} and denote situations that imply volition and instigation by the agent, involve physical action of the agent upon her/himself, or imply movement of the body.


The following examples involve the “body care” verbs \textit{wás} ‘wash’ \REF{ex:key:1297}, \textit{báf} ‘bathe’ \REF{ex:key:1298}, and \textit{wɛ́r} ‘dress (up)’ \REF{ex:key:1231}. Note that \textit{wɛ́r} takes an object in \REF{ex:key:1300} and still implies reflexivity:



\ea%1297
    \label{ex:key:1297}
    \gll Dɛn  de  kán    sé    dɛn  kán    \textbf{wás}.\\
\textsc{3pl}  \textsc{ipfv}  come  \textsc{quot}    \textsc{3pl}  come  wash\\

\glt ‘They come to wash themselves.’ [nn07fn 145]
\z


\ea%1298
    \label{ex:key:1298}
    \gll Yu  dɔ́n  \textbf{báf}?\\
\textsc{2sg}  \textsc{prf}  bathe\\

\glt ‘Have you bathed?’ [dj07ae 377]
\z


\ea%1299
    \label{ex:key:1299}
    \gll A    \textbf{wɛ́r}.\\
\textsc{1sg.sbj}  wear\\

\glt ‘I’m dressed up.’ [ye05ae 233]
\z


\ea%1300
    \label{ex:key:1300}
    \gll Na  lɛk  if  yu  \textbf{wɛ́r}    \textbf{sɔ́t}    di  gud-sáy
wet    di  rɔn-sáy.\\
\textsc{foc}  like  if  \textsc{2sg}  wear  shirt  \textsc{def}  good\textsc{.cpd}{}-side
with    \textsc{def}  wrong\textsc{.cpd}{}-side\\

\glt ‘That’s like if you put on a shirt the right way or inside out.’ [au07se 049]
\z

In principle, these verbs may also occur with a reflexive pronoun, although they do so less frequently. Compare the usage of \textit{wás} ‘wash (oneself)’ and \textit{wɛ́r} ‘dress (up)’ in the following sentences: 


\ea%1301
    \label{ex:key:1301}
    \gll \textbf{Wás}    yu  \textbf{skín}!\\
wash  \textsc{2sg}  body\\

\glt ‘Wash yourself!’ [dj07ae 504]
\z


\ea%1302
    \label{ex:key:1302}
    \gll Djunais  \textbf{wɛ́r}    in    \textbf{sɛ́f}.\\
\textsc{name}  wear  \textsc{3sg.poss}  self\\

\glt ‘Djunais has dressed up.’ [dj07ae 375]
\z

The basic posture verbs \textit{slíp} ‘lie (down), sleep’, \textit{tínap} ‘stand (up)’ and \textit{sidɔ́n} ‘sit (down)’ are never encountered with a reflexive pronoun in the corpus (cf. \sectref{sec:8.1.3} for an extensive treatment). In contrast, verbs denoting less prototypical postures, e.g. \textit{líng} ‘lean over’ and \textit{bɛ́n} ‘bend (over)’ in (\ref{ex:key:1303}–\ref{ex:key:1304}), as well as those denoting other types of body-related events, e.g. \textit{háyd} ‘hide’ in (\ref{ex:key:1305}–\ref{ex:key:1306}) are found with or without reflexive pronouns:


\ea%1303
    \label{ex:key:1303}
    \gll E    de  wáka  e    \textbf{bɛ́n}.\\
\textsc{3sg.sbj}  \textsc{ipfv}  walk  \textsc{3sg.sbj}  bend\\

\glt ‘He is walking stooped over.’ [ra07se 080]
\z


\ea%1304
    \label{ex:key:1304}
    \gll Sé    dɛn  \textbf{líng}    dɛn  sɛ́f  ɔ  fɔ  lɛk  háw  
dɛn  \textbf{bɛ́n}    dɛn  sɛ́f?\\
\textsc{quot}    \textsc{3pl}  lean    \textsc{3pl}  self  or  \textsc{prep}  like  how  
\textsc{3pl}  bend  \textsc{3pl}  self\\

\glt ‘That they’re leaning (onto something) or how 
they’re stooped over?’ [dj07re 026]
\z


\ea%1305
    \label{ex:key:1305}
    \gll A    kán  \textbf{háyd}  ínsay  hós.\\
\textsc{1sg.sbj}  \textsc{pfv}  hide    inside  house\\

\glt ‘(Then) I hid in the house.’ [dj07ae 382]
\z


\ea%1306
    \label{ex:key:1306}
    \gll A    \textbf{háyd}  mi    sɛ́f  na  hós.\\
\textsc{1sg.sbj}  hide    \textsc{1sg.poss}  self  \textsc{loc}  house\\

\glt ‘I hid myself in the house.’ [dj07ae 383]
\z

Other verbs in this group that occur with or without reflexive pronouns are the synonymous verbs \textit{bló} ‘rest’ or \textit{rɛ́s} ‘rest’:


\ea%1307
    \label{ex:key:1307}
    \gll A    de  \textbf{bló}    ɔ  a    de  \textbf{rɛ́s}.\\
\textsc{1sg.sbj}  \textsc{ipfv}  relax  or  \textsc{1sg.sbj}  \textsc{ipfv}  rest\\

\glt ‘I’m relaxing or I’m resting.’ [dj07ae 030]
\z


\ea%1308
    \label{ex:key:1308}
    \gll A    wánt  gó  \textbf{rɛ́s}  \textbf{mi}    \textbf{sɛ́f}.\\
\textsc{1sg.sbj}  want  go  rest  \textsc{1sg.poss}  self\\

\glt ‘I want to go rest.’ [dj07ae 379]
\z


\ea%1309
    \label{ex:key:1309}
    \gll A    wánt  gó  \textbf{bló}    \textbf{mi}    \textbf{sɛ́f}.\\
\textsc{1sg.sbj}  want  go  relax  \textsc{1sg.poss}  self\\

\glt ‘I want to go rest.’ [dj07ae 380]
\z

Verbs with an inherently reciprocal meaning may appear with or without the reflexive and reciprocal anaphor \textit{sɛ́f} ‘self’. Consider the use of reciprocal \textit{sɛ́f} with the sexual act denoting verbs \textit{nák} ‘knock’ \REF{ex:key:1310} and \textit{slíp} ‘sleep with’ \REF{ex:key:1311}, as well as the unexpressed reciprocal pronoun in \REF{ex:key:1312}. These examples also illustrate that sexual act denoting verbs, including highly transitive ones like \textit{nák}, do not imply a male agent in Pichi:


\ea%1310
    \label{ex:key:1310}
    \gll \op...\cp{}  wi  \textbf{nák}    \textbf{wi}  \textbf{sɛ́f}.\\
  {} \textsc{1pl}  knock  \textsc{1pl}  self\\

\glt ‘(...) we knocked each other.’ [dj07ae 300]
\z


\ea%1311
    \label{ex:key:1311}
    \gll \'{I}nsay  di  motó,  na  dé    unu  de  \textbf{slíp}    \textbf{unu}  \textbf{sɛ́f}?\\
inside  \textsc{def}  car    \textsc{foc}  there  \textsc{2pl}  \textsc{ipfv}  sleep  \textsc{2pl}  self\\

\glt \textit{‘}In the car, that’s where you sleep with each other?’ [ro05rt 020]
\z


\ea%1312
    \label{ex:key:1312}
    \gll Una    \textbf{slíp}?\\
\textsc{2pl}    sleep\\

\glt ‘You slept (with each other)?’ [fr03wt 028]
\z

Conversely, the inherently reciprocal verbs of social interaction \textit{mít} ‘meet’ and \textit{mítɔp} ‘meet’ do not normally occur with the anaphor \textit{sɛ́f} (\ref{ex:key:1313}–\ref{ex:key:1314}): 


\ea%1313
    \label{ex:key:1313}
    \gll E    tɛ́l  mí    sé    wi  kin  \textbf{mítɔp}  ínsay  wán    motó.\\
\textsc{3sg.sbj}  tell  \textsc{1sg.indp}  \textsc{quot}    \textsc{1pl}  \textsc{hab}  meet  inside  one    car\\

\glt ‘He told me “we usually meet inside a car”.’ [ro05rt 019]
\z


\ea%1314
    \label{ex:key:1314}
    \gll \'{A}fta    wi  kán  \textbf{mít}    layk    wán  seis    años  después.\\
then  \textsc{1pl}  \textsc{pfv}  meet  like    one  six    years  afterwards\\

\glt ‘Then we met some six years later.’ [fr03ft 191]
\z

Nevertheless, like other inherently reciprocal verbs, \textit{mít} and \textit{mítɔp} may take part in a reciprocal alternation (cf. also \textit{fɛ́t} ‘fight’ in \ref{ex:key:1090}). The two participants may be expressed as coordinate subject{\fff}s in an intransitive clause while reciprocity is understood. Compare the transitive use of \textit{mít} ‘meet’ in \REF{ex:key:1315}, with its intransitive use with two coordinate subjects in \REF{ex:key:1316}:


\ea%1315
    \label{ex:key:1315}
    \gll Pero    e    \textbf{mít}    \textbf{mi}    \textbf{gran-má}.\\
but    \textsc{3sg.sbj}  meet  \textsc{1sg.poss}  grand-ma\\

\glt ‘But he met my grandmother.’ [fr03ft 085]
\z


\ea%1316
    \label{ex:key:1316}
    \gll \textbf{Mí}    \textbf{wet}    \textbf{Djunais}  wi  mítɔp.\\
\textsc{1sg.indp}  with    \textsc{name}  \textsc{1pl}  meet\\

\glt ‘Me and Djunais (we) met.’ [dj07ae 092]
\z

A further example for this alternation is provided with \textit{fíba} ‘resemble’ in the following transitive and intransitive sentences:


\ea%1317
    \label{ex:key:1317}
    \gll \textbf{Djunais}  fíba      \textbf{Boyé}.\\
\textsc{name}  resemble    \textsc{name}\\

\glt ‘Djunais resembles Boyé.’ [dj08ae 397]\is{reflexivity}
\z


\ea%1318
    \label{ex:key:1318}
    \gll \textbf{Djunais}  \textbf{wet}    \textbf{Boyé}  dɛn  fíba.\\
\textsc{name}  with    \textsc{name}  \textsc{3pl}  resemble\\

\glt ‘Djunais and Boyé (they) resemble (each other).’ [dj07ae 393]\is{reciprocity}
\z

\subsection{Causative constructions}\label{sec:9.4.4}

A lexically restricted means of expressing causation in Pichi is the use of labile verbs\is{labile verbs} in transitive clauses (cf. \sectref{sec:9.2.3}). Pichi also features inherently causative verbs like \textit{kíl} ‘kill’, which pairs with \textit{dáy} ‘die’ in a semantic relation of causation. In this section, we are, however, only concerned with fully productive means of causative expression in Pichi. 


Pichi causative constructions are periphrastic and involve the use of subordinate predication. Hence, the causative verb is realised as a main verb to a subordinate predicate of effect. \tabref{tab:key:9.12} summarises the majority patterns of causative formation in Pichi. Minor variations to these patterns are discussed below.


%%please move \begin{table} just above \begin{tabular
\begin{table}
\caption{Causative constructions}
\label{tab:key:9.12}
\small
\begin{tabularx}{\textwidth}{XlQQ}
\lsptoprule

Function & Causative verb & Expression of causee & Expression of effect \\
\midrule
Causative & \textstyleTablePichiZchn{mék} ‘make’ & Subject of \textsc{sbjv} clause & Subjunctive clause \\
\tablevspace
Permissive causative & \textit{lɛ́f} ‘leave’ & Object of \textit{lɛ́f} and simultaneously subject of \textsc{sbjv} clause & Subjunctive clause \\
\tablevspace
Resultative causative & \textit{lɛ́f} ‘leave’ & Object of \textit{lɛ́f} & Resultative complement\is{resultative constructions}\\
\lspbottomrule
\end{tabularx}
\end{table}
Causative and permissive constructions are formed with the two verbs \textit{mék} ‘make’ and \textit{lɛ́f} ‘leave, permit’. Examples (\ref{ex:key:1319}–\ref{ex:key:1320}) present their use in non-causative transitive clauses:


\ea%1319
    \label{ex:key:1319}
    \gll Yu  fít  \textbf{mék}    mí    wán    café?\\
\textsc{2sg}  can  make  \textsc{1sg.indp}  one    coffee\\

\glt ‘Can you make me a coffee?’ [ye07ga 034]
\z


\ea%1320
    \label{ex:key:1320}
    \gll A    sé    a    nó  fít  \textbf{lɛ́f}=an.\\
\textsc{1sg.sbj}  \textsc{quot}    \textsc{1sg.sbj}  \textsc{neg}  can  leave=\textsc{3sg.obj}\\

\glt ‘I said I can’t leave her (behind).’ [ab03ay 143]
\z

Two types of causative constructions can be distinguished on formal grounds (cf. \citealt{Yakpo2012a,Yakpo2017}). The most common type of causative construction in Pichi inolves a “balanced” structure \citep{Cristofaro2003}. The causative event is expressed in two finite clauses and the causative verb and the verb-of-effect are linked in a relation of subordination\is{subordination}. Sentence \REF{ex:key:1321} below features the (inanimate) causer NP \textit{lotería} ‘lottery’, the causative main verb \textit{mék} ‘make’, the causee NP \textit{mi mɔní} ‘my money’, and the subordinate verb-of-effect \textit{bɔkú} ‘be much’. The subordinate status of the effect situation is evident through its appearance in a subjunctive clause introduced by the modal complementiser and subjunctive marker \textit{mék} ‘\textsc{sbjv}’:


\ea%1321
    \label{ex:key:1321}
    \gll Lotería  dɔ́n  \textbf{mék}    \textbf{mék}    mi    mɔní  \textbf{bɔkú}.\\
lottery  \textsc{prf}  make  \textsc{sbjv}    \textsc{1sg.poss}  money  be.much\\

\glt ‘The lottery has made my money become a lot.’ [dj07ae 198]
\z

The second type of causative construction involves a “deranked” \citep{Cristofaro2003} or “reduced” \citep{Lehmann1988} structure and argument sharing. The causee (here \textit{=an} \textsc{‘3sg.obj’}) is the syntactic object of the causative main verb \textit{mék} and at once the notional subject of the subordinate verb-of-effect \textit{gó} as in \REF{ex:key:1322}. This construction is marginal in terms of frequency, and only attested with Group 1 speakers (cf. \sectref{sec:1.3}). I could not identify any semantic differences between the two types of caustive constructions:


\ea%1322
    \label{ex:key:1322}
    \gll A    go  \textbf{mék}=an      \textbf{gó}  tumɔ́ro.\\
\textsc{1sg.sbj}  \textsc{pot}  make=\textsc{3sg.obj}    go  tomorrow\\

\glt ‘I’ll make him go tomorrow.’ [to05fn 030]
\z

Both transitive and intransitive verbs may be causativised. Example \REF{ex:key:1323} features a causative construction with the intransitive verb of effect \textit{bɛ́lch} ‘belch’ and \REF{ex:key:1324} one with the transitive verb \textit{wích} ‘bewitch’. Like all complement clauses, the subjunctive clause in these constructions can optionally be introduced by the quotative marker \textit{sé} ‘\textsc{quot}’ in addition to \textit{mék} ‘\textsc{sbjv}’ \REF{ex:key:1324}:


\ea%1323
    \label{ex:key:1323}
    \gll A    níd    fɔ  drink  sɔn    tín    wé  de  mék
\textbf{mék}    a    \textbf{bɛ́lch}\\
\textsc{1sg.sbj}  need  \textsc{prep}  drink  some  thing  \textsc{sub}  \textsc{ipfv}  make
\textsc{sbjv}    \textsc{1sg.sbj}  belch\\

\glt ‘I need to drink something that will make me belch.’ [ye07ga 029]
\z


\ea%1324
    \label{ex:key:1324}
    \gll Na  ín    mék    \textbf{sé}    \textbf{mék}    dɛn  \textbf{wích}=an.\\
\textsc{foc}  \textsc{3sg.indp}  make  \textsc{quot}    \textsc{sbjv}    \textsc{3pl}  bewitch=\textsc{3sg.obj}\\

\glt ‘That’s why he was bewitched.’ [ru03wt 011]
\z

Sentence \REF{ex:key:1325} illustrates the two options for rendering causative meaning with labile verbs\is{labile verbs}. Before the comma, the verb \textit{drɔ́ngo} ‘be/get drunk’ is used as a transitive and causative verb followed by the patient\is{patient} object pronoun \textit{=an} ‘\textsc{3sg.obj}’. In the second half of the sentence, causative meaning is expressed periphrastically through the \textit{mék} causative construction. When the second option is used, the speaker may want to express that causation is less direct. Meanwhile, the use of the transitive variant of a labile verb implies a direct, possibly even physical implication of the causer:


\ea%1325
    \label{ex:key:1325}
    \gll A    \textbf{drɔ́ngor}=an,    a    \textbf{mék}    mék    e    \textbf{drɔ́ngo}.\\
\textsc{1sg.sbj}  get.drunk=\textsc{3sg.obj}  \textsc{1sg.sbj}  make  \textsc{sbjv}    \textsc{3sg.sbj}  be.drunk\\

\glt ‘I got him drunk, I made him drunk.’ [dj07ae 053]
\z

The following example illustrates the causative use of the ditransitive transfer verb \textit{gí} ‘give’ in a double-object construction\is{double-object construction}: 


\ea%1326
    \label{ex:key:1326}
    \gll E    bin  mék    \textbf{mék}    a    \textbf{gí}  di  gɛ́l  di  plantí.\\
\textsc{3sg.sbj}  \textsc{pst}  make  \textsc{sbjv}    \textsc{1sg.sbj}  give  \textsc{def}  girl  \textsc{def}  plantain\\

\glt ‘She made me give the girl the plantain.’ [dj05be 003]
\z

There are no restrictions on negation\is{negation} in causative constructions. The causative verb in the main clause \REF{ex:key:1327} as well as the verb of effect in the subordinate clause \REF{ex:key:1328} may be negated: 


\ea%1327
    \label{ex:key:1327}
    \gll Pút  di  watá  pero  \textbf{nó}  \textbf{mék}    mék    e    \textbf{fɔdɔ́n}  nado.\\
put  \textsc{def}  water  but    \textsc{neg}  make  \textsc{sbjv}    \textsc{3sg.sbj}  fall    outside\\

\glt ‘Put the water (inside) but don’t make it fall outside (the vessel).’ [dj05be 169]
\z


\ea%1328
    \label{ex:key:1328}
    \gll Fít  sifta    ín    sóté    tú  tɛ́n    \textbf{mék}    mék
dán  smɔ́l  smɔ́l  watá  dɛn  \textbf{nó}  \textbf{lɛ́f}.\\
can  sieve  \textsc{3sg.indp}  until  two  time    make  \textsc{sbjv}
that  small  small  water  \textsc{pl}  \textsc{neg}  remain\\

\glt ‘(You) can sift it up to two times to make none of that little
water remain.’ [dj03do 008]
\z

There are instances in which TMA marking in the subjunctive clause of effect is not reduced as it usually is in a subjunctive clause (cf. \sectref{sec:10.5.1}). These instances involve the idiomatic expressions \textit{na ín mék}\textit{\textup{/}}\textit{na di tín mék} ‘that’s why’ and the question phrase \textit{wétin mék} ‘why’.


Hence, the subordinate clauses in \REF{ex:key:1329} and \REF{ex:key:1330} feature regular TMA marking via \textit{dɔ́n} ‘\textsc{prf}’ and \textit{de} ‘\textsc{ipfv}’, respectively, instead of subjunctive marking. Nonetheless, even these idioms are occasionally conceived of as regular causative constructions with the reduced TMA marking characteristic of subjunctive subordinate clauses (cf. \ref{ex:key:1324} above):



\ea%1329
    \label{ex:key:1329}
    \gll Na  ín    \textbf{mék}    dɔtí    \textbf{dɔ́n}  plɛ́nte.\\
\textsc{foc}  \textsc{3sg.indp}  make  dirty  \textsc{prf}  plenty\\

\glt ‘That’s why the dirt has become so much.’ [hi03cb 033]
\z


\ea%1330
    \label{ex:key:1330}
    \gll Wétin  \textbf{mék}    yu  nó  \textbf{de}  wók    tidé?\\
what  make  \textsc{2sg}  \textsc{neg}  \textsc{ipfv}  work  today\\

\glt ‘Why aren’t you working today?’ [ro05ee 016]
\z

The subjunctive marker \textit{mék} also introduces the complement clauses\is{complement clauses} of other main verbs, which – like the causative verb \textit{mék} ‘make’ – induce deontic modality over their subordinate clauses. One such main verb is \textit{wánt} ‘want’\textit{} \REF{ex:key:1331} (cf. \sectref{sec:10.5.5} for a full treatment of the functions of \textit{mék} ‘\textsc{sbjv’} in subordinate clauses): 


\ea%1331
    \label{ex:key:1331}
    \gll \'{U}s=sáy  yu  \textbf{wánt}  \textbf{mék}    di  smók  kɔmɔ́t?\\
\textsc{q}=side  \textsc{2sg}  want  \textsc{sbjv}    \textsc{def}  smoke  come.out\\

\glt ‘Where do you want the smoke to come out?’ [ye07fn 123]
\z

Besides that, \textit{mék} ‘\textsc{sbjv}’ introduces purpose\is{purpose clauses} and certain types of consecutive clauses (cf. \sectref{sec:10.7.6}) as well as imperative\is{imperatives}s and other types of directive main clauses (cf. \sectref{sec:6.7.3.3}). The conflation of these functions in the element \textit{mék} represents a case in which the semantic linkages within a functional domain are actually instantiated in a single form (cf. \citealt[213–30]{PerkinsPagliuca1994}; \citealt[25–33]{Song2001}).


The verb \textit{lɛ́f} ‘leave, remain’ is employed as a causative verb in the formation of permissive causatives. This type of causative is usually formed differently from the causative proper, i.e. constructions featuring the causative verb \textit{mék} ‘make’. The effect situation is also expressed in a subjunctive clause. Yet, it is commonplace to express the causee as the object of \textit{lɛ́f} and reiterate it as the subject{\fff} of the subordinate subjunctive clause. 



Consider the following two permissives and compare them with a causative construction like \REF{ex:key:1323} above. In \REF{ex:key:1323}, the causative verb \textit{mék} takes no object pronoun \textit{mí} ‘\textsc{1sg.indp}’ that is co-referential with the subject \textit{a} ‘\textsc{1sg.sbj’} of the subjunctive clause.



\ea%1332
    \label{ex:key:1332}
    \gll A    \textbf{lɛ́f}    mi    pikín  \textbf{mék}    e    gó  Panyá. \\
\textsc{1sg.sbj}  leave  \textsc{1sg.poss}  child  \textsc{sbjv}    \textsc{3sg.sbj}  go  Spain\\

\glt ‘I allowed my child to go to Spain.’ [dj07ae 443]
\z


\ea%1333
    \label{ex:key:1333}
    \gll Seis  años,  \textbf{lɛ́f}=an    \textbf{mék}    e    wɛ́r    klós,
mék    e    gó  báy  in    brɛ́d.\\
six  year.\textsc{pl}  leave=\textsc{3sg.obj}  \textsc{sbjv}    \textsc{3sg.sbj}  wear  clothing
\textsc{sbjv}    \textsc{3sg.sbj}  go  buy  \textsc{3sg.poss}  bread\\

\glt ‘(At) six years, let him dress up (by himself), let him go buy his (own) bread.’ [ab03ab 151]\is{subjunctive mood}
\z

The verb \textit{lɛ́f} ‘leave, remain’ is also employed in the formation of resultative causatives{\fff}. Resultative causative constructions serve to causativise stative situations denoted by property items, as well as stative situatons denoted by the identity copulas \textit{na/nóto} and \textit{bí} and their complements in equative clauses{\fff}. Resultative causative constructions do not feature a subordinate clause. Instead, the effect situation is expressed as a resultative complement to the causative verb \textit{lɛ́f} (cf. \sectref{sec:11.3} for resultative adjuncts in secondary predicate constructions).


Sentence \REF{ex:key:1334} features the property item \textit{yún} ‘be young’. The resultative causative equivalent in \REF{ex:key:1335} features the causer \textit{e} ‘\textsc{3sg.sbj}’ = ‘it’ (i.e. ‘the clothing’), the causee \textit{yú} ‘\textsc{2sg.indp}’, which is an object to \textit{lɛ́f} ‘leave’, as well as the resultative complement \textit{yún} ‘young’. The verb \textit{lɛ́f} in these constructions may either be used as an inchoative-stative verb, as in \REF{ex:key:1335}, or a dynamic verb, as in \REF{ex:key:1339} below, where \textit{lɛ́f} is specified by \textit{de} ‘\textsc{ipfv}’: 



\ea%1334
    \label{ex:key:1334}
    \gll Dís  húman  \textbf{yún}      yét.\\
this  woman  be.young    yet\\

\glt ‘This woman is still young.’ [ro05fe 014]
\z


\ea%1335
    \label{ex:key:1335}
    \gll E    \textbf{lɛ́f}    yú    \textbf{yún}.\\
\textsc{3sg.sbj}  leave  \textsc{2sg.indp}  be.young\\

\glt ‘It makes/made you (appear) young.’ [dj07ae 197]
\z

Example \REF{ex:key:1336} presents a non-causative predication involving the inchoative-stative property item \textit{kɔrɛ́t} ‘be correct’. The resultative causative counterpart in \REF{ex:key:1337} features the force{\fff} causer{\fff} \textit{gɔ́d} ‘God’, the causative verb \textit{lɛ́f} ‘leave’, and the resultative complement \textit{kɔrɛ́t} ‘(be) correct’:


\ea%1336
    \label{ex:key:1336}
    \gll Dí  wán    nó  \textbf{kɔrɛ́t}.\\
this  one    \textsc{neg}  be.correct\\

\glt ‘This one is not correct.’ [dj07ae 188]
\z


\ea%1337
    \label{ex:key:1337}
    \gll Gɔ́d  go  \textbf{lɛ́f}    di  mán    \textbf{kɔrɛ́t}.\\
God  \textsc{pot}  leave  \textsc{def}  man    be.correct\\

\glt ‘God will make this man righteous.’ [dj07ae 202]
\z

Sentence \REF{ex:key:1338} is an equative clause{\fff} featuring the identity copula/focus marker \textit{na} ‘\textsc{foc}’. The causative equivalent in \REF{ex:key:1339} once more features the resultative causative verb \textit{lɛ́f}, as well as the compound noun and resultative complement \textit{yun-bɔ́y} ‘young.\textsc{cpd}-boy’:


\ea%1338
    \label{ex:key:1338}
    \gll Di  húman  \textbf{na}  \textbf{yun-gɛ́l}.\\
\textsc{def}  woman  \textsc{foc}  be.young.\textsc{cdp}{}-girl\\

\glt ‘The woman is a young woman.’ [ro05fe 013]
\z


\ea%1339
    \label{ex:key:1339}
    \gll Di  klós    dɛn  de  \textbf{lɛ́f}    yú    \textbf{yun-bɔ́y}.\\
\textsc{def}  clothing  \textsc{pl}  \textsc{ipfv}  leave  \textsc{2sg.indp}  be.young\textsc{.cpd}{}-boy\\

\glt ‘These clothes make you (appear) a young man.’ [dj07ae 196]
\z

An interesting semantic aspect of the use of resultative causatives is that they are not attested with human causers occupying the agent\is{agent} role. All recorded instances of resultative causatives feature inanimate force\is{force} causers in the subject\is{subjects} position. I assume that speakers prefer to employ causative constructions featuring \textit{mék} ‘make’ where the causer is human, or where they intend to convey a notion of strong agency on the part of the causer even if it is inanimate (e.g. sentence \REF{ex:key:1321} above with the force causer \textit{lotería} ‘lottery’ and the property item \textit{b}\textit{ɔ}\textit{kú} ‘be much’ as a verb of effect).


The verb \textit{pút} ‘put’ is also used as a causative verb in a few instances in the corpus. In \REF{ex:key:1340} below, \textit{pút} is employed like \textit{lɛ́f} in \REF{ex:key:1335} and \REF{ex:key:1339} above in order to express the resultative causative equivalent of a non-causative equative clause{\fff}. The sentence contains the non-causative equative clause \textit{yu húman na bíg húman} ‘your wife is an important woman’ and the causative equivalent \textit{pút yu sɛ́f bíg mán} ‘make yourself an important man’:{\fff}



\ea%1340
    \label{ex:key:1340}
    \gll Ɛf  yu  húman  \textbf{na}  \textbf{bíg}  \textbf{húman},  e    hád    fɔ
\textbf{pút}     yu  sɛ́f  \textbf{bíg}  \textbf{mán}.\\
if  \textsc{2sg}  woman  \textsc{foc}  big  woman  \textsc{3sg.sbj}  hard  \textsc{prep}
put    \textsc{2sg}  self  big  man\\

\glt ‘If your wife is an important woman, it is difficult to make yourself 
an important man.’ [ma03hm 083]\is{causative constructions}
\z

\subsection{Impersonal constructions}\label{sec:9.4.5}

A backgrounding passive may be formed by using impersonal \textit{dɛn} ‘\textsc{3pl’} in the subject position. To begin with, the \textsc{3pl} personal pronoun \textit{dɛn} may be used generically to refer to a loosely specified collective. Example \REF{ex:key:1341} features the generic, impersonal use of \textit{dɛn} in a transitive clause: \is{agent}


\ea%1341
    \label{ex:key:1341}
    \gll \textbf{Dɛn}  de  wɛ́r    wáyt  ɔ́p  violeta  dɔ́n.\\
\textsc{3pl}  \textsc{ipfv}  wear  white  up  violet  down\\

\glt ‘They [the pupils] wear white up (and) violet down.’ [ma03hm 032]
\z

The pronoun \textit{dɛn} is also used impersonally with verbs characterised by a higher degree of semantic transitivity. In clauses with verbs that presuppose a volitional, instigating, and animate agent and an affected patient\is{patient}, impersonal use of \textit{dɛn} serves to background a non-specific agent:


\ea%1342
    \label{ex:key:1342}
    \gll Esto    na  wán    ɔ́da    kɔ́ntri,  \textbf{dɛn}  go  púl    yú    inmediatamente,
\textbf{dɛn}  de  púl    yú    wók.\\
this    \textsc{loc}  one    other  country  \textsc{3pl}  \textsc{pot}  remove  \textsc{2sg.indp}  immediately
\textsc{3pl}  \textsc{ipfv}  remove  \textsc{2sg.indp}  work\\

\glt ‘This in another country, they would remove you immediately, they
would remove you from your job.’ [ye03cd 077]
\z

The following two sentences exemplify the pragmatic and syntactic rearrangements which go along with the use of the labile\is{labile verbs} property item \textit{strét} ‘be straight, straighten’ in an intransitive \REF{ex:key:1343} and a transitive clause \REF{ex:key:1344}, respectively. In the intransitive clause, the subject\is{subjects} \textit{ród} ‘road’ is patient\is{patient} to the inchoative-stative verb \textit{strét}. In the transitive clause, impersonal \textit{dɛn} in subject position denotes the backgrounded agent, while the patient \textit{ród} is now in object position: 


\ea%1343
    \label{ex:key:1343}
    \gll Di  \textbf{ród}    strét.\\
\textsc{def}  road    be.straight\\

\glt ‘The road is straight.’ [dj07ae 122]
\z


\ea%1344
    \label{ex:key:1344}
    \gll \textbf{Dɛn}    dɔ́n    strét    di  \textbf{ród}.\\
\textsc{3pl}    \textsc{prf}    straighten  \textsc{def}  road\\

\glt ‘The road has been straightened.’ [dj07ae 123]
\z

Impersonal \textit{dɛn} always refers to an unspecified group of animate, usually human agents. The lower the agent is on the animacy scale, and hence its capacity of volition and instigation, the less likely it is to be referred to by impersonal \textit{dɛn}. For example, \REF{ex:key:1345} sounds awkward, since the backgrounded agent is construed as animate and human. A situation involving a non-human agent like \textit{snék} ‘snake’ is therefore more likely to be expressed through an ‘active’ clause with a foregrounded agent in subject position \REF{ex:key:1346}: \is{animacy}


\ea[?]{%1345
    \label{ex:key:1345}
    \gll \textbf{Dɛn}  \textbf{bɛ́t}=an    na  fám.\\
 \textsc{3pl}    bite=\textsc{3sg.obj}  \textsc{loc}  farm\\
\glt ?‘She was bitten on the farm.’ [li07fn 098]
}\z


\ea%1346
    \label{ex:key:1346}
    \gll \textbf{Snék}  \textbf{bɛ́t}=an    na  fám.\\
snake  bite=\textsc{3sg.obj}  \textsc{loc}  farm\\

\glt ‘A snake bit her on the farm.’ [li05fn 099]
\z

However, impersonal \textit{dɛn} does not retain its plural reference by default. Sentence \REF{ex:key:1347} was elicited by means of the “caused positions” video clip series of the Language and Cognition Group of the Max Planck Insitute for Psycholinguistics in Nijmegen. In all preceding clips, the agent of a series of actions had been a single individual. Nonetheless, the following sentence was given in response to a still image showing a pot lying upside down on a table: 


\ea%1347
    \label{ex:key:1347}
    \gll \textbf{Dɛn}  \textbf{pút}=an    mɔ́t    dɔ́n    fɔ  di  tébul.\\
\textsc{3pl}  put=\textsc{3sg.obj}  mouth  down  \textsc{prep}  \textsc{def}  table\\

\glt ‘It has been put mouth-down on the table.’ [li07pe 089]
\z

Impersonal \textit{dɛn} is subject to some morphosyntactic restrictions inherent to the non-specific nature of the pronoun. Impersonal \textit{dɛn} may not be focused, relativised, or subjected to other operations which require specific reference.\is{passive} 


Agent-backgrounding may also be achieved via the use of generic, non-specific, and non-referential nouns like \textit{pɔ́sin} ‘person’ and \textit{mán} ‘man, human-being’. The generic noun \textit{pɔ́sin} ‘person, human-being’ may occur as an agent subject in transitive clauses and function like impersonal \textit{dɛn} ‘\textsc{3pl}’. The noun \textit{pɔ́sin} refers to a backgrounded non-specific human agent. Compare the use of \textit{pɔ́sin} and \textit{dɛn} in these two near-identical sentences:



\ea%1348
    \label{ex:key:1348}
    \gll \textbf{Pɔ́sin}  go  entiende    bɔt  e    nó  dé    bien.\\
person  \textsc{pot}  understand  but  \textsc{3sg.sbj}  \textsc{neg}  \textsc{be.loc}  good\\

\glt ‘One would understand but it’s not correct.’ [dj05be 043]
\z


\ea%1349
    \label{ex:key:1349}
    \gll \textbf{Dɛn}    go  hía    ín    bɔt  e    nó  só    dé    claro.\\
\textsc{3pl}    \textsc{pot}  hear    \textsc{3sg.indp}  but  \textsc{3sg.sbj}  \textsc{neg}  like.that  \textsc{be.loc}  clear\\

\glt ‘It would be understood but it’s not so clear.’ [ye0502e2 050]
\z

In addition to \textit{dɛn} ‘\textsc{3pl}’, other personal pronouns are also sometimes used with weak reference. Example \REF{ex:key:1350} features the use of \textit{wi} ‘\textsc{1pl}’ in the idiom which serves as a response to the enquiry ‘how are you?’. Also compare the use of \textit{wi} in \REF{ex:key:1351}:


\ea%1350
    \label{ex:key:1350}
    \gll \textbf{Wi}  de  pús=an.\\
\textsc{1pl}  \textsc{ipfv}  push=\textsc{3sg.obj}\\

\glt ‘I’m managing.’ [\textit{Lit}. ‘We’re pushing it.’] [ur07fn 100]
\z


\ea%1351
    \label{ex:key:1351}
    \gll Na  lɛkɛ    \textbf{wí}    náw,  \textbf{wi}  de  tɔ́k  Panyá,  \textbf{wi}  go
nó    sé    dís  pɔ́sin,  na  nigeriano.\\
\textsc{foc}  like    \textsc{1pl.indp}  now    \textsc{1pl}  \textsc{ipfv}  talk  Spanish  \textsc{1pl}  \textsc{pot}
know  \textsc{quot}    this  person  \textsc{foc}  Nigerian\\

\glt ‘It’s like with us now, (if) we spoke Spanish, we would know
that this person, is Nigerian.’ [ma03hm 045]
\z

Likewise, the impersonal backgrounded use of \textit{yu} ‘\textsc{2sg}’ is common in procedural texts \REF{ex:key:1352}:


\ea%1352
    \label{ex:key:1352}
    \gll Dé,    ɛ́ni    káyn  tín    na  mɔní,  \textbf{yu}  fít  mék
ɛ́ni    káyn  tín    \textbf{yu}  go  sí  mɔní.\\
there  every  kind    thing  \textsc{foc}  money  \textsc{2sg}  can  make
every  kind    thing  \textsc{2sg}  \textsc{pot}  see  money\\

\glt ‘There, everything is money, you can do anything (and) 
you’ll earn money.’ [ma03hm 054]
\z

Finally, the copula and focus marker \textit{na} ‘\textsc{foc}’ may be used to construct purpose-like clauses \is{purpose clauses}with impersonal reference with an obligation reading in combination with the prepositions \textit{fɔ} ‘\textsc{prep}’ or \textit{to} ‘to’ and a subsequent verb without person-marking \REF{ex:key:1353}:


\ea%1353
    \label{ex:key:1353}
    \gll \textbf{Na}  \textbf{fɔ} tík=an      mɔ́.\\
\textsc{foc}  \textsc{prep}  thicken=\textsc{3sg.obj}  more\\

\glt ‘It has to be thickened more.’ [dj07ae 151]
\z


\ea%1354
    \label{ex:key:1354}
    \gll \textbf{Na}  \textbf{to} inicia  ín.\\
\textsc{foc}  to  initiate  \textsc{3sg.indp}\\

\glt ‘He has to be initiated [to social life in Malabo].’
\z

