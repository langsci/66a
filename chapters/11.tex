\chapter{Multiverb constructions}

I employ the term “multiverb constructions” (MVCs) as a generous cover term for serial verb constructions (SVCs), secondary predication, and clause chaining in Pichi. Multiverb constructions all have in common that there is some form of semantic dependence of one or more predicates with another, which is reflected in some form of reduction, restriction, or merging of elements of one or several predicates. Nonetheless, some of the constructions described in this chapter exhibit varying degrees of resemblance with some of the multiclausal structures covered in chapter \ref{sec:10}. As a consequence, the classification as “multiverb” or “multiclausal” is sometimes difficult to make.

\section{General characteristics}\label{sec:11.1}

Multiverb constructions form a continuum of structures involving predicates that are strung together in various ways. The area covered by MVCs stretches from tightly integrated verb strings to clause chains, i.e. structures that can barely be distinguished from a series of fully finite clauses. In the middle range of the continuum, we find secondary predication, which is characterised by more flexibility than SVCs, both in the types of verbs that may enter the construction as well as in the ways of paraphrasing them. SVCs are the most integrated MVCs. I use the term SVC only for constructions where “[o]ne verb is from a relatively large, open, or otherwise unrestricted class, and another from a semantically or grammatically restricted (or closed) class” \citep[21]{Aikhenvald2006}.


The verb from the restricted class in SVCs is henceforth referred to as the “minor verb” and the open-class verb as the “major verb” \citep{Durie1997}. The relative position of verbs in SVCs is indicated by V1, \textsc{V2,} \textsc{VX} irrespective of their function as minor or major verbs. Whether (and which) SVCs constitute monoclausal or multiclausal structures in Pichi is left to future research to determine.



SVCs are less central to event integration in Pichi than the variety of constructions might suggest. SVCs constitute somewhere between five to twenty per cent of the clause linkage types in a given text. Equally, older (50+ years) speakers tend to use \textsc{SVC}s more frequently. Maybe increased language contact between Pichi and the non-serialising languages Spanish and Bube has led to the reinforcement of already existing, non-serialising strategies of clause linkage in Pichi (see \citealt{Hajek2006} on contact-induced “deserialisation”). This observation concerns in particular argument-introducing SVCs, a prominent type of SVC in serialising languages of the region. Equally, there is a tendency towards the lexicalisation of SVCs involving particular verbs. This characteristic warrants analysing at least some of these SVCs as compound verbs\index{}. 


\section{Serial verb constructions}\label{sec:11.2}

\tabref{tab:key:11.1} lists all types of SVCs identified in the corpus. The table lists the minor verbs of each construction. The semantic class of the major verb is indicated in the V1 or V2 column, e.g. “dynamic verb”. Where there is no significant semantic restriction on the semantic class of the major verb, the row simply contains the entry “verb”. The possibility of using switch-function (pro)nouns is listed in the table and discussed below where it applies.

%%please move \begin{table} just above \begin{tabular
\begin{table}
\caption{Serial verb constructions}
\label{tab:key:11.1}
\small
\begin{tabularx}{\textwidth}{lp{1.8cm}lQQ}
\lsptoprule

Type of SVC & V1 & V2 & Description & Switch-function?\\
\midrule
Motion-direction & Motion verb & \itshape gó \textup{‘go’} & Motion away & Yes\\
& Motion verb & \itshape kán \textup{‘come’} & Motion towards & Yes\\
& Motion verb & \itshape kɔmɔ́t \textup{‘go out’} & Motion outwards & Yes\\
& Motion verb & \itshape rích \textup{‘reach’} & Motion up to & Yes\\

\tablevspace
Motion-action & \textstyleTablePichiZchn{gó} ‘go’ & Dynamic verb & Motion away/purpose \& action & No\\
& \textstyleTablePichiZchn{kán} ‘come’ & Dynamic verb & Motion to/purpose \& action & No\\

\tablevspace
Participant-introducing  & \textstyleTablePichiZchn{ték} ‘take’ & Dynamic verb & Instrument\is{instrument SVC}; theme\index{} & No\\
& \textstyleTablePichiZchn{fála} ‘follow’ & Motion verb & Comitative \is{comitative SVC} & No\\
& Verb & \itshape pás \textup{‘(sur)pass’} & Comparative\is{comparative SVC} & No\\


\tablevspace
Complementation & \textstyleTablePichiZchn{hía} ‘hear’,\newline  \textstyleTablePichiZchn{sí} ‘see’ & Verb & Immediate perception & Yes\\
\lspbottomrule
\end{tabularx}
\end{table}


Not included in \tabref{tab:key:11.1} are structures involving the following words with highly grammaticalised functions: \textit{go} \textsc{‘pot’} (<\textit{gó} ‘go’), \textit{kán} \textsc{‘pfv’} (<\textit{kán} ‘come’), \textit{mék} \textsc{‘sbjv’} (<\textit{mék} ‘make’), and \textit{sé} \textsc{‘quot’} (< \textit{sé} ?‘say’). Comparative SVCs featuring the minor verb \textit{pás} ‘(sur)pass’ are covered in detail in \sectref{sec:6.9}.


Verbs that participate in SVCs may not be separated by juncture markers such as declarative intonation,\is{declarative intonation} pauses, and continuative intonation\is{continuative intonation}, nor adverbial clause linkers and complementisers. Equally, the V2 may not be negated separately from the V1, while the negation\is{negation} of V1 has scope over the entire construction. Compare the following examples involving a motion-direction SVC:



\ea[*]{%1540
    \label{ex:key:1540}
    \gll Yu    kɛ́r=an    yu  \textbf{nó}  gó  hospital?\\
  \textsc{2sg}    carry=\textsc{3sg.obj}  \textsc{2sg}  \textsc{neg}  go  hospital\\
\glt Intended: ‘Didn’t you take him to hospital?’ [pa07me 006]
}\z


\ea%1541
    \label{ex:key:1541}
    \gll Yu  \textbf{nó}  kɛ́r=an    gó  hospital?\\
\textsc{2sg}  \textsc{neg}  carry=\textsc{3sg.obj}  go  hospital\\
\glt ‘Didn’t you take him to hospital?’ [pa07me 005]
\z

Further, the V2 of an SVC does not appear with TMA marking, since it acquires its TMA specifications from the V1. Only the V1 is marked for tense\is{tense}, mood, and aspect. Hence, the second translation of \REF{ex:key:1542} as a motion-direction SVC is rejected. Instead, the construction may only be interpreted as involving a depictive secondary predicate (cf. \sectref{sec:11.3.2}). The inability to be independently marked for TMA also distinguishes SVCs from the verbal complements of aspectual and modal auxiliaries, some of which may be preceded by the imperfective aspect marker as well (cf. \sectref{sec:10.5.3}). 


\ea%1542
    \label{ex:key:1542}
    \gll Yu  \textbf{de}  kɛ́r=an    \textbf{de}  gó  hospital.\\
\textsc{2sg}  \textsc{ipfv}  carry=\textsc{3sg.obj}  \textsc{neg}  go  hospital\\

\glt ‘You’re carrying him while going to the hospital.’ [pa07me 009]\\
but not ‘You’re taking him to the hospital.’
\z

\subsection{Motion-direction SVCs}\label{sec:11.2.1}

Motion-direction SVCs involve one of the four motion{\fff} verbs listed in \tabref{tab:key:11.1} as minor verbs and V2s. These verbs contribute direction to the motion expressed by the V1. The construction is only attested with a total of eight motion verbs in the major verb, V1 position (cf. \tabref{tab:key:8.5} for a summary of some of their semantic and syntactic characteristics). Of these verbs, four denote locomotion (i.e. \textit{wáka} ‘walk’, \textit{rɔ́n} ‘run’, \textit{fláy} ‘fly’, and \textit{fála} ‘follow’), while the remaining four (\textit{ték} ‘take’, \textit{kɛ́r} ‘take, carry’, \textit{bríng} ‘bring’, and \textit{sɛ́n} ‘send’) include direction, manner, and causation as part of their meaning. 


The V1 position is therefore not open to other potential candidates with similar meanings (e.g. \textit{drɛ́b} ‘drive’, \textit{ɛ́nta} ‘enter’, or \textit{pús} ‘push’), and the use of other motion verbs usually involves non-serial strategies of expressing direction. Indeed, the lexical specialisation of this SVC may justify an analysis of the construction as involving compound verbs {\fff}rather than more open structures created by syntactic processes. 



The following example presents a motion-direction SVC involving the V1 \textit{rɔ́n} ‘run’ and the V2 \textit{gó} ‘go’, which expresses motion\is{motion verbs} away from the ground. \sectref{sec:8.1.5} contains an extensive treatment of goal and source-marking in combination with motion-direction SVCs and other constructions involving spatial relations:



\ea%1543
    \label{ex:key:1543}
    \gll E    sé    “mɔ́mi  mɔ́mi,  yu  nó  de  sí  dán  mán    wé  e
\textbf{rɔ́n}    \textbf{gó}  abuela    in    rúm?”\\
\textsc{3sg.sbj}  \textsc{quot}     mum  mum  \textsc{2sg}  \textsc{neg}  \textsc{ipfv}  see  that  man    \textsc{sub}  \textsc{3sg.sbj}
run    go  grandmother  \textsc{3sg.poss}  room\\

\glt ‘He said “mum, mum, don’t you see that man who ran into 
grandmother’s room?”’ [ab03ab 053]
\z

The goal of the motion may be expressed as an object of the V2 motion verb as in \REF{ex:key:1543} above. The goal may also be instantiated in a prepositional phrase introduced by \textit{na} ‘\textsc{loc}’ \REF{ex:key:1544}. Motion{\fff}-direction SVCs featuring a transitive V1 can involve a “switch-function{\fff}” (pro)noun (\citealt[14–15]{Aikhenvald2006}), in which case the object \textit{=an} ‘\textsc{3sg.obj}’ of the V1 \textit{kɛ́r} ‘carry’ may be analysed as the subject{\fff} of the V2 \textit{gó} ‘go’ in the following example:


\ea%1544
    \label{ex:key:1544}
    \gll A    kɛ́r=\textbf{an}      gó  na  comedor.\\
\textsc{1sg.sbj}  carry=\textsc{3sg.obj}    go  \textsc{loc}  dining-room\\

\glt ‘I carried him to the dining-room.’ [ab03ab 091]
\z

A string of two verbs may be followed by additional serial verbs. Example \REF{ex:key:1545} illustrates multiple serialisation{\fff} with the verb string \textit{kɛ́r-gó-wáka} ‘carry-go-walk’. The construction is an overlap of a motion-direction SVC (\textit{kɛ́r-gó}) and a motion-action SVC (\textit{gó-wáka}): 


\ea%1545
    \label{ex:key:1545}
    \gll Di  bíg  wán,  a    bin  de  \textbf{kɛ́r}=an    \textbf{gó}  \textbf{wáka}
na  nɛ́t    wet    Tokobé. \\
\textsc{def}  big  one    \textsc{1sg.sbj}  \textsc{pst}  \textsc{ipfv}  carry=\textsc{3sg.obj}  go  walk
\textsc{loc}  night  with    \textsc{name}\\

\glt ‘As for the big one, I was carrying it off travelling by night with Tokobé.’ [ab03ab 006]
\z

The V2 \textit{kán} ‘come’ expresses motion towards a ground \REF{ex:key:1546}. Strings involving the verb \textit{kɔmɔ́t} ‘go out’ as the V2 express evacuation, i.e. motion out of a ground \REF{ex:key:1547}. Note the presence of the prepositions \textit{fɔ} ‘\textsc{prep’}\is{associative preposition} and \textit{na} ‘\textsc{loc’}\is{general} which mark goal and source,\index{} respectively:


\ea%1546
    \label{ex:key:1546}
    \gll Kɛ́r    di  motó  yu  \textbf{bríng}  \textbf{kán}    \textbf{fɔ}  yá.\\
take    \textsc{def}  car    \textsc{2sg}  bring  come  \textsc{prep}  here\\

\glt ‘Take the car and bring it here.’ [ro05de 036]
\z


\ea%1547
    \label{ex:key:1547}
    \gll E    kán  \textbf{rɔ́n}  \textbf{kɔmɔ́t}  \textbf{na} kɔ́ntri,    \op...\cp{}\\
\textsc{3sg.sbj}  \textsc{pfv}  run  go.out  \textsc{loc}  country\\

\glt ‘She fled from the [her] home town, (...)’ [ed03sb 035]
\z

The notion of ‘movement up to’ is formed with the verb \textit{rích} ‘arrive’ in the V2 position as in \REF{ex:key:1548}. This construction is, however, rare: 


\ea%1548
    \label{ex:key:1548}
    \gll A    tínk    sé    e    gɛ́t  treinta  y  ocho  años  náw
wé  e    dɔ́n  de  \textbf{gó}  \textbf{rích}    cuarenta.\\
\textsc{1sg.sbj}  think  \textsc{quot}    \textsc{3sg.sbj}  get  thirty  and  eight  years  now
\textsc{sub}  \textsc{3sg.sbj}  \textsc{prf}  \textsc{ipfv}  go  reach  forty\\

\glt ‘I think that he’s thirty-eight years old now and is already going towards forty.’ [fr03ft 146]
\z

The situation expressed by motion-direction SVCs is more often expressed in non-serial structures featuring prepositional phrases\is{prepositional phrases} as in \REF{ex:key:1549}. In these constructions, context and common sense disambiguate the potentially locative (i.e. ‘in the pharmacy’) and goal (‘to the pharmacy’) meanings of the PP introduced by the general locative preposition\is{general} \textit{na} ‘\textsc{loc}’:


\ea%1549
    \label{ex:key:1549}
    \gll Dɛn  rɔ́n    \textbf{na}  \textbf{farmacia},  receta    de  mɛ́rɛsin.\\
\textsc{3pl}  run    \textsc{loc}  pharmacy  prescription  of  medicine\\

\glt ‘They ran to the pharmacy, (to get a) prescription for medicine.’ [ab03ab 123]
\z

Motion-direction SVCs and alternative ways of expressing the events they denote are also treated extensively in section \sectref{sec:8.1.5}.\is{motion-direction SVCs}

\subsection{Motion-action SVCs}

Motion-action SVCs involve the motion verbs \textit{gó} ‘go’ and \textit{kán} ‘come’ as minor verbs in the V1 position. This SVC denotes movement and subsequent action. It often has an underlying purposive meaning best translated as ‘go/come in order to’. The construction is the most frequent SVC in the corpus and involves a large variety of minor verbs in the V2 position.


The construction may involve another motion verb as V2 \REF{ex:key:1550}, or any other dynamic verb \REF{ex:key:1551}. Motion-action SVCs are only attested with a dynamic V2:



\ea%1550
    \label{ex:key:1550}
    \gll Di  pikín  dɔ́n  gɛ́t  sɛ́ven  hía,    e    go  wánt  \textbf{gó}  \textbf{wáka},
“hɛ́,  nó  kɔmɔ́t  na  hós!”\\
\textsc{def}  child  \textsc{prf}  get  seven  year    \textsc{3sg.sbj}  \textsc{pot}  want  go  walk
\textsc{intj}  \textsc{neg}  go.out  \textsc{loc}  house\\
\glt ‘(When) the child is seven years old, she will want to go walk [roam around], 
[then you tell her], “don’t you leave the house!”’ [ab03ay 115]
\z


\ea%1551
    \label{ex:key:1551}
    \gll \'{A}pás  tumɔ́ro    a    go  \textbf{gó}  \textbf{sí}  mi    mamá.\\
after  tomorrow  \textsc{1sg.sbj}  \textsc{pot}  go  see  \textsc{1sg.poss}  mother\\

\glt ‘After tomorrow, I will go see my mother.’ [ro05ee 131]
\z

Below follow motion-action SVCs involving the minor verb \textit{kán} ‘come’ as the V1. Like \textit{gó}{}-SVCs, \textit{kán}{}-SVCs are encountered with \REF{ex:key:1552} and without \REF{ex:key:1553} resumptive\is{resumptive pronouns} subject\is{subjects} marking with the V2:


\ea%1552
    \label{ex:key:1552}
    \gll \textbf{Yu}  kán    \textbf{yu}  púl=an.\\
\textsc{2sg}  come  \textsc{2sg}  remove=\textsc{3sg.obj}\\

\glt ‘You came and removed it.’ [ro05ee 094]
\z


\ea%1553
    \label{ex:key:1553}
    \gll Na  ín    \textbf{e}    de  kán    púl    mí    dán  torí.\\
\textsc{foc}  \textsc{3sg.indp}  \textsc{3sg.sbj}  \textsc{ipfv}  come  remove  \textsc{1sg.indp}  that  story\\

\glt ‘That’s when she was coming to tell me that story.’ [ab03ab 073]
\z

SVCs involving the use of \textit{kán} as a verb in a motion-action SVC like \REF{ex:key:1553} need to be distinguished from the use of \textit{kán} as a narrative perfective\is{perfective aspect} aspect marker in a sentence like \REF{ex:key:1554}. There are two ways of making the distinction. Firstly, in \REF{ex:key:1553}, the lexical verb \textit{kán} ‘come’ may be marked by TMA markers like any other Pichi verb. On the contrary, the narrative perfective marker kán ‘\textsc{pfv}’ is subject to co-occurrence restrictions. For example, the TMA marker sequence *\textit{de kán} ‘\textsc{ipfv} \textsc{pfv}’ in \REF{ex:key:1553} above would be ungrammatical in \REF{ex:key:1554}.\is{aspect}


\ea%1554
    \label{ex:key:1554}
    \gll   Na  ín    e    \textbf{kán}  \textbf{vɛ́ks},    e    \textbf{kán}  \textbf{gó}.\\
\textsc{foc}  \textsc{3sg.indp}  \textsc{3sg.sbj}  \textsc{pfv}  be.angry    \textsc{3sg.sbj}  \textsc{pfv}  go\\

\glt ‘That’s why he got angry, (and) he left.’ [fr03ft 190]
\z

Secondly, speakers may employ resumptive subject\is{subjects} marking with the V2 in sentences like \REF{ex:key:1552} above in order to avoid the potential ambiguity between a motion-action SVC and a verb marked for narrative perfective aspect (i.e. \textit{yu kán púl=an} ‘\textsc{2sg} come remove=\textsc{3sg.obj}’ = ‘(then) you removed it)’. The same strategy is employed in \REF{ex:key:1555} below. In both examples, the bare lexical verb \textit{kán} ‘come’ is likely to be interpreted as the narrative perfective marker \textit{kán} ‘\textsc{pfv’} if the sequence were not interrupted by the personal pronoun \textit{yu} ‘\textsc{2sg}’. That said, these two uses of \textit{kán} are often very similar and appear to be diachronically related:


\ea%1555
    \label{ex:key:1555}
    \gll Porque  if  yu  mék,  yu  sí  dán  polvo,  e    de
pút=an    ínsay,  \textbf{yu}  \textbf{kán}    \textbf{yu}  \textbf{dríng},  \op...\cp{}\\
because  if  \textsc{2sg}  make  \textsc{2sg}  see  that  powder  \textsc{3sg.sbj}  \textsc{ipfv}
put=\textsc{3sg.obj}  inside  \textsc{2sg}  come  \textsc{2sg}  drink\\

\glt ‘Because if you make, you see that powder, he’s putting it inside, 
you come and drink (...)’ [ed03sb 099]
\z

Motion-action SVCs frequently involve the use of resumptive\is{resumptive 'go'} \textit{gó} and \textit{kán}. \is{resumptive 'kan'}In \REF{ex:key:1556}, the verb string is interrupted by the adverbial phrase \textit{na peluquería} ‘to the hairdresser’s’, after which we find a resumptive \textit{gó}. Example \REF{ex:key:1557} features resumptive \textit{kán} after the adverbial phrase \textit{wán dé ‘}one day’: 


\ea%1556
    \label{ex:key:1556}
    \gll Ɛf  yu  wánt bába,  yu  wánt  \textbf{gó}  na  peluquería  \textbf{gó}  \textbf{kɔ́t}  yu  hía.\\
if  \textsc{2sg}  want  cut.hair  \textsc{2sg}  want  go  \textsc{loc}  hairdresser’s  go  cut  \textsc{2sg}  hair\\

\glt ‘If you want to have a hair-cut, you want to go cut your hair at the hairdresser’s.’
[ro05fe 031]
\z


\ea%1557
    \label{ex:key:1557}
    \gll Dán  mán    fít  \textbf{kán}    wán  dé  \textbf{kán}    \textbf{ték}    yú    sé   
 “kán  wi  gó”,  \op...\cp{}\\
that  man    can  come  one  day  come  take    \textsc{2sg.indp}  \textsc{quot}     
  come  \textsc{1pl}  go\\

\glt ‘That man can come take you one day (and) say “let’s go” (...)’ [hi03cb 196]
\z

A more literal motion meaning may give way to a purposive meaning. In \REF{ex:key:1558}, movement to the speakers hometown has already occurred before the motion-action SVC \textit{a gó bɔ́n} ‘I went to give birth’ follows. There is no prosodic juncture between the two clauses:


\ea%1558
    \label{ex:key:1558}
    \gll A    \textbf{gó}  fɔ  kɔ́ntri  a    \textbf{gó}  \textbf{bɔ́n}.\\
\textsc{1sg.sbj}  go  \textsc{prep}  country  \textsc{1sg.sbj}  go  give.birth\\

\glt ‘I went to my home town in order to give birth.’
\z

In \REF{ex:key:1559}, the literal meaning of the V1 \textit{gó} recedes further behind a purposive sense. In this example, we see how motion through space instantiated by \textit{kɛ́r} ‘bring’, the motion metaphor of the purpose clause{\fff} introduced by \textit{fɔ} ‘\textsc{prep}’, and the motion/purpose reading of \textit{gó} itself harmonise:


\ea%1559
    \label{ex:key:1559}
    \gll Dɛn  kán  \textbf{kɛ́r}    mí    na  Madrid  \textbf{fɔ}  mék    dɛn  \textbf{gó}  opera  mí.\\
\textsc{3pl}  \textsc{pfv}  carry  \textsc{1sg.indp}  \textsc{loc}  \textsc{place}  \textsc{prep}  \textsc{sbjv}    \textsc{3pl}  go  operate  \textsc{1sg.indp}\\

\glt ‘They took me to Madrid in order to go and operate me.’ [fr03ft 026]
\z

The motion-action SVC in \REF{ex:key:1560} does not involve directed motion through space either. The SVC \textit{a gó a púl di trɔsis} ‘I (went and) removed the trousers’ involves no motion other than removing the pair of trousers:


\ea%1560
    \label{ex:key:1560}
    \gll A    púl    in    camiseta,  a    pút=an    pantáp  béd,
\textbf{a}    \textbf{gó}  \textbf{a}    \textbf{púl}    \textbf{di}  \textbf{trɔsis}  a    híb=an
ínsay  di  bañera.\\
\textsc{1sg.sbj}  remove  \textsc{3sg.poss}  singlet    \textsc{1sg.sbj}  put=\textsc{3sg.obj}  top    bed
\textsc{1sg.sbj}  go  \textsc{1sg.sbj}  remove  \textsc{def}  trousers  \textsc{1sg.sbj}  throw=\textsc{3sg.obj}
inside  \textsc{def}  bath.tub\\

\glt ‘I removed his singlet, I put him on the bed, I (went and) removed his trousers, 
I heaved him into the bath tub.’ [ab03ab 083]
\z

Example \REF{ex:key:1560} also points towards a difference in meaning that may arise between motion-action serialisation without resumptive\is{resumptive pronouns} subject marking (cf. e.g. \ref{ex:key:1559} above) and motion-action SVCs, in which the V2 has an overt subject\is{subjects} pronoun (cf. e.g. \ref{ex:key:1560}). While the former type tends to extend metaphorically into the expression of purpose relations, the latter tends to focus the action designated by V2. Motion-action SVCs involving \textit{kán} also lend themselves to less literal interpretations. Compare \REF{ex:key:1555} above, where the V1 \textit{kán} also focuses the following V2 \textit{dríng} ‘drink’.\is{motion-action SVCs} 

\subsection{Participant-introducing SVCs}\label{sec:11.2.3}

In participant-introducing SVCs, a noun appears as the syntactic object of the minor verb, and this object may occupy diverse semantic roles. One type of participant-introducing SVC involves the verb \textit{ték} ‘take’. \textit{Ték}{}-SVCs may in turn be divided into two types. 


In the first type, the object of the V1 \textit{ték} ‘take’ is the instrument\is{instrument SVC}\index{} or means used for performing V2. Compare \textit{wán blák lapá} ‘a black cloth’ in \REF{ex:key:1561}. The instrument may also be an abstract noun like \textit{páwa} ‘power’ \REF{ex:key:1562} or \textit{papá} \textit{gɔ́d} ‘God’ in the idiom in \REF{ex:key:1563}:



\ea%1561
    \label{ex:key:1561}
    \gll E    kin  dé    lɛk  sé    dɛn  ték  \textbf{wán}  \textbf{blák}    \textbf{lapá}   dɛn  kɔ́ba  yú.\\
\textsc{3sg.sbj}  \textsc{hab}  \textsc{be.loc}  like  \textsc{quot}    \textsc{3pl}  take  one  black  cloth  \textsc{3pl}  cover  \textsc{2sg.indp}\\

\glt ‘It is usually so that they cover you with a black cloth.’ [ed03sb 119]
\z


\ea%1562
    \label{ex:key:1562}
    \gll Yu  fít  gó  sé    “bueno  a    ték    \textbf{páwa}  gó”  \op...\cp{}\\
\textsc{2sg}  can  go  \textsc{quot}    good  \textsc{1sg.sbj}  take    power  go\\

\glt ‘You can go and say, “well, I leave by my own authority” (...)’ [hi03cb 194]
\z


\ea%1563
    \label{ex:key:1563}
    \gll A    ték    \textbf{papá}  \textbf{gɔ́d}  bɛ́g=an.\\
\textsc{1sg.sbj}  take    father  God  ask=\textsc{3sg.obj}\\

\glt ‘I implored him in the name of God.’ [sa07fn 297]
\z

In the second type, the object of the V1 \textit{ték} ‘take’ is the theme of the V2. This type of \textit{ték}{}-SVC is far more frequent than the one involving an instrument role. Equally, in this type, the theme is always reiterated by a resumptive object pronoun following V2, and very frequently it additionally involves resumptive subject marking. These two characteristics may make such \textit{ték}{}-SVCs difficult to distinguish from clause chaining\is{clause chaining} when the first subevent of the situation denoted by the SVC may actually involve “taking” in a literal sense (cf. \sectref{sec:11.4}).


Compare the alternative translations of \REF{ex:key:1564} and \REF{ex:key:1565}. Note the use of a resumptive\is{resumptive pronouns} object pronoun alone in the first example, and the use of both a resumptive object and subject pronoun in the second one:



\ea%1564
    \label{ex:key:1564}
    \gll A    ték=\textbf{an}    pút=\textbf{an}    pantáp  mi    bɛlɛ́.\\
\textsc{1sg.sbj}  take=\textsc{3sg.obj}  put=\textsc{3sg.obj}  top    \textsc{1sg.poss}  belly\\

\glt ‘I (took him and) put him onto my belly.’ [ab03ab 067]
\z


\ea%1565
    \label{ex:key:1565}
    \gll \textbf{Yu}  ték   di  maíz  \textbf{yu}  hól=an.\\
\textsc{2sg}  take    \textsc{def}  maize  \textsc{2sg}  hold=\textsc{3sg.obj}\\

\glt ‘You take the maize (and) you hold it.’ [dj03do 003]
\z

However, a theme object of \textit{ték} need not be an entity that can be “taken” in a literal sense. The following example once more involves resumptive object and subject pronouns. With an object like \textit{yáy} ‘eye’, no literal interpretation of \textit{ték} as ‘take’ is possible here: 


\ea%1566
    \label{ex:key:1566}
    \gll A    tɛ́l  yú    sé    mi    mán  ték    ín    \textbf{yáy}  
e    pút=an    bɔtɔ́n  grɔ́n    só.\\
\textsc{1sg.sbj}  tell  \textsc{2sg.indp}  \textsc{quot}    \textsc{1sg.poss}  man    take    \textsc{3sg.poss}  eye  
\textsc{3sg.sbj}  put\textsc{3sg.obj}  bottom  ground  like.that\\

\glt ‘I tell you that my husband diverted his eye [gaze] down like this.’ [ro05rt 011]
\z

When the theme object of \textit{ték} is human, it may also receive a comitative ‘together with’ interpretation. This occurs with the object \textit{di gɛ́l} ‘the girl’ in the relative construction{\fff} in \REF{ex:key:1567}: 


\ea%1567
    \label{ex:key:1567}
    \gll Porque  e    fíba    sé    \textbf{di}  \textbf{gɛ́l}  [wé  e    bin  de  \textbf{ték}    \textbf{kɔmɔ́t}],
e    bin  gɛ́t  bɔkú  bɔ́y  dɛn.\\
because  \textsc{3sg.sbj}  seem  \textsc{quot}    \textsc{def}  girl   \phantom{[}\textsc{sub}  \textsc{3sg.sbj}  \textsc{pst}  \textsc{ipfv}  take    go.out
\textsc{3sg.sbj}  \textsc{pst}  get  much  boy  \textsc{pl}\\

\glt ‘Because it seems that the girl that he was going out with, she had many boyfriends.’
[fr03ft 127]\is{theme}
\z

Example \REF{ex:key:1567} above is also noteworthy because it shows what happens when the object of \textit{ték} is relativised. The object \textit{di gɛ́l} ‘the girl’ is placed in the head noun position, while the relativised position may remain empty, which leads to V1 and V2 occurring next to each other. Contiguity of \textit{ték} and the V2 is also found when the object of \textit{ték} is fronted in content questions{\fff}. Sentence \REF{ex:key:1568} features the questioned concrete noun \textit{plɛ́nk} ‘board’ and \REF{ex:key:1569} the abstract noun \textit{stáyl} ‘manner’: 


\ea%1568
    \label{ex:key:1568}
    \gll \'{U}s=káyn  plɛ́nk  dɛn  \textbf{ték}    \textbf{bíl}    di  hós?\\
\textsc{q}=kind  board  \textsc{3pl}  take    build  \textsc{def}  house\\

\glt ‘What kind of board did they build the house with?’ [dj05ce 104]
\z


\ea%1569
    \label{ex:key:1569}
    \gll Na  ús=káyn  stáyl  yu  \textbf{ték}    \textbf{kán}    na  yá?\\
\textsc{foc}  \textsc{q}=kind  style  \textsc{2sg}  take    come  \textsc{loc}  here\\

\glt ‘How did you come here.’ [ro05ee 005]
\z

SVCs involving \textit{ték} are less frequent than equivalent combinations of verbs and prepositions. A PP involving \textit{wet} ‘with’ is more commonly employed to express the semantic role of instrument\index{} \REF{ex:key:1570}. Comitative \textit{ték}{}-serialisations are even less common. Speakers usually resort to a PP introduced by the preposition \textit{wet} ‘with’ as in \REF{ex:key:1575} further below: 


\ea%1570
    \label{ex:key:1570}
    \gll Dɛn  \textbf{bíl}   di  strít    \textbf{wet}    caterpillar.\\
\textsc{3pl}  build  \textsc{def}  street  with    caterpillar\\

\glt ‘The street was built with a caterpillar.’ [dj05be 078]
\z

The competition between the serial and prepositional strategies of participant-marking is manifest in the rather exceptional sentences \REF{ex:key:1571} and \REF{ex:key:1572} elicited from two different speakers. Here, the questioning of the instrument noun produced redundant marking of the question phrase \textit{ús=káyn tín} ‘\textsc{q}=kind thing’ = ‘\textsc{what’} with both a preposition and a \textit{ték}{}-SVC. Non-interrogative double uses of this kind were not found, however:


\ea%1571
    \label{ex:key:1571}
    \gll \textbf{Wet}    ús=káyn  tín    dɛn  \textbf{ték}  \textbf{bíl}   di  hós?\\
with    \textsc{q}=kind  thing  \textsc{3pl}  take  build  \textsc{def}  house\\

\glt ‘(With) what did they build the house with?’ [ye05ce 106]
\z


\ea%1572
    \label{ex:key:1572}
    \gll \textbf{Wet}    ús=káyn  stík    yu  bin  \textbf{ték}    \textbf{bíl}   di  hós?\\
with    \textsc{q}=kind  wood  \textsc{2sg}  \textsc{pst}  take    build  \textsc{def}  house\\

\glt ‘(With) what kind of wood did you build the house with?’ [ro05de 050]
\z

The verb \textit{fála} ‘follow, accompany’ participates as a V1 in the expression of a comitative role. The object of \textit{fála} is the accompanee of the situation denoted by the V2. The object of \textit{fála} is usually human and placed between V1 and V2:


\ea%1573
    \label{ex:key:1573}
    \gll Yɛ́s,    Concha  \textbf{fála}    \textbf{Princess}    \textbf{gó}  viaje.\\
yes    \textsc{name}  follow  \textsc{name}    go  voyage\\

\glt ‘Yes, Concha went on the voyage together with Princess.’ [dj05be 097]
\z

Once more, most speakers prefer to express accompaniment through non-serial alternatives. One possibility is the use of the verb \textit{jwɛ́n} ‘join’, followed by the nominalised reference verb{\fff} as in \REF{ex:key:1574}. The most common means involves a comitative prepositional phrase introduced by \textit{wet} ‘with’ \REF{ex:key:1575}:


\ea%1574
    \label{ex:key:1574}
    \gll A    \textbf{jwɛ́n}  Boyé  \textbf{fɔ}  \textbf{chɔ́p}.\\
\textsc{1sg.sbj}  join    \textsc{name}  \textsc{prep}  eat\\

\glt ‘I ate together with Boyé.’ [ur05fn 045]
\z


\ea%1575
    \label{ex:key:1575}
    \gll E    gó  \textbf{wet}    in    mamá?\\
\textsc{3sg.sbj}  go  with    \textsc{3sg.poss}  mother\\

\glt ‘Did he go with his mother?’ [fr03do 033]\is{comitative}
\z

A final type of participant-introducing SVC is the comparative construction \is{comparative SVC}featuring the verb \textit{pás} ‘(sur)pass’ \REF{ex:key:1576}. The object of \textit{pás} is the standard of comparison. Comparative SVCs are covered in detail in section \sectref{sec:6.9.1}:


\ea%1576
    \label{ex:key:1576}
    \gll Lage    de  \textbf{tɔ́k}    Bubɛ  \textbf{pás}    mí.\\
\textsc{name}  \textsc{ipfv}  talk    Bube  pass    \textsc{1sg.indp}\\

\glt ‘Lage speaks Bube better than me.’ [fr03ab 012]\is{participant-introducing SVCs}
\z

\subsection{Complementation SVCs}

This type of SVC features a verb of immediate perception as a minor verb and V1. In the corpus, this construction is attested with \textit{sí} ‘see’ and \textit{hía} ‘hear’ as V1. The construction features a switch-function (pro)noun\is{switch-function (pro)nouns}. In \REF{ex:key:1577}, the object of \textit{sí} ‘see’, i.e. \textit{sɔn wáyt pambɔ́d ‘}a white bird’, functions as the notional subject\is{subjects} of the V2 \textit{kán} ‘come’: 


\ea%1577
    \label{ex:key:1577}
    \gll \MakeUppercase{A}   \textbf{sí}  sɔn    wáyt  pambɔ́d  \textbf{de}  \textbf{kán}\textbf{\textmd{.}}\\
\textsc{1sg.sbj}  see  some  white  bird    \textsc{ipfv}  come\\

\glt ‘I saw a white bird coming.’ [ed03sb 174]
\z

Apart from participant overlap via switch-function, a defining feature of complementation SVCs is the temporal overlap between V1 and V2. Hence, in the example above, the dynamic verb \textit{kán} ‘come’ is marked for imperfective aspect\is{imperfective aspect}, which signals simultaneity with the situation denoted by the factative marked V1 \textit{sí} ‘see.’ The appearance of imperfective aspect to indicate simultaneity is also found with depictive secondary predicates (cf. \sectref{sec:11.3.2}). Complementation SVCs are, however, syntactically more integrated; they involve switch-function (pro)nouns\is{switch-function (pro)nouns} while secondary predication may not.\is{aspect}


When the V2 in a complementation SVC is an inchoative-stative property item, the V2 may appear with an overt subject\is{subjects} \textit{e} ‘\textsc{3sg.sbj}’, which is coreferential with the preceding object pronoun \textit{=an} ‘\textsc{3sg.obj}’, as in the example below. Without the V2 subject, the property item \textit{fáyn} would be interpreted as an adverbial modifier of \textit{sí} ‘see’. This structure is now in fact identical to some of the depictive secondary predications covered in section \sectref{sec:11.3}, e.g. \REF{ex:key:1601}. Complementation structures are therefore not so clear-cut cases of SVCs, and it is debatable whether they should not be seen as “overlapping clauses” \citep{Ameka2006}, hence multiclausal structures. 



\ea%1578
    \label{ex:key:1578}
    \gll If  yu  gó  fɔ  di  máred,  yu  \textbf{sí=an}    \textbf{e}    \textbf{fáyn}.\\
if  \textsc{2sg}  go  \textsc{prep}  \textsc{def}  marry  \textsc{2sg}  see=\textsc{3sg.obj}  \textsc{3sg.sbj}  be.fine\\

\glt ‘If you go to the marriage, you see it (to be) nice.’ [\textit{Lit}. ‘(...) it is nice.’] [hi03cb 006]
\z

The more common alternative to complementation SVCs is for the perceived situation to be expressed as a complement clause \is{complement clauses}introduced by \textit{sé} ‘\textsc{quot}’, as in the following example: 


\ea%1579
    \label{ex:key:1579}
    \gll Yu  jɔ́s  \textbf{hía}    \textbf{sé}    pɔ́sin  dɛn  bin  de  tɔ́k  bɔt  yu  nó  lístin.\\
\textsc{2sg}  just  hear    \textsc{quot}    person  \textsc{pl}  \textsc{pst}  \textsc{ipfv}  talk  but  \textsc{2sg}  \textsc{neg}  listen\\

\glt ‘You just heard that people were talking but you didn’t listen.’ [au07se 109]
\z

\subsection{Adverbial SVCs}\label{sec:11.2.5}

Two verbs in the corpus appear as minor verbs in adverbial SVCs. In these structures the V1 provides a modification that is temporal in nature. The verb \textit{lás} ‘be the last to, end up’ enters into an adverbial SVC as a minor verb \REF{ex:key:1580}. Proof for the verbal status of \textit{lás} comes from \REF{ex:key:1581}: \textit{lás} may not appear in the postverbal adverbial position. In contrast, the word \textit{fɔ́s} ‘first’ which also expresses temporal meanings may, since it is an adverb \REF{ex:key:1582}: 


\ea%1580
    \label{ex:key:1580}
    \gll \MakeUppercase{A}   \textbf{lás}    \textbf{chɔ́p}.\\
\textsc{1sg.sbj}  be.last  eat\\

\glt ‘I was the last to eat/I ended up eating.’ [eb07fn 130]
\z


\ea[*]{%1581
    \label{ex:key:1581}
    \gll Na  mí    chɔ́p  \textbf{lás}.\\
  \textsc{foc}  \textsc{1sg.indp}  eat    last\\
\glt Intended: ‘I ate last.’ [ra07ve 025]
}\z


\ea%1582
    \label{ex:key:1582}
    \gll A    wás    \textbf{fɔ́s}.\\
\textsc{1sg.sbj}  wash  first\\

\glt ‘I washed (myself) first.’ [ra07ve 023]
\z

The dynamic verb \textit{sté} ‘stay’ is employed as the V1 in an SVC in order to express (excessive) duration. This SVC is frequently used in a context of current relevance, where it commonly appears together with the perfect marker \textit{dɔ́n}: 


\ea%1583
    \label{ex:key:1583}
    \gll Yu  dɔ́n  \textbf{sté}  \textbf{kán}?\\
\textsc{2sg}  \textsc{prf}  stay  come\\

\glt ‘Did you come long ago?’ [ge07fn 164]
\z

Many speakers instead prefer to express duration through a biclausal structure with co-referential subjects \REF{ex:key:1584} or an expletive subject to \textit{sté} \REF{ex:key:1585}. The latter use is once more similar to secondary predication covered below in section \sectref{sec:11.3}:


\ea%1584
    \label{ex:key:1584}
    \gll \textbf{A}    \textbf{sté}  wé  \textbf{a}    nɛ́va  chɔ́p.\\
\textsc{1sg.sbj}  stay  \textsc{sub}  \textsc{1sg.sbj}  \textsc{neg}.\textsc{prf}  eat\\

\glt ‘It’s been long since I haven’t eaten.’ [au07ec 081]
\z


\ea%1585
    \label{ex:key:1585}
    \gll \textbf{E}    nó \textbf{sté}   a    recibe  di  carta,
di  tín    wé  a    bɛ́g.\\
\textsc{3sg.sbj}  \textsc{neg}  stay    \textsc{1sg.sbj}  receive  \textsc{def}   letter
\textsc{def}  thing  \textsc{sub}  \textsc{1sg.sbj}  ask.for\\

\glt ‘It wasn’t long and I received the letter, the thing I (had) asked for.’ [ed03sb 214]
\z

\section{Secondary predication}\label{sec:11.3}

Pichi deploys reduced clauses as adjuncts to clauses fully specified for person and TMA. In the following, I refer to the predicator of the reduced clause as the secondary predicate, and to that of the full clause as the primary predicate (\citealt{BerndtHimmelmann2004,HimmelmannBerndt2006}). Secondary predicates may range in complexity from fully-fledged clauses to reduced clauses consisting of the secondary predicate alone. In Pichi, there is therefore no clear-cut distinction between structures involving secondary predication and some of the time, manner, and result clauses covered in section \sectref{sec:10.7}.


There are two types of secondary predication in Pichi, namely depictives (\sectref{sec:11.3.2}) and resultatives (\sectref{sec:11.3.3}). The difference between the two types is both semantic and formal. Resultative secondary predicates instantiate an end-state and can therefore be seen to stand in a relation of temporal sequentiality or posteriority to the primary predicate. In formal terms, only inchoative-stative property items can function as resultative secondary predicates.



Depictive secondary predicates are in a temporal relation of simultaneity to the primary predicate and therefore contribute manner or temporal readings to the primary predicate. It makes little sense to distinguish further in Pichi between secondary predicates commonly referred to as depictives and those known as circumstantials (cf. \citealt{HimmelmannBerndt2006}). The semantic and formal differences that we find between individual constructions are due to differences in the lexical aspect class, degree and type of transitivity, and other semantic features (e.g. animacy) of the primary and secondary predicates. These features also co-determine whether a secondary predicate is subject- or object-oriented. With depictives, the lexical aspect class of the secondary predicate also determines whether the secondary predicate is marked for imperfective aspect, by factative TMA (i.e. with inchoative-stative verbs) or by the use of \textit{de} ‘\textsc{ipfv}’ (i.e. with dynamic verbs).


\subsection{Secondary predication vs. serial verb constructions}

Secondary predicates can be distinguished from SVCs on formal grounds. For one, the secondary predicate is connected to the primary predicate in a loose way, i.e. via adjunction. The secondary predicate can therefore be paraphrased by fuller clauses with sometimes only slight modifications to the sentence (cf. \ref{ex:key:1590}).


A second distinguishing feature is that secondary predicate constructions do not involve switch-function (pro)nouns{\fff}. In the following motion-direction SVC, \textit{=an} ‘\textsc{3sg.obj}’, the object of the V1 \textit{kɛ́r} ‘carry’, simultaneously functions as the notional subject of the V2 \textit{kán} ‘come’. In fact, the overt expression of a subject{\fff} pronoun with the V2 would be ungrammatical (i.e. *\textit{a kɛ́r=an e gó na comedor} ‘I carried him, he went to the dining-room’). 



\ea%1586
    \label{ex:key:1586}
    \gll A    kɛ́r=\textbf{an}      gó  na  comedor.\\
\textsc{1sg.sbj}  carry=\textsc{3sg.obj}    go  \textsc{loc}  dining-room\\

\glt ‘I carried him to the dining-room.’ [ab03ab 091]
\z

The following secondary predicate construction is therefore rejected. The object \textit{mí} ‘\textsc{1sg.indp}’ of the primary predicate \textit{mít} ‘meet’ may not simultaneously serve as the subject of the secondary predicate \textit{kúk} ‘cook’:


\ea[*]{%1587
    \label{ex:key:1587}
    \gll E    mít    \textbf{mí}    de  kúk.\\
  \textsc{3sg.sbj}  meet  \textsc{1sg.indp}  \textsc{ipfv}  cook\\
\glt Intended: ‘He came across me while (I was) cooking.’ [pa07me 017]
}\z

In such cases of object-subject identity involving an animate participant, the secondary predicate must rather have an explicit subject\is{subjects}, even if the primary predicate object and the secondary predicate subject are co-referential: 


\ea%1588
    \label{ex:key:1588}
    \gll E    mít    \textbf{mí}    \textbf{a}    de  kúk    sɛ́f.\\
\textsc{3sg.sbj}  meet  \textsc{1sg.indp}  \textsc{1sg.sbj}  \textsc{ipfv}  cook  \textsc{emp}\\

\glt ‘He came across me while I was actually cooking.’ [ro05de 023]
\z

Further, the V2 of an SVC acquires its TMA specification from the V1; the V2 may not be independently marked for tense\is{tense}, mood,\is{mood} and aspect (cf. \ref{ex:key:1542}). In contrast, depictive secondary predicates must be marked for simultaneity by imperfective aspect, either via factative TMA or via \textit{de} ‘\textsc{ipfv’.} Compare the imperfective-marked secondary predicate \textit{chɔ́p} ‘eat’ in this example: \is{aspect}


\ea%1589
    \label{ex:key:1589}
    \gll Yu  pikín  sidɔ́n  \textbf{de}  \textbf{chɔ́p}  dɛn  tú  brɛ́d.\\
\textsc{2sg}  child  sit    \textsc{ipfv}  eat    \textsc{3pl}  two  bread\\

\glt ‘Your child was sitting (there) eating those two loaves of bread.’ [ab03ab 128]
\z

Many secondary predicates in the data do not feature overt subjects either and in that, they resemble the V2s of SVCs like \REF{ex:key:1586} above. However, contrary to the SVC in \REF{ex:key:1586}, the notional subject of the secondary predicate may optionally be expressed. Secondary predicates may therefore be expanded into fuller clauses. 


The following sequence of near-identical resultative constructions graphically shows the progression from the reduced clause typical of secondary predication to a biclausal structure involving overt clause linkage:\is{resultative constructions}



\ea%1590
    \label{ex:key:1590}
\ea{
    \gll A    lɛ́f    di  domɔ́t  \textbf{ópin}.\\
  \textsc{1sg.sbj}  leave  \textsc{def}  door  be.open\\
\glt   ‘I left the door open.’ [pa07me 029]
}\ex{\gll
A    lɛ́f    di  domɔ́t  \textbf{e}    \textbf{ópin}.\\
  \textsc{1sg.sbj}  leave  \textsc{def}  door  \textsc{3sg.sbj}  be.open\\
\glt   ‘I left the door open.’ [pa07me 030]
}\ex{\gll
A    lɛ́f    di  domɔ́t  \textbf{sé}    \textbf{e}    \textbf{ópin}.\\
  \textsc{1sg.sbj}  leave  \textsc{def}  door  \textsc{quot}    \textsc{3sg.sbj}  be.open\\
\glt   ‘I left the door open.’ [pa07me 031]
}\z\z

\subsection{Depictives}\label{sec:11.3.2}

In formal terms, there are two types of depictive secondary predicates. One type features a bare verb with a stative interpretation, the other a dynamic verb marked for imperfective aspect. Both types are therefore marked for simultaneous taxis with the primary predicate – the bare inchoative-stative verb by default via factative TMA\is{factative TMA}, and the dynamic verb via explicit imperfective aspect marking. Further, depictive secondary predications can be differentiated according to their participant orientation. Subject-oriented predicates predicate a situation relating to the subject, object-oriented ones relate a situation relating to the object. 


Transitive verbs denoting various types of use or manipulation are prone to occuring with object-oriented depictive predicates. For example, affected-agent verbs\index{} like the verbs of ingestion \textit{dríng} ‘drink’ and \textit{chɔ́p} ‘eat’ appear with object-oriented secondary predicates with a depictive function:



\ea%1591
    \label{ex:key:1591}
    \gll E    \textbf{dríng}  di  watá  \textbf{kól}.\\
\textsc{3sg.sbj}  drink  \textsc{def}  water  be.cold\\

\glt ‘He drank the water (and it was) cold.’ [ra07ve 004]
\z


\ea%1592
    \label{ex:key:1592}
    \gll Dɛn  \textbf{chɔ́p}  di  banána  \textbf{grín}.\\
\textsc{3pl}  eat    \textsc{def}  banana  be.green\\

\glt ‘They ate the banana green [unripe].’ [dj05be 108]
\z

Another group that appears with object-oriented depictives are verbs of handling and manipulation (e.g. \textit{bay} ‘buy’, \textit{kɛ́r} ‘carry’, \textit{sɛ́l} ‘sell’, \textit{yús} ‘use’). The following example illustrates this usage by means of \textit{kɛ́r} ‘carry’ and the secondary predicate \textit{ɛ́nti} ‘be empty’:


\ea%1593
    \label{ex:key:1593}
    \gll A    \textbf{kɛ́r}    di  bokit-pán    \textbf{ɛ́nti}.\\
\textsc{1sg.sbj}  carry  \textsc{def}  bucket.\textsc{cpd}{}-pan  be.empty\\

\glt ‘I carried the bucket empty.’ [pa07me 039]
\z

Subject\is{subjects}{}-oriented depictives occur in intransitive clauses with various types of intransitive or low-transitivity primary predicates. A prominent group of primary predicates encompasses locomotion verbs like \textit{kɔmɔ́t} ‘go/come out’ as in this example: 


\ea%1594
    \label{ex:key:1594}
    \gll E    \textbf{kɔmɔ́t}  na  rúm    \textbf{nékɛd}.\\
\textsc{3sg.sbj}  go.out  \textsc{loc}  room  be.naked\\

\glt ‘He left the room naked.’ [ra07ve 001]
\z

Some depictive secondary predications may alternatively be expressed through nominal depictives. One strategy involves the use of a prepositional phrase introduced by the multifunctional preposition \textit{wet} ‘with’ (cf. also \ref{ex:key:888} and \ref{ex:key:885}–\ref{ex:key:886} in \sectref{sec:7.7.2}): 


\ea%1595
    \label{ex:key:1595}
    \gll E    \textbf{kɔmɔ́t}  na  wók    \textbf{wet}    \textbf{hángri}.\\
\textsc{3sg.sbj}  go.out  \textsc{loc}  work  with    hunger\\

\glt ‘He left work hungry.’ [ra07ve 073]
\z

A common subject-oriented depictive construction in the data involves the expression of “associated posture” \citep{Enfield2002}: The secondary predicate denotes a situation that holds while the subject\is{subjects} assumes a posture denoted by the primary predicate. The secondary predicate is therefore both participant- (the subject) and event-oriented (the primary predicate). When associated posture verbs co-occur with a dynamic secondary predicate, temporal simultaneity is marked overtly by imperfective marking. This is the case in \REF{ex:key:1596} where the posture verb \textit{sidɔ́n} ‘sit (down)’ is followed by the imperfective\is{imperfective aspect} marked dynamic verb \textit{chɔ́p} ‘eat’: \is{aspect}


\ea%1596
    \label{ex:key:1596}
    \gll Yu  pikín  \textbf{sidɔ́n}  \textbf{de}  \textbf{chɔ́p}  dɛn  tú  brɛ́d.\\
\textsc{2sg}  child  sit    \textsc{ipfv}  eat    \textsc{3pl}  two  bread\\

\glt ‘Your child was sitting there eating those two loaves of bread.’ [ab03ab 128]
\z

The secondary predicate in an associated posture construction may also be another locative verb\is{locative verbs} that elaborates on the type of posture taken by the subject. In \REF{ex:key:1597}, the posture verb \textit{sidɔ́n} ‘sit (down)’ is followed by the inchoative-stative locative verb \textit{ráwn} ‘form a circle’. Since \textit{ráwn} is not dynamic, the situation of temporal overlap is not marked by means of the imperfective aspect. It is rather marked by factative TMA, hence the bare verb (cf. \sectref{sec:6.1}). The use of a co-referential subject\is{subjects} pronoun with the V2 (the second \textit{dɛn} ‘\textsc{3pl’} in the example) is the norm if the secondary predicate is not dynamic:


\ea%1597
    \label{ex:key:1597}
\gll
        Dɛn    \textbf{sidɔ́n}  \textbf{dɛn}    \textbf{ráwn}    di  fáya.\\
\textsc{3pl}    sit    \textsc{3pl}    surround  \textsc{def}  fire\\

\glt ‘They’re sitting around the fire.’ or ‘They sat down around the fire.’\textstylePichiglossZchn{ [ro05ee 115]}
\z

The following example also involves associated posture, this time featuring the locative-existential copula \textit{dé} \textsc{‘be.loc’} serving as a primary predicate. The general locative meaning of the copula allows various interpretations of associated posture. The use of \textit{dé} \textsc{‘be.loc’} together with the adverbial complement \textit{dé} ‘there’ in such a construction also conveys affective nuances like negligence or irritation with the situation denoted by the secondary predicate:


\ea%1598
    \label{ex:key:1598}
    \gll Di  pikín  \textbf{dé}    dé    \textbf{de}  \textbf{kráy}.\\
\textsc{def}  child  \textsc{be.loc}  there  \textsc{ipfv}  cry\\

\glt ‘The child is just (standing/sitting/lying) there crying.’ [pa07me 027]\is{posture verbs}
\z

A second, equally common subject-oriented secondary predicate features a dynamic locomotion verb as the primary predicate. The secondary predicate provides information about the subject as well as the event denoted by the primary predicate itself. In the example below, both verbs are dynamic, hence, imperfective marking is again used to express the temporal overlap of the two predicates. Note the optional use of a resumptive\is{resumptive pronouns} subject pronoun with the secondary predicate: 


\ea%1599
    \label{ex:key:1599}
    \gll Dɛn  \textbf{de}  \textbf{fála}    dɛn  sɛ́f  dɛn  \textbf{de}  \textbf{rɔ́n}.\\
\textsc{3pl}  \textsc{ipfv}  follow  \textsc{3pl}  self  \textsc{3pl}  \textsc{ipfv}  run\\

\glt ‘They’re following each other running.’ [dj07re 005]
\z

The construction in \REF{ex:key:1600} features the locomotion verb \textit{wáka} ‘walk’ as primary predicate and the idiomatic reflexive construction \textit{ópin in sɛ́f} ‘(to) boast’ as secondary predicate. Note the presence of the resumptive subject{\fff} pronoun \textit{e} ‘\textsc{3sg.sbj}’ in this example as well: 


\ea%1600
    \label{ex:key:1600}
    \gll E    nó  gɛ́t  mɔní,  wétin  e    de  \textbf{wáka}
e    de  \textbf{ópin}  in    sɛ́f  so?\\
\textsc{3sg.sbj}  \textsc{neg}  get  money  what  \textsc{3sg.sbj}  \textsc{ipfv}  walk
\textsc{3sg.sbj}  \textsc{ipfv}  open  \textsc{3sg.poss}  self  like.that\\

\glt ‘He doesn’t have money, why does he go around boasting like that?’ [ye07je 132]
\z

Animacy provides additional cues to the meaning of constructions involving secondary predication. When the object of a transitive verb has an animate object as in \REF{ex:key:1601} (i.e. \textit{mí} ‘\textsc{1sg.indp’),} the secondary predicate may be interpreted as either subject- or object-oriented. In such cases, the secondary predicate requires a subject pronoun in order to establish reference with either of the two participants:\is{adjuncts}


\ea%1601
    \label{ex:key:1601}
    \gll Pero    dɛn  kán  \textbf{dú}  \textbf{mí}    \textbf{a}    de  sté  na  Móka,  
dɛn  kán  dú  mí    na  Móka.\\
but    \textsc{3pl}  \textsc{pfv}  do  \textsc{1sg.indp}  \textsc{1sg.sbj}  \textsc{ipfv}  stay  \textsc{loc}  \textsc{place}  
\textsc{3pl}  \textsc{pfv}  do  \textsc{1sg.indp}  \textsc{loc}  \textsc{place}\\

\glt ‘But they did it to me while I was staying in Moka, they did it to 
me in Moka.’ [ab03ay 071]\is{secondary predicates}
\z

\subsection{Resultatives}\label{sec:11.3.3}

Resultative secondary predicates express resultant states, hence, they also involve stative(ly interpreted) property items. Resultative meaning arises in sentences featuring highly transitive effected-object verbs\index{} as primary predicates and property items as secondary predicates. Resultatives are invariably object-oriented.


In \REF{ex:key:1602}, the verb of production \textit{pént} ‘paint’ is followed by the patient\is{patient} object \textit{hós} ‘house’ and the secondary predicate \textit{blák} ‘black’. The secondary predicated is an object-oriented resultative adjunct that denotes the resultant state of the event denoted by the primary predicate:



\ea%1602
    \label{ex:key:1602}
    \gll E    \textbf{pént}  di  hós    \textbf{blák}.\\
\textsc{3sg.sbj}  paint  \textsc{def}  house  black\\

\glt ‘He painted the house black.’ [pa07me 037]
\z

Sentence \REF{ex:key:1603} features the effected-object verb\is{} \textit{mék} ‘make, prepare’ as a primary predicate. Note that the secondary predicate takes the subject\index{subjects} pronoun \textit{e} ‘\textsc{3sg.sbj}’, which is co-referential with the primary predicate object \textit{café} ‘coffee’. The overt subject pronoun is not necessary here because the resultative predicate is clearly object-oriented (unlike the primary predicate presented in \ref{ex:key:1588} above). I assume that an explicit subject pronoun is nevertheless employed because of the presence of the preverbal degree adverb \textit{tú} ‘too (much)’. This makes the secondary predicate more complex and motivates the use of a finite resultative clause featuring an overt subject: 


\ea%1603
    \label{ex:key:1603}
    \gll Dɛn    \textbf{mék}    di  café    \textbf{e}    \textbf{tú}  \textbf{swít}.\\
\textsc{3pl}    make  \textsc{def}  coffee  \textsc{3sg.sbj}  too  be.sweet\\

\glt ‘They prepared the coffee (it’s) too sweet.’ [ra07ve 064]
\z

Resultatives may be paraphrased by employing a nominal strategy. The resultative secondary predicate in \REF{ex:key:1604}, i.e. the property item \textit{wɔwɔ́} ‘be ugly, messed up’ may be vaguely paraphrased via an NP in which it appears as a prenominal modifier to the generic noun \textit{stáyl} ‘manner’ \REF{ex:key:1605}. The generic noun \textit{stáyl} ‘manner’ is also used in modifications of manner (cf. e.g. \ref{ex:key:880} in \sectref{sec:7.7.2}) and in manner question words (cf. e.g. \ref{ex:key:629} in \sectref{sec:7.3.2}), hence in \REF{ex:key:1605}, it is ambiguous between a participant-oriented resultative reading and an event-oriented manner reading:


\ea%1604
    \label{ex:key:1604}
    \gll Dɛn  \textbf{bíl}   di  ród    \textbf{wɔwɔ́}.\\
\textsc{3pl}  build  \textsc{def}  road    be.ugly\\

\glt ‘They built the road (and it’s) shoddy.’ [ra07ve 059]
\z


\ea%1605
    \label{ex:key:1605}
    \gll Dɛn  \textbf{bíl}=an    \textbf{wɔwɔ́}  \textbf{stáyl}.\\
\textsc{3pl}  build=\textsc{3sg.obj}  ugly    style\\

\glt ‘They built it (and it’s) shoddy.’ or ‘They built it shoddily.’ [ra07ve 060]
\z

Pichi resultative constructions are object-oriented and require the secondary predicate to be an inchoative-stative property item. Neither inchoative-stative verbs from other semantic classes nor dynamic verbs are employed as resultative secondary predicates. In contrast, Pichi’s West African sister languages have object-oriented resultative SVC\is{resultative SVC}s featuring dynamic secondary predicates as in the Krio example below, and subject-oriented resultatives featuring change-of-state secondary predicates as in Ghanaian Pidgin English. 


\ea%1606
    \label{ex:key:1606}
\textsc{Krio}\il{Krio}\\
    \gll Di  wúmán  \textbf{kúk}    rɛ́s    \textbf{sɛ́l}.\\
\textsc{def}  woman  cook  rice    sell\\

\glt ‘The woman cooked rice and sold it.’ \citep[72]{Finney2004}
\z


\ea%1607
    \label{ex:key:1607}
\textsc{Ghanaian Pidgin English}\\
    \gll A    \textbf{chɔ́p}  \textbf{táya}.\\
\textsc{1sg.sbj}  eat    be.tired  \\

\glt ‘I ate (until I was) tired (of it).’ (Own knowledge)
\z

Accordingly, Pichi also does not have a resultative completive aspect construction featuring the dynamic verb \textit{fínish} ‘finish’ as a secondary predicate, as in the following example. In Pichi, completive aspect is instead expressed via an auxiliary construction and a verbal complement (cf. \sectref{sec:6.4.3}):


\ea%1608
    \label{ex:key:1608}
\textsc{Ghanaian Pidgin English}\\
    \gll A    \textbf{chɔ́p}  \textbf{fínish}.\\
\textsc{1sg.sbj}  eat    finish  \\

\glt ‘I’ve finished eating/I’m done eating.’ (Own knowledge)
\z

Resultant situations like the ones above must therefore be expressed through fuller clauses in Pichi. When the secondary predicate is not a property item and subject-oriented, a clause linker like \textit{sóté} ‘until’ may be sufficient. When the secondary predicate is not a property item and object-oriented, a chained clause with person-marking is required:


\ea%1609
    \label{ex:key:1609}
    \gll A    viaja  *(\textbf{sóté}\textbf{\textmd{)}}  \textbf{táya}.\\
\textsc{1sg.sbj}  travel   \phantom{*(}until  be.tired\\

\glt ‘I travelled until (I was) tired (of it).’ [ju07ae 531]
\z


\ea%1610
    \label{ex:key:1610}
    \gll Bɔt  wi  fít  de  plé,  a    \textbf{jám}        yú    yu  \textbf{fɔdɔ́n}.\\
but  \textsc{1pl}  can  \textsc{ipfv}  play  \textsc{1sg.sbj}  make.contact    \textsc{2sg.indp}  \textsc{2sg}  fall\\

\glt ‘But we could be playing, I hit you (and) you fall.’ [au07se 178]\is{resultative constructions}
\z

\section{Clause chaining}\label{sec:11.4}

Clause chaining is utilised in narrative discourse to describe tightly-knit situations that take place in sequence. In chained clauses, speakers use one predicate after the other without pausing or placing clause linkers between them. However, chained predicates invariably feature resumptive personal pronouns, and the subject\is{subjects} is repeated with each verb in the series. Verbs that participate in clause chaining are always dynamic and form part of foregrounded sections of narrative discourse (cf. \sectref{sec:6.8.1}).


TMA\is{tense}\is{aspect} marking is reduced in chained clauses. Tense, aspect, and mood marking is overtly expressed with the initial one or two verb(s) in order to provide orientation and grounding. Subsequent verbs, however, remain bare. Clause chaining is therefore different from linkage involving fuller clauses through the absence of prosodic juncture marking and the reduction of TMA marking. At the same time, chained clauses differ from SVCs because they exhibit overt person marking.



The clause chain below features the initial verbs \textit{rɛdí} ‘be/make ready’ and \textit{mék} ‘make’, which are both fully finite and marked for potential mood. The verbs following \textit{mék}, i.e. \textit{ték} ‘take’, \textit{pút} ‘put’, \textit{sɛ́n} ‘send’, and \textit{gó} ‘go’ are all left bare without TMA marking. Instead, they form part of a clause chain, in which the initial two verbs alone provide the temporal, aspectual, and modal frame of reference. Note that the bare verbs in the clause chain cannot be interpreted as being marked for factative TMA,{\fff} since the temporal and modal frame of the paragraph is provided by the potential mood marked on \textit{rɛdí} ‘prepare’ and \textit{mék} ‘make’:



\ea%1611
    \label{ex:key:1611}
    \gll Dɛn  \textbf{go}  \textbf{rɛdí}    yú    dɛn  \textbf{go}  \textbf{mék}    lɛk  háw    dɛn  de  mék  
fɔ  wích,  dɛn  \textbf{ték}   yú    dɛn  \textbf{pút}  yú    na  avión
dɛn  \textbf{sɛ́n}    yú    fɔ  ɔ́da    kɔ́ntri   yu  \textbf{gó}  wók    mɔní.\\
\textsc{3pl}  \textsc{pot}  prepare  \textsc{2sg.indp}  \textsc{3pl}  \textsc{pot}  make  like  how    \textsc{3pl}  \textsc{ipfv}  make
\textsc{prep} sorcery  \textsc{3pl}  take    \textsc{2sg.indp}  \textsc{3pl}  put  \textsc{2sg.indp}   \textsc{loc}  plane  
\textsc{3pl}  send  \textsc{2sg.indp}  \textsc{prep}  other  country  \textsc{2sg}  go  work  money\\

\glt ‘They would prepare you like the way it’s done by sorcery, they’ll take you, 
put you into a plane, and send you to another country (and) you’ll go earn 
(them) money.’ [ed03sb 104]\is{aspect}
\z

The following example illustrates how the difference between clause chaining and the linkage of fully finite clauses may hinge on intonation when a series of dynamic verbs are marked for factative TMA.\is{resumptive pronouns} In \REF{ex:key:1612}, the verbs \textit{ték} ‘take’\textit{, pé} ‘pay’,\textit{ kɔmɔ́t ‘}go out’,\textit{ rích} ‘arrive’, and \textit{pé} ‘pay’ are iconically ordered along the time axis and describe successive events. However, they are separated by pauses. Additionally, the last constituent of each clause bears continuative intonation\is{continuative intonation} (indicated by a comma), which alerts the hearer to the existence of a clausal boundary. For these reasons, \REF{ex:key:1612} does not involve clause chaining: 



\ea%1612
    \label{ex:key:1612}
    \gll Lúk=an,    di  dé  wé  dís  Paquita  in    papá  bin  kán    \textbf{ték}=an,
e    \textbf{pé}  avioneta,      \textbf{kɔmɔ́t}  Alemania,  \textbf{rích}    na  Douala,
\textbf{pé}  avioneta,      e    \textbf{kán}    na  yá    só.\\
look=\textsc{3sg.obj}  \textsc{def}  day  \textsc{sub}  this  \textsc{name}  \textsc{3sg.poss}  father  \textsc{pst}  come  take=\textsc{3sg.obj}
\textsc{3sg.sbj}  pay  small.aircraft    go.out  \textsc{place}    reach  \textsc{loc}  \textsc{place}
pay  small.aircraft    \textsc{3sg.sbj}  come  \textsc{loc}  here    like.that\\

\glt ‘Look at her, the day that Paquita’s father came to take her, he paid (a ticket for) a small plane, 
left Germany, got to Douala, paid (a ticket for) a small plane, (and) came here.’ [ab03ay 140]\is{clause chaining}
\z

