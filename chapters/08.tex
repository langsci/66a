\chapter{Spatial and temporal relations}

Location in space is expressed by elements from diverse word classes and through a large variety of constructions. Some of the means employed for the expression of spatial relations are carried over into the expression of temporal relations but there are also independent ways of expressing location in time. 

\section{Spatial relations}\label{sec:8.1}

Prepositions, locative nouns, and locative verbs play a part in expressing spatial relations. Other items involved are motion verbs – verbs whose meanings include a motion component. The relation between “figure” and “ground” may be mediated through various types of structures. The expression of source and goal is of particular interest in the disucssion because it may involve the use of various competing structures.

\subsection{Locative prepositions}\label{sec:8.1.1}

Prepositions are employed to express the location and direction of motion of an entity (the “figure”) in relation to a place (the “ground”). Locative prepositions and locative nouns (cf. \sectref{sec:8.1.2}) belong to separate word class\is{word class}es, but some shared characteristics make the distinction less clear-cut. \tabref{tab:key:8.1} contains the Pichi inventory of prepositions. There are no postpositions in Pichi. Non-locative roles expressed by prepositions are covered in \sectref{sec:9.1.3}. Note that Pichi also has the two temporal prepositions \textit{ápás} ‘after’ and \textit{síns} ‘since’ (cf. \sectref{sec:8.2}).

%%please move \begin{table} just above \begin{tabular
\begin{table}
\caption{Locative uses of prepositions}
\label{tab:key:8.1}

\begin{tabularx}{\textwidth}{lllQ}
\lsptoprule
Preposition & Gloss & Location/direction & Other semantic roles/uses\\
\midrule 
\itshape na & ‘\textsc{loc}’ & General location (at rest) & {}---\\
\itshape fɔ & ‘\textsc{prep}’ & General location (at rest) & Various non-locative roles\\
\itshape pan & ‘on’ & Superior location & ‘in addition to’\\
\itshape frɔn & ‘from’ & Source & ‘since (temporal)’\\
\itshape sóté & ‘up to’ & Extent & ‘until (temporal), extremely (\textsc{adv})’\\
\itshape to & ‘to’ & Goal & Complementiser\\
\lspbottomrule
\end{tabularx}
\end{table}
Locative prepositions introduce adverbial prepositional phrases\is{prepositional phrases}. Prepositions differ from locative nouns because they cannot be employed in the syntactic position of nouns. Prepositions require explicit mention of the ground, which is usually a nominal complement \is{complements}(\ref{ex:key:900}–\ref{ex:key:901}). The prepositions \textit{fɔ} ‘\textsc{prep}’, \textit{pan} ‘on’, and \textit{wet} ‘with’ may however be stranded\is{stranding} in questions, cf. (\ref{ex:key:594}–\ref{ex:key:595}), as well as in relative clauses.


\ea%900
    \label{ex:key:900}
    \gll Di  pépa  dé    \textbf{na}  \textbf{tébul}.\\
\textsc{def}  paper  \textsc{be.loc}  \textsc{loc}  table\\

\glt ‘The paper is on the table.’ [dj05be 190]
\z


\ea%901
    \label{ex:key:901}
    \gll \op...\cp{}  e    lɛ́f    dɛ́n    \textbf{pan}  di  tébul.\\
{}  \textsc{3sg.sbj}  leave  \textsc{3pl.indp}  on  \textsc{def}  table\\

\glt ‘(...) she left them on the table.’ [li07pe 020]
\z

Next to full nouns, locative adverbs may also function as complements to prepositions. Take note of the temporal meaning of the locative adverb \textit{dé} ‘there’ in \REF{ex:key:902}:


\ea%902
    \label{ex:key:902}
    \gll Wé  in    mamá  dáy,  na  \textbf{frɔn}    \textbf{dé}    e    bigín  krés.\\
\textsc{sub}  \textsc{3sg.poss}  mother  die  \textsc{foc}  from  there  \textsc{3sg.sbj}  begin  be.crazy\\

\glt ‘When his mother died, that’s when he began to go insane.’ [dj07ae 103]
\z


\ea%903
    \label{ex:key:903}
    \gll E    kán    fɔdɔ́n  \textbf{sóté}    \textbf{yá}.\\
\textsc{3sg.sbj}  \textsc{pfv}    fall    until  here\\

\glt ‘(And then) it fell up to here.’ [li07pe 090]
\z

The general locative preposition \textit{na} ‘\textsc{loc}’ and the general associative preposition\is{associative preposition} \textit{fɔ} ‘\textsc{prep}’ take the locative adverb \textit{yá (só)} ‘(right) here’ as a complement but are not attested with \textit{dé} ‘there’ or\textit{ yandá} ‘yonder’ as a complement: 


\ea%904
    \label{ex:key:904}
    \gll \op...\cp{}  na  di  tín    a    kán    \textbf{na}  \textbf{yá}.\\
{}  \textsc{foc}  \textsc{def}  thing  \textsc{1sg.sbj}  come  \textsc{loc}  here\\

\glt ‘(...) that’s why I came here.’ [ed03sb 087]
\z


\ea%905
    \label{ex:key:905}
    \gll \textbf{Fɔ}  \textbf{yá}    \textbf{só},    pípul  fɔ  isla    dɛn  de  pé  líka.\\
\textsc{prep}  here    like.that  people  \textsc{prep}  island  \textsc{3pl}  \textsc{ipfv}  pay  alcohol\\

\glt ‘As for here, people of the island pay [the bride price] in alcohol.’ [hi03cb 004]
\z

It is also common to find the generic noun\is{generic nouns} \textit{sáy} ‘side, place’ and a demonstrative\is{demonstratives} as a complement to \textit{na} or \textit{fɔ} instead of a deictic locative adverb: 


\ea%906
    \label{ex:key:906}
    \gll Na  só    dɛn  de  mék    café    \textbf{na}  \textbf{dí}  \textbf{sáy}.\\
\textsc{foc}  like.that  \textsc{3pl}  \textsc{ipfv}  make  coffee  \textsc{loc}  this  side  \\

\glt ‘That’s how they make coffee here.’ [ye07ga 038]
\z

Personal pronouns do not normally occur as complements to the general (locative) prepositions \textit{na} ‘\textsc{loc’} and \textit{fɔ} ‘\textsc{prep’}. Pichi employs other means of expressing the relevant notions. For example, the ground may be named more specifically as in \REF{ex:key:907} or an idiomatic expression may be used, as in \REF{ex:key:991} further below: 


\ea%907
    \label{ex:key:907}
    \gll E    bin  pás  \textbf{na}  \textbf{mi}    \textbf{hós}.\\
\textsc{3sg.sbj}  \textsc{pst}  pass  \textsc{loc}  \textsc{1sg.poss}  house\\

\glt ‘She passed by my house [to see me].’ [ro05ee 078]
\z

The preposition \textit{to} ‘to’ is rare. It is employed with a locative function to mark a goal \REF{ex:key:908}. The following sentences represent two of altogether four occurrences of this preposition in the corpus. I point out that in \REF{ex:key:909}, the preposition \textit{to} is used to mark the goal in a motion-direction SVC in the same position as \textit{na} ‘\textsc{loc}’ or \textit{fɔ} ‘\textsc{prep}’ (cf. e.g. \ref{ex:key:957}). The use of \textit{to} as a complementiser\is{complementisers} is even more marginal (cf. \ref{ex:key:1280} for an example involving the main verb of cognition \textit{nó} ‘know (how to)) and is not common with the vast majority of speakers:


\ea%908
    \label{ex:key:908}
    \gll Yu  gó  \textbf{to}  yu  kɔ́mpin  yu  sé    “chico  dán  gɛ́l  de
    bɔ́t        mi”.\\
    \textsc{2sg}  go  to  \textsc{2sg}  friend  \textsc{2sg}  \textsc{quot}    boy    that  girl  \textsc{ipfv} 
hit.with.head    \textsc{1sg.indp}\\

\glt ‘You go to your friend (and) you say “man, that girl is rejecting me”.’ [au07se 066]
\z


\ea%909
    \label{ex:key:909}
    \gll Wé  dɛn  bin  kɛ́r=an    gó  \textbf{to}  dɔ́kta,  \op...\cp{}\\
\textsc{sub}  \textsc{3pl}  \textsc{pst}  carry=\textsc{3sg.obj}  go  to  doctor\\

\glt ‘When they took her to the doctor, (...)’ [ab03ay 121]
\z

The preposition \textit{na} ‘\textsc{loc}’ expresses location in the most general way. Depending on context, \textit{na} may denote superior \REF{ex:key:900}, interior, proximate, or lateral \REF{ex:key:906} location. The associative preposition \textit{fɔ} ‘\textsc{prep}’ is employed as a general locative preposition in ways similar to \textit{na} (cf. e.g. \ref{ex:key:935}, \ref{ex:key:955}, \ref{ex:key:956} and \ref{ex:key:983}). But compared to \textit{na} ‘\textsc{loc}’, the preposition ‘\textit{fɔ} ‘\textsc{prep}’ is only employed in a minority of instances for the expression of general location.\is{locative prepositions}

\subsection{Locative nouns}\label{sec:8.1.2}
\tabref{tab:key:8.2} presents the repertoire of locative nouns. The distribution of these elements (cf. \tabref{tab:key:8.3} further below) reflects their heterogeneity and intermediary status between noun and preposition. Circumferential location is expressed via the locative verb \textit{ráwn} ‘surround’ (cf. \ref{ex:key:937}) and distal location by means of the multifunctional word \textit{fá(wé)} ‘(be) far’ (cf. e.g. \ref{ex:key:895}). In Pichi, body part nouns such as \textit{bák} ‘back’ or \textit{fés} ‘face’ are not usually employed to express location roles.

%%please move \begin{table} just above \begin{tabular
\begin{table}
\caption{Locative nouns}
\label{tab:key:8.2}

\begin{tabularx}{\textwidth}{lllQ}
\lsptoprule
Locative noun & Translation & Type of location & Other uses\\
\midrule 
\itshape nía & ‘near, in contact with’ & Proximate; lateral & Verb: ‘be near’\\
\itshape kɔ́na & ‘next to’ & Proximate; lateral & Noun: ‘corner’\\
\itshape ínsay & ‘inside’ & Interior & Temporal: ‘during, within’\\
\itshape nadó & ‘outside’ & Exterior & —\\
\itshape bifó & ‘front, before’ & Anterior & Temporal: ‘before’\\
\itshape bihɛ́n & ‘rear, behind’ & Posterior & —\\
\itshape pantáp; ɔntɔ́p & ‘top, on’ & Superior (contact) & \mbox{‘in addition to’}\\
\itshape ɔ́p & ‘up(per side)’ & Superior & —\\
\itshape bɔtɔ́n & ‘bottom, under’ & Inferior (contact) & —\\
\itshape dɔ́n & ‘down (side)’ & Inferior & —\\
\itshape míndul & ‘middle, amidst’ & Medial & —\\
\lspbottomrule
\end{tabularx}
\end{table}
Locative nouns have characteristics in common with ordinary nouns. They may occur in the position of \textsc{NP}s, for example as subjects \REF{ex:key:910} or as goal objects\is{objects} of movement verbs, like \textit{rích} ‘arrive’ \REF{ex:key:911}. In both cases, an explicit mention of the ground is not required:


\ea%910
    \label{ex:key:910}
    \gll \op...\cp{}  mék    yu  tɔ́n=an,    porque  \textbf{bɔtɔ́n}  go  rós.\\
{}  \textsc{sbjv}    \textsc{2sg}  turn=\textsc{3sg.obj}  because  bottom  \textsc{pot}  burn\\

\glt ‘(...) turn it, because the bottom might burn.’ [dj03do 055]
\z


\ea%911
    \label{ex:key:911}
    \gll Yu  de  klém  fɔ  \textbf{rích}    \textbf{pantáp}.\\
\textsc{2sg}  \textsc{ipfv}  climb  \textsc{prep}  arrive  top\\

\glt ‘You’re climbing in order to reach the top.’ [au07se 086]
\z

In the same vein, a locative noun can appear as the adverbial complement of the locative-existential copula \textit{dé} \textsc{‘be.loc’} \REF{ex:key:912}:\is{complements:copula complements}


\ea%912
    \label{ex:key:912}
    \gll \op...\cp{}  e    \textbf{dé}    \textbf{ɔ́p},  gó  só!\\
{}  \textsc{3sg.sbj}  \textsc{be.loc}  up  go  like.that\\

\glt ‘(...) it’s [farther] up, go this way!’ [ma03ni 011]
\z

All locative nouns except \textit{nía} ‘near’, \textit{kɔ́na} ‘next to’, and \textit{nadó} ‘outside’ may also be preceded by the definite article \textit{di} ‘\textsc{def}’ as in the following example: \is{article}


\ea%913
    \label{ex:key:913}
    \gll \textbf{Di}  \textbf{dɔ́n}    na  violeta\\
 \textsc{def}  down  \textsc{foc}  violet\\

\glt ‘The lower part is violet.’ [ma03hm 034]\\
\z

In addition, all locative nouns except \textit{nía} ‘near’, \textit{kɔ́na} ‘next to’, and \textit{nadó} ‘outside’ may also be preceded by the general locative preposition\is{general} \textit{na} ‘\textsc{loc}’ like any ordinary noun. In the data, such constructions are, however, very rare, and none of these locative nouns is preceded by the general associative preposition \textit{fɔ} ‘\textsc{prep}’ instead of \textit{na} ‘\textsc{loc}’: 


\ea%914
    \label{ex:key:914}
    \gll E    púl=an      \textbf{na}  \textbf{pantáp}  di  béd.\\
\textsc{3sg.sbj}  remove=\textsc{3sg.obj}  \textsc{loc}  top    \textsc{def}  bed\\

\glt ‘She took him from the bed.’ [ab03ab 079]
\z


\ea%915
    \label{ex:key:915}
    \gll Na  fɔ  mék    nó  gó  \textbf{na}  \textbf{dɔ́n}.\\
\textsc{foc}  \textsc{prep}  make  \textsc{neg}  go  \textsc{loc}  down\\

\glt ‘It’s in order (for us) not to go down.’ [ma03hm 003]
\z

The locative nouns \textit{nía} ‘near’, \textit{kɔ́na} ‘next to’, \textit{nadó} ‘outside’, and \textit{bifó} ‘before, front’ are not normally found as complements to \textit{na} ‘\textsc{loc}’ in prepositional phrases\is{prepositional phrases} like the ones above. The peculiar distribution of \textit{nía} and \textit{kɔ́na} may be due to their multifunctionality. \textit{Nía} also functions as a locative verb ‘be near’ (cf. \ref{ex:key:939}), \textit{kɔ́na} as a common noun ‘corner’, and \textit{bifó} as a time clause linker ‘before’ (cf. \sectref{sec:10.7.3}). In \REF{ex:key:933} below, \textit{kɔ́na} is employed as a locative noun, in the following example \REF{ex:key:916}, as a common noun: 


\ea%916
    \label{ex:key:916}
    \gll E    de  sɛ́l  e    de  pút  smɔ́l  smɔ́l  wán  fɔ  \textbf{kɔ́na} 
    mék    e    fít  bák    dán  mán    in    mɔní.\\
    \textsc{3sg.sbj}  \textsc{ipfv}  sell  \textsc{3sg.sbj}  \textsc{ipfv}  put  small  \textsc{red}    one  \textsc{prep}  corner
\textsc{sbjv}    \textsc{3sg.sbj}  can  return  that  man    \textsc{3sg.poss}  money\\

\glt ‘(...) she’s selling (and) she’s putting a bit at the side in order to be able to give
that man back his money.’ [hi03cb 220]
\z

In turn, \textit{nadó} is a lexicalised collocation, in which the locative preposition \textit{na} already serves as the first component. The second component is the rare noun \textit{do} ‘door’ (the more current word for ‘door’ is \textit{do.mɔ́t} ‘door.mouth’). Although it is lexicalised, the prepositional phrase which constitutes this collocation therefore has a residual meaning of its own. I assume that this results in the ungrammaticality of a sequence like \textit{*na nadó} ‘\textsc{loc} outside’.


When the locative nouns \textit{bifó} ‘before’, \textit{bihɛ́n} ‘behind’, \textit{ɔ́p} ‘upperside’, and \textit{dɔ́n} ‘downside’ appear in a nominal position, speakers tend to employ an associative construction featuring the generic place noun \textit{sáy} ‘side, place’ \REF{ex:key:918} and sometimes \textit{pát} ‘part, place’ \REF{ex:key:919} as a modified noun and the locative noun as a modifier noun. This construction, which serves to derive a nominal structure, is favoured with these nouns when a ground is not mentioned. Compare \REF{ex:key:917} with an explicit ground (i.e. \textit{di hós} ‘the house’) and the two sentences thereafter without mention of a ground:



\ea%917
    \label{ex:key:917}
    \gll E    dé    \textbf{bifó}    di  \textbf{hós}.\\
\textsc{3sg.sbj}  \textsc{be.loc}  before  \textsc{def}  house\\

\glt ‘She’s in front of the house.’ [ye07de 026]
\z


\ea%918
    \label{ex:key:918}
    \gll E    dé    \textbf{bifó}    \textbf{sáy}.\\
\textsc{3sg.sbj}  \textsc{be.loc}  before  side\\

\glt ‘She’s at the front.’ [ye07de 025]
\z


\ea%919
    \label{ex:key:919}
    \gll Di  pambɔ́d  gó  bihɛ́n  dí  bíg  stón    yá,    \textbf{bifó}    \textbf{pát},
e    gó  dé.\\
\textsc{def}  bird    go  behind  this  big  stone  here    before  part
\textsc{3sg.sbj}  go  there\\

\glt ‘The bird went behind this big stone here, the front part, it went there.’ [ed03sb 174]
\z

However, when the ground is explicitly mentioned, most locative nouns participate in a construction that is structurally equivalent to a prepositional phrase featuring a preposition and an object complement. Compare \REF{ex:key:901} above with (\ref{ex:key:920}–\ref{ex:key:922}) below:


\ea%920
    \label{ex:key:920}
    \gll Di  béd  dé    \textbf{míndul}  \textbf{di}  \textbf{rúm}.\\
\textsc{def}  bed  \textsc{be.loc}  middle  \textsc{def}  room\\

\glt ‘The bed is in the middle of the room.’ [ro05ee 118]
\z


\ea%921
    \label{ex:key:921}
    \gll Boyé  sidɔ́n  \textbf{bihɛ́n}  \textbf{dís}  \textbf{hós}.\\
\textsc{name}  stay    behind  this  house\\

\glt ‘Boyé lives behind this house.’ [ro05ee 073]
\z


\ea%922
    \label{ex:key:922}
    \gll E    de  cruza-cruza    \textbf{bifó}    \textbf{di}  \textbf{domɔ́t},  e    de  dú
lɛk  sé    e    de  fɛ́n    sɔn    tín.\\
\textsc{3sg.sbj}  \textsc{ipfv}  cross.\textsc{cpd}{}-cross  before  \textsc{def}  door  \textsc{3sg.sbj}  \textsc{ipfv}  do
like  \textsc{quot}    \textsc{3sg.sbj}  \textsc{ipfv}  look.for  some  thing\\

\glt ‘He’s walking back and forth in front of the door, he’s pretending to be 
looking for something.’ [ne07fn 170]
\z

The same holds for the locative nouns \textit{nía} ‘near’ and \textit{kɔ́na} ‘next to’, which behave differently from other locative nouns in other contexts: 


\ea%923
    \label{ex:key:923}
    \gll Yu  fít  tɔ́k  sé    “dɛn    sidɔ́n  \textbf{nía}    \textbf{di}  \textbf{fáya}”.\\
\textsc{2sg}  can  talk  \textsc{quot}    \textsc{3pl}    sit    near    \textsc{def}   fire\\

\glt ‘You can say “they’re sitting by the fire”.’ [ro05ee 112]
\z


\ea%924
    \label{ex:key:924}
    \gll A    sidɔ́n  \textbf{kɔ́na}    \textbf{di}  \textbf{aeropuerto}.\\
\textsc{1sg.sbj}  stay    next.to    \textsc{def}  airport\\

\glt ‘I stay next to the airport.’ [dj05be 213]
\z

The ground need not be marked for definiteness\is{definiteness} as it is in the two examples above. Three sentences follow without overt definiteness marking. In this respect, the same principles of definiteness marking apply as they do for other objects. Note that the locative nouns \textit{ɔntɔ́p} ‘top, on’ \REF{ex:key:925} and \textit{pantáp} ‘top, on’ \REF{ex:key:914} above are absolute synonyms and equally frequent.


\ea%925
    \label{ex:key:925}
    \gll Di  pépa  dé    \textbf{ɔntɔ́p}  \textbf{tébul}.\\
\textsc{def}  paper  \textsc{be.loc}  top    table\\

\glt ‘The paper is on the table.’ [ro05ee 091]
\z


\ea%926
    \label{ex:key:926}
    \gll Discoteca  dɛn  dé    \textbf{bɔtɔ́n}  \textbf{grɔ́n}  ɛ́n.\\
club      \textsc{pl}  \textsc{be.loc}  bottom  ground  \textsc{intj}\\

\glt ‘(The) clubs are under the ground, you know.’ [ed03sb 217]
\z


\ea%927
    \label{ex:key:927}
    \gll Dán  skúl    e    dé    \textbf{nía}    \textbf{bɛrin-grɔ́n},    nɔ́?\\
that  school  \textsc{3sg.sbj}  \textsc{be.loc}  near    burial.\textsc{cpd}{}-ground  \textsc{intj}\\

\glt ‘That school is near the cemetery, right?’ [ma03hm 018]
\z

The locative noun \textit{nadó} ‘outside’ behaves differently in this respect. The ground may only be expressed in a possessive construction\is{possessive constructions}, namely a \textit{fɔ}{}-prepositional phrase:


\ea%928
    \label{ex:key:928}
    \gll Pɔ́sin  dɛn  dé    \textbf{nadó}  \textbf{fɔ} di  avión.\\
person  \textsc{pl}  \textsc{be.loc}  outside  \textsc{prep}  \textsc{def}  plane\\

\glt ‘People are outside the plane.’ [dj05be 165]
\z

The expression of the ground by way of a \textit{fɔ}-prepositional phrase as in \REF{ex:key:928} above is not accepted with other locative nouns, i.e. *\textit{míndul fɔ di rúm} \{middle \textsc{prep def} room\} ‘in the middle of the room’, *\textit{bihɛ́n fɔ dís hós} \{behind \textsc{prep} this house\} ‘behind this house’. This also holds for the locative associative constructions described further below in \REF{ex:key:930}. Compare the ungrammatical example \REF{ex:key:929}, which involves such a structure: 


\ea[*]{%929
    \label{ex:key:929}
    \gll E    dé    \textbf{bifó}    \textbf{sáy}  \textbf{fɔ} di  hós.\\
 \textsc{3sg.sbj}  \textsc{be.loc}  before  side  \textsc{prep}  \textsc{def}  house\\
\glt Intended: ‘She’s in front of the house.’ [ye07de 024]
}\z

Furthermore, \textit{dɔ́n} ‘down’ does not normally occur in clauses with an explicit ground at all. An explicit ground may, however, be included in the clause by making use of another possessive structure, namely an associative construction. \textit{Dɔ́n} enters into a recursive collocation with the generic noun \textit{sáy} ‘side, place’, which in turn functions as the modifier to the ground in yet another associative construction. Compare the following example: \is{associative}


\ea%930
    \label{ex:key:930}
    \gll \textbf{Dɔ́n}    \textbf{sáy}  Santa  Teresita.\\
down  side  \textsc{place}\\

\glt ‘(At) the lower side (of) Santa Teresita.’ [ye07de 021]
\z

All locative nouns except \textit{nadó} ‘outside’ may be followed by locative adverbs as in the following two examples featuring \textit{dɔ́n} ‘down’ and \textit{bɔtɔ́n} ‘under’: 


\ea%931
    \label{ex:key:931}
    \gll Wi  de  dú=an    \textbf{dɔ́n}    \textbf{yá}   na  mi    kɔ́ntri.\\
\textsc{1pl}  \textsc{ipfv}  do=\textsc{3sg.obj}  down  here    \textsc{loc}  \textsc{1sg.poss}  country\\

\glt ‘We do it down here in my hometown.’ [ab03ay 070]
\z


\ea%932
    \label{ex:key:932}
    \gll E    sé    mí    nó  de  mék    e    slíp    \textbf{bɔtɔ́n}  \textbf{dé}.\\
\textsc{3sg.sbj}  \textsc{quot}    \textsc{1sg.indp}  \textsc{neg}  \textsc{ipfv}  make  \textsc{3sg.sbj}  sleep  under  there\\

\glt ‘She said I [\textsc{emp}] don’t make him sleep under there [the mosquito net].’ [ab03ab 139]
\z

Moreover, all locative nouns except \textit{nadó} ‘outside’, \textit{dɔ́n} ‘down’ and \textit{ɔ́p} ‘up’ may appear with personal pronouns as the ground in the same way as prepositions like \textit{fɔ} ‘\textsc{prep’} and \textit{wet} ‘with’ (hence prepositions that are not [exclusively] used for the expression of locative roles). This sets the locative nouns to which this applies apart from locative prepositions: 


\ea%933
    \label{ex:key:933}
    \gll E    pás    \textbf{kɔ́na}  \textbf{mí}.\\
\textsc{3sg.sbj}  pass    next.to  \textsc{1sg.indp}\\

\glt ‘He went past next to me.’ [dj05be 212]
\z


\ea%934
    \label{ex:key:934}
    \gll Motó  de  kɔmɔ́t    \textbf{bihɛ́n}  \textbf{yú}    pan  yu  lɛf-hán.\\
car    \textsc{ipfv}  come.out  behind  \textsc{2sg.indp}  on  \textsc{2sg}  left.\textsc{cpd}{}-hand\\

\glt ‘A car is coming out behind you on your left.’ [ro05ee 108]
\z

The distribution of the locative nouns discussed is summarised in \tabref{tab:key:8.3}.

%%please move \begin{table} just above \begin{tabular
\begin{table}
\caption{Distribution of locative nouns}
\label{tab:key:8.3}

\begin{tabularx}{\textwidth}{QCCCQ}
\lsptoprule
Locative noun &
  \raggedright Can be pre-ceded by \textstyleTablePichiZchn{di} ‘\textsc{def’} and \textstyleTablePichiZchn{na} ‘\textsc{loc’} & 
  \raggedright Can be followed by \textstyleTablePichiZchn{yá} ‘here’ and \textstyleTablePichiZchn{dé} ‘there’ & 
  \raggedright Can be modifier to \textstyleTablePichiZchn{sáy} ‘side’ or \textstyleTablePichiZchn{pát} ‘part’ & Relation of ground to locative noun\\
\midrule 
\itshape nadó &  &  &  & \textstyleTablePichiZchn{fɔ}{}-PP\\
\itshape nía &  & x &  & Complement\\
\itshape kɔ́na &  & x &  & Complement\\
\itshape ínsay & x & x &  & Complement\\
\itshape míndul & x & x &  & Complement\\
\itshape bɔtɔ́n & x & x &  & Complement\\
\itshape pantáp, ɔntɔ́p & x & x &  & Complement\\
\itshape bifó & x & x & x & Complement\\
\itshape bihɛ́n & x & x & x & Complement\\
\itshape ɔ́p & x & x & x & Complement\\
\itshape dɔ́n & x & x & x & \textstyleTablePichiZchn{dɔ́n sáy} + ground\\
\lspbottomrule
\end{tabularx}
\end{table}
In sum, locative nouns are diverse in nature. All locative nouns differ from prepositions in that they do not require an explicit complement. Some locative nouns cannot be preceded by the determiner or the locative preposition \textit{na} ‘\textsc{loc}’, and hence lack a decisive diagnostic feature of “nouniness” in Pichi (i.e. \textit{nía} ‘near’, \textit{kɔ́na} ‘next to’, and \textit{nadó} ‘outside’). 


Other locative nouns are, in contrast, “nouny”. They may not only be preceded by the definite article \textit{di} and the preposition \textit{na}, i.e. \textit{bifó} ‘before’, \textit{bihɛ́n} ‘behind’, \textit{ɔ́p} ‘up(per side)’, \textit{bɔtɔ́n} ‘bottom’, \textit{dɔ́n} ‘down (side)’, and \textit{míndul} ‘middle’. Many of them may also enter as modifier nouns into associative constructions with the generic place nouns \textit{sáy} ‘side, place’ and \textit{pát} ‘part, place’.{\fff} 



Except \textit{nadó} ‘outside’ and \textit{dɔ́n} ‘down’, however, all locative nouns also appear in the same syntactic position as prepositions when relating a figure to an explicitly mentioned ground. In this respect, these two locative nouns are therefore similar in their distribution to the deictic adverbs \textit{yá}, ‘here’, \textit{dé} ‘there’, and \textit{yandá} ‘yonder’.\is{adverbial phrases}


\subsection{Locative verbs} \label{sec:8.1.3}
\tabref{tab:key:8.4} below provides an overview of the most common locative verbs. These verbs serve to express the manner in which a figure is located with respect to a ground. The column entitled ‘manner of location’ groups these verbs into three classes (cf. \citealt{Ameka2007}).

%%please move \begin{table} just above \begin{tabular
\begin{table}
\caption{Locative verbs}
\label{tab:key:8.4}

\begin{tabularx}{\textwidth}{lQl}
\lsptoprule

Verbs & Stative \& dynamic gloss & Manner of location\\
\midrule
\itshape dé & \textsc{‘be.loc’} & Location\\
\itshape ráwn & ‘be round, form a circle, surround’ & \\
\itshape lɛ́f & ‘remain at, leave at’ & \\
\itshape nía & ‘be near to, bring near’ & \\

\tablevspace
\itshape sidɔ́n & ‘sit, seat’ & Posture\\
\itshape tínap & ‘stand, stand up ‘ & \\
\itshape slíp & ‘sleep, lie, lay’ & \\
\itshape lé & ‘lie, lay’ & \\

\tablevspace
\itshape jám & ‘be in/make contact’ & Adhesion and attachment\\
\itshape hɛ́ng & ‘be hung onto, hang onto’ & \\
\itshape pín & ‘stuck to/in, stick to/in’ & \\
\itshape líng & ‘lean against, be leaning against’ & \\
\lspbottomrule
\end{tabularx}
\end{table}
With the exception of the locative-existential copula \textit{dé} \textsc{‘be.loc’,} all other verbs listed above are labile\is{labile verbs} verbs. Hence they may be used as (inchoative-)stative verbs in intransitive clauses and as dynamic verbs in transitive clauses. In intransitive clauses, the figure is the theme\is{theme} subject \REF{ex:key:935}, and in transitive clauses, the figure is the patient\is{patient} object \REF{ex:key:936}. The ground is expressed as a locative adverb(ial phrase) in both alternations:


\ea%935
    \label{ex:key:935}
    \gll Dɛn  \textbf{líng}    fɔ  dán  butaca.\\
\textsc{3pl}  lean    \textsc{prep}  this  armchair\\

\glt ‘They’re sitting reclined in that armchair.’ [befn07 207]\\
\z

\ea%936
    \label{ex:key:936}
    \gll E    \textbf{líng}=an    dé.\\
\textsc{3sg.sbj}  lean=\textsc{3sg.obj}  there\\

\glt ‘He leaned it there.’ [li07pe 063]
\z

The copula\is{copula:locative-existential} \textit{dé} \textsc{‘be.loc’} expresses existence in a location or in a manner in its most general sense (cf. \sectref{sec:7.6.1}). More specific nuances of location are expressed by other locative verbs. Compare the stative use of \textit{ráwn} ‘surround’ in the intransitive clause in \REF{ex:key:937}:


\ea%937
    \label{ex:key:937}
    \gll Di  ríba    e    \textbf{ráwn}    di  hós.\\
\textsc{def}  river  \textsc{3sg.sbj}  surround  \textsc{def}  house\\

\glt ‘The river flows around the house.’ [dj05be 228]
\z

Next to its use as a locative noun \REF{ex:key:938}, the multifunctional item \textit{nía} ‘near’ may be employed as a an inchoative-stative \REF{ex:key:939} or dynamic verb \REF{ex:key:940} like any other locative verb, although the latter usage is rare: 


\ea%938
    \label{ex:key:938}
    \gll Di  glás    dé    \textbf{nía}.\\
\textsc{def}  glass  \textsc{be.loc}  near\\
\glt ‘The glass is near.’ [dj07ae 193]\\
\z

\ea%939
    \label{ex:key:939}
    \gll Di  glás    \textbf{nía}    di  domɔ́t.\\
\textsc{def}  glass  near    \textsc{def}  door\\

\glt ‘The glass is near the door.’ [dj07ae 194]
\z


\ea%940
    \label{ex:key:940}
    \gll \textbf{Nía}    di  glás,    a    bɛ́g.\\
near    \textsc{def}  glass  \textsc{1sg.sbj}  beg\\

\glt ‘Bring the glass near, please.’ [dj07ae 195]
\z

Some locative verbs select specific figures according to the criterion of animacy. For example, \textit{sidɔ́n} ‘sit (down)’ generally implies an animate (e.g.\textit{ pikín} ‘child’) and \textit{pín} ‘stick (into)’ an inanimate (e.g. \textit{stík} ‘tree’) figure. Consider \REF{ex:key:941} and \REF{ex:key:942} respectively: \is{animacy}


\ea%941
    \label{ex:key:941}
    \gll E    \textbf{sidɔ́n}  \textbf{di}  \textbf{pikín}  na  butaca.\\
\textsc{3sg.sbj}  seat    \textsc{def}  child  \textsc{loc}  armchair\\

\glt ‘She seated the child in (the) armchair.’ [dj07ae 234]
\z


\ea%942
    \label{ex:key:942}
    \gll E    \textbf{pín}    \textbf{di}  \textbf{stík}  na  grɔ́n.\\
\textsc{3sg.sbj}  stick  \textsc{def}  tree  \textsc{loc}  ground\\

\glt ‘He stuck the stick in (the) ground.’ [li07pe 092]
\z

In contrast, all the other verbs listed in \tabref{tab:key:8.4} exhibit no such restrictions. This includes verbs that denote other, typically human postures. For example, \textit{tínap} ‘stand (up)’ may appear with an inanimate \REF{ex:key:943} or animate \REF{ex:key:944} figure as well as in intransitive and transitive \REF{ex:key:945} clauses alike:


\ea%943
    \label{ex:key:943}
    \gll Di  \textbf{kasára}  \textbf{tínap}  míndul  tú  stík.\\
\textsc{def}  cassava  stand  middle  two  tree\\

\glt ‘The cassava is standing upright between two trees.’ [li07pe 081]
\z


\ea%944
    \label{ex:key:944}
    \gll \textbf{Di}  \textbf{mán}  \textbf{tínap}  míndul  pípul  dɛn.\\
\textsc{def}  man    stand  middle  people  \textsc{pl}\\

\glt ‘The man is standing amidst people.’ [ye05ce 282]
\z


\ea%945
    \label{ex:key:945}
    \gll E    \textbf{tín}\textbf{ap}  \textbf{di}  \textbf{kasára}  míndul  tú  stík.\\
\textsc{3sg.sbj}  stand.up  \textsc{def}  cassava  middle  two  tree\\

\glt ‘He stood up the cassava between two trees.’ [li07pe.082]
\z

Also compare the intransitive use of \textit{slíp} ‘sleep, lie, lay’ in \REF{ex:key:946} with the transitive use of \textit{slíp} in \REF{ex:key:947}. Both sentences involve the inanimate figure \textit{bɔ́tul} ‘bottle’: 


\ea%946
    \label{ex:key:946}
    \gll \textbf{Di}  \textbf{bɔ́tul}  \textbf{slíp}   pantáp  di  tébul  bikɔs  \textbf{di}  \textbf{bɔ́tul}  \textbf{lé}  dé.\\
\textsc{def}  bottle  sleep  top    \textsc{def}  table  because  \textsc{def}  bottle  lie  there\\

\glt ‘The bottle is lying [in a horizontal position] on the table because the bottle is lying 
there.’ [li07pe 075]
\z


\ea%947
    \label{ex:key:947}
    \gll E    \textbf{slíp}    \textbf{di}  \textbf{bɔ́tul}  pantáp  di  tébul.\\
\textsc{3sg.sbj}  sleep  \textsc{def}  bottle  top    \textsc{def}  table\\

\glt ‘He laid the bottle on the table [in a horizontal position].’ [li07pe 072]\is{posture verbs}
\z

The verb \textit{jám} ‘make/be in contact’ denotes contact between figure and ground. The meaning of \textit{jám} contains no connotation with respect to the type of contact. Hence intransitive \textit{jám} means ‘be in contact’ in \REF{ex:key:948}. Note the use of the \ili{Spanish}-derived verb \textit{para} ‘stand’ as a labile locative verb just like its Pichi equivalent \textit{tínap} ‘stand (up)’ in \REF{ex:key:945} above:


\ea%948
    \label{ex:key:948}
    \gll Dɛn  \textbf{para}  di  búk    dɛn  sé    dɛn  \textbf{jám}        dɛn  sɛ́f.\\
\textsc{3pl}  stand.up  \textsc{def}  book  \textsc{pl}  \textsc{quot}    \textsc{3pl}  make.contact    \textsc{3pl}  self\\

\glt ‘The books were stood up [in such way] that they’re in contact with each 
other.’ [dj07re 044]
\z

When \textit{jám} is used transitively, context may imply a sudden or forceful contact, as in the following sentence: 


\ea%949
    \label{ex:key:949}
    \gll So  di  mán  kán  pás  nía  ín,    e    \textbf{jám}=an, 
di  plét    fɔdɔ́n  na  grɔ́n.\\
so  \textsc{def}  man  \textsc{pfv}  pass  near  \textsc{3sg.indp}  \textsc{3sg.sbj}  make.contact=\textsc{3sg.obj} 
\textsc{def}  plate  fall    \textsc{loc}  ground\\

\glt ‘So the man passed near her, he bumped into her, the plate fell 
to the ground.’ [au07se 013]
\z

The following two examples involve the stative/dynamic alternation of the verb of adhesion and attachement \textit{hɛ́ng} ‘be hung onto, hang onto’:


\ea%950
    \label{ex:key:950}
    \gll Di  písis        \textbf{hɛ́ng}  na  di  stík  nɔ́,
bikɔs  nó  mán    nó  pút=an.\\
\textsc{def}  piece.of.cloth    hang  \textsc{loc}  \textsc{def}  tree  \textsc{intj}
because  \textsc{neg}  man    \textsc{neg}  put=\textsc{3sg.obj}\\

\glt ‘The piece of cloth is hanging onto the stick, right, because 
nobody put it there.’ [li07pe 058]
\z


\ea%951
    \label{ex:key:951}
    \gll E    táy  di  kasára  wet  róp    áfta    e    \textbf{hɛ́ng}=an.\\
\textsc{3sg.sbj}  tie  \textsc{def}  cassava  with  rope  then  \textsc{3sg.sbj}  hang=\textsc{3sg.obj}\\

\glt ‘He tied the cassava with a rope and then he hung it up.’ [li07pe 078]\is{locative verbs}
\z

\subsection{Motion verbs}\label{sec:8.1.4}

Besides the locative verbs discussed in \sectref{sec:8.1.3}, Pichi features verbs of diverse semantic types whose meanings also include a change of location, and hence motion. A selection of the most common ones in the corpus is provided in \tabref{tab:key:8.5}. Some of these verbs contain the additional meaning components of direction (e.g. \textit{gó} ‘go (away)’) and/or manner of motion{\fff} (e.g. \textit{júmp} ‘jump’). Further, some verbs denote self-motion of the figure subject, hence are lexically intransitive (e.g. \textit{wáka} ‘walk’), or preponderantly appear in intransitive clauses (e.g. \textit{ɛ́nta} ‘enter’). Others involve motion caused by the figure subject and are therefore more likely to occur in transitive clauses with an overt ground object than in intransitive clauses without one (e.g. \textit{drɛ́b} ‘drive’, \textit{pút} ‘put’).


Moreover, the verbs listed in \tabref{tab:key:8.5} differ in the way the ground is expressed as a participant in the clause. Hence we find the ground expressed as prepositional phrases\is{prepositional phrases} (PP), objects\is{objects} (O), and as objects or prepositional phrases in serial verb constructions (SVCs).


%%please move \begin{table} just above \begin{tabular
\begin{table}
\caption{Motion verbs}
\label{tab:key:8.5}

\begin{tabularx}{\textwidth}{lX ccc l}
\lsptoprule
Verb & Gloss & Direction & Manner & Causation & Ground\\
\midrule 
\itshape gó & ‘go’ & x &  &  & PP, O\\
\itshape kán & ‘come’ & x &  &  & PP, O\\
\itshape kɔmɔ́t & ‘go/come out’ & x &  &  & PP, O\\
\itshape rích & ‘arrive’ & x &  &  & PP, O\\
\itshape ɛ́nta & ‘enter’ & x &  &  & PP\\
\itshape baja & ‘go down’ & x &  &  & PP\\
\itshape sube & ‘go up’ & x &  &  & PP\\
\itshape fɔdɔ́n & ‘fall’ &  & x &  & PP, O\\
\itshape júmp & ‘jump’ &  & x &  & PP, O\\
\itshape pás & ‘pass’ &  & x &  & PP, O\\
\itshape klém & ‘climb’ &  & x &  & PP, O\\
\itshape wáka & ‘walk’ &  & x &  & PP, SVC\\
\itshape rɔ́n & ‘run’ &  & x &  & PP, SVC\\
\itshape fláy & ‘fly’ &  & x &  & PP, SVC\\
\itshape fála & ‘follow’ &  & x &  & PP, SVC\\
\itshape drɛ́b & ‘drive’ &  & x & x & PP\\
\itshape bɔ́t & ‘cause to rebound’ &  & x & x & PP\\
\itshape flíng & ‘fling’ &  & x & x & PP\\
\itshape pús & ‘push’ &  & x & x & PP\\
\itshape híb & ‘throw’ &  & x & x & PP\\
\itshape ték & ‘take’ &  &  & x & PP, SVC, O\\
\itshape kɛ́r & ‘carry, take (to)’ &  &  & x & PP, SVC, O\\
\itshape bríng & ‘bring’ & x &  & x & PP, SVC, O\\
\itshape sɛ́n & ‘throw, send’ & x & x & x & PP, SVC, O\\
\itshape pút & ‘put’ & x & x & x & PP, O\\
\itshape púl & ‘remove’ & x & x & x & PP\\
\lspbottomrule
\end{tabularx}
\end{table}
The most commonly employed verbs to simultaneously encode motion and direction are \textit{gó} ‘go (away)’, \textit{kán} ‘come’, \textit{kɔmɔ́t} ‘go/come out of’, and \textit{rích} ‘arrive (at)’. These verbs also function as V2 in motion-direction SVCs. With any of these four motion verbs, the ground (i.e. the source or goal of the motion) may be expressed as an object of a transitive clause \REF{ex:key:952} or as a prepositional phrase in an intransitive clause \REF{ex:key:953}. The second alternative is, however, attested in the majority of cases:\is{noun phrase adverbials}


\ea%952
    \label{ex:key:952}
    \gll A    \textbf{kɔmɔ́t}    \textbf{colegio},  \op...\cp{}\\
\textsc{1sg.sbj}  come.out  college\\

\glt ‘I came out of college (...)’ [ab03ay 132]
\z


\ea%953
    \label{ex:key:953}
    \gll A    de  \textbf{kɔmɔ́t}    \textbf{na}  \textbf{tɔ́n}    náw    náw.\\
\textsc{1sg.sbj}  \textsc{ipfv}  come.out  \textsc{loc}  town  now    \textsc{rep}\\

\glt ‘I’m coming from town right now.’  [ro05ee 076]
\z

The preference for a prepositional phrase rather than an object also holds when the ground is a named place\is{named place}, such as Malabo, the capital of Equatorial Guinea \REF{ex:key:954}. A PP is also the favoured option when the ground occurs in a motion-direction SVC featuring one of the motion verbs listed above as a V2 (cf. \sectref{sec:11.2.1}).\is{general}


\ea%954
    \label{ex:key:954}
    \gll Bɔt  wé  e    \textbf{kán}    \textbf{na}  \textbf{Malábo},  ɛ́ni    nɛ́t
in    abuela    kin  kán    hambɔ́g=an.\\
but  \textsc{sub}  \textsc{3sg.sbj}  come  \textsc{loc}  Malabo  every  night 
\textsc{3sg.poss}  grandmother  \textsc{hab}  come  bother=\textsc{3sg.obj}\\

\glt ‘But when she came to Malabo, every night her grandmother
would come bother her.’ [ed03sb 042]
\z

In principle, the preposition\is{associative preposition} \textit{fɔ} ‘\textsc{prep}’ may introduce the inanimate goal of a motion verb instead of \textit{na} ‘\textsc{loc}’ \REF{ex:key:955}. In practice, the use of the general locative preposition \textit{na} instead of \textit{fɔ} ‘\textsc{prep’} as in \REF{ex:key:953} above is pervasive. Nevertheless, \textit{fɔ} must be used when the goal (or any other locative role) is animate\is{animacy} \REF{ex:key:956}:


\ea%955
    \label{ex:key:955}
    \gll Wi  kɔmɔ́t  dé    wi  kán  gó  \textbf{fɔ,}  \textbf{fɔ}  \textbf{Akébé Ville}.\\
\textsc{1pl}  go.out  there  \textsc{1pl}  \textsc{pfv}  go  \textsc{prep}  \textsc{prep}  \textsc{place}\\

\glt ‘We left that place (and then) went to, to Akebeville.’ [ma03hm 039]
\z


\ea%956
    \label{ex:key:956}
    \gll Yu  gɛ́fɔ    \textbf{gó}  \textbf{fɔ}  \textbf{yu}  \textbf{fámbul}.\\
\textsc{2sg}  have.to  go  \textsc{prep}  \textsc{2sg}  family\\

\glt ‘You had to go to your family.’ [ab03ab 035]
\z

All other verbs in \tabref{tab:key:8.5} whose goals may be expressed as a PP, an SVC, and an O exhibit the same pattern with respect to ground marking. This applies to locomotion verbs, such as \textit{wáka} ‘walk’, \textit{rɔ́n} ‘run’, or \textit{pás} ‘pass (by)’, to the caused location verb \textit{pút} ‘put’, or the caused motion verbs \textit{ték} ‘take’, \textit{bríng} ‘bring’, and \textit{kɛ́r} ‘carry, take’. The following three examples featuring the verb \textit{kɛ́r} once more present the PP \REF{ex:key:957}, the object \REF{ex:key:958} and the SVC alternatives \REF{ex:key:959}. Again the PP option is the most common one. Note that the goal object \textit{hospital} ‘hospital’ in \REF{ex:key:958} is positioned to the right of the patient object \textit{di pikín} ‘the child’: 


\ea%957
    \label{ex:key:957}
    \gll Di  cemento,  estaba  dicho  que    na  fɔ  kɛ́r=an
directamente  \textbf{na}  {\textbf{Ela} \textbf{Nguema}}.\\
\textsc{def}  cement    was    said    that    \textsc{foc}  \textsc{prep}  carry=\textsc{3sg.obj} 
directly    \textsc{loc}  \textsc{place}\\

\glt ‘The cement, it was said that it was to be taken directly to Ela Nguema.’ [ye03cd 008]
\z


\ea%958
    \label{ex:key:958}
    \gll A    kɛ́r    di  pikín  \textbf{hospital}.{\fff}\\
\textsc{1sg.sbj}  carry  \textsc{def}  child  hospital\\

\glt ‘I took the child to hospital.’ [dj07ae 343]
\z


\ea%959
    \label{ex:key:959}
    \gll Dɛn  kɛ́r=an    gó  \textbf{fɔ}  \textbf{pólis}.\\
\textsc{3pl}  carry=\textsc{3sg.obj}  go  \textsc{prep}  police\\

\glt ‘They took him to the police.’ [ma03sh 009]
\z

The manner-of-motion verbs \textit{wáka} (also \textit{wɔ́k}) ‘walk’, \textit{rɔ́n} ‘run’, and \textit{fláy} ‘fly’ are intransitive. Speakers univocally reject these verbs in grammaticality judgments featuring an undergoer or goal object (cf. \sectref{sec:9.2.1} for more details). 


Next to these, we find the manner-of-motion verbs \textit{fɔdɔ́n} ‘fall’ and \textit{pás} ‘pass’. These two verbs allow for the ground to be expressed as an O or a PP without any difference in meaning. Compare \textit{fɔdɔ́n} ‘fall’ in the following two examples: 



\ea%960
    \label{ex:key:960}
    \gll E    \textbf{fɔdɔ́n}  \textbf{di}  \textbf{béd}.\\
\textsc{3sg.sbj}  fall    \textsc{def}  bed\\

\glt ‘He fell from the bed.’ [pa07me 042]
\z


\ea%961
    \label{ex:key:961}
    \gll Di  bolí  \textbf{fɔdɔ́n}  \textbf{frɔn}    \textbf{di}  \textbf{tébul}.\\
\textsc{def}  pen  fall    from  \textsc{def}  table\\

\glt ‘The pen fell off the table.’ [dj05be 204]
\z

In turn, the use of either the PP or the O strategy of ground marking is accompanied by a change in meaning with the two manner-of-motion verbs \textit{júmp} ‘jump’ \REF{ex:key:962} and \textit{klém} ‘climb’ \REF{ex:key:963}. When the ground is expressed as an object, a clause featuring these two verbs is usually interpreted as involving locomotion (hence motion with a path) of the figure, as in the following two sentences:


\ea%962
    \label{ex:key:962}
    \gll Di  húman,  e    de  \textbf{júmp}  \textbf{di}  \textbf{wínda}.\\
\textsc{def}  woman  \textsc{3sg.sbj}  \textsc{ipfv}  jump  \textsc{def}  window\\

\glt ‘The woman is jumping through the window.’ [ra07se 068]
\z


\ea%963
    \label{ex:key:963}
    \gll E    stíl  butú  yét  wé  e    de  \textbf{klém}  \textbf{di}  \textbf{chía}.\\
\textsc{3sg.sbj}  still  stoop  yet  \textsc{sub}  \textsc{3sg.sbj}  \textsc{ipfv}  climb  \textsc{def}  chair\\

\glt ‘She’s still stooped over while she’s climbing the chair.’ [au07se 088]
\z

When the ground is, however, encoded as a PP, these two verbs may denote motion without a path, or locomotion with a path. Compare the alternative translations of \REF{ex:key:964}, featuring \textit{júmp} ‘jump’: 


\ea%964
    \label{ex:key:964}
    \gll Miguel  Ángel  de  \textbf{júmp}  \textbf{pantáp}  \textbf{di}  \textbf{béd}.\\
\textsc{name}  \textsc{name}  \textsc{ipfv}  jump  top    \textsc{def}  bed\\

\glt ‘Miguel Ángel is jumping on/onto the bed.’ [dj07ae 019]
\z

Likewise, speaker (au) finds \REF{ex:key:965} unacceptable, because he interprets the clause featuring \textit{klém} ‘climb’ as involving motion without a path on the ground\textit{ chía} ‘chair’:


\ea%965
    \label{ex:key:965}
    \gll Nóto  “e    \textbf{klém}  \textbf{pantáp}  \textbf{di}  \textbf{chía}.”\\
\textsc{neg}.\textsc{foc}  \textsc{3sg.sbj}  climb  top    \textsc{def}  chair\\

\glt ‘Not “he climbed [being] on the chair”.’ [au07se 085]\is{manner of motion}
\z

The “propulsion verbs\is{propulsion verbs}” (\citealt[200]{Longacre1996}) \textit{híb} ‘throw’ and \textit{flíng} ‘fling’ are caused-motion verbs without a direction component in their meaning. Here, the ground is preferably expressed as a PP or an equivalent locative adverbial\is{locative adverbials} as in the following examples: 


\ea%966
    \label{ex:key:966}
    \gll Dɛn  \textbf{híb}=an    \textbf{dɔ́n}.\\
\textsc{3pl}  throw=\textsc{3sg.obj}  down\\

\glt ‘It was thrown down.’ [dj07fn 136]
\z


\ea%967
    \label{ex:key:967}
    \gll A    \textbf{flíng}=an    \textbf{na}  \textbf{solwatá}.\\
\textsc{1sg.sbj}  fling=\textsc{3sg.obj}  \textsc{loc}  sea\\

\glt ‘I flung it into the sea.’ [nn03fn 002]
\z

The propulsion verb \textit{sɛ́n} equally involves caused motion without direction when used with the sense ‘throw (with aim)’. However, \textit{sɛ́n} additionally involves the notion of aim, hence has a manner component in its meaning:


\ea%968
    \label{ex:key:968}
    \gll E    de  \textbf{sɛ́n}    di  bɔ́l  fɔ  mék  e    nák  di  cartón.\\
\textsc{3sg.sbj}  \textsc{ipfv}  send  \textsc{def}  ball  \textsc{prep}  \textsc{sbjv}  \textsc{3sg.sbj}  hit  \textsc{def}  carton\\

\glt ‘He’s throwing the ball with aim in order to hit the cardboard box.’ [ra07se 175]
\z

In contrast, when sɛ́n occurs as a transfer verb in a double-object construction, it acquires the sense ‘throw to, send’, and therefore also features a direction component. In such double-object constructions, the ground, a usually animate recipient{\fff}, is only expressed as an object, not as a PP: 


\ea%969
    \label{ex:key:969}
    \gll E    \textbf{sɛ́n}=an    di  bɔ́l.\\
\textsc{3sg.sbj}  send=\textsc{3sg.obj}  \textsc{def}  ball\\

\glt ‘He threw the ball to him.’ [ra07se 093]
\z

Another motion verb which may appear in double-object constructions and has a direction, manner, and causation component is \textit{pút} ‘put’ (covered in detail in \sectref{sec:9.3.4}).\is{motion verbs}

\subsection{Expressing source and goal}\label{sec:8.1.5}

The foregoing sections have shown that the prepositions \textit{na} ‘\textsc{loc}’ and \textit{fɔ} ‘\textsc{prep}’ have a very general meaning and participate in various types of clauses expressing spatial relations. We have seen that these two prepositions may also mark the ground in clauses with a motion-to and a motion-from component. For example, in \REF{ex:key:953} above \textit{na} marks the source of \textit{kɔmɔ́t} ‘go/come out of’, and in \REF{ex:key:956} above \textit{fɔ} the goal of \textit{gó} ‘go’. 


In fact, any preposition or locative noun\is{locative nouns} that may serve to express an ‘at rest’ location role does not contribute any meaning to the motion component of the spatial relation. Instead, these elements specify the part of the ground where the figure is located (cf. \citealt{Essegbey2005}). Compare the locative nouns \textit{ɔntɔ́p} ‘top’ \REF{ex:key:970} and \textit{nía} ‘near’ \REF{ex:key:971}, which both express ‘at rest’ location and appear with motion verbs in these two sentences: 



\ea%970
    \label{ex:key:970}
    \gll Di  pambɔ́d  de  \textbf{fláy}  \textbf{ɔntɔ́p}  di  stík.\\
\textsc{def}  bird    \textsc{ipfv}  fly  top    \textsc{def}  tree\\

\glt ‘The bird is flying over/above the tree.’ [ro05ee 099]
\z


\ea%971
    \label{ex:key:971}
    \gll A    nó  nó    wétin  mék    Anto  \textbf{púl}
Reina  náw    \textbf{nía}    Tokobé.\\
\textsc{1sg.sbj}  \textsc{neg}  know  what  make  \textsc{name}  pull
\textsc{name}  now    near    \textsc{name}\\

\glt ‘I don’t know how come Anto pulled Reina away from Tokobé.’ [ab03ab 157]
\z

Hence, when a motion verb lacks a directional sense, it is the combined meaning of the verb, the preposition, and the complement \is{complements}that provides the meaning of the entire construction. The following sentences featuring the prepositions \textit{na} ‘\textsc{loc}’ and \textit{fɔ} \textit{\textup{‘}}\textit{\textsc{prep}}\textit{\textup{’}} are therefore not interpreted as involving ‘at rest’ location. Instead, the compositional meaning suggests a goal sense: \is{general}


\ea%972
    \label{ex:key:972}
    \gll Dɛn    \textbf{rɔ́n}    \textbf{na}  farmacia,  receta    dé  mɛ́rɛsin.\\
\textsc{3pl}    run    \textsc{loc}  pharmacy  prescription  of  medicine\\

\glt ‘They ran to [*in] the pharmacy, [to get a] prescription for medicine.’ [ab03ab 123]
\z


\ea%973
    \label{ex:key:973}
    \gll Dɛn  \textbf{pús}    di  motó  \textbf{na} garaje.\\
\textsc{3pl}  push  \textsc{def}  car    \textsc{loc}  garage\\

\glt ‘They pushed the car into [*in] the garage.’ 
\z

Sometimes, however, there may be room for ambiguity between a motion and a location reading as in \REF{ex:key:974}, featuring the propulsion verb \textit{sút} ‘shoot’, which lacks a directional sense. The ground PP introduced by the locative noun \textit{bifó} ‘before’ may be interpreted as a location (at rest), a source (motion-from) or a goal (motion-to): 


\ea%974
    \label{ex:key:974}
    \gll Di  soldado  \textbf{sút}    \textbf{bifó}    di  hós.\\
\textsc{def}  soldier  shoot  before  \textsc{def}  house\\

\glt ‘The soldier shot in front of/at/from the front of the house.’ [dj05be 188]
\z

Any potential ambiguity between the goal and source senses of \textit{na} and \textit{fɔ} may be eliminated by employing the directional prepositions \textit{frɔn} ‘from, since’ \REF{ex:key:975} and \textit{sóté} ‘until, up to’ \REF{ex:key:976}:


\ea%975
    \label{ex:key:975}
    \gll Di  bolí  \textbf{fɔdɔ́n}  \textbf{frɔn}    di  tébul.\\
\textsc{def}  pen  fall    from  \textsc{def}  table\\

\glt ‘The pen fell from the table’ [dj05be 204]
\z


\ea%976
    \label{ex:key:976}
    \gll E    kán  \textbf{fɔdɔ́n}  \textbf{sóté}    yá.\\
\textsc{3sg.sbj}  \textsc{pfv}  fall    until  here\\

\glt ‘(And then) it fell up to here.’ [li07pe 090]
\z

Alternatively, a motion-direction SVC may be employed to mark a goal with verbs permitting such use as in \REF{ex:key:977}. A biclausal structure featuring a modifying purpose\is{purpose clauses} or other adverbial clause may also serve the same end:


\ea%977
    \label{ex:key:977}
    \gll Dɛn  bin  de  \textbf{rɔ́n}  \textbf{gó}  \textbf{na} ɔspítul  la  una  de  la  noche.\\
\textsc{3pl}  \textsc{pst}  \textsc{ipfv}  run  go  \textsc{loc}  hospital  the  one   of  the  night\\

\glt ‘They were running to hospital at one o’clock in the night.’ [ab03ab 137]
\z


\ea%978
    \label{ex:key:978}
    \gll Dɛn  \textbf{pús}    di  motó  mék    e    \textbf{ɛ́nta}  \textbf{na}  garaje.\\
\textsc{3pl}  push  \textsc{def}  car    \textsc{sbjv}    \textsc{3sg.sbj}  enter  \textsc{loc}  garage\\

\glt ‘They pushed the car in order for it to enter the garage.’
\z

Nevertheless, even in clauses featuring inherently directional verbs where no such ambiguity could possibly arise, the goal or source is sometimes additionally marked with a directional preposition. Compare the following example, in which the motion-from sense of \textit{kɔmɔ́t} ‘come out of’ is reiterated by the ablative motion-from preposition \textit{frɔn} ‘from’:


\ea%979
    \label{ex:key:979}
    \gll Olinga  \textbf{kɔmɔ́t}    \textbf{frɔn}    bɔtɔ́n.\\
\textsc{name}  come.out  from  bottom\\

\glt ‘Olinga comes from the bottom [worked himself up from the bottom].’ [ye03cd 068]
\z

The general locative preposition\is{general locative preposition} \textit{na} ‘\textsc{loc}’ may also additionally mark the ground when preceded by the directional prepositions \textit{frɔn} ‘from’ and \textit{sóté} ‘until, up to’. This usage is not attested with the associative preposition \textit{fɔ} ‘\textsc{prep’}:


\ea%980
    \label{ex:key:980}
    \gll E    kɔ́l  \textbf{frɔn}    \textbf{na} plataforma,  e    kɔ́l  dɔ́n    yá.\\
\textsc{3sg.sbj}  call  from  \textsc{loc}  oil.rig    \textsc{3sg.sbj}  call  down  here\\

\glt ‘He called from the platform, he called down here.’ [to03gm 006]
\z


\ea%981
    \label{ex:key:981}
    \gll \op...\cp{}  mék  e    fít  de  rích    ɔ́l  sáy  \textbf{sóté}    \textbf{na} Riaba.\\
{}  \textsc{sbjv}  \textsc{3sg.sbj}  can  \textsc{ipfv}  arrive  all  side  until  \textsc{loc}  \textsc{place}\\

\glt ‘(...) so that he should be able to get everywhere (even) up to Riaba.’ [fr03cd 070]
\z

The use of the preposition \textit{fɔ} ‘\textsc{prep}’ may open up another space of ambiguity\is{associative preposition}. \textit{F}\textit{ɔ} may mark an animate source or beneficiary. Hence, the meaning of clauses featuring verbs which may assign both animate source and beneficiary roles are potentially ambiguous. Compare \textit{recibe} ‘receive’ and \textit{báy} ‘buy’ below: \is{beneficiary}


\ea%982
    \label{ex:key:982}
    \gll \op...\cp{}  e    \textbf{recibe}  wán  regalo  \textbf{fɔ} in    mamá.\\
{}  \textsc{3sg.sbj}  receive  one  present  \textsc{prep}  \textsc{3sg.poss}  mother\\

\glt ‘(...) she received a present for/from her mother.’ [dj05be 067]
\z


\ea%983
    \label{ex:key:983}
    \gll A    bin  \textbf{báy}  wán  motó  \textbf{fɔ} mi    mása.\\
\textsc{1sg.sbj}  \textsc{pst}  buy  one  car    \textsc{prep}  \textsc{1sg.poss}  boss\\

\glt ‘I bought a car for/from my boss.’ [dj05be 073]
\z

Speakers may resort to other means of expressing these relations in pursuit of disambiguation. Example \REF{ex:key:982} above and \REF{ex:key:984} below were both elicited by means of the Spanish sentence \textit{recibió un regalo dé su mamá} ‘she received a present from her mother’. In the sentence below, speaker (ro) prefers to employ the transfer verb \textit{dás} ‘give as present’ which assigns an agent instead of a theme\is{theme} subject: 


\ea%984
    \label{ex:key:984}
    \gll Mi    mamá  bin  \textbf{dás}        mí    sɔn    regalo.\\
\textsc{1sg.poss}  mother  \textsc{pst}  give.as.present  \textsc{1sg.indp}  some  present\\

\glt ‘My mother gave me a present.’ [ro05ee 055]
\z

Speaker (ro) also employs a partitive\is{partitive} possessive construction\is{possessive constructions} in \REF{ex:key:985} below in order to render the meaning of Spanish \textit{compré un coche dé mí jefe} ‘I bought a car from my boss.’ Compare \REF{ex:key:985} below to \REF{ex:key:983} above, where speaker (dj) uses the \textit{fɔ}{}-possessive construction instead (which is structurally similar to the Spanish \textit{de}{}-possessive construction):


\ea%985
    \label{ex:key:985}
    \gll A    bin  báy  \textbf{wán} mi    másta \textbf{in} \textbf{motó}\textbf{\textmd{.}}\\
\textsc{1sg.sbj}  \textsc{pst}  buy  one    \textsc{1sg.poss}  boss    \textsc{3sg.poss}  car\\

\glt ‘I bought one of my boss’s cars.’ [ro05ee 057]
\z

The manner-of-motion verb \textit{pás} ‘pass (by)’ is employed to express motion-past a ground. The ground is normally expressed as a PP introduced by a locative preposition\is{locative prepositions} \REF{ex:key:986} or locative noun \REF{ex:key:987}: 


\ea%986
    \label{ex:key:986}
    \gll \'{I}n    bin  \textbf{pás}  \textbf{na} mi    hós.\\
\textsc{3sg.indp}  \textsc{pst}  pass  \textsc{loc}  \textsc{1sg.poss}  house\\
\glt ‘He [\textsc{emp}] passed (by/through) my house.’ [dj05be.143]
\z

\ea%987
    \label{ex:key:987}
    \gll Di  motó  \textbf{pás}  \textbf{ɔntɔ́p}  di  rayt-hán.\\
\textsc{def}  car    pass  top    \textsc{def}  right.\textsc{cpd}{}-hand\\

\glt ‘The car passed (by) on the right hand side.’ [ro05ee 104]
\z

The nature of a spatial relation may be specified in detail by making use of the appropriate combination of motion verbs\is{motion verbs}, locative prepositions\is{locative prepositions}, locative nouns,\is{locative nouns} and SVCs.


For example, the situation in \REF{ex:key:988} involves a figure (the theme\is{theme} \textit{pikín} ‘child’) which undergoes a change-of-location (denoted by \textit{fɔdɔ́n} ‘fall’) in a motion-from along a path (specified by \textit{frɔn} ‘from’) out of the specific part (the superior location \textit{ɔ́p} ‘upperside’) of the ground (the source \textit{stík} ‘tree’):



\ea%988
    \label{ex:key:988}
    \gll Di  pikín  \textbf{fɔdɔ́n}  \textbf{frɔn}    \textbf{ɔ́p}  di  \textstylePichiexamplenumberZchnZchn{stík.}\\
\textsc{def}  child  fall    from  up  \textsc{def}  tree\\

\glt ‘The child fell from up in the tree.’ [dj05be 201]
\z

In \REF{ex:key:989}, the figure (\textit{wi} ‘\textsc{1pl}’) instigates a motion-from (denoted by \textit{kɔmɔ́t} ‘go out’) out of the specific part (the anterior location \textit{bifó} ‘before’) of the ground (the source \textit{chɔ́ch} ‘church’): 


\ea%989
    \label{ex:key:989}
    \gll Wi  \textbf{kɔmɔ́t}  \textbf{bifó}    di  chɔ́ch.\\
\textsc{1pl}  go.out  before  \textsc{def}  church\\

\glt ‘We went away from the front side of the church.’ [dj05be 179]
\z

Sentence \REF{ex:key:990} features a change-of-location (denoted by the manner-of-motion verb \textit{fláy} ‘fly, rush’) in a motion-to (expressed through the V2 \textit{gó} ‘go’ of a motion-direction SVC) into the specific part (the interior location \textit{ínsay} ‘inside’) of the ground (the goal \textit{Ela Nguema}, a quarter of Malabo):


\ea%990
    \label{ex:key:990}
    \gll Chico,  a    wánt  \textbf{fláy}  \textbf{gó}  \textbf{ínsay}  {Ela  Nguema}  náw    só.\\
boy    \textsc{1sg.sbj}  want  fly  go  inside  \textsc{place}    now    like.that\\

\glt ‘Man, I’m about to rush to Ela Nguema right now.’ [dj07ae 360]
\z

Additional dimensions that may add to the complexity of a spatial relation are manner modifications to the clause, reciprocity and animacy. \is{animacy}For example, the idiomatic expression \textit{na} \textit{X hán}, literally ‘in X’s hand’ (where X is the possessor) encodes an animate\is{animacy} source as in the following example (cf. \sectref{sec:7.6.4} for the use of this idiom in possessive clauses):


\ea%991
    \label{ex:key:991}
    \gll Dɛn  púl    di  motó  \textbf{na}  \textbf{in}    \textbf{hán}.\\
\textsc{3pl}  remove  \textsc{def}  car    \textsc{loc}  \textsc{3sg.poss}  hand\\

\glt ‘They seized the car from him.’ [to07fn 206]
\z

The locative noun\is{locative nouns} \textit{nía} ‘near, next to’ expresses various degrees of proximity to the ground including contact with it. Compare the use of \textit{nía} with the verb of adhesion \textit{jám} ‘be in/make contact with’ in \REF{ex:key:992}. \textit{Nía}, as well as \textit{kɔ́na} ‘next to’ are also used to express a reciprocal\is{reciprocity} spatial relation, in which figure and ground are ground and figure to each other \REF{ex:key:993}: 


\ea%992
    \label{ex:key:992}
    \gll E    \textbf{jám}=an        \textbf{nía}    wán    stík  wé  e    tínap.\\
\textsc{3sg.sbj}  make.contact=\textsc{3sg.obj}  near    one    tree  \textsc{sub}  \textsc{3sg.sbj}  stand\\

\glt ‘He placed it next to [and in contact with] a tree that’s standing.’ [li07pe 050]
\z


\ea%993
    \label{ex:key:993}
    \gll Dɛn  \textbf{sidɔ́n}  \textbf{nía}    dɛn  sɛ́f.\\
\textsc{3pl}  sit    near    \textsc{3pl}  self\\

\glt ‘They’re sitting next to each other.’ [dj07re 028]
\z

Clauses which express spatial relations can be modified further for manner independently of the meaning of the verb. This may be done through adverbial clauses introduced by \textit{sé} ‘\textsc{quot}’ (cf. \sectref{sec:10.7.2}), \textit{wé} ‘\textsc{sub}’ (cf. \sectref{sec:10.7.1}), or secondary predication (cf. \sectref{sec:11.3}).


The sentence \REF{ex:key:994} exhibits a complex spatial relation featuring the figure \textit{e} ‘\textsc{3sg.sbj}’ that has carried out a motion-past (i.e. \textit{pás} ‘pass by’) the proximity (i.e. \textit{kɔ́na} ‘next to’) of the ground \textit{chía} ‘chair’. The clause is followed by the secondary predicate \textit{dé wáka} ‘\textsc{ipfv} walk’ which provides information about the manner of movement. The secondary predicate is in turn modified by the compound adverbial \textit{rɔn-sáy} ‘backwards’:\is{secondary predicates}



\ea%994
    \label{ex:key:994}
    \gll E    pás \textbf{kɔ́na} chía de \textbf{wáka}  \textbf{rɔn-sáy}.\\
\textsc{3sg.sbj}  pass    next.to  chair  \textsc{ipfv}  walk  wrong.\textsc{cpd}{}-side\\

\glt ‘She passed by next to (a) chair walking backwards.’ [au07se 051]\is{adverbial phrases}
\z

\section{Temporal relations}\label{sec:8.2}

The clause-internal temporal relations of location in time, duration, and iteration are established through adverb(ial)s, quantifier\is{quantifiers}s, prepositions, and lexicalised phrases featuring verbs. The expression of standard time units is characterised by a high incidence of conventionalised codemixing.

\subsection{Standard time units}

In Pichi, the two equal halves of the day are split into \textit{dé} ‘day’ and \textit{nɛ́t} ‘night’. The conventionalised associative constructions \textit{mɔ́nin tɛ́n} ‘morning time’ = ‘morning’, \textit{sán tɛ́n} ‘sun time’ = ‘midday, noon’ \REF{ex:key:995}, \textit{ívin tɛ́n} ‘evening time’ = ‘afternoon, evening’, \textit{míndul nɛ́t} ‘middle night’ = ‘midnight’ \REF{ex:key:996} denote the central points of the twenty-four hour day:{\fff}


\ea%995
    \label{ex:key:995}
    \gll E    kán    \textbf{sán}  \textbf{tɛ́n}.\\
\textsc{3sg.sbj}  come  sun  time\\

\glt ‘She came (at) noon/in the afternoon.’ [dj05ce 050]
\z


\ea%996
    \label{ex:key:996}
    \gll E    kán    \textbf{míndul}  \textbf{nɛ́t}.\\
\textsc{3sg.sbj}  come  middle  night\\

\glt ‘He came (at) midnight.’ [dj05ce 053]
\z

The expression \textit{áftanun} ‘afternoon’ is occasionally heard in the speech of Group 2 speakers (cf. \sectref{sec:1.3}) in the greeting formula \textit{gúd áftanun} ‘good afternoon’. However this word is not usually employed to denote the corresponding period of the day.


The concept ‘dawn’ may be expressed by means of paraphrase, i.e. via emphatic repetition of the modifier noun \textit{mɔ́nin} ‘morning’ as in \REF{ex:key:997} or the use of another emphatic element (here the quantifier\is{quantifiers} \textit{sósó} ‘only’), with or without repetition for emphasis \REF{ex:key:998}:



\ea%997
    \label{ex:key:997}
    \gll Tumɔ́ro    \textbf{mɔ́nin}  \textbf{mɔ́nin}  \textbf{tɛ́n}    lɛk  háw  yu  gráp,
bifó    yu  nɔ́ba  chɔ́p.\\
tomorrow  morning  \textsc{rep}    time    like  how  \textsc{2sg}  get.up 
before  \textsc{2sg}  \textsc{neg}.\textsc{prf}  eat\\
\glt ‘Tomorrow very early in the morning, as soon as you get up, 
before you have eaten.’ [ro05ee 144]
\z


\ea%998
    \label{ex:key:998}
    \gll \op...\cp{},  dís  \textbf{sósó}    \textbf{mɔ́nin}  \textbf{tɛ́n},  dís  sósó    sósó    mɔ́nin  tɛ́n.\\
{}  this  only    morning  time  this  only    \textsc{rep}    morning  time\\

\glt ‘(...) early this morning, very early this morning.’ [ye05ce 048]
\z

An additional way of expressing ‘dawn’ is through a clause featuring the subject \textit{mɔ́nin} and the verb \textit{brék} ‘(to) dawn’ \REF{ex:key:999}, or simply, by way of the Spanish noun \textit{madrugada} ‘dawn’\is{loan words}:


\ea%999
    \label{ex:key:999}
    \gll E    kán    wé  di  \textbf{mɔ́nin}  de  \textbf{brék}.\\
\textsc{3sg.sbj}  come  \textsc{sub}  \textsc{def}  morning  \textsc{ipfv}  dawn\\

\glt ‘He came while morning was breaking.’ [dj05ce 049]
\z


\ea%1000
    \label{ex:key:1000}
    \gll E    kán    \textbf{madrugada}.\\
\textsc{3sg.sbj}  come  dawn\\

\glt ‘She came (at) dawn.’ [dj05ce 050]
\z

When telling the time of day, Spanish lexical items are fit into a conventionalised codemixed construction, which does not have an exact equivalent in Spanish (cf. also \sectref{sec:13.3.1} on codemixing). There is no other generally accepted way of telling the time:


\ea%1001
    \label{ex:key:1001}
    \gll So  yu  wánt  dé    dé    las    cuatro,  wi  dɔ́n  \textbf{dé} 
\textbf{las}    \textbf{tres}    \textbf{y}  \textbf{veinte}.\\
so  \textsc{2sg}  want  \textsc{be.loc}  there  the.\textsc{pl}  four    \textsc{1pl}  \textsc{prf}  \textsc{be.loc}
the.\textsc{pl}  three  and  twenty\\

\glt ‘So you want to be there at four (and) we’re already here
at three twenty. [ma03ni 005]
\z

The Pichi day names \textit{mɔ́nde} ‘Monday’, \textit{tyúsde} ‘Tuesday’, \textit{wɛ́nsde} ‘Wednesday’, \textit{tɔ́sde} ‘Thursday’, \textit{fráyde} ‘Friday’, \textit{satidé} ‘Saturday’, and \textit{sɔ́nde} ‘Sunday’ are (falling) out of use. Instead, the vast majority of speakers employ the corresponding \ili{Spanish} day names \textit{lunes}, \textit{martes}, \textit{miércoles}, \textit{jueves}, \textit{viernes}, \textit{sábado}, and \textit{domingo} at all times. The codemixed sentences in \REF{ex:key:1002} reflect typical usage. 


\ea%1002
    \label{ex:key:1002}
\ea{
    \gll
\'{U}s=dé  yu  de  gó,  \textbf{viernes}?\\
  \textsc{q}=day  \textsc{2sg}  \textsc{ipfv}  go  Friday\\

\glt   ‘Which day are you going, (on) Friday?’ [fr07se 166]
}\ex{
\gll
Una    gó  na  di  sén    avión,  \textbf{sábado}!\\
  \textsc{2pl}    go  \textsc{loc}  \textsc{def}  same  plane  Saturday\\

\glt   ‘Go [\textsc{pl}] in the same plane, (on) Saturday!’ [fr07se 167]
}\z\z

The \ili{Spanish} noun phrase \textit{fin de semana} is also usually recruited to express ‘weekend’:


\ea%1003
    \label{ex:key:1003}
    \gll A    go  lɛ́f    na  Lubá  sóté    \textbf{fin de semana}.\\
\textsc{1sg.sbj}  \textsc{pot}  remain  \textsc{loc}  \textsc{place}  until  weekend\\

\glt ‘I’ll remain in Luba until the weekend.’ [ye05ce 010]
\z

The following Spanish designations for the months of the year are in use: \textit{enero} ‘January’, \textit{febrero} ‘February’, \textit{marzo} ‘March’, \textit{abril} ‘April’, \textit{mayo} ‘May’, \textit{junio} ‘June’, \textit{julio} ‘July’, \textit{agosto} ‘August’, \textit{septiembre} ‘September’, \textit{octubre} ‘October’, \textit{noviembre} ‘November’, \textit{diciembre} ‘December’. Hence, dates are also exclusively expressed in codemixed structures like the following one: \is{loan words}


\ea%1004
    \label{ex:key:1004}
    \gll \textbf{El}  \textbf{diez}  \textbf{de}  \textbf{agosto},  bay  gɔ́d  in    páwa,  a    go  pás  na  yá.\\
the  ten  of  August  by  God  \textsc{3sg.poss}  power  \textsc{1sg.sbj}  \textsc{pot}  pass  \textsc{loc}  here\\

\glt ‘(On) the tenth of August, by the grace of God, I’ll pass by this place.’ [ab07fn 113]
\z

The two seasons of the year may be designated by the compounds{\fff} \textit{ren-sísin} \{rain.\textsc{cpd}-season\} ‘rainy season’ \REF{ex:key:1005} and \textit{dray-sísin} \{dry.\textsc{cpd}-season\} ‘dry season’. An alternative designation for the rainy season is the phrasal expression \textit{tɛ́n fɔ rén} \REF{ex:key:1006}: 


\ea%1005
    \label{ex:key:1005}
    \gll Dís  dé  dɛn  \textbf{ren-sísin}    go  bigín.\\
this  day  \textsc{pl}  rain.\textsc{cpd}{}-season  \textsc{pot}  begin\\

\glt ‘These days, the rainy season should begin.’ [dj05ce 059]
\z


\ea%1006
    \label{ex:key:1006}
    \gll Wi  dé    \textbf{tɛ́n}    \textbf{fɔ}  \textbf{rén}.\\
\textsc{1pl}  \textsc{be.loc}  time    \textsc{prep}  rain\\

\glt ‘We’re in the rainy season.’ [ro05ee 116]
\z

The noun \textit{amatán} stands for ‘harmattan’, the dry and dusty seasonal weather condition throughout West Africa (between November and March): 


\ea%1007
    \label{ex:key:1007}
    \gll Wí    de  kɔ́l  yá    só    \textbf{amatán}    dán,    lɛk  sé 
e    kin  dé    lɛkɛ  niebla.\\
\textsc{1pl.indp}  \textsc{ipfv}  call  here    like.that  harmattan  that    like  \textsc{quot}
\textsc{3sg.sbj}  \textsc{hab}  \textsc{be.loc}  like  fog\\

\glt ‘Here, we call harmattan that, like it’s usually like fog.’ [ye05ce 062]\is{noun phrase adverbials}
\z

\subsection{Temporal deixis}\label{sec:8.2.2}

Adverb(ial)s, quantifier\is{quantifiers}s, prepositions, and lexicalised phrases featuring verbs are recruited for the expression of temporal deixis within the clause. These means are summarised in \tabref{tab:key:8.6} below with respect to the temporal relations of location, duration, and iteration. \is{adverbs}


In \tabref{tab:key:8.6}, the letter “X” stands for a compatible time-unit, like \textit{tɛ́n} ‘time’, \textit{lunes} ‘Monday’, \textit{tú dé} ‘two days’, \textit{wán wík} ‘one week’, \textit{tú mún} ‘moon, month’, or \textit{wán hía} ‘year’. Optional elements are in parentheses. There is considerable flexiblity with regard to TMA marking, the expression of participants, and the use of prepositions or locative nouns{\fff} in the phrasal expressions in the column entitled “temporal expressions”. Therefore, I limit myself to including the most common alternative in the table, and only provide a free translation. Exact glosses of these phrases can be found in the examples further below.


%%please move \begin{table} just above \begin{tabular
\begin{table}
\caption{Temporal deixis}
\label{tab:key:8.6}

\begin{tabularx}{\textwidth}{lXX}
\lsptoprule

Temporal relation & \multicolumn{2}{c}{Temporal expressions}\\
Location &  & \\
\midrule 
  Future & \itshape tumɔ́ro/tumára & ‘tomorrow’\\
& \itshape ápás tumɔ́ro/tumára & ‘the day after tomorrow’\\
& \itshape nɛ́ks X & ‘next X’\\
& \itshape ínsay X & ‘in X’\\
& \itshape X wé e de kán & ‘coming X’\\

\tablevspace
Present & \itshape náw (só) & ‘(right) now’\\
& \itshape tidé/tudé & ‘today’\\

\tablevspace
Past & \itshape yɛ́stadé & ‘yesterday’\\
& \itshape ápás yɛ́stadé & ‘the day before yesterday’\\
& \itshape lás X & ‘last X’\\
& \itshape las-nɛ́t & ‘last night’\\
& \itshape ínsay X & ‘in X’\\
& \itshape lɔ́n tɛ́n & ‘long ago’\\
& \itshape (lás) X wé pás (bihɛ́n) & ‘X ago’\\
& \itshape (wé) X fínis & ‘at the end of X’\\
& \itshape (wé) X dɔ́n & ‘at the end of X’\\

\tablevspace
Anterior & \itshape bifó X & ‘before X’\\
& \itshape ápás X & ‘before X’\\

\tablevspace
Posterior & \itshape ápás X & ‘after X’\\

\tablevspace
Duration & \itshape fɔ X & ‘for X’\\
& \itshape síns X & ‘since X’\\
& \itshape frɔn X & ‘since X’\\
& \itshape sóté X & ‘until X\\
& \itshape frɔn X sóté X & ‘from X to X’\\
& \itshape bɔkú tɛ́n & ‘for a long time’\\
& \itshape pás bɔkú tɛ́n wé/sé X & ‘be a long time’ ‘since X’\\
& \itshape kɛ́r X & ‘(to) last X, stay for X’\\
& \itshape sté (fɔ) X & ‘stay for X’\\
& \itshape sté (wé) & ‘be a long time (that)’\\
& \itshape sté + V2 & ‘be a long time since V2’\\

\tablevspace
Iteration & \itshape ɛ́ni X & ‘every X’\\
\lspbottomrule
\end{tabularx}
\end{table}
A relation between event time and a point of reference in the present, future, and past can be established by combining an element from \tabref{tab:key:8.6} with absolute time reference (i.e. time points like \textit{las dos} ‘two o’clock’ and \textit{sán tɛ́n} ‘(after)noon’ or calendric units like \textit{viernes} ‘Friday’) with the appropriate TMA marking. Compare \REF{ex:key:995}, \REF{ex:key:1001}, and \REF{ex:key:1002} above. 


Some items lexically incorporate time reference to the present, past, or future. Compare the time adverb \textit{náw} ‘now’ \REF{ex:key:1008} and the temporal nouns \textit{tidé/tudé} ‘today’ \REF{ex:key:1009}. Note that the reference point of \textit{tidé/tudé} is event time, not absolute time. Hence \textit{tidé} in \REF{ex:key:1009} may refer to ‘today’, the actual day on which the sentence was uttered, or to ‘that day’, the day on which speaker (ye) conversed with the subject \textit{e} ‘\textsc{3sg.sbj}’:



\ea%1008
    \label{ex:key:1008}
    \gll \textbf{Náw}    a    dɔ́n  sí  di  tín    wé  yu  níd.\\
now    \textsc{1sg.sbj}  \textsc{prf}  see  \textsc{def}  thing  \textsc{sub}  \textsc{2sg}  need\\

\glt ‘Now I’ve seen what you need.’ [au07se 003]
\z


\ea%1009
    \label{ex:key:1009}
    \gll E    sé    \textbf{ɔ́l}  \textbf{tidé}    e    bin  de  kɔ́l  yú,
yu  nó  ték    teléfono.\\
\textsc{3sg.sbj}  \textsc{quot}    all  today  \textsc{3sg.sbj}  \textsc{pst}  \textsc{ipfv}  call  \textsc{2sg.indp}
\textsc{2sg}  \textsc{neg} take    telephone\\

\glt ‘He said the whole of today [that day], he had been calling you (and) you didn’t 
pick up the telephone.’ [ye03cd 021]
\z

The equally synonymous temporal nouns \textit{tumɔ́ro/tumára} ‘tomorrow’ incorporate future reference to a day ahead of event time \REF{ex:key:1010}. When \textit{tumɔ́ro} is combined with the temporal preposition \textit{ápás} ‘after’, the resulting collocation means ‘the day after tomorrow’ and denotes a point of reference two days into the future ahead of event time \REF{ex:key:1011}:


\ea%1010
    \label{ex:key:1010}
    \gll \textbf{Tumára}    a    go  sí  mi    mamá.\\
tomorrow  \textsc{1sg.sbj}  \textsc{pot}  see  \textsc{1sg.poss}  mother\\

\glt ‘Tomorrow, I’ll see my mother.’ [dj05ce 045] 
\z


\ea%1011
    \label{ex:key:1011}
    \gll \textbf{\'{A}pás}  \textbf{tumɔ́ro}    a    go  sí  mi    mamá.\\
after  tomorrow  \textsc{1sg.sbj}  \textsc{pot}  see  \textsc{1sg.poss}  mother\\

\glt ‘The day-after-tomorrow, I’ll see my mother.’ [ye05ce 046]
\z

The temporal noun \textit{yɛ́stadé} ‘yesterday’ relates event time to a reference point one day back into the past \REF{ex:key:1012}. The temporal preposition \textit{ápás} ‘after’ also combines with \textit{yɛ́stadé} ‘yesterday’ in the collocation \textit{ápás yɛ́stadé} ‘the day before yesterday’ \REF{ex:key:1013}: 


\ea%1012
    \label{ex:key:1012}
    \gll \textbf{Yɛ́stadé}    a    sí  mi    mamá.\\
yesterday  \textsc{1sg.sbj}  see  \textsc{1sg.poss}  mother\\

\glt ‘Yesterday, I saw my mother.’ [dj05ce 033]
\z


\ea%1013
    \label{ex:key:1013}
    \gll \textbf{\'{A}pás}  \textbf{yɛ́stadé}    a    sí  mi    mamá.\\
after  yesterday  \textsc{1sg.sbj}  see  \textsc{1sg.poss}  mother\\

\glt ‘The day before yesterday, I saw my mother.’ [dj05ce 043]
\z

The temporal nouns \textit{tumɔ́ro/tumára} and \textit{yɛ́stadé} express relative time reference in the same way as \textit{tidé/tudé} above. Depending on context, they may therefore also be translated as ‘one day after event time’ and ‘one day before event time’, respectively. Examples \REF{ex:key:1011} and \REF{ex:key:1013} above also show that the preposition \textit{ápás} ‘after’ may be used to indicate both a posterior and an anterior temporal relation. \textit{\'{A}pás} may therefore be combined with \textit{tumɔ́ro} ‘tomorrow’ as well as \textit{yɛ́stadé} ‘yesterday’. The “spatial frame of reference” \citep[24]{Levinson2003} of temporal posteriority is characterised by a mirror-like “reflection” (\citealt{BenderBannardo2005}:222) of the speaker’s vantage point into both directions of the time stream.{\fff}


Temporal deixis involving time units other than two days in either direction from event time is achieved through a variety of means. The quantifier{\fff} \textit{nɛ́ks} ‘next’ may modify the Pichi nouns \textit{wík} ‘week’, \textit{mún} ‘month’, and \textit{hía} ‘year’ and thereby remove the reference point from event time into the future by one unit. Compare \REF{ex:key:1014} and also note the use of the spatial and temporal preposition \textit{sóté} ‘until, up to’ which expresses extent: 



\ea%1014
    \label{ex:key:1014}
    \gll A    de  lɛ́f    na  Lubá  \textbf{sóté}    di  nɛ́ks  wík.\\
\textsc{1sg.sbj}  \textsc{ipfv}  leave  \textsc{loc}  \textsc{place}  until  \textsc{def}  next    week\\

\glt ‘I’m remaining in Luba until the next week.’ [ye05ce 014]
\z

The quantifier \textit{lás} ‘last’ mirrors the time reference of \textit{nɛ́ks} ‘next’. \textit{Lás} ‘last’ pushes a reference point into the past by one unit from event time as in (\ref{ex:key:1015}–\ref{ex:key:1016}). Note the presence of the definite article \textit{di} ‘\textsc{def}’ in \REF{ex:key:1015}: {\fff}


\ea%1015
    \label{ex:key:1015}
    \gll Boyé  kɔmɔ́t  na  tɔ́n    \textbf{di}  \textbf{lás} \textbf{mún}.\\
\textsc{name}  go.out  \textsc{loc}  town  \textsc{def}  last  month\\

\glt ‘Boyé left town last month.’ [dj05ce 027]
\z


\ea%1016
    \label{ex:key:1016}
    \gll Ɛf  e    nó  bin  gó  \textbf{lás} \textbf{hía},    e    bin  fɔ    dé
wet    wí    na  yá    só.\\
if  \textsc{3sg.sbj}  \textsc{neg}  \textsc{pst}  go  last  year    \textsc{3sg.sbj}  \textsc{pst}  \textsc{cond}    \textsc{be.loc}
with    \textsc{1pl.indp}  \textsc{loc}  here    like.that\\

\glt ‘If she hadn’t gone last year, she’d be with us right here.’ [dj05ae 059]
\z

In Pichi, the expression of punctual location in time does not require the use of a locative preposition{\fff} or locative noun{\fff} (e.g. \textit{na} ‘\textsc{loc}’, \textit{fɔ} ‘\textsc{prep}’, or \textit{ínsay} ‘inside’) if the temporal expression is inherently time deictic. This is the case in various examples throughout this section featuring relational items like \textit{nɛ́ks} ‘next’ and \textit{tumára} ‘tomorrow’ above or \textit{lás} ‘last’. 


The collocation \textit{lɔ́n tɛ́n} ‘long time ago’ is also inherently relational. Rather than expressing duration (i.e. *for a long time), its meaning includes an unspecified reference point in the past:



\ea%1017
    \label{ex:key:1017}
    \gll E    bin  dɔ́n  pás  \textbf{lɔ́n}  \textbf{tɛ́n},    nóto  \textbf{lɔ́n}    \textbf{lɔ́n}  \textbf{tɛ́n}.\\
\textsc{3sg.sbj}  \textsc{ipfv}  \textsc{prf}  pass  long  time    \textsc{neg}.\textsc{foc}  long    \textsc{red}  time\\

\glt ‘It happened long ago, not very long ago.’ [ma03sh 001]
\z

The collocation \textit{las-nɛ́t} ‘last night’ is a compound \REF{ex:key:1018}. The lexicalisation of this collocation distinguishes it from other time expressions featuring \textit{lás} ‘last’ (cf. e.g. \textit{lás hía} ‘last year’ in \ref{ex:key:1016}), which are not usually subjected to the tonal derivation characteristic of compounding{\fff}: 


\ea%1018
    \label{ex:key:1018}
    \gll \textbf{Las-nɛ́t}    a    chakrá  mi    sɛ́ns.\\
last.\textsc{cpd}{}-night  \textsc{1sg.sbj}  destroy  \textsc{1sg.poss}  brain\\

\glt ‘Last night, I drank myself senseless.’ [ra07fn 060]
\z

Spatial expressions are, however, used to encode temporal relations if the temporal expression in the clause is not inherently time deictic. This may apply to temporal location as in \REF{ex:key:1019}, where the locative noun\textit{ ínsay} ‘inside’ fulfills this function. 


\ea%1019
    \label{ex:key:1019}
    \gll A    de  wét  sé    mék  a    gó  \textbf{ínsay}  \textbf{tú}  \textbf{dé}.\\
\textsc{1sg.sbj}  \textsc{ipfv}  wait  \textsc{quot}    \textsc{sbjv}  \textsc{1sg.sbj}  go  inside  two  day\\

\glt ‘I’m hóping to go in two days.’ [dj05ae 033]
\z

Neither the associative preposition{\fff} \textit{fɔ} ‘\textsc{prep}’, nor the general locative preposition{\fff} \textit{na} ‘\textsc{loc}’ are generally employed to mark adverbial phrases with a location-in-time sense. An exception in the data is the presence of \textit{na} ‘\textsc{loc}’ in the lexicalised collocation \textit{na nɛ́t} ‘at night’{\fff}. All other standard periods of the day are expressed through associative constructions featuring the generic noun{\fff} \textit{tɛ́n} ‘time’ \REF{ex:key:1020}. In view of the limited number of \textit{tɛ́n} ‘time’ collocations in Pichi and their often idiosyncratic meanings (cf. \ref{ex:key:103}), even these expressions may be seen as lexicalised structures:{\fff}


\ea%1020
    \label{ex:key:1020}
    \gll Bɔ́y  dɛn  dé    dé,    \textbf{mɔ́nin}  \textbf{tɛ́n}    \textbf{sán}  \textbf{tɛ́n}    \textbf{na}  \textbf{nɛ́t},  
na  Píchi  dɛn  de  tɔ́k  Píchi.\\
boy  \textsc{pl}  \textsc{be.loc}  there  morning  time    sun  time    \textsc{loc}  night
\textsc{foc}  Pichi  \textsc{3pl}  \textsc{ipfv}  talk  Pichi\\

\glt ‘(The) guys are there, in the morning, at day time, at night, 
it’s only Pichi that they talk.’ [au07se 257]
\z

The extension of spatial notions into the temporal domain is also reflected in the means employed to encode the temporal relation of anteriority by means of the locative noun \textit{bifó} ‘before’. In contrast to \textit{ápás} ‘after’, which may express anteriority or posteriority, the use of \textit{bifó} in \REF{ex:key:1021} incorporates an “intrinsic” (\citealt[221]{BenderBannardo2005}) temporal perspective. The intrinsic beginning or end of the time unit itself provides the temporal reference point. Contrary to the “reflection” perspective inherent to \textit{ápás} ‘after’ \REF{ex:key:1022}, a relational linkage with the vantage point of the speaker is not expressed: 


\ea%1021
    \label{ex:key:1021}
    \gll Kofí    bin  dé    yá    só    \textbf{bifó}    lás  hía. \\
\textsc{name}  \textsc{pst}  \textsc{be.loc}  here    like.that  before  last  year\\

\glt ‘Kofi was here before last year [the year before last].’ [ro05ee 130]
\z


\ea%1022
    \label{ex:key:1022}
    \gll Dɛn  go  tɔ́n  bák    \textbf{ápás}  di  nɛ́ks    wík.\\
\textsc{3pl}  \textsc{pot}  turn  back  after  \textsc{def}  next    week\\

\glt ‘They’ll return the week after next [in two weeks].’ [he07fn 209]
\z

Duration in time for a specific period is expressed by means of the general associative preposition \textit{fɔ} ‘\textsc{prep}’ followed by a time expression: 


\ea%1023
    \label{ex:key:1023}
    \gll Yu  go  moja  di  rɛ́s  na  watá,  \textbf{fɔ}  \textbf{tidé},  \textbf{tú}  \textbf{dé} \op...\cp{}\\
\textsc{2sg}  \textsc{pot}  soak    \textsc{def}  rice  \textsc{loc}  water  \textsc{prep}  today  two  day\\

\glt ‘You soak the rice in water, for today [one day], (for) two days (...)’ [dj03do 019]
\z

An equally common way of expressing duration for a specified period is by means of the verb \textit{kɛ́r} ‘carry, take, last’. The “figure” enduring in time is expressed as the subject of the clause and may be inanimate \REF{ex:key:1024} or animate \REF{ex:key:1025}, while the specified time period is the object of \textit{kɛ́r}: 


\ea%1024
    \label{ex:key:1024}
    \gll \op...\cp{}  pero  di  fíba    bin   \textbf{kɛ́r}    wán    dé  dásɔl.\\
{}  but    \textsc{def}  fever  \textsc{pst}  carry  one    day  only\\

\glt ‘(...) but the fever only lasted for a day.’ [ru03wt 062]
\z


\ea%1025
    \label{ex:key:1025}
    \gll Háw  mɔ́ch  tɛ́n    yu  go   \textbf{kɛ́r}    na  kɔ́ntri?\\
how    much  time    \textsc{2sg}  \textsc{pot}  carry  \textsc{loc}  country\\

\glt ‘How long are you going to stay in (your) hometown?’ [lo07he 046]
\z

Aside from that, elements that express motion through space are put to use for establishing temporal relations of duration. Firstly, the allative motion{}-to preposition/clause linker \textit{sóté} ‘up to, until’ also expresses temporal duration-to \REF{ex:key:1026}. 


\ea%1026
    \label{ex:key:1026}
    \gll A    de  lɛ́f    na  Lubá  \textbf{sóté}    wík    fínis.\\
\textsc{1sg.sbj}  \textsc{ipfv}  remain  \textsc{loc}  \textsc{place}  until  week  finish\\

\glt ‘I’m staying in Luba until the end of the week.’ [ro05ee 128]
\z

Secondly, example \REF{ex:key:1027} and \REF{ex:key:1026} illustrate the use of \textit{sóté} together with the lexicalised (factative-marked) clausal structures \textit{mún dɔ́n} ‘month done’ = ‘at the end of the month’ and \textit{wík fínis} ‘week finish’ = ‘at the end of the week’. Both expressions establish a punctual and past temporal reference point:


\ea%1027
    \label{ex:key:1027}
    \gll Mék  e    wét     \textbf{sóté}     \textbf{mún}   \textbf{dɔ́n},    wé  a    gɛ́t
di  mɔní  a    go  báy  di  chɔ́p.\\
\textsc{sbjv}  \textsc{3sg.sbj}  wait    until  month  done  \textsc{sub}  \textsc{1sg.sbj}  get
\textsc{def}  money  \textsc{1sg.sbj}  \textsc{pot}  buy  \textsc{def}  food\\

\glt ‘Let him wait until the month is over, when I get the money, I’ll buy the food.’ [hi03cb 214]
\z

The multifunctional item \textit{sóté} ‘up to, until’ may also introduce finite adverbial extent clauses, in which the subordinate verb may take the full range of TMA and person marking \REF{ex:key:1028}. Next to that, \textit{sóté} also appears as a temporal preposition directly followed by a verb as in \REF{ex:key:1029}. The resulting combination acquires a resultative sense and means that the situation denoted by the verb has been attained. Since \textit{sóté} is also a preposition, it may also take nominal complements. For example, the complement\is{complements} \textit{táya} ‘be tired’ in \REF{ex:key:1029} is a non-finite, deverbal noun and appears without TMA or person marking: 


\ea%1028
    \label{ex:key:1028}
    \gll A    chɔ́p  frijoles  \textbf{sóté}    \textbf{a}    \textbf{táya}.\\
\textsc{1sg.sbj}  eat    bean.\textsc{pl}  until  \textsc{1sg.sbj}  be.tired\\

\glt ‘I ate beans until I was tired (of them).’ [ed03sp 121]
\z


\ea%1029
    \label{ex:key:1029}
    \gll A    chɔ́p  \textbf{sóté}    \textbf{táya}.\\
\textsc{1sg.sbj}  eat    until  be.tired\\

\glt ‘I ate to my full satisfaction.’ [dj07ae 523]\is{resultative constructions}
\z

The ablative preposition \textit{frɔn} ‘from, since’ marks a source when used with a spatial sense. In the temporal domain, \textit{frɔn} expresses duration-from a reference point \REF{ex:key:1030}. The period of duration may be further specified by employing both \textit{frɔn} ‘from’ and \textit{sóté} ‘until’ as in \REF{ex:key:1031}. I draw attention to the optional use of another lexicalised clausal structure in the second example, namely\textit{ e gó} \{\textsc{3sg.sbj} go\textsc{\}} ‘going to’ in order to provide an additional allative sense:


\ea%1030
    \label{ex:key:1030}
    \gll Dɛn  nó  nó    dɛn  sɛ́f   \textbf{frɔn}     \textbf{bɔkú}   \textbf{tɛ́n}.\\
\textsc{3pl}  \textsc{neg}  know  \textsc{3pl}  self  from  much  time\\

\glt ‘They don’t know each other for a long time.’ [ch07fn 210]
\z


\ea%1031
    \label{ex:key:1031}
    \gll \textbf{Frɔn}  las    doce,  \textbf{sóté}    e    \textbf{gó}  las    seis,
na  “gúd    ívin”.\\
from  the.\textsc{pl}  twelve  until  \textsc{3sg.sbj}  go  the.\textsc{pl}  six  
\textsc{foc}  good  evening\\

\glt ‘From twelve to six o’clock, its “good evening”.’ [ye07je 011]
\z

The temporal preposition \textit{síns} ‘since’ is specialised to expressing duration-from but its use is marginal when compared with the frequency of \textit{frɔn} ‘from’: 


\ea%1032
    \label{ex:key:1032}
    \gll Wi  dé    yá    \textbf{síns}    las    dos.\\
\textsc{1pl}  \textsc{be.loc}  here    since  the.\textsc{pl}  two\\

\glt ‘We’re here since two o’clock.’ [ab07fn 242]
\z

The transfer of spatial concepts into the temporal domain is also reflected in the kind of verbs employed. Location in the future features the ablative motion verb\is{motion verb} \textit{kán} ‘come’, that of past location and duration the motion verb \textit{pás} ‘pass (by)’ -- hence time is conceived as moving and the reference point as fixed:


\ea%1033
    \label{ex:key:1033}
    \gll A    de  lɛ́f    na  Lubá  sóté    di  wík
\textbf{wé}  \textbf{e}    \textbf{de}  \textbf{kán}.\\
\textsc{1sg.sbj}  \textsc{ipfv}  leave  \textsc{loc}  \textsc{place}  until  \textsc{def}  week
\textsc{sub}  \textsc{3sg.sbj}  \textsc{ipfv}  come\\

\glt ‘I’m remaining in Luba until the coming week.’ [dj05ce 015]
\z


\ea%1034
    \label{ex:key:1034}
    \gll Djunais  bin  lɛ́f    na  Lubá  sóté    di  wík
\textbf{wé}  \textbf{e}    \textbf{bin}  \textbf{pás}.\\
\textsc{name}  \textsc{pst}  leave  \textsc{loc}  \textsc{place}  until  \textsc{def}  week
\textsc{sub}  \textsc{3sg.sbj}  \textsc{pst}  pass\\

\glt ‘Djunais remained in Luba until last week.’ [dj05ce 016]
\z

The verb \textit{sté} ‘stay, be a long time’ inherently expresses lengthy duration, so no further specification of the length of the period is required \REF{ex:key:1035}. The verb is versatile in its syntactic behaviour. Firstly, it may appear as the only verb of a sentence like \REF{ex:key:1035} or participate as a V1 in an adverbial SVC (cf. \sectref{sec:11.2.5} for details):


\ea%1035
    \label{ex:key:1035}
    \gll Na  wán  hós    wé  e    dɔ́n  \textbf{sté}    nɔ́?\\
\textsc{foc}  one  house  \textsc{sub}  \textsc{3sg.sbj}  \textsc{prf}  remain  \textsc{intj}\\

\glt ‘It’s a house that’s been around for a long time, right?’ [dj05ae 161]
\z

Secondly, the verb \textit{sté} may also appear with an expletive subject followed by an adverbial time clause which specifies the relevant time period:


\ea%1036
    \label{ex:key:1036}
    \gll E    dɔ́n  \textbf{sté},  a    tínk    sé    e    dɔ́n  \textbf{sté}
wé  una  bin  gɛ́t  insecticida  yá.\\
\textsc{3sg.sbj}  \textsc{prf}  stay  \textsc{1sg.sbj}  think  \textsc{quot}    \textsc{3sg.sbj}  \textsc{prf}  stay
\textsc{sub}  \textsc{2pl}  \textsc{pst}  get  insecticide  here\\

\glt ‘It’s been long, I think it’s been long since you [\textsc{pl}] have had insecticide 
[sprayed] here.’ [fr03wt 059]
\z

The quantifier{\fff} \textit{ɛ́ni} ‘every’ expresses iteration of the time unit it refers to \REF{ex:key:1037}. Time units are generally conceived as countable, \textit{ɛ́ni} is semantically compatible with any time unit including units of the clock \REF{ex:key:1038}:


\ea%1037
    \label{ex:key:1037}
    \gll A    bin  de  chénch  húman  \textbf{ɛ́ni}    \textbf{síks}  \textbf{mún}.\\
\textsc{1sg.sbj}  \textsc{pst}  \textsc{ipfv}  change  woman  every  six  month\\

\glt ‘I was changing women every six months.’ [ed03sp 033]
\z


\ea%1038
    \label{ex:key:1038}
    \gll Bikɔs  ín    de  sé,    \textbf{ɛ́ni}    \textbf{las}    \textbf{doce}  na  ín
in    abuela    kin  kán    kɔ́l=an.\\
because  \textsc{3sg.indp}  \textsc{ipfv}  \textsc{quot}    every  the.\textsc{pl}  twelve  \textsc{foc}  \textsc{3sg.indp}
\textsc{3sg.poss}  grandmother  \textsc{hab}  come  call=\textsc{3sg.obj}\\

\glt ‘Because she [\textsc{emp}] would say, always at twelve o’clock, that’s when her 
grandmother used to come and call her.’ [ed03sb 150]
\z

\tabref{tab:key:8.7} contains all locative nouns\is{locative nouns} and prepositions that participate in expressing temporal relations in Pichi. The table complements the inventory of locative and non-locative prepositions presented in \tabref{tab:key:8.1} and \tabref{tab:key:9.1}, respectively.

%%please move \begin{table} just above \begin{tabular
\begin{table}
\caption{Temporal (uses of) prepositions and locative nouns}
\label{tab:key:8.7}

\begin{tabularx}{\textwidth}{lllQ}
\lsptoprule

Element & Temporal use & Temporal relation & Other semantic roles/uses\\
\midrule
\itshape ínsay & ‘inside’ & Location & Locative noun\\
\itshape bifó & ‘before’ & Location (anterior) & Locative noun; time clause linker\\
\itshape bihɛ́n & ‘after’ & Location (posterior) & Locative noun\\
\itshape ápás & ‘after’ & Location (posterior) & {}---\\
\itshape fɔ & ‘for’ & Duration & General associative preposition\\
\itshape frɔn & ‘since’ & Duration (from) & source (locative)\\
\itshape síns & ‘since’ & Duration (from) & \textit{síns wé}: time clause linker\\
\itshape sóté & ‘until’ & Duration (to) & Extent (locative); time clause linker\\
\lspbottomrule
\end{tabularx}
\end{table}
