\addchap{Symbols and abbreviations}
\label{sec:abbr}
\begin{tabularx}{.45\textwidth}{lQ}
 {}- & morpheme boundary \\
 = & clitic morpheme boundary \\
 ! & directive clause; vocative \\
 * & ungrammatical example \\
 , & continuative intonation and pause \\
 . & utterance-final: declarative intonation \\
 . & word-medial: morpheme boundary in \\
& derived compound \\
 (…) & untranscribed part of utterance \\
{}[ ] & explanation of translated elements \\
 / & speech interruption \\
 ? & final: question intonation \\
 ? & initial: grammaticality dubious \\
{}[á] & IPA transcription \\
 /a/ & phoneme \\
 <a> & grapheme \\
 á & high tone diacritic \\
 à & low tone diacritic \\
 \% & boundary tone \\
 1, 2, 3 & first, second, third person \\
 \textsc{abl} & abilitive mood marker \\
 \textsc{adv} & adverbial(ising suffix) \\
 \textsc{be} & identity copula \\
 \textsc{be.loc} & locative-existential copula \\
 \textsc{bt} & boundary tone \\
 \textsc{cpd} & tone deletion in compounding \\
\end{tabularx}
\begin{tabularx}{.45\textwidth}{lQ}
 \textsc{def} & definite article \\
 \textsc{emp} & emphatic \\
 \textsc{f} & feminine gender \\
 \textsc{fn} & first name \\
 \textsc{foc} & focus marker and identity copula \\
 \textsc{h} & high tone(d syllable) \\
 \textsc{hab} & habitual marker \\
 \textsc{ideo} & ideophone \\
 \textsc{indf} & indefinite \\
 \textsc{indp} & independent/emphatic pronoun \\
 \textsc{intj} & interjection \\
 \textsc{intr} & intransitive \\
\textsc{ipfv} & imperfective aspect marker \\
 \textsc{l} & low tone(d syllable) \\
 \textsc{l.h} & low-high tone sequence over two adjacent syllables \\
\textsc{lh} & rising contour tone over same syllable\\
\textsc{ln} & last name\\
\textsc{loc} & locative preposition\\
\textsc{lt} & lexical tone\\
\textsc{mvc} & multiverb construction\\
n.a. & not applicable\\
\textsc{name} & personal name\\
\textsc{neg} & negative/negator\\
\textsc{np} & noun phrase\\
\textsc{nspc} & non-specific\\
\textsc{obj} & object (case)\\
\textsc{obl} & obligative mood marker\\
\textsc{pfv} & narrative perfective marker\\
\textsc{pl} & plural(iser)\\
\end{tabularx}

\newpage 
\begin{tabularx}{.45\textwidth}{lQ}
\textsc{place} & place name\\
\textsc{poss} & possessive (case)\\
\textsc{pot} & potential mood marker\\
\textsc{pp} & prepositional phrase\\
\textsc{prep} & associative preposition\\
\textsc{prf} & perfect tense-aspect\\
\textsc{pst} & past tense marker\\
\textsc{q} & question particle\\
\textsc{qnt} & quantifier\\
\textsc{quot} & quotative marker\\
\textsc{red} & reduplicant in reduplication\\
\textsc{rep} & repeated word in repetition\\
\end{tabularx}
\begin{tabularx}{.45\textwidth}{lQ}
\textsc{sbj} & subject (case)\\
\textsc{sbjv} & subjunctive marker\\
\textsc{sg} & singular\\
\textsc{skt} & “suck teeth”\\
\textsc{sp} & sentence particle\\
\textsc{spec} & specific\\
\textsc{sub} & subordinator\\
\textsc{svc} & serial verb construction\\
\textsc{tma} & tense-mood-aspect\\
\textsc{tr} & transitive\\
\textsc{v1} & initial verb in \textsc{MVC}\\
\textsc{v2} & second verb in \textsc{MVC}\\
\textsc{vp} & verb phrase\\
\end{tabularx}