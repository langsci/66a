\chapter{The verbal system}

Pichi verbs fall into three lexical aspect classes. The verbal system of Pichi is characterised by the use of preverbal particles, which modify the verb for tense, aspect, and modality. These three grammatical categories are interlocked in various ways, which transpire best when larger stretches of discourse are analysed. The system also includes numerous aspectual and modal auxiliary constructions. Verbs, and those denoting properties in particular, may be modified for degree in comparative constructions.

\section{Lexical aspect}\label{sec:6.1}

Pich\textstyleannotationreference{i} verbs fall into three lexical aspect\is{lexical aspect} classes: stative, inchoative-stative, and dynamic. Most subclasses of inchoative-stative verbs may receive a stative or a dynamic interpretation in the right context, but the reverse is not the case, hence my use of the term “lexical” aspect. In this chapter and others, I employ “situation” as a cover term for events denoted by dynamic verbs as well as states denoted by (inchoative-)stative verbs and predicate adjectives. When a situation is construed as stative, it has no inherent boundaries, e.g. \textit{e dé} ‘\textsc{3sg.sbj} \textsc{be.loc}’ = ‘s/he/it exists’. 


When a situation is construed as inchoative-stative, it encompasses the entry-into-state (inchoative), as well as the ensuing state (stative), e.g. \textit{e chák ‘}\textsc{3sg.sbj} get.drunk’ = ‘he got drunk’. Since inchoative-stative verbs may also be read with a stative meaning, the preceding clause may also be translated as ‘he is drunk’. Situations denoted by dynamic verbs are conceived of as being bounded; they have an inherent beginning and end (\textit{wi chɔ́p} ‘\textsc{1pl} eat’ = ‘wé ate’) \citep{Sasse1991}.



I expressly avoid the terms telic (with an inherent endpoint) and atelic (without an inherent endpoint) (\citealt{Comrie1976}:44–51) in the description of lexical aspect. The telic–atelic distinction blurs the boundaries between lexical aspect (as part of the meaning of the verb), grammatical aspect (expressed e.g. in the perfective–imperfective opposition), and clausal aspect (expressed e.g. by clausal transitivity and temporal adverbs), and is therefore of limited usefulness in this regard. 



The inherent temporal structure of Pichi verbs co-determines the meanings that arise when aspect markers co-occur with a verb (cf. \sectref{sec:9.2.3} for further valency-related effects of lexical aspect). Therefore, I apply two distributional criteria for delineating the three lexical aspect classes: firstly, co-occurrence with the imperfective marker \textit{de} ‘\textsc{ipfv’} and secondly, co-occurrence with the aspectual/phasal verb \textit{bigín} ‘begin’ in an ingressive aspect \is{ingressive aspect}auxiliary construction \citep[8]{Sasse1991}. The latter criterion is particularly useful, because the imperfective marker \textit{de} ‘\textsc{ipfv’} optionally intervenes between \textit{bigín} and the following verb.



The corpus contains only a handful of verbs that can be classified as stative with sufficient certainty. These are listed in \tabref{tab:key:6.1} together with the semantic classes they belong to: 


%%please move \begin{table} just above \begin{tabular
\begin{table}
\caption{Stative verbs}
\label{tab:key:6.1}

\begin{tabularx}{.8\textwidth}{XXX}
\lsptoprule
Semantic class & Verbs & \\
\midrule
Modal & \itshape fít & ‘can’\\
& \itshape hébul & ‘be capable’\\
& \itshape lɛ́k & ‘like’\\
& \itshape mín & ‘mean (to)’\\
& \itshape níd & ‘need’\\
& \itshape wánt & ‘want’\\

\tablevspace
Existence & \itshape bí & \textsc{‘be’}\\
& \itshape dé & \textsc{‘be.loc’}\\
& \itshape blánt & ‘reside’\\
& \itshape fíba & ‘resemble; seem’\\

\tablevspace
Cognition & \itshape tínk & ‘think’\\
\lspbottomrule
\end{tabularx}
\end{table}
Stative verbs do not co-ocur with the imperfective marker \textit{de} ‘\textsc{ipfv’}. Secondly, they do not normally appear with the aspectual/phasal verb \textit{bigín} ‘begin (to)’. For most speakers, a clause like the following one is therefore ungrammatical: 


\ea[*]{%300
    \label{ex:key:300}
    \gll A    \textstylePichiexamplebold{bigín}   (de)     \textstylePichiexamplebold{hébul}    dú=an.\\
  \textsc{1sg.sbj}  begin  \textsc{ipfv}    be.capable  do=\textsc{3sg.obj}\\
\glt *I began to be capable of doing it. [to07fn 226]
}
\z

The two modal verbs lɛ́k ‘like’ and wánt ‘want’ are ambivalent in their lexical aspect. I suggest that wánt is ambivalent between a dynamic and a stative sense, while lɛ́k vacillates between a stative and an inchoative-stative sense. Most of the time, these two verbs do not co-ocur with de ‘ipfv’ in imperfective situations. They sometimes do, however, and they are also attested in phasal constructions involving bigín ‘begin’: 


\ea%301
    \label{ex:key:301}
    \gll Na  ín    a    bigín  de  lɛ́k=an.\\
\textsc{foc}  \textsc{3sg.indp}  \textsc{1sg.sbj}  begin  \textsc{ipfv}  like\textsc{=3sg.obj}\\

\glt ‘That’s when I began liking her.’ [he07fn 228]
\z

The class of inchoative-stative verbs includes three semantic classes that belong to the large group of labile verbs\is{labile verbs} (cf. \sectref{sec:9.2.3} for details): change-of-state verbs, locative verbs,\is{locative verbs} and property items. It also includes two verbs of possession\is{possession verbs}, two verbs of cognition, a verb of perception, and a verb denoting existence in time and space. The class of inchoative-stative verbs is therefore much larger than that of stative verbs, which only has a few members. 

In this, I concur with analyses that posit a similar distribution of lexical aspect classes in other Afro-Caribbean English-lexifier Creoles (e.g. \citealt{Winford1993}; \citealt{Migge2000}). \tabref{tab:key:6.2} below lists the relevant (groups of) verbs:

%%please move \begin{table} just above \begin{tabular
\begin{table}
\caption{Inchoative-stative verbs}
\label{tab:key:6.2}

\begin{tabularx}{.8\textwidth}{Xll}
\lsptoprule
Semantic class & Verbs & \\
\midrule
Change of state; & \multicolumn{2}{l}{Labile verbs}\\
Property items; & \\
Locative verbs & \\

\tablevspace
Possession & \itshape gɛ́t & ‘get; have’\\
& \itshape hól & ‘seize; keep’\\

\tablevspace
Cognition & \itshape sabí & ‘(get to) know’\\
& \itshape nó & ‘(get to) know’\\

\tablevspace
Perception & \itshape sí & ‘see; catch sight of’\\

\tablevspace
Existence & \itshape kɔmɔ́t & ‘come from; hail from’\\
\lspbottomrule
\end{tabularx}
\end{table}

All inchoative-stative verbs may potentially be interpreted as stative or inchoative in the absence of disambiguating information. This is for example the case when these verbs remain unmarked in basic intransitive clauses (cf. \sectref{sec:6.3.1}). However, such ambivalence between an ongoing state (stative) and an entry-into-state (inchoative) reading occurs with differing likelihood with the relevant semantic classes. 


Within the group of labile verbs, property items are far more likely to be interpreted as stative than inchoative when left unmarked in an intransitive clause. In contrast, most change-of-state verbs and locative verbs may receive a stative and an inchoative interpretation with equal likelihood (cf. \sectref{sec:9.2.3}). This also holds for inchoative-stative cognition, possession, and perception verbs. 



Inchoative-stative verbs are compatible with the imperfective marker\is{imperfective aspect} \textit{de} ‘\textsc{ipfv’} \REF{ex:key:302} The use of \textit{de} ‘\textsc{ipfv’} with these verbs renders an inchoative meaning, which is in the present tense\is{present tense} in relation to event time (cf. \sectref{sec:6.3.4} for details). Likewise, inchoative-stative verbs may combine with the verb \textit{bigín} ‘begin’. The resulting ingressive aspect\is{ingressive aspect} construction highlights the inchoative, entry-into-state meaning component of the verb \REF{ex:key:303}:



\ea%302
    \label{ex:key:302}
    \gll Dís  bɔ́y,    ɛ́ni    dé  e    de  fáyn  mɔ́-ɛn-mɔ́.\\
this  boy    every  day  \textsc{3sg.sbj}  \textsc{ipfv}  be.fine  more-and-more\\

\glt ‘This boy is getting more and more handsome every day.’ [ro05ee 046]
\z


\ea%303
    \label{ex:key:303}
    \gll Wi  bigín  de  nó    wi  sɛ́f.\\
\textsc{1pl}  begin  \textsc{ipfv}  know  \textsc{1pl}  self\\

\glt ‘We began to get to know each other.’ [ye07fn 019]
\z

The inchoative-stative posture verbs \textit{sidɔ́n} ‘sit (down)’, \textit{slíp} ‘lie down; sleep’ and \textit{tínap} ‘stand (up)’ may co-ocur with the imperfective marker without necessarily acquiring the usual inchoative sense. These verbs appear to vacillate in their lexical aspect between an inchoative-stative and a dynamic sense. Consider the use of \textit{slíp} ‘lie sleep’ as an inchoative-stative verb in \REF{ex:key:304} and as a dynamic verb in \REF{ex:key:305}:


\ea%304
    \label{ex:key:304}
    \gll Yu  de  respira,  yu  sɛ́ns    de  lɔ́s,  e    dé
lɛk    sé    yu  \textstylePichiexamplebold{slíp}.\\
\textsc{2sg}  \textsc{ipfv}  breathe  \textsc{2sg}  mind  \textsc{ipfv}  lose  \textsc{3sg.sbj}  \textsc{be.loc}
like    \textsc{quot}    \textsc{2sg}  sleep\\

\glt ‘You’re breathing, your mind is slipping away, it is as if you’re sleeping.’ [ed03sb 120]
\z


\ea%305
    \label{ex:key:305}
    \gll Di  dɔ́g  \textstylePichiexamplebold{de}  \textstylePichiexamplebold{slíp}  bɔtɔ́n  di  tébul.\\
\textsc{def}  dog  \textsc{ipfv}  slíp  under  \textsc{def}  table\\

\glt ‘The dog is sleeping/lying under the table.’ [ro05ee 072]
\z

The verb \textit{tínap} ‘stand (up)’ may also be used as a dynamic verb. However, it is then also usually employed with the different meaning of ‘begin to stand (of a toddler)’. Compare the following two uses of this posture verb: 


\ea%306
    \label{ex:key:306}
    \gll E    tínap  bihɛ́n  di  hós.\\
\textsc{3sg.sbj}  stand.up  behind  \textsc{def}  house\\

\glt ‘He’s standing behind the house.’ [ye0502e2 181]
\z


\ea%307
    \label{ex:key:307}
    \gll E    \textstylePichiexamplebold{de} \textstylePichiexamplebold{tínap},  smɔ́l  pikín  wé  e    de  tráy
fɔ    tínap  yet.\\
\textsc{3sg.sbj}  \textsc{ipfv}  stand.up  small  child  \textsc{sub}  \textsc{3sg.sbj}  \textsc{ipfv}  try
\textsc{prep}    stand.up  yet\\

\glt ‘He is beginning to stand, a small child that’s still trying to stand.’ [\is{posture verbs}dj0502e2 219]
\z

A semantic specialisation of the inchoative vs. the dynamic meanings of the verb is also present with the verb \textit{kɔmɔ́t}. When unmarked, it is left to context to disambiguate the meanings ‘come from’ (dynamic) and ‘hail from’ (inchoative-stative) from each other. This is illustrated in \REF{ex:key:308} and \REF{ex:key:309}, respectively:


\ea%308
    \label{ex:key:308}
    \gll Wi  \textstylePichiexamplebold{kɔmɔ́t}  dé,    wi  kán  gó  fɔ,  fɔ  Akebeville.\\
\textsc{1pl}  go.out  there  \textsc{1pl}  \textsc{pfv}  go  \textsc{prep}  \textsc{prep}  \textsc{place}\\

\glt ‘(When) we left there, we went to, to Akebeville.’ [ma03hm 039]
\z


\ea%309
    \label{ex:key:309}
    \gll \'{U}s=sáy  yu  \textstylePichiexamplebold{kɔmɔ́t}?\\
\textsc{q}=side  \textsc{2sg}  come.from\\

\glt ‘Where do you come from? \textsc{or} ‘Where did you exit [e.g. the market]?’ [dj050e3 167]
\z

A comparison of \REF{ex:key:309} and \REF{ex:key:310} shows that ambiguity does not arise once \textit{kɔmɔ́t} is marked for imperfective aspect:


\ea%310
    \label{ex:key:310}
    \gll Yu  \textstylePichiexamplebold{de} \textstylePichiexamplebold{kɔmɔ́t}    ús=sáy?\\
\textsc{2sg}  \textsc{ipfv}  come.out  \textsc{q}=side\\

\glt ‘Where are you coming from?’ [dj05ce 170]
\z


\ea%311
    \label{ex:key:311}
    \gll Mí    gɛ́t  dán  problema  wet    bɔ́y  dɛn  wé  dɛn  kɔmɔ́t
Bata    nɔ́,  sé    ‘no  Pichi  es  un  dialecto.’\\
\textsc{1sg.indp}  get  that  problem    with    boy  \textsc{pl}  \textsc{sub}  \textsc{3pl}  come.out
Bata    \textsc{intj}  \textsc{quot}    no  Pichi  it.is  a  dialect\\

\glt ‘I have that problem with guys who are from Bata, right, [they] say 
“no, Pichi is a dialect [not a language].”’ [au07se 219]
\z

The data contains a large number of dynamic verbs from a wide range of semantic classes. Dynamic verbs may appear freely with the imperfective marker \textit{de} ‘\textsc{ipfv’} \REF{ex:key:312} and in ingressive\is{ingressive aspect} auxiliary constructions featuring the aspectual/phasal verb \textit{bigín} ‘begin’ \REF{ex:key:313}. The use of the imperfective marker\is{imperfective aspect} renders a progressive or habitual aspect\is{habitual aspect} reading with dynamic verbs. Note that labile\is{labile verbs} inchoative-stative verbs may also be used as dynamic verbs in transitive clauses (cf. \sectref{sec:9.2.3} for further details):\is{imperfective aspect}


\ea%312
    \label{ex:key:312}
    \gll Dɛn  de  sláp  dɛn  sɛ́f.\\
\textsc{3pl}  \textsc{ipfv}  slap  \textsc{3pl}  self\\

\glt ‘They’re slapping each other.’ [dj07re 020]
\z


\ea%313
    \label{ex:key:313}
    \gll \MakeUppercase{A}   \textstylePichiexamplebold{bigín}  \textstylePichiexamplebold{gó}  skúl.\\
\textsc{1sg.sbj}  begin  go  school\\

\glt ‘I began going to school.’ [fr03ft 018]\is{lexical aspect}
\z

\section{The TMA system}\label{sec:6.2}

Pichi has a core and a non-core system of tense\is{tense}{}-mood-aspect (TMA) marking. The core system is constituted by TMA particles which express central TMA notions. These particles (henceforth TMA markers) may be combined with each other, share phonological characteristics, such as monosyllabicity, and form a unit with the verb between which only a small group of preverbal adverbs\is{preverbal adverbs} may intervene. In the non-core system, auxiliary verbs express aspectual and modal notions as minor verbs in serial verb constructions. Besides TMA markers and auxiliary verbs, Pichi also makes use of complementisers in order to express modality. 


The markers of the core TMA system and their linear order relative to each other and the verb root are provided in the following figure. The figure shows that all TMA markers are found to the left of the root. The modal complementiser \textit{mék} ‘\textsc{sbjv}’ is the only TMA marker found to the left of the dependent subject pronoun in a position occupied by clause linkers. It should also be borne in mind that factative TMA is achieved via the bare, unmarked verb, hence involves no overt marker: 


\begin{figure}
\caption{Ordering of TMA markers\is{predicate structure}}
\label{fig:key:6.1}
{\footnotesize\begin{tabularx}{\textwidth}{XXXX QQQ XX}
\lsptoprule

Mood & Pronoun & Negation & Tense & Mood & Aspect & Aspect & Stem & Root\\
\midrule
{\itshape mék}

\textstyleTableEnglishZchn{\textsc{sbjv}} &  \textit{yu}

\textsc{2sg} & {\itshape nó}

\textsc{neg} & {\itshape bin}

\textsc{pst} & {\itshape go}

\textsc{pot}

{\itshape fɔ}

\textsc{obl/cond}

{\itshape mɔs}

\textsc{obl} & {\itshape dɔ́n}

\textsc{prf}

{\itshape nɛ́a}

\textsc{neg}.\textsc{prf}

{\itshape kin}

\textsc{hab/abl} & {\itshape \textit{de}}

\textsc{ipfv}

{\itshape kán}

\textsc{pfv} & \textsc{red}{}- & verb\\
\lspbottomrule
\end{tabularx}}
\end{figure}

The markers that express the two basic aspect categories of imperfective (i.e. \textit{de} ‘\textsc{ipfv}’) and narrative perfective (i.e. \textit{kán} ‘\textsc{pfv}’) are closest to the verb root. The marker \textit{kin} ‘\textsc{hab;} \textsc{abl}’ has the same position when used in its habitual\is{habitual aspect} function as with its (marginal) function as a modality marker of ability. The same holds for \textit{fɔ} when it instantiates conditional or obligative mood. When it occurs with the abilitive function it is glossed as ‘\textsc{abl’}. There are co-occurrence restrictions for the expression of composite TMA categories (cf. \figref{fig:key:6.2}). 

%%please move \begin{table} just above \begin{tabular

\tabref{tab:key:6.3} presents the focal functions of TMA categories that are expressed when markers occur on their own. Factative TMA is included under all relevant categories in recognition of the multiple functions the unmarked verb plays in the TMA system. Factative TMA is indicated by a dash ({}---) in the column headed by “Marker”.

%%please move \begin{table} just above \begin{tabular
\begin{table}
\caption{Functions of TMA markers}
\label{tab:key:6.3}

\begin{tabularx}{\textwidth}{llX}
\lsptoprule

Category & Marker & Function\\
\midrule
Tense\is{tense} & \textit{{}---} {{(factative TMA)}} & Past \\
& \textit{bin} ‘\textstyleTableEnglishZchn{\textsc{pst’}} & Past \\

\tablevspace
Mood & \textit{{}---} {{(factative TMA)}} & Realis\\
& \textit{go} ‘\textstyleTableEnglishZchn{\textsc{pot’}} & Potential \\
& \textit{mék} \textstyleTableEnglishZchn{\textsc{‘sbjv’}} & Subjunctive; complementiser \\
& \textit{fɔ} ‘\textstyleTableEnglishZchn{\textsc{prep;} \textsc{cond’}} & Obligative, complementiser; conditional\\
& \textit{mɔs} ‘\textstyleTableEnglishZchn{\textsc{obl’}} & Obligative \\

\tablevspace
Aspect & {}--- {{(factative TMA)}} & Perfective\\
& \textit{kán} \textstyleTableEnglishZchn{\textsc{‘pfv’}} & Narrative perfective \\
& \textit{de} \textstyleTableEnglishZchn{\textsc{‘ipfv’}} & Imperfective \is{imperfective aspect}\\
& \textit{dɔ́n} \textstyleTableEnglishZchn{\textsc{‘prf’}} & Perfect\\
& \textit{kin} \textstyleTableEnglishZchn{\textsc{‘hab’}} & Habitual \\

\tablevspace
& Reduplication \textsc{‘red’} & Iterative\\
\lspbottomrule
\end{tabularx}
\end{table}

Combinations of the TMA markers listed above may render composite TMA categories. All attested combinations are listed in \figref{fig:key:6.2} below. TMA markers follow the linear order established in \figref{fig:key:6.1} where possible. Crossferences to examples featuring uses of composite categories are provided in the first column.

\begin{sidewaysfigure}
\caption{Composite TMA categories}

\label{fig:key:6.2}
\begin{tabularx}{\textwidth}{lllllllllll}
\lsptoprule
Ex. & \textsc{sbjv} & \textsc{pst} & \textsc{pot} & \textsc{cond} & \textsc{prf} & \textsc{hab} & \textsc{ipfv} & \textsc{pfv} & \textsc{red} & Function\\
\midrule
    \REF{ex:key:427}





         & \textit{mék} &  &  &  &  &  & \textit{de} &  &  & Subjunctive imperfective\\
    \REF{ex:key:1529}





         &  & \textit{bin} &  & \textit{fɔ} &  &  &  &  &  & Counterfactual\\
    \REF{ex:key:396}





         &  & \textit{bin} &  & \textit{fɔ} & \textit{dɔ́n} &  &  &  &  & Counterfactual perfect\\
    \REF{ex:key:663}





        (a) &  & \textit{bin} &  &  &  &  &  & \textit{kán} &  & Past perfective\\
    \REF{ex:key:401}





         &  & \textit{bin} &  &  & \textit{dɔ́n} &  &  &  &  & Past perfect\\
    \REF{ex:key:456}





        (e) &  & \textit{bin} &  &  &  &  & \textit{de} &  &  & Past imperfective\\
    \REF{ex:key:402}





         &  & \textit{bin} &  &  & \textit{dɔ́n} &  & \textit{de} &  &  & Past perfect imperfective\\
    \REF{ex:key:742}





         &  &  &  &  & \textit{dɔ́n} &  & \textit{de} &  &  & Perfect imperfective\\
    \REF{ex:key:397}





         &  &  & \textit{go} &  & \textit{dɔ́n} &  &  &  &  & Potential/future perfect\\
 {{\REF{ex:key:391}}} &  &  & \textit{go} &  &  &  & \textit{de} &  &  & Potential/future imperfective\\
    \REF{ex:key:139}





         &  &  & \textit{go} &  &  &  &  &  & \textsc{red} & Iterative potential/future\\
    \REF{ex:key:138}





         &  &  &  &  &  &  & \textit{de} &  & \textsc{red} & Iterative imperfective\\
    \REF{ex:key:142}





         &  &  &  &  &  & \textit{kin} & \textit{de} &  & \textsc{red} & Iterative habitual\\
    \REF{ex:key:343}





         &  &  &  &  &  & \textit{kin} & \textit{de} &  &  & Habitual \\
\lspbottomrule
\end{tabularx}
\end{sidewaysfigure}

In the corpus, the maximal number of TMA markers encountered in one clause is three (e.g. \REF{ex:key:402}). The markers \textit{bin} ‘\textsc{pst}’, \textit{go} ‘\textsc{pot}’, and \textit{kin} ‘\textsc{hab’} are mutually exclusive. The imperfective marker \textit{de} has the widest distribution and co-occurs with all markers except \textit{fɔ} ‘\textsc{cond’}. In contrast, the narrative perfective marker \textit{kán} has a far more restricted distribution. It only co-occurs with \textit{bin} ‘\textsc{pst’}, and it does so only in two instances in the corpus. Iterative aspect, expressed by reduplication, is most compatible with the imperfective senses expressed by \textit{de} ‘\textsc{ipfv’} and \textit{kin} ‘\textsc{hab’}, but it is also compatible with the tense/mood markers \textit{bin} ‘\textsc{pst’} and \textit{gó} ‘\textsc{pot’}. The imperfective aspect harmonises with the habitual aspect. When \textit{de} ‘\textsc{ipfv}’ and \textit{kin} ‘\textsc{hab}’ co-occur, there is no additional composite sense. The co-occurrence of subjunctive \textit{mék} ‘\textsc{sbjv’} and imperfective \textit{de} ‘\textsc{ipfv’} is very rare in the corpus. 


The markers nɛ́a ‘neg.prf’ and mɔs ‘obl’ are not attested with any other marker and hence not included. Also unattested is the co-occurrence of fɔ ‘cond’ with the potential mood marker go ‘pot’. 


\section{Aspect}\label{sec:6.3}

Sections \sectref{sec:6.3.1} to \sectref{sec:6.4.4} cover aspect marking by means of TMA markers and auxiliary verbs. 

\subsection{The unmarked verb}\label{sec:6.3.1}

Pichi employs factative TMA marking, a phenomenon well known from other languages in the region (cf. \citealt{Welmers1973}: 348). When the unmarked verb occurs in an intransitive main clause and the clause contains no additional information that may have an effect on the interpretation of TMA, it acquires default interpretations of tense\is{tense}, aspect, and modality in accordance with its lexical aspect. The effect of factative TMA marking (or absence of marking) is summarised in \tabref{tab:key:6.4}. Bearing in mind that tense is relational, a factative marked (inchoative-)stative verb is interpreted as “present tense\is{present tense}” with respect to event time not speech time (cf. \sectref{sec:6.5.1}). 


As \tabref{tab:key:6.4} shows, inchoative-stative verbs may receive an imperfective\is{imperfective aspect} interpretation if focus is on the end-state, i.e. the stative meaning component of the verb. Alternatively, these verbs may receive a perfective interpretation, if focus is on the entry-into-state, i.e. the inchoative meaning component of the verb. The modality reading “realis” indicates that factative TMA in an intransitive clause does not normally render meanings associated with the irrealis\is{irrealis modality}\is{irrealis modality} domain, i.e. future tense or subjunctive and potential mood. 


%%please move \begin{table} just above \begin{tabular
\begin{table}
\caption{Default readings of factative TMA according to lexical aspect class}
\label{tab:key:6.4}

\begin{tabularx}{\textwidth}{Qp{1.3cm}p{1.9cm}lp{2.3cm}}
\lsptoprule
Lexical aspect & Tense & Aspect & Modality & Example\\
\midrule
Stative & Present & Imperfective\is{imperfective aspect} & Realis\is{realis modality} & \textit{A hébul}\newline ‘I am capable’\\
\tablevspace
Inchoative-stative & Present,\newline past & Imperfective,\newline perfective & Realis & \textit{A chák}\newline {‘I am drunk;}\newline {I got drunk’}\\
\tablevspace
Dynamic & Past & Perfective\is{perfective aspect} & Realis & \textit{A gó}\newline ‘I went’\\
\lspbottomrule
\end{tabularx}
\end{table}

The unmarked verb also occurs in contexts that are removed from the immediate function of signalling aspect relations. Hence, the unmarked verb occurs in contexts of reduced finiteness (cf. \sectref{sec:10.5.3})\is{finiteness}. It occurs in the \textsc{if-}clauses of conditionals (cf. \sectref{sec:10.7.11}) and with non-initial verbs in clause chaining\is{clause chaining} (cf. \sectref{sec:11.4}). Equally, verbs in subjunctive clauses usually appear devoid of TMA marking (cf. e.g. \sectref{sec:10.5.5}). The unmarked verb also occurs in singular imperatives (cf. \sectref{sec:6.7.3.3})\is{imperatives}.


Perfective aspect via factative TMA marking with dynamic verbs yields an interpretation of the situation as bounded and terminated, hence past by default. Compare \textit{pás} ‘pass’ and \textit{gó} ‘go’ in \REF{ex:key:314}: 



\ea%314
    \label{ex:key:314}
    \gll A    pás    di  domɔ́t  bihɛ́n  sáy  a    gó  fɛ́n    sigá.\\
\textsc{1sg.sbj}  pass    \textsc{def}  door  behind  side  \textsc{1sg.sbj}  go  look.for  cigarette\\

\glt ‘I passed through the entrance at the back, I went to look for a cigarette.’ [ro05rt 016]
\z

Since stative verbs have no inherent boundaries, the unmarked stative verb receives an imperfective, i.e. present tense or present state reading \REF{ex:key:315}. Tense is relational in Pichi, so a stative verb like \textit{wánt} ‘want’ is in the present tense in relation to “event time” (\citealt{ChungTimberlake1985}), which is past tense in this example:


\ea%315
    \label{ex:key:315}
    \gll ɛ́ni    sáy  wé  pɔ́sin  wánt  sidɔ́n,  dɛn  de  sidɔ́n.\\
every  side  \textsc{sub}  person  want  stay    \textsc{3pl}  \textsc{ipfv}  stay\\

\glt ‘Anywhere that a person wanted to stay, they stayed.’ [ma03hm 042]
\z

Given the right context, all factative-marked inchoative-stative verbs may be interpreted as stative or inchoative (hence denoting entry-into-state). While \textit{sabí} ‘(get to) know’ in \REF{ex:key:316} may be interpreted as either stative or inchoative in the absence of disambiguating cues, an inchoative reading is forced upon the factative marked verb \textit{évi} ‘be heavy’ in \REF{ex:key:317}. This is due to the presence of a relational element, namely the temporal clause linker \textit{bifó} ‘before’, which induces an implicit comparison with the prior empty state of the bag:


\ea%316
    \label{ex:key:316}
    \gll \MakeUppercase{A}   \textstylePichiexamplebold{sabí}    sɔn    kápinta    dɛn.\\
\textsc{1sg.sbj}  know  some  carpenter  \textsc{pl}\\

\glt ‘I know some carpenters.’ \textsc{or} ‘I got to know some carpenters.’ [ro05fe 001]
\z


\ea%317
    \label{ex:key:317}
    \gll \textstylePichiexamplebold{Bifó}    wi  rích    fɔ  carretera  di  bolsa  \textstylePichiexamplebold{évi}.\\
before  \textsc{1pl}  arrive  \textsc{prep}  road      \textsc{def}  bag    be.heavy\\

\glt ‘Before we reached the road the bag had become heavy.’ [ed03sb 198]
\z

However, when labile\is{labile verbs} inchoative-stative verbs occur in transitive clauses they automatically acquire a dynamic reading, in which case they receive a perfective, bounded and past tense interpretation like any other dynamic verb. Compare the meaning of the labile change of state verb \textit{brók ‘}break; be broken’ in this example: 


\ea%318
    \label{ex:key:318}
    \gll Dán  húman  e    \textstylePichiexamplebold{brók}  \textstylePichiexamplebold{di}   \textstylePichiexamplebold{plét}\\
that  woman  3\textsc{sg.sbj}  break  \textsc{def}   plate\\

\glt ‘That woman (she) broke the plate.’ [au07se 006]
\z

In addition, even in intransitive clauses, adverbials, and preceding tense-aspect marking in the same sentence, paragraph or text will usually disambiguate an inchoative from a stative interpretation. In \REF{ex:key:319}, for example, factative marking with the dynamic verb \textit{ték} ‘take’ leads to an entry-into-state interpretation of the following factative marked inchoative-stative verb \textit{sidɔ́n} ‘sit (down)’:


\ea%319
    \label{ex:key:319}
    \gll \MakeUppercase{A}   \textstylePichiexamplebold{ték}    di  trí    chía    dɛn,    dán    butaca    ɔ́p  sáy,
mí    \textstylePichiexamplebold{sidɔ́n}  dé,    e    \textstylePichiexamplebold{sidɔ́n}  dís  pát.\\
\textsc{1sg.sbj}  take    \textsc{def}  three  chair  \textsc{pl}    that    elbow.chair  up  side
\textsc{1sg.indp}  sit.down  there  \textsc{3sg.sbj}  sit.down  this  part\\

\glt ‘I took the three chairs, that elbow chair up there, I \textsc{[emp]} sat down 
there, he sat down on this side.’ [ro05rt 006]
\z

Beyond the expression of aspect taxis, the factative perfective aspect expresses conditional modality in the \textsc{if-}clause of conditionals with dynamic \REF{ex:key:320} and stative verbs alike \REF{ex:key:321}: \is{conditional clauses}


\ea%320
    \label{ex:key:320}
    \gll E    go  \textstylePichiexamplebold{dé}   fáyn    ɛf  e    \textstylePichiexamplebold{kán}.\\
\textsc{3sg.sbj}  \textsc{pot}  \textsc{be.loc}  fine    if  \textsc{3sg.sbj}  come\\

\glt ‘It will be nice, if he comes.’ [dj05ae 205]
\z


\ea%321
    \label{ex:key:321}
    \gll If  yu  wánt,  a    fít  sɛ́l  yu  mi    hós.\\
if  \textsc{2sg}  want  \textsc{1sg.sbj}  can  sell  \textsc{2sg}  \textsc{1sg.poss}  house\\

\glt ‘If you want, I can sell you my house.’ [dj07ae 342]
\z

Beyond that, factative marking is encountered in procedural texts in contexts that suggest a habitual reading. In the following excerpt, speaker (dj) is asked by (ge) to explain how \textit{ógi} ‘corn porridge’ is prepared. The dynamic verb \textit{pút} ‘put’ in (b) has a habitual\is{habitual aspect} sense but remains bare. Note that imperative\is{imperatives} clauses are not formed with \textsc{2sg} personal pronouns:


\ea%322
    \label{ex:key:322}
    \ea{
    \gll \'{A}fta    háw    fɔ  mék    di  ógi?\\
  then  how    \textsc{prep}  make  \textsc{def}  corn.porridge\\

\glt   ‘Then how do you make the corn porridge?’ [ge03do 050]
	}
	\ex{
\gll
Yu  fít  ték    náw,  wán,  wán  smɔ́l  kɔ́p  nɔ́,  \textstylePichiexamplebold{yu}  \textstylePichiexamplebold{pút}=an
  na  fáya,  ínsay  di  pɔ́t.\\
  \textsc{2sg}  can  take    now    one    one  small  cup  \textsc{intj}  \textsc{2sg} put=\textsc{3sg.obj}
  \textsc{loc}  fire    inside  \textsc{def}  pot\\

\glt 
  ‘Now you can take, a, a small cup, right, you put it on the fire, inside 
the pot.’ [dj03do 051]
}
\z
\z

\subsection{Perfective and imperfective aspect}\label{sec:6.3.2}

The Pichi system of aspect marking represents a typologically widespread type in which the expression of perfective and imperfective aspect is not fully symmetrical (\citealt{Dahl1985}:69–102). The system features a general imperfective aspect marker \textit{de}. Its function is to suppress the inherent boundaries of a situation (\citealt{Breu1985}; \citealt{Sasse1991b,Sasse1991a}. Although Pichi has other markers that encode imperfective notions (e.g. \textit{kin} ‘\textsc{hab}’), the marker \textit{de} ‘\textsc{ipfv}’ alone may cover their functions, as well as those of others generally associated with the imperfective domain (e.g. future tense). 


At the same time, the expression of perfective aspect is less uniform. On the one hand, perfective aspect is covered by factative TMA for dynamic verbs. Factative marking activates the inherent boundaries of dynamic verbs and thereby expresses perfective aspect by default. However, factative marked (inchoative-)stative verbs do not receive the corresponding perfective reading of entry-into-state by default. Instead, factative marking with stative verbs yields an imperfective reading, namely present or ongoing state, while inchoative-stative verbs are not automatically interpreted with an entry-into-state meaning either. 



The narrative perfective marker \textit{kán} ‘\textsc{pfv}’, rather than factative TMA\is{factative TMA}, is therefore a better candidate for the expression of perfective meanings. As shown in section \sectref{sec:6.3.3}, the use of \textit{kán} ‘\textsc{pfv’} yields typically perfective meanings with all lexical aspect classes. The difference between the marking of dynamic verbs with perfective aspect by factative TMA and by narrative perfective aspect emerges most clearly in their uses in narrative discourse (cf. section \sectref{sec:6.8}).



Elements like the perfect marker dɔ́n and its negative counterpart nɛ́a, as well as ingressive, egressive and completive{\fff} aspect auxiliaries also express various perfective readings. The following table provides an overview of the formal means of core perfective and imperfective marking and their readings in the three lexical aspect classes. The default tense{\fff} interpretation of each aspect reading is provided in parentheses (prs = present tense{\fff}, pst = past tense{\fff}): 


%%please move \begin{table} just above \begin{tabular
\begin{table}
\caption{Perfective and imperfective readings according to lexical aspect class}
\label{tab:key:6.5}

\begin{tabularx}{\textwidth}{llQQ}
\lsptoprule
 & Stative verbs & {{Inchoative-stative verbs}} & {{Dynamic verbs}}\\
\midrule 
Factative & Stative (\textsc{prs}) & Stative (\textsc{prs}), inchoative (\textsc{pst}) & Bounded (\textsc{pst})\\
\tablevspace
\textit{kán} ‘\textsc{pfv’}\is{perfective aspect} & Inchoative (\textsc{pst}) & Inchoative (\textsc{pst}) & Bounded (\textsc{pst})\\
\tablevspace
\textit{de} ‘\textsc{ipfv’}\is{imperfective aspect} & {}--- & Inchoative (\textsc{prs}) & Progressive (\textsc{prs}), continuative (\textsc{prs}),\is{continuative aspect} habitual (\textsc{prs})\is{habitual aspect}, future\is{future tense}, hypothetical, non-finiteness\is{infinitive} \\
\lspbottomrule
\end{tabularx}
\end{table}
\subsection{Narrative perfective}\label{sec:6.3.3}

The marker \textit{kán} ‘\textsc{pfv}’ expresses narrative perfective aspect\is{narrative perfective aspect} (cf. \citealt{Jaggar2006}). It encodes perfective aspect and consequently, past tense by default. \textit{Kán} ‘\textsc{pfv}’ occurs in salient [+high] foreground sequences of narrative discourse, while factative perfective marking is employed for less salient [-high] foreground sequences (cf. \sectref{sec:6.8.1}). The narrative perfective marker therefore shares its functional space with factative TMA, and hence falls short of functioning as a general perfective marker. Although \textit{kán} ‘\textsc{pfv}’ is homophonous with its lexical source verb \textit{kán} ‘come’, there is no restriction on its co-occurrence with directional verbs, such as \textit{gó} ‘go’ \REF{ex:key:323} or \textit{kán} ‘come’ \REF{ex:key:324}:


\ea%323
    \label{ex:key:323}
    \gll Dán    mán    e    bin  kán  gó  na  jél  lɔ́n    tɛ́n.\\
that    man    \textsc{3sg.sbj}  \textsc{pst}  \textsc{pfv}  go  \textsc{loc}  jail  long    time\\

\glt ‘That man went to jail a long time ago.’ [ma03sh 015]
\z


\ea%324
    \label{ex:key:324}
    \gll E    gí  di  papá  di  pikín,  kɔmɔ́t,  e    \textstylePichiexamplebold{kán}
\textstylePichiexamplebold{kán}   na  Malábo.\\
\textsc{3sg.sbj}  give  \textsc{def}  father  \textsc{def}  child  go.out  \textsc{3sg.sbj}  \textsc{pfv}
come  \textsc{loc}  \textsc{place}\\

\glt ‘She gave her child to the father, left, (and then) she came to Malabo.’ [ed03sb 036]
\z


The marker \textit{kán} ‘\textsc{pfv}’ is largely specialised to use in the foregrounded main line of narrative discourse. Here, it usually marks consecutive and bounded events denoted by dynamic verbs. In this function, the narrative perfective overlaps with perfective marking via factative TMA\is{factative TMA}. But contrary to the latter, narrative perfective marking is employed in foregrounded sequences containing particularly salient, important information. \textit{Kán} is preferred to factative perfective marking when new events unfold. In that, \textit{kán} serves to highlight and focus the event denoted by the verb it refers to. 


The three sentences below are an excerpt from a personal narrative. The speaker relates how she went to stay with her paternal uncle during a critical illness. This new information is provided in clauses \REF{ex:key:325}(a) and (b), and the relevant verbs (\textit{gó} ‘go’ and \textit{dé} \textsc{‘be.loc’)} are marked by narrative perfective. In (c), the speaker reverts to factative TMA because the sentence now contains given information. Note that the same stative verb \textit{dé} \textsc{‘be.loc’,} which occurs with narrative perfective marking in the foregrounded sentence (b), appears with factative TMA in the backgrounded sentence in (c): 



\ea%325
    \label{ex:key:325}
    \ea{\label{ex:key:325a}
    \gll  A    \textstylePichiexamplebold{kán} \textstylePichiexamplebold{gó}  na  mi    ɔnkúl  in    papá  in    lét  brɔ́da.\\
  \textsc{1sg.sbj}  \textsc{pfv}  go  \textsc{loc}  \textsc{1sg.poss}  uncle  \textsc{3sg.poss}  father  \textsc{3sg.poss}  late  brother\\

\glt   ‘I went to my uncle’s father’s late brother.’ [ab03ay 098]
}
\ex{\label{ex:key:325b}
\gll
Mi    lét  papá  in    brɔ́da,  a    \textstylePichiexamplebold{kán} \textstylePichiexamplebold{dé}    na  in    hós.\\
  \textsc{1sg.poss}  late  father  \textsc{3sg.poss}  brother  \textsc{1sg.sbj}  \textsc{pfv}  \textsc{be.loc}  \textsc{loc}  \textsc{3sg.poss}  house\\

\glt   ‘My late father’s brother, I came to be in his house.’ [ab03ay 099]
}
\ex{
\gll 
Na  dé    a    \textstylePichiexamplebold{dé}    wán  hía    a    nó  \textstylePichiexamplebold{fít}  dú  nó  nátin.\\
  \textsc{foc}  there  \textsc{1sg.sbj}  \textsc{be.loc}  one  year    \textsc{1sg.sbj}  \textsc{neg}  can  do  \textsc{neg}  nothing\\

\glt   ‘It’s there that I was (for) one year, I couldn’t do anything at all.’ [ab03ay 100]

}
\z
\z

The narrative perfective marker \textit{kán}, even though specialised to narrative discourse, is a typical perfective marker (cf. \tabref{tab:key:6.5}). Irrespective of the lexical class of the verb, \textit{kán} always activates the potential boundaries of a situation. With dynamic verbs, the situation is bounded and seen as a whole, hence past tense\is{past tense} by default (cf. \REF{ex:key:325} above). The consistent meaning associated with the narrative perfective marker \textit{kán} ‘\textsc{pfv’} therefore contrasts with diametrically opposed meanings that arise through factative TMA marking with stative and dynamic verbs respectively.


The use of kán with stative (cf. dé ‘be.loc’ in \REF{ex:key:325}(b) above) and inchoative-stative verbs (cf. \REF{ex:key:326} below) activates the initial boundary of the situation and focuses the ensuing state. Hence, it yields an inchoative (entry-into-state) meaning with a past tense interpretation in relation to event time. The different meanings that arise when a stative verb like lɛ́k ‘like; love’ is marked for perfective aspect and for factative aspect respectively, is shown by comparison of \REF{ex:key:327} and \REF{ex:key:335} further below. 



\ea%326
    \label{ex:key:326}
    \gll Pero    ɛf  di  tín    \textstylePichiexamplebold{kán} \textstylePichiexamplebold{bɔkú}  mɔ́    pás  di  watá,
e    go  lɛ́f    wán    pasta.\\
but    if  \textsc{def}  thing  \textsc{pfv}  much  more  pass  \textsc{def}  water
\textsc{3sg.sbj}  \textsc{pot}  leave  one    paste\\

\glt ‘But if the thing has become more than the water, a paste will remain.’ [dj03do 059]
\z


\ea%327
    \label{ex:key:327}
    \gll E    kán  lɛ́k  ɔ́da    húman.\\
\textsc{3sg.sbj}  \textsc{pfv}  like  other  woman\\

\glt ‘(Then) he fell in love with another woman.’ [ma03ni 022]
\z

Like factative TMA, the narrative perfective is sometimes employed, albeit rarely, in contexts other than aspect taxis. In \REF{ex:key:328}, \textit{kán} appears in the \textsc{if-}clause of a past conditional (cf. also \REF{ex:key:320}). Maybe this usage reflects a tendency for \textit{kán} to extend its function even further to that of a generalised perfective marker: \is{conditional clauses}


\ea%328
    \label{ex:key:328}
    \gll Ɛf  yu  \textstylePichiexamplebold{bin}  \textstylePichiexamplebold{kán} bigín  lás  semana,  yu  bin  fɔ    dɔ́n  fínis    tidé.\\
if  \textsc{2sg}  \textsc{pst}  \textsc{pfv}  begin  last  week  \textsc{2sg}  \textsc{pst}  \textsc{cond}    \textsc{prf}  finish  today\\

\glt ‘If you had begun last week, you would have finished by today.’ [dj05ae 057]\is{perfective aspect}
\z

\subsection{Imperfective}\label{sec:6.3.4}

The general imperfective marker \textit{de} ‘ipfv’ encodes various aspectual readings associated with the imperfective domain (cf. \tabref{tab:key:6.5}). Imperfective marking may express progressive aspect with dynamic verbs and present tense{\fff} by default. Compare smɛ́l ‘smell’ and kúk ‘cook’ in \REF{ex:key:329}: 


\ea%329
    \label{ex:key:329}
    \gll A    de  smɛ́l  di  sɛ́nt    fɔ  lɛk  háw    e    de  kúk    plantí.\\
\textsc{1sg.sbj}  \textsc{ipfv}  smell  \textsc{def}  scent  \textsc{prep}  like  how    \textsc{3sg.sbj}  \textsc{ipfv}  cook  plantain\\

\glt ‘I smell the scent of him cooking plantain.’ [dj05ae 025]
\z

Context may force a habitual interpretation on imperfective marked dynamic verbs. In \REF{ex:key:330}, the habitual reading of chɔ́p ‘eat’ is signalled through the presence of the time adverbial ɛ́ni dé ‘every day’:


\ea%330
    \label{ex:key:330}
    \gll ɛ́ni    dé  dɛn  de  chɔ́p  rɛ́s,  ɛ́ni    dé.\\
every  day  \textsc{3pl}  \textsc{ipfv}  eat    rice  every  day\\

\glt ‘Every day they eat rice, every day.’ [ed03sp 117]
\z

Certain human propensities and body states that may potentially be conceived as stative are expressed as dynamic verbs in Pichi. These include property items such \textit{krés} ‘be crazy’ and \textit{sík} ‘be sick’ (cf. \sectref{sec:5.2.1} and \sectref{sec:7.6.5} for more details). These verbs also take imperfective marking when progressive, continuous, or habitual aspect is to be expressed: 


\ea%331
    \label{ex:key:331}
    \gll Yu  \textstylePichiexamplebold{de} \textstylePichiexamplebold{krés}.\\
\textsc{2sg}  \textsc{ipfv}  be.crazy\\
\glt ‘You are crazy.’ [ro05ee 038]
\z


\ea%332
    \label{ex:key:332}
    \gll E    \textstylePichiexamplebold{de} \textstylePichiexamplebold{sík}  malérya.\\
\textsc{3sg.sbj}  \textsc{ipfv}  sick  malaria\\

\glt ‘He is sick with malaria.’ [dj05be 091]
\z

The imperfective marker does not normally co-occur with stative verbs. Yet de ‘ipfv’ is sometimes found with non-dynamic verbs. In \REF{ex:key:333} and \REF{ex:key:334} the inchoative-stative verb gɛ́t ‘get; have’ and the stative verb lɛ́k ‘like’ take the imperfective marker without acquiring an inchoative sense. This usage appears limited to modal verbs and verbs of possession like the following two:


\ea%333
    \label{ex:key:333}
    \gll \'{A}fta    dɛn  de  gɛ́t  fisionomía    fɔ,  fɔ  \'{A}frika  dɛn.\\
then  \textsc{3pl}  \textsc{ipfv}  get  physiognomy    \textsc{prep}  \textsc{prep}  Africa  \textsc{pl}\\

\glt ‘Then they have the physiognomy of, of Africans.’ [ed03sp 031]
\z


\ea%334
    \label{ex:key:334}
    \gll A    nó,  a    nó  de  lɛ́k=an    mɔ́,    nó.\\
\textsc{1sg.sbj}  \textsc{neg}  \textsc{1sg.sbj}  \textsc{neg}  \textsc{ipfv}  like=\textsc{3sg.obj}  more  \textsc{neg}\\

\glt ‘I don’t, I don’t love him any longer, no.’ [ma03ni 037]
\z

The conventional way of expressing imperfective aspect with (inchoative-)stative verbs is, however, by way of factative TMA{\fff}. In \REF{ex:key:335} the stative verb lɛ́k ‘like’ remains unmarked, hence is imperfective by default:{\fff} 


\ea%335
    \label{ex:key:335}
    \gll Dɛn    nó  lɛ́k  pɔ́sin,  dɛn  tú  badhát.\\
\textsc{3pl}    \textsc{neg}  like  person  \textsc{3pl}  too  be.mean\\

\glt ‘They don’t like people, they’re too mean.’ [ma03hm 012]
\z

In contrast, \textit{de} ‘\textsc{ipfv}’ is regularly made use of with most inchoative-stative verbs in order to express an inchoative reading with a present tense\is{present tense} interpretation in relation to event time. Compare the following two examples, as well as \REF{ex:key:371} below: 


\ea%336
    \label{ex:key:336}
    \gll In    mɔní  \textstylePichiexamplebold{de} \textstylePichiexamplebold{bɔkú}.\\
\textsc{3sg.poss}  money  \textsc{ipfv}  be.much\\

\glt ‘Her money is getting more.’ [ro05ee 047]
\z


\ea%337
    \label{ex:key:337}
    \gll In    mɔní  \textstylePichiexamplebold{de} \textstylePichiexamplebold{smɔ́l}.\\
\textsc{3sg.poss}  money  \textsc{ipfv}  be.small\\

\glt ‘His money is getting less.’ [ro05ee 048]
\z

Besides its use for expressing aspectual relations, the functions of \textit{de} reach into the domain of modality and overlap with those of the potential marker \textit{go} ‘\textsc{pot’}. The imperfective marker may express future tense in combination with an appropriate time adverbial \REF{ex:key:338}. It can also express conditional modality in \textsc{then-}clauses and hypothetical statements contingent upon inferred conditions \REF{ex:key:339}:\is{conditional clauses}


\ea%338
    \label{ex:key:338}
    \gll A    de  lɛ́f    na  Lubá  sóté    di  nɛ́ks    wík.\\
\textsc{1sg.sbj}  \textsc{ipfv}  remain  \textsc{loc}  \textsc{place}  until  \textsc{def}  next    week\\

\glt ‘I’m staying in Luba until next week.’ [dj05ce 014]
\z


\ea%339
    \label{ex:key:339}
    \gll \MakeUppercase{A}   de \textstylePichiexamplebold{ték} mi    pikín  gó  na  hospital  claro.\\
\textsc{1sg.sbj}  \textsc{ipfv}  take  \textsc{1sg.poss}  child  go  \textsc{loc}  hospital  clear\\

\glt ‘I would take my child to hospital, of course.’ [hi03cb 140]
\z

We also encounter the imperfective marker in environments characterised by reduced finiteness{\fff}. Thus, de optionally intervenes between certain aspectual auxiliaries (cf. \sectref{sec:6.4.1}) and modal verbs and the verbs that follow them (cf. \sectref{sec:10.5.3} for more details). Compare the following modal verbs gɛ́fɔ ‘have to’ \REF{ex:key:340} and wánt ‘want’ \REF{ex:key:341}:


\ea%340
    \label{ex:key:340}
    \gll Yu  gɛ́fɔ    de  tɔ́n=an.\\
\textsc{2sg}  have.to  \textsc{ipfv}  turn=\textsc{3sg.obj}\\

\glt ‘You need to be stirring it.’ [dj03do 057]
\z


\ea%341
    \label{ex:key:341}
    \gll Yu  \textstylePichiexamplebold{wánt} \textstylePichiexamplebold{de}  gó?\\
\textsc{2sg}  want  \textsc{ipfv}  go\\


\glt ‘Do you want to go?’ [nn07fn 227]\is{imperfective aspect}\\
\z
\subsection{Habitual}\label{sec:6.3.5}

The central function of the marker \textit{kin} ‘\textsc{hab}’ is to express the imperfective reading of habitual aspect. Next to that, \textit{kin} is also employed to express iterative aspect (cf. \sectref{sec:6.3.6}), and it marginally functions as a modal verb of ability (cf. \REF{ex:key:411}). The marker either appears alone in preverbal position \REF{ex:key:342} or is optionally followed by the imperfective marker\is{imperfective aspect} \textit{de} if the reference verb is dynamic \REF{ex:key:343}. There is no discernible semantic difference between \textit{kin} and \textit{kin} \textit{de}. The optional co-occurrence of the two can be seen as a form of aspectual harmony or mutual reinforcement:


\ea%342
    \label{ex:key:342}
    \gll E    tɛ́l  mí    sé    ‘wi  kin  mítɔp  ínsay  wán  motó’.\\
\textsc{3sg.sbj}  tell  \textsc{1sg.indp}  \textsc{quot}    \textsc{1pl}  \textsc{hab}  meet  inside  one  car\\


\glt ‘He told me (that) “we would meet inside a car.”   [ro05rt 019]
\z

\ea%343
    \label{ex:key:343}
    \gll Nit  na  in    éks  dɛn  wé  e    \textstylePichiexamplebold{kin} \textstylePichiexamplebold{de} \textstylePichiexamplebold{pút}.\\
nit  \textsc{foc}  \textsc{3sg.poss}  egg  \textsc{pl}  \textsc{sub}  \textsc{3sg.sbj}  \textsc{hab}  \textsc{ipfv}  put\\

\glt ‘The nits are the eggs that it lays.’ [ye05ce 293]
\z

Since stative verbs are not normally marked by means of \textit{de} ‘\textsc{ipfv’}, an important function of \textit{kin} ‘\textsc{hab’} is therefore to overtly mark stative verbs for habitual aspect. The habitual marker is therefore compatible with all lexical aspect classes. When used with (inchoative-)stative verbs, \textit{kin} may additionally emphasise\is{emphasis} the habitual nature of the situation. Examples follow with the stative copula \textit{dé} \textsc{‘be.loc’} \REF{ex:key:344} and the inchoative-stative verb \textit{nó} ‘(get to) know’ \REF{ex:key:345}:


\ea%344
    \label{ex:key:344}
    \gll Sé    ús=tín  \textstylePichiexamplebold{kin} \textstylePichiexamplebold{dé}    ínsay  dé?\\
\textsc{quot}    \textsc{q}=thing  \textsc{hab}  \textsc{be.loc}  inside  there\\

\glt ‘(She) said “what is usually in there?”‘ [ed03sb 052]
\z


\ea%345
    \label{ex:key:345}
    \gll Dɛn    nó  kin  nó    sɛ́f.\\
\textsc{3pl}    \textsc{neg}  \textsc{hab}  know  \textsc{emp}\\

\glt ‘They didn’t even use to know.’ [bo03cb 118]
\z

The habitual marker is also employed in generic statements, such as the following one:


\ea%346
    \label{ex:key:346}
    \gll Dɔ́g    kin  bɛ́t.\\
dog    \textsc{hab}  bite\\

\glt ‘Dogs bite.’ [dj07ae 371]
\z

The habitual marker does not co-occur with the tense marker \textit{bin} ‘\textsc{pst’} or the potential mood and future tense marker \textit{go} ‘\textsc{pot’}. Like the imperfective marker \textit{de} ‘\textsc{ipfv}’, \textit{kin} ‘\textsc{hab}’ is itself unspecified for tense. Accordingly, sentence \REF{ex:key:345} above is translated as past habitual, because the time frame of the corresponding discourse context suggests so.\is{habitual aspect} 

\subsection{Iterative}\label{sec:6.3.6}

The reduplication of dynamic verbs yields the imperfective reading of iterative aspect when the reduplicated verb serves as the predicate of a clause. I refer the reader to section \sectref{sec:4.5.1} for a detailed treatment of the phonology, morphosyntax, and semantics of reduplication.


Sentence \REF{ex:key:347} shows a typical context in which an iterative reading of reduplication arises. The reduplicated verb is accompanied by imperfective marking\is{imperfective aspect} and co-occurs with the plural count noun object \textit{nɔ́mba dɛn} ‘numbers’:



\ea%347
    \label{ex:key:347}
    \gll Wétin  yu  \textbf{de}  \textbf{chench-chénch}  nɔ́mba  dɛn  só?\\
what  \textsc{2sg}  \textsc{ipfv}  \textsc{red.cpd-}change  number  \textsc{pl}  like.that\\

\glt ‘Why do you constantly change (telephone) numbers like that?’ [ye03cd 131]\is{reduplication}
\z

In a small number of cases in the corpus, the habitual\is{habitual aspect} marker \textit{kin} also expresses iterative aspect by itself without additional reduplication. The speaker in the two consecutive sentences in \REF{ex:key:348} narrates how she repeatedly felt the temperature of her sick grandchild: 


\ea%348
    \label{ex:key:348}
    \ea{\label{ex:key:348a}
    \gll Wé  a    \textstylePichiexamplebold{kin}  \textstylePichiexamplebold{mék}   só,    a    nó  de  fíl  hɔ́t.\\
  \textsc{sub}  \textsc{1sg.sbj}  \textsc{hab}  make  like.that  \textsc{1sg.sbj}  \textsc{neg}  \textsc{ipfv}  feel  heat\\

\glt   ‘Anytime I would do like this, I wouldn’t feel heat.’ [ab03ab 065]
}
\ex{\label{ex:key:348b}
\gll
Pero    wé  a    \textstylePichiexamplebold{kin} \textstylePichiexamplebold{tɔ́ch}    in    fút,  in    hán    dé,
  na  só    dɛn  [kó:::l].\\
  but    \textsc{sub}  \textsc{1sg.sbj}  \textsc{hab}  touch  \textsc{3sg.poss}  foot  \textsc{3sg.poss}  hand  there
  \textsc{foc}  like.that  \textsc{3pl}  be.cold.\textsc{emp}\\
\glt 
  ‘When I would touch his leg, his hand there, that’s how terribly cold they were.’
[ab03ab 066]\is{iterative aspect}
}
\z
\z

\section{Aspectual auxiliaries}\label{sec:6.4}
\is{aspect}
A specific set of verbs and adverbs function as auxiliaries\is{aspectual verbs} in constructions that express aspectual notions. These constructions involve the verbs \textit{bigín} ‘begin’ (ingressive\is{ingressive aspect}), \textit{kɔmɔ́t} ‘go out’ (egressive), \textit{fínis} ‘finish’\textit{} (completive), \textit{wánt} ‘want’ (prospective\is{prospective aspect}), and the Spanish-origin verb \textit{sigue} ‘continue’ (continuative\is{continuative aspect}). The expression of egressive\is{egressive aspect} and continuative aspect may also involve the preverbal adverbs\is{preverbal adverbs} \textit{jɔ́s/jís} ‘just’ and \textit{stíl} ‘still’, either in conjunction with the corresponding auxiliary verbs or alone. 


These auxiliary verbs function as main verbs to complement verbs that are, in turn, specified for an aspect reading by the auxiliary verb. I analyse these aspectual (and modal, cf. \sectref{sec:6.7.1}) structures as involving complementation rather than verb serialisation. This is because the imperfective marker\is{imperfective aspect} \textit{de} may optionally intervene between the main and complement verb in some of these structures (cf. \sectref{sec:10.5.1}). When this the case, \textit{de} \textsc{‘ipfv’} functions as a complementiser while emphasising the continuous nature of the situation denoted by the complement verb. The following table provides an overview of the functions of aspectual auxiliaries. Optional elements are in parentheses: 


%%please move \begin{table} just above \begin{tabular
\begin{table}
\caption{Functions of aspectual auxiliaries}
\label{tab:key:6.6}

\begin{tabularx}{\textwidth}{XXX}
\lsptoprule

Aspect reading & Auxiliary & Translation\\
\midrule
Ingressive\is{ingressive aspect} & \itshape bigín (de) & ‘begin’\\
\tablevspace
Egressive\is{egressive aspect} & \itshape (jís/jɔ́s) kɔmɔ́t & ‘just have’\\
& \itshape jís/jɔ́s & ‘just’\\
\tablevspace
Completive\is{completive aspect} & \itshape fínis & ‘finish’\\
\tablevspace
Continuative\is{continuative aspect} & \itshape (stíl) sigue & ‘continue’\\
& \itshape stíl (de) & ‘still’\\
\tablevspace
Prospective\is{prospective aspect} & \itshape wánt (de) & ‘be about to’\\
\lspbottomrule
\end{tabularx}
\end{table}
Not included in the table are constructions involving the verbs \textit{sté} ‘stay’ and \textit{lás} ‘end up’. These verbs participate in adverbial SVCs with a certain degree of aspectual meaning (cf. \sectref{sec:11.2.5}). However, these constructions are more specialised in their meaning and not as grammaticalised to warrant being seen as aspectual auxiliaries in the same way as the ones covered in this section.

\subsection{Ingressive}\label{sec:6.4.1}

The aspectual verb verb \textit{bigín} ‘begin’ expresses ingressive aspect. The function of \textit{bigín} as a transitive dynamic verb is exemplified in \REF{ex:key:349}, where it is followed by the object \textsc{NP} \textit{di wók} ‘\textsc{def} work’ = ‘the work’: 


\ea%349
    \label{ex:key:349}
    \gll A    \textstylePichiexamplebold{bigín}  \textstylePichiexamplebold{di}  \textstylePichiexamplebold{wók}   wé  yu  dɔ́n  gó.\\
\textsc{1sg.sbj}  begin  \textsc{def}  work  \textsc{sub}  \textsc{2sg}  \textsc{prf}  go\\

\glt ‘I began the work when you had gone.’ [ro05de 024]
\z

Ingressive aspect highlights the crossing of the initial boundary of a situation \REF{ex:key:350}. When employed as an aspectual auxiliary, \textit{bigín} may be immediately followed by a lexical verb \REF{ex:key:350} or optionally be followed by the imperfective\is{imperfective aspect} marker \textit{de} \REF{ex:key:351}, which stresses the continuous or extended nature of the transition to the situation denoted by the lexical verb:


\ea%350
    \label{ex:key:350}
    \gll \MakeUppercase{A}   \textstylePichiexamplebold{bigín}  \textstylePichiexamplebold{gó}  skúl.\\
\textsc{1sg.sbj}  begin  go  school\\

\glt ‘I began going to school.’ [fr03ft 018]
\z


\ea%351
    \label{ex:key:351}
    \gll A    \textstylePichiexamplebold{bigín} \textstylePichiexamplebold{de}  \textstylePichiexamplebold{lás},    a    de  kɔ́stɔn.\\
\textsc{1sg.sbj}  begin  \textsc{ipfv}  endure  \textsc{1sg.sbj}  \textsc{ipfv}  get.used\\

\glt ‘I began enduring (it), I was getting used (to it).’ [ed03sp 110]
\z

The auxiliary \textit{bigín} itself can be marked by tense-aspect markers like any other dynamic verb. In \REF{ex:key:352}, \textit{bigín} cooccurs with the narrative perfective aspect marker \textit{kán} ‘\textsc{pfv’}. The auxiliary \textit{bigín} is not attested with stative verbs. But it may combine with inchoative-stative verbs, in order to highlight the entry-into-state meaning of verbs from this lexical aspect class \REF{ex:key:353}: 


\ea%352
    \label{ex:key:352}
    \gll Dɛn  \textstylePichiexamplebold{kán}  \textstylePichiexamplebold{bigín}  \textstylePichiexamplebold{kɔ́l}  mí    Francisca.\\
\textsc{3pl}  \textsc{pfv}  begin  call  \textsc{1sg.indp}  \textsc{name}\\

\glt ‘They began to call me Francisca.’ [fr03ft 095]
\z


\ea%353
    \label{ex:key:353}
    \gll Wi  bigín  de  nó    wi  sɛ́f.\\
\textsc{1pl}  begin  \textsc{ipfv}  know  \textsc{1pl}  self\\

\glt ‘We began to get to know each other.’ [ye07fn 019]\is{ingressive aspect}
\z

\subsection{Egressive}\label{sec:6.4.2}

The verb \textit{kɔmɔ́t} expresses egressive aspect. The egressive highlights the crossing of the terminal boundary of the situation described by the verb. This auxiliary construction is not attested with stative verbs. The egressive aspect neither carries a connotation of completion like the completive, nor does it establish a relation to reference time like the perfect. The auxiliary \textit{kɔmɔ́t} may optionally be preceded by the prev\is{preverbal adverbs}erbal adverb \textit{jís/jɔ́s} ‘just’ and is immediately followed by the complement verb. 


\ea%354
    \label{ex:key:354}
    \gll E    tɛ́l  mí    sé    dán    papá  wé  e    jɔ́s  kɔmɔ́t
\textstylePichiexamplebold{cobra}   in    mɔní  fɔ  cacao,  salút=an!\\
\textsc{3sg.sbj}  tell  \textsc{1sg.indp}  \textsc{quot}    that    father  \textsc{sub}  \textsc{3sg.sbj}  just  come.out
receive  \textsc{3sg.poss}  money  \textsc{prep}  cocoa  greet=\textsc{3sg.obj}\\

\glt ‘He said to me “that elderly man that just received the money for his cocoa, 
greet him!”’ [ed03sb 196]
\z

The verb \textit{kɔmɔ́t} has various meanings ranging from more lexical to more grammatical (cf. e.g. uses as a copula verb in \sectref{sec:7.6.2} and as a directional verb in motion\is{motion verbs}{}-direction SVCs in \sectref{sec:11.2.1}). In the following sentence, \textit{kɔmɔ́t} is used with its presumably focal spatial meaning of ‘go/come out’: 


\ea%355
    \label{ex:key:355}
    \gll Di  gɛ́l  kán  kɔmɔ́t  dé.\\
\textsc{def}  girl  \textsc{pfv}  go.out  there\\

\glt ‘The girl left there [that place].’ [ed03sb 030]
\z

In other instances, the meaning of \textit{kɔmɔ́t} is intermediary between a spatial and a more grammatical sense. In \REF{ex:key:356}, it is the presence of the locative question word \textit{ús=sáy} ‘where’ that creates ambiguity between the literal and the egressive senses of \textit{kɔmɔ́t}. In sentence \REF{ex:key:357}, semantic ambiguity is produced by the presence of \textit{wók} which may mean ‘work’ (the noun) or ‘to work’ (the verb). If the former translation is preferred, \textit{wók} is analysed as the (source) object of \textit{kɔmɔ́t}. With the latter translation \textit{wók} is a complement verb:


\ea%356
    \label{ex:key:356}
    \gll \'{U}s=sáy  yu  \textstylePichiexamplebold{kɔmɔ́t} \textstylePichiexamplebold{chák}   só?\\
\textsc{q}=side  \textsc{2sg}  come.out  get.drunk  like.that\\

\glt ‘Where do you come from drunk like this?’ [ye07fn 126]
\z

The verb \textit{kɔmɔ́t} may co-occur with any TMA marker compatible with its status as a dynamic verb. Compare its appearance with the habitual\is{habitual aspect} marker \textit{kin} ‘\textsc{hab}’ in \REF{ex:key:357}:


\ea%357
    \label{ex:key:357}
    \gll Wé  e    \textstylePichiexamplebold{kin} \textstylePichiexamplebold{kɔmɔ́t}    wók,  a    kin
mék=an    so,    lɛk  háw  mún    fínis.\\
\textsc{sub}  \textsc{3sg.sbj}  \textsc{hab}  come.out  work  \textsc{1sg.sbj}  \textsc{hab}
make=\textsc{3sg.obj}  like.that  like  how  month  finish\\

\glt ‘When he comes from work/ when he has barely finished working, I do like 
this to him [stretches out hand], as soon as the month is over.’ [ro05rt 042]
\z

The synonymous and equally common adverbials \textit{jís} and \textit{jɔ́s} can express an egressive notion by themselves when they appear in the preverbal adverb\is{preverbal adverbs} position \REF{ex:key:358}, and thereby be functionally equivalent to egressive\is{egressive aspect} \textit{kɔmɔ́t}. The adverb \textit{jís}{{/}}\textit{jɔ́s} may be preceded by a TMA marker and be followed by the lexical verb that it modifies. Note the occurrence of resumptive imperfective\is{imperfective aspect} marking \is{resumptive imperfective marking}in \REF{ex:key:359} (cf. also \REF{ex:key:98}:


\ea%358
    \label{ex:key:358}
    \gll A    \textstylePichiexamplebold{jɔ́s} \textstylePichiexamplebold{báy} sɔn.\\
\textsc{1sg.sbj}  just  buy  some\\

\glt ‘I just bought some.’ [ma03hm 072]
\z


\ea%359
    \label{ex:key:359}
    \gll Náw    dɛn  \textstylePichiexamplebold{de}  \textstylePichiexamplebold{jís}  \textstylePichiexamplebold{de}  kán.\\
now    \textsc{3pl}  \textsc{ipfv}  just  \textsc{ipfv}  come\\

\glt ‘Now they’re just coming.’ [ye07je 179]
\z

I analyse \textit{jís}{{/}}\textit{jɔ́s} as an adverb rather than a preverbal TMA marker or a verb since it occasionally also occurs in the sentence-initial adverbial position with no difference in meaning \REF{ex:key:360}. The adverb \textit{jís}{{/}}\textit{jɔ́s} is also used with no temporal meaning at all \REF{ex:key:361}: 


\ea%360
    \label{ex:key:360}
    \gll \textstylePichiexamplebold{Jɔ́s}  e    kɔmɔ́t    na  Baney  (...)\\
just  \textsc{3sg.sbj}  come.out  \textsc{loc}  \textsc{place}\\

\glt ‘She had just left Baney (...)’ [ab03ay 079]
\z


\ea%361
    \label{ex:key:361}
    \gll Yu  nó  gɛ́fɔ    pút=an    fɔ  plástik  yu  jɔ́s  gó
na  bús    yu  trowé=an.\\
\textsc{2sg}  \textsc{neg}  have.to  put=\textsc{3sg.obj}  \textsc{prep}  plastic  \textsc{2sg}  just  go
\textsc{loc}  forest  \textsc{2sg}  throw=\textsc{3sg.obj}\\

\glt ‘You don’t have to put it into a plastic (bag), you just go to the 
forest and throw it away.’ [hi03cb 034]\is{egressive aspect}
\z

\subsection{Completive}\label{sec:6.4.3}

The verb \textit{fínis} ‘finish’ expresses completive aspect. The use of \textit{fínis} as a lexical verb with the meaning ‘finish’ is exemplified in \REF{ex:key:362}: 

\ea%362
    \label{ex:key:362}
    \gll Bɔt  dá  mɔní  de  \textstylePichiexamplebold{fínis}    kwík.\\
but  that  money  \textsc{ipfv}  finish  quickly\\

\glt ‘But that money used to finish quickly.’ [ed03sp 088]
\z

The completive indicates the crossing of the terminal boundary of a situation and adds the nuance of completion. Compare \REF{ex:key:363}:


\ea%363
    \label{ex:key:363}
    \gll E    fínis  bɛ́n    di  písis        fáyn.\\
\textsc{3sg.sbj}  finish  bend  \textsc{def}  piece.of.cloth    fine\\

\glt ‘She has finished folding the piece of cloth real nice.’ [li07pe 043]
\z

The completive may signal a thorough consumption of the subject by the situation \REF{ex:key:364}. This is particularly so when \textit{fínis} co-occurs with the perfect marker \textit{dɔ́n} or with the emphatic imperfective \textit{dɔ́n de} ‘\textsc{pfv} \textsc{ipfv}’: 


\ea%364
    \label{ex:key:364}
    \gll Náw  a    dɔ́n    de  fínis  sém      fɔ  wɛ́r    dán    sús,
ɛf  a    bin  nó    a    fɔ    kɛ́r    ɔ́da    sús.\\
now  \textsc{1sg.sbj}  \textsc{prf}    \textsc{ipfv}  finish  be.ashamed  \textsc{prep}  wear  that    shoe
if  \textsc{1sg.sbj}  \textsc{pst}  know  \textsc{1sg.sbj}  \textsc{cond}    carry  other  shoes\\

\glt ‘Now I am really ashamed to be wearing those shoes, if I had known I would have 
brought other shoes.’ [ma03hm 021]
\z

The completive auxiliary \textit{fínis} ‘finish’ also co-occurs with the narrative perfective\is{perfective aspect} marker \textit{kán}:


\ea%365
    \label{ex:key:365}
    \gll Di  prɔ́blɛm  dɛn  dɔ́n  tú  mɔ́ch,  kán  fínis  {tɛ́l= àn}    sé    ‘lɛ́f’.\\
\textsc{def}  problem  \textsc{pl}  \textsc{prf}  too  much  \textsc{pfv}  finish  tell=3\textsc{sg.obj}  \textsc{quot}    leave\\

\glt ‘The problems had become too much, (I) then finally told him “leave”.’ [ma0313ni 035]\is{completive aspect}
\z

\subsection{Continuative}\label{sec:6.4.4}

The Spanish-origin dynamic verb \textit{sigue} ‘continue’ expresses continuative aspect. The continuative construction is usually encountered with dynamic verbs and inchoative-stative verbs with inherently more dynamic meanings (i.e. with change-of-state verbs but not with property items): 


\ea%366
    \label{ex:key:366}
    \gll A    sigue    plé    bɔ́l  sóté    ívin    tɛ́n.\\
\textsc{1sg.sbj}  continue    play    ball  until  evening  time\\

\glt ‘I continued playing ball until the evening.’ [be07fn 189]
\z

Alternatively, the preverbal temporal adverb \textit{stíl} ‘still’ may function as an auxiliary in its own right to express continuative aspect. Contrary to \textit{sigue}, the adverb \textit{stíl} is also found to modify stative verbs like the copula \textit{dé} in \REF{ex:key:367}:


\ea%367
    \label{ex:key:367}
    \gll Mi    gran-má    wet    mi    gran-pá    wé
dɛn  \textstylePichiexamplebold{stíl}  \textstylePichiexamplebold{dé}    láyf,    dɛn-ɔ́l    dɛn  dé    na  Panyá.\\
\textsc{1sg.poss}  grand-ma  with    \textsc{1sg.poss}  grand-pa    \textsc{sub}
\textsc{3pl}  still  \textsc{be.loc}   life    \textsc{3pl.cpd-}all  \textsc{3pl}  \textsc{be.loc}  \textsc{foc}  Spain\\

\glt ‘My grandmother and my grandfather, when they were still alive, 
they were all in Spain.’ [fr03ft 038]
\z

When \textit{stíl} co-occurs with a dynamic verb, the verb is normally marked for imperfective\is{imperfective aspect} aspect \REF{ex:key:368}:


\ea%368
    \label{ex:key:368}
    \gll Ɛf  yu  \textstylePichiexamplebold{stíl} \textstylePichiexamplebold{de} smók,  yu  go  sík.\\
if  \textsc{2sg}  still  \textsc{ipfv}  smoke  \textsc{2sg}  \textsc{pot}  sick\\

\glt ‘If you continue smoking, you’ll be sick.’ [ro05ee 041]
\z

A negative continuative meaning is generally expressed by means of discontinous negation involving the degree and temporal adverb \textit{mɔ́} ‘again; more’ as in \REF{ex:key:369}:


\ea%369
    \label{ex:key:369}
    \gll E    \textstylePichiexamplebold{nó}  dé\textstylePichiexamplebold{}   \textstylePichiexamplebold{mɔ́}.\\
\textsc{3sg.sbj}  \textsc{neg}  \textsc{be.loc}  more\\

\glt ‘He’s no longer (here/there).’ [ye03cd 155]
\z

Like the preverbal adverb\is{preverbal adverbs} \textit{jís} ‘just’ (cf. \sectref{sec:6.4.2}), \textit{stíl} may also be preceded by TMA markers. Also like the former adverb, the latter appears with resumptive imperfective\is{resumptive imperfective marking} marking \REF{ex:key:370}: 


\ea%370
    \label{ex:key:370}
    \gll E    \textstylePichiexamplebold{de} \textstylePichiexamplebold{stíl} \textstylePichiexamplebold{de}  wáka.\\
\textsc{3sg.sbj}  \textsc{ipfv}  still  \textsc{ipfv}  walk\\

\glt ‘He’s still walking.’ [dj05ae 050]
\z

A gradual and inherently comparative \is{comparative constructions}nuance of the continuative aspect can be expressed by employing the quantifying adverb \textit{mɔ́-ɛn-mɔ́} ‘more and more’ \REF{ex:key:371}: 


\ea%371
    \label{ex:key:371}
    \gll Dís  bɔ́y,    ɛ́ni    dé  e    de  fáyn    mɔ́-ɛn-mɔ́.\\
this  boy    every  day  \textsc{3sg.sbj}  \textsc{ipfv}  be.fine  more-and-more\\

\glt ‘This boy, everyday he is getting more handsome.’ [ro05ee 046]\is{continuative aspect}
\z

\subsection{Prospective} 

The lexical verb \textit{wánt} ‘want’ participates in an auxiliary construction that expressess “prospective aspect”\is{prospective aspect} (\citealt{Comrie1976}:64–65), also referred to as “proximative” \citep[36]{Heine1994}. The prospective aspect denotes imminence of a situation:


\ea%372
    \label{ex:key:372}
    \gll Layk  háw    dɛn  wánt  kɛ́r    yu  na  hospital
yu  dɔ́n    dáy.\\
like    how    \textsc{3pl}  want  carry  \textsc{2sg}  \textsc{loc}  hospital
\textsc{2sg}  \textsc{prf}    die\\

\glt ‘As they’re about to carry you to hospital, you’re already dead.’ [ed03sb 100]
\z

The modal readings of desire and intention (cf. \sectref{sec:6.7.2.2}) and the aspectual reading of prospective are related in their meanings. Hence the difference between modal and prospective \textit{wánt} is not always clear-cut.


For example, \REF{ex:key:373} is uttered when the speaker looks at a photograph of a father and his daughter, who is very tall for her young age. A desire reading of \textit{wánt} as ‘want to’ is conceivable if \textit{lɔ́n} ‘be long; tall’ is seen as a property that can be controlled by the speaker (even if humorously). However, a prospective reading denoting imminence appears more reasonable. Note that the prospective aspect reading of \textit{wánt} triggers an imminent entry-into-state interpretation of the inchoative-stative verb \textit{lɔ́n} ‘be long; tall’:



\ea%373
    \label{ex:key:373}
    \gll E    \textstylePichiexamplebold{wɔ́nt} \textstylePichiexamplebold{lɔ́n}    lɛkɛ  in    papá.\\
\textsc{3sg.sbj}  want  be.long  like  \textsc{3sg.poss}  father\\

\glt ‘She’s about to become as tall as her father.’ [ma03fn 003]\is{auxiliaries}
\z

\section{Tense}\label{sec:6.5}

The tense system of Pichi is relational, and in principle, bipartite. There is only one form – the past marker \textit{bin} ‘\textsc{pst’} – which has the focal function of a tense marker. Past tense can be expressed by means of \textit{bin} ‘\textsc{pst}’ with any verb irrespective of its lexical aspect. Next to the past marker, the narrative perfective\is{perfective aspect} marker \textit{kán} ‘\textsc{pfv}’, factative\is{factative TMA} marking and other perfective aspectual readings (i.e. perfect\is{perfect tense-aspect}, egressive,\is{egressive aspect} and completive\is{completive aspect}) express past tense\is{past tense} by default. 


In contrast, there is no single form to mark non-past tense. Non-past marking is taken care of by a variety of means, none of which exclusively serves the expression of tense. Hence, the potential mood marker \textit{go} and the imperfective\is{imperfective aspect} marker \textit{de} express future tense next to their respective modal and aspectual functions. Present tense\is{present tense} arises by default through imperfective marking, either via factative TMA with (inchoative-)stative verbs, or through overt marking by markers that express imperfective readings (i.e. \textit{de} ‘\textsc{ipfv}’ and \textit{kin} ‘\textsc{hab}’). 


\tabref{tab:key:6.7} summarises the overt and default basic tense readings that arise through the use of core TMA marking with the three lexical aspect classes. Non-basic, mixed tense-aspect readings (i.e. past/future perfect, past/future progressive) are taken up in the relevant sections (cf. also \tabref{tab:key:6.5}):

%%please move \begin{table} just above \begin{tabular
\begin{table}
\caption{Overt and default tense marking}
\label{tab:key:6.7}

\begin{tabularx}{\textwidth}{QlQQQ}
\lsptoprule
Class/Tense & Past-before-past\is{past-before-past tense} & Past\is{past tense} & Present\is{present tense} & Future\is{future tense}\\
\midrule 
Stative & \textit{bin} ‘\textsc{pst}’ & \textit{bin} ‘\textsc{pst}’, \textit{kán} ‘\textsc{pfv}’, \textit{dɔ́n} ‘\textsc{prf’} & \textit{kin} ‘\textsc{hab’}, factative & \textit{go} ‘\textsc{pot}’\\
\tablevspace
Inchoative-stative & \textit{bin} {{‘}}\textit{\textsc{pst’}} & \textit{bin} {{‘}}\textit{\textsc{pst’}}\textit{, }\textit{kán} {{‘}}\textit{\textsc{pfv’}}\textit{, }\textit{dɔ́n} {{‘}}\textit{\textsc{prf’}}{{, factative}} & \textit{kin} ‘\textsc{hab’},\newline  \textit{de} ‘\textsc{ipfv’}, factative & \textit{go} ‘\textsc{pot’},\newline  \textit{de} ‘\textsc{ipfv’}\\
\tablevspace
Dynamic & \textit{bin} ‘\textsc{pst}’ & \textit{bin} ‘\textsc{pst}’, \textit{kán} ‘\textsc{pfv}’, \textit{dɔ́n} ‘\textsc{prf’}, factative & \textit{de} ‘\textsc{ipfv’}, \textit{kin} ‘\textsc{hab’} & \textit{go} ‘\textsc{pot}’,\newline \textit{de} ‘\textsc{ipfv}’\\
\lspbottomrule
\end{tabularx}
\end{table}
The following sections provide an overview of the general characteristics of tense marking in Pichi (\sectref{sec:6.5.1}) as well as the expression of past (\sectref{sec:6.5.2}), present (\sectref{sec:6.5.3}), and future tense (\sectref{sec:6.5.4}). The potential mood and future tense marker \textit{go} is covered in \sectref{sec:6.7.4.1} in the section on modality. In order to do justice to the workings of relative tense in Pichi (cf. \sectref{sec:6.5.1}), I use the labels “anterior”, “simultaneous”, and “posterior” interchangeably with “past”, “present”, and “future”, respectively, where necessary.

\subsection{Relational tense}\label{sec:6.5.1}

Tense is relational (or “relative”) in Pichi. Overt or default tense is assigned in relation to “event time” (\citealt{ChungTimberlake1985}) rather than speech time. Relational tense manifests itself in two ways. Firstly, in complex sentences, a subordinate clause is assigned tense in relation to the tense value of the main clause, and there is no need for corresponding overt tense or mood marking in the subordinate clause. Hence, there is no \textit{consecutio} \textit{temporum} in Pichi.


In \REF{ex:key:374}, the main clause is marked for past tense by \textit{bin} ‘\textsc{pst}’. The subordinate clause (which begins with \textit{wé} ‘\textsc{sub}’), although simultaneous with the main clause, is not also marked for past. Instead, the factative\is{factative TMA} marked stative verb \textit{dé} \textsc{‘be.loc’} is assigned present tense\is{present tense} by default, hence it is interpreted as simultaneous to the main clause verb \textit{sí} ‘see’: 



\ea%374
    \label{ex:key:374}
    \gll \MakeUppercase{A}   nó  \textstylePichiexamplebold{bin}  \textstylePichiexamplebold{sí}  mi    gran-má    wé  e  \textstylePichiexamplebold{  dé} láyf.\\
\textsc{1sg.sbj}  \textsc{neg}  \textsc{pst}  see  \textsc{1sg.poss}  grand-ma  \textsc{sub}  \textsc{3sg.sbj}  \textsc{be.loc}  life\\

\glt ‘I didn’t see my grandmother while she was alive.’ [ro05ee 147]
\z

In \REF{ex:key:375}, the main clause is also marked for past tense. This time, the subordinate clause (which begins with \textit{sé} ‘\textsc{quot}’) is posterior to the main clause. Posteriority is expressed via the use of the potential marker \textit{go}. Yet there is no additional past tense marking in the subordinate clause, and indeed, it would be ungrammatical. This in spite of the fact that both the main clause and the subordinate clause are set in the past from the vantage point of the speaker. Hence, the event in the main clause, not speech time, is the reference point for the tense assignment of the subordinate clause:


\ea%375
    \label{ex:key:375}
    \gll A    bin  chɛ́k  sé    rén  go  fɔ́l.\\
\textsc{1sg.sbj}  \textsc{pst}  check  \textsc{quot}    rain  \textsc{pot}  rain\\

\glt ‘I thought it might rain.’ [ma03hm 022]
\z

A second manifestation of relational tense in Pichi is the absence of explicit tense marking whenever context offers enough information on tense anchoring. Contextual information may be provided by time adverbials as in \REF{ex:key:376}. Here, yɛ́stadé náyt ‘yesterday night’ anchors time reference in the past. Consequently, the imperfective marked verb kɔ́l ‘call’ receives a present tense/simultaneous interpretation in relation to past tense anchoring. Further marking by bin is unnecessary, although possible (cf. \sectref{sec:6.5.2}):


\ea%376
    \label{ex:key:376}
    \gll Yɛ́stadé    náyt  wé  a    de  kɔ́l  yú,    yɛ́stadé    náyt,
nɔ́  dís  mɔ́nin  náw.\\
yesterday  night  \textsc{sub}  \textsc{1sg.sbj}  \textsc{ipfv}  call  \textsc{2sg.indp}  yesterday  night
\textsc{neg}  this  morning  now\\

\glt ‘Yesterday night, when I was calling you, yesterday night,
no this morning.’ [hi03cb 083]
\z

In \REF{ex:key:377}, past tense reference is established through the adverbial wán ívin tɛ́n ‘one evening’ and the factative marked, perfective, hence past tense dynamic verbs kɔmɔ́t ‘go out’ and gó ‘go’. The imperfective marked verb rích ‘arrive’ in the subsequent clause remains unspecified for tense and receives a simultaneous reading, once more in relation to the past tense anchoring provided by the preceding adverbial and factative-marked dynamic verbs:


\ea%377
    \label{ex:key:377}
    \gll Wán    ívin    tɛ́n    a    kɔmɔ́t  mɔ́,    a    gó  wáka,
wé  a    \textstylePichiexamplebold{de} \textstylePichiexamplebold{rích}   na  hós,    hía    Djunais  (...)\\
one    evening  time    \textsc{1sg.sbj}  go.out  more  \textsc{1sg.sbj}  go  walk
\textsc{sub}  \textsc{1sg.sbj}  \textsc{ipfv}  reach  \textsc{loc}  house  hear    \textsc{name}\\

\glt ‘One evening, I went out again, I went for a stroll, when I was arriving 
at the house, [I] hear Djunais [say that…].’ [ro05rt 001]
\z

\subsection{Past}\label{sec:6.5.2}

Two types of past tense expression exist in Pichi. The principal means of expressing past tense by default are factative marking (cf. e.g. \REF{ex:key:314}) and the use of the narrative perfective{\fff} marker kán ‘pfv’ (cf. \REF{ex:key:323}–\REF{ex:key:324}). With (inchoative-)stative verbs, factative TMA gives rise to present tense{\fff} reference by default. This is illustrated in \REF{ex:key:378} with the stative verb fíba ‘resemble’ and the inchoative-stative verbs lɛ́k ‘like’ and sabí ‘(get to) know’ in \REF{ex:key:378}. 


\ea%378
    \label{ex:key:378}
    \gll Mí    nó  sabí,  e    fíba    sé    e    nó  lɛ́k
tín     dɛn  fɔ  súp.\\
\textsc{1sg.indp}  \textsc{neg}  know  \textsc{3sg.sbj}  seem  \textsc{quot}    \textsc{3sg.sbj}  \textsc{neg}  like
thing   \textsc{pl}  \textsc{prep}  soup\\

\glt ‘I \textsc{[emp]} don’t know, it seems that she doesn’t like soupy things.’ [ma03hm 059]
\z

If, on the contrary, pragmatic context suggests an inchoative reading of inchoative-stative verbs, a past tense interpretation is also possible. The change-of-state verb \textit{brók ‘}break; be broken‘ has factative TMA in the following example. Without contextual information the clause \textit{e brók} could mean either ‘it broke’ or ‘it is broken’. However, in this example, factative past tense marking on the preceding dynamic verbs \textit{ték} ‘take’ and \textit{nák} ‘hit’ only allows the first translation of \textit{di plét brók}: 


\ea%379
    \label{ex:key:379}
    \gll E    \textstylePichiexamplebold{ték}    di  háma,  e    \textstylePichiexamplebold{nák}  ɔntɔ́p  di  tébul,
di  plét    \textstylePichiexamplebold{brók}.\\
\textsc{3sg.sbj}  take    \textsc{def}  hammer  \textsc{3sg.sbj}  hit  on    \textsc{def}  table
\textsc{def}  plate  break\\

\glt ‘She took the hammer, she hit [it] on the table, (and) the plate broke.’ [ra07se 023]
\z

Factative-marked stative verbs have a default present tense reference in relation to event time. Hence past tense reference can only be established for stative verbs by means of explicit past tense marking (i.e. via bin ‘pst’) or by means of contextual cues in the clause. In \REF{ex:key:380}, the time adverbial dán tɛ́n ‘that time’ anchors time reference in the past, so the stative copula dé is interpreted as simultaneous to this tense anchor:


\ea%380
    \label{ex:key:380}
    \gll Dán    tɛ́n    a    dé    fáyn.\\
that    time    \textsc{1sg.sbj}  \textsc{be.loc}  fine\\

\glt ‘That time, I was fine.’ [ru03wt 024]\is{factative TMA}
\z

Secondly, past tense may be explicitly marked by means of the past marker \textit{bin} ‘\textsc{pst’}, which encodes relational past tense. \textit{Bin} is not obligatory in clauses with past reference. Instead, its use depends on discourse-pragmatic factors. The past marker is generally employed in temporally remote, backgrounded, orienting, and supportive sections of narratives. In this function, \textit{bin} ‘\textsc{pst}’ is diametrically opposed to the narrative perfective marker \textit{kán} ‘\textsc{pfv}’. It should therefore come as no surprise that \textit{bin} has a default imperfective reading next to its function as a past tense marker. Consider sentence \REF{ex:key:381}: 

\ea%381
    \label{ex:key:381}
    \gll Mí    \textstylePichiexamplebold{bin} \textstylePichiexamplebold{dé}    dé,    a    \textstylePichiexamplebold{bin} {\textup{\textsuperscript{1}}\textstylePichiexamplebold{mék}}  dásɔl,  dís,
a    \textstylePichiexamplebold{de} {\textup{\textsuperscript{2}}\textstylePichiexamplebold{mék}}  fínga  dɛn,    manicura.\\
\textsc{1sg.indp}  \textsc{pst}  \textsc{be.loc}  there  \textsc{1sg.sbj}  \textsc{pst}  make  only    this
\textsc{1sg.sbj}  \textsc{ipfv}  make  finger  \textsc{pl}    manicure\\

\glt ‘(As for) me, (when) I was there, I only made, this, 
I used to make fingers, manicure.’ [ma03hm 055]
\z

Sentence \REF{ex:key:381} above is part of an orienting section of a narrative and provides background information to a story. The stative copula \textit{dé} and the first dynamic verb (marked by superscript as \textsuperscript{1}\textit{mek} ‘make’) are overtly marked for past tense with \textit{bin}. Once the use of \textit{bin} with these two verbs has anchored the sentence (and in fact, the entire following narrative) in the past, overt past tense marking is unnecessary with subsequent verbs as is the case with the second dynamic verb (marked by superscript as \textsuperscript{2}\textit{mek} in \REF{ex:key:381}). 


The fact that \textit{bin} also incorporates imperfective aspect transpires in the TMA marking choices of the sentence. All three verbs denote situations simultaneous to each other, an aspect relation that usually requires imperfective marking with dynamic verbs. However, \textsuperscript{1}\textit{mek} is only marked for past tense with \textit{bin}, whereas \textsuperscript{2}\textit{mek}, which is devoid of past tense marking, must be marked for imperfective aspect via \textit{de} in order to express simultaneity of the situation.\is{imperfective aspect}



While past reference may be established by factative TMA alone with dynamic verbs, overt past tense marking is often encountered with stative verbs where the occurrence of the unmarked form would give rise to ambiguity. In \REF{ex:key:382}, \textit{wánt} ‘want’ is explicitly marked for past tense by \textit{bin}, because the unmarked form would favour a present tense\is{present tense}, simultaneous reading. The same holds for the copula verb \textit{dé} in \REF{ex:key:383}: 



\ea%382
    \label{ex:key:382}
    \gll Mí    dú=an    fɔséko  sé    a    bin  wánt  hɛ́lp=an.\\
\textsc{1sg.indp}  do=\textsc{3sg.obj}  due.to  \textsc{quot}    \textsc{1sg.sbj}  \textsc{pst}  want  help=\textsc{3sg.obj}\\

\glt ‘I \textsc{[emp]} did it because I wanted to help him.’ [ro05ee 069]
\z


\ea%383
    \label{ex:key:383}
    \gll \MakeUppercase{A}   kán  kɔmɔ́t  na  dán    hós    wé  a    \textstylePichiexamplebold{bin} \textstylePichiexamplebold{dé}.\\
\textsc{1sg.sbj}  \textsc{pfv}  go.out  \textsc{loc}  that    house  \textsc{sub}  \textsc{1sg.sbj}  \textsc{pst}  \textsc{be.loc}\\

\glt ‘I left that house where I had been.’ [ab03ay 097]
\z

\textit{Bin} can also express past-before-past tense \is{past-before-past tense}when specifying a situation that is set in the past. In \REF{ex:key:384}(a), perfect marking with the dynamic verb \textit{dáy} ‘die’ anchors time reference in the past. The subsequent clause (b) featuring the stative copula verb \textit{dé} is marked for \textit{bin.} Hence, the situation referred to by \textit{dé} \textsc{‘be.loc’} is anterior to \textit{dáy} ‘die’ in the preceding clause. 


\ea%384
    \label{ex:key:384}
\ea{\label{ex:key:384a}
    \gll Náw    e    dɔ́n  dáy  sɛ́f.\\
  now    \textsc{3sg.sbj}  \textsc{prf}  die  \textsc{emp}\\

\glt   ‘Now he’s even dead.’ [ma03sh 016]
}
\ex{\label{ex:key:384b}
\gll
E    \textstylePichiexamplebold{bin} \textstylePichiexamplebold{dé}    na  jél.\\
  \textsc{3sg.sbj}  \textsc{pst}    \textsc{be.loc}  \textsc{loc}  jail\\
\glt \textstylePichitranslationZchn{  He had been to prison.’ [ma03sh 017]}
}
\z
\z

\textit{Bin} marks past-before-past in the same way in \REF{ex:key:385}. Here, \textit{pás} ‘pass’ in (b) is anterior to the past tense point of reference provided by \textit{sík} ‘be sick’ in (a). In this example, we once more witness relational tense at work:


\ea%385
    \label{ex:key:385}
\ea{\label{ex:key:385a}
    \gll Wán    dé  wán    pikín  \textstylePichiexamplebold{bin} \textstylePichiexamplebold{de} \textstylePichiexamplebold{sík}.\\
  one    day  one    child  \textsc{pst}  \textsc{ipfv}  sick\\

\glt   ‘One day a child was sick.’ [fr03cd 071]
}
\ex{\label{ex:key:385b}
\gll
A    nó  sabí    ús=káyn  tín    \textstylePichiexamplebold{bin} \textstylePichiexamplebold{pás}.\\
  \textsc{1sg.sbj}  \textsc{neg}  know  \textsc{q}=kind  thing  \textsc{pst}  pass\\

\glt   ‘I don’t know what had happened.’ [fr03cd 072]\is{past tense}
}
\z
\z

The past marker also plays an important role as a modal element. \textit{Bin} is used as a conditional modality marker in the \textsc{if-} and \textsc{then-}clauses of past (counterfactual) conditionals (cf. \sectref{sec:10.7.11}).

\subsection{Present}\label{sec:6.5.3}

Present tense\is{present tense} is not expressed by means of elements specialised to this function. Instead, present tense reference is established by default through a variety of means. Bare stative verbs (cf. \REF{ex:key:315}) and in the appropriate context inchoative-stative verbs \REF{ex:key:316} are assigned present tense by default when marked for factative\is{factative TMA} TMA. Present tense reference is also established with inchoative-stative verbs via the use of the imperfective aspec\is{imperfective aspect}t marker \textit{de} (cf. \REF{ex:key:336}) and with both lexical aspect classes by the use of the habitual\is{habitual aspect} marker \textit{kin} (cf. \REF{ex:key:344}–\REF{ex:key:345}). Dynamic verbs are assigned present tense by default when they appear with the imperfective marker \textit{de} (cf. \REF{ex:key:329}) and the habitual aspect marker \textit{kin} (cf. \REF{ex:key:342}). 

\subsection{Future}\label{sec:6.5.4}

Future tense may be expressed explicitly by means of the potential mood marker \textit{go} \textsc{‘pot’.} The marker can be used indiscriminately with stative \REF{ex:key:386}, inchoative-stative (cf. \textit{máred} ‘marry, be married’ in \REF{ex:key:390} below) and dynamic verbs \REF{ex:key:387}:


\ea%386
    \label{ex:key:386}
    \gll Mí    \textstylePichiexamplebold{go}  bí  dɔ́kta.\\
\textsc{1sg.indp}  \textsc{pot}  \textsc{be}  doctor\\

\glt ‘I’ll be doctor.’ [ro05ee 025]
\z


\ea%387
    \label{ex:key:387}
    \gll \'{I}n    go  chɔ́p=an,    e    nó  gɛ́t  nó  problema.\\
\textsc{3sg.indp}  \textsc{pot}  eat=\textsc{3sg.obj}  \textsc{3sg.sbj}  \textsc{neg}  get  \textsc{neg}  problem\\

\glt ‘He \textsc{[emp]} will/would eat it, he has no problem whatsoever 
[with this kind of food].’ [ro05rt 066]
\z

The expression of future tense is part of a field of interrelated mood and tense-marking functions (cf. \sectref{sec:6.7.4.1}). I assume that the expression of epistemic possibility is a central function of \textit{go}, which is reflected in the gloss ‘\textsc{pot}’. Nevertheless, the function of \textit{go} also leans strongly towards that of a future tense marker in certain contexts.


When a situation is set in a hypothetical frame, hence based on an inferred or explicit condition, the meaning of \textit{go} is modal. When context provides no such frame, the meaning of \textit{go} tilts towards a tense reading. This is particularly the case in the presence of time adverbials (e.g. \textit{tumɔ́ro} ‘tomorrow’ in \REF{ex:key:388}) or where an intention of the speaker may be deduced from context \REF{ex:key:389}:



\ea%388
    \label{ex:key:388}
    \gll E    \textstylePichiexamplebold{go} \textstylePichiexamplebold{púl}  yú=an      \textstylePichiexamplebold{tumɔ́ro}.\\
\textsc{1sg.sbj}  \textsc{pot}  pull  \textsc{2sg.indp}=\textsc{3sg.obj}  tomorrow\\

\glt ‘He’ll tell it [the story] to you tomorrow.’ [ye07de 018]
\z


\ea%389
    \label{ex:key:389}
    \gll Lɛ́f=an,    a    go  chɔ́p,  áfta    a    go  dríng.\\
leave=\textsc{3sg.obj}  \textsc{1sg.sbj}  \textsc{pot}  eat    then  \textsc{1sg.sbj}  \textsc{pot}  drink\\

\glt ‘Leave it, I will eat, then I will drink.’ [ye03cd 080]
\z

Relational tense marking in Pichi allows a future projection from a speaker’s vantage point in the past\is{past tense} without the tense or mood change characteristic of \textit{consecutio} \textit{temporum} in languages with absolute tense systems. In \REF{ex:key:390}, the verb in the main clause is marked for past tense\is{past tense}. The verb in the subordinate clause introduced by \textit{sé} ‘\textsc{quot}’ is marked for future, not future-in-the-past: 


\ea%390
    \label{ex:key:390}
    \gll A    bin  de    chɛ́k  sé    a    go  máred.\\
\textsc{1sg.sbj}  \textsc{pst}  \textsc{ipfv}    check  \textsc{quot}    \textsc{1sg.sbj}  \textsc{pot}  marry\\

\glt ‘I was thinking that I would marry/get married.’ [fr03ft 165]\is{tense}
\z

The marker \textit{go} may also combine with \textit{de} ‘\textsc{ipfv’} to form a future imperfective\textit{,} {{as in \REF{ex:key:391} below,}}\textit{} and\textit{} with \textit{dɔ́n} ‘\textsc{prf}’ to form a future perfect (cf. e.g. \REF{ex:key:400}). \textit{Go} may also precede any of the aspectual auxiliaries covered in section \sectref{sec:6.4}.


\ea%391
    \label{ex:key:391}
    \gll Dɛn  tɛ́l  mí    sé    ɛf  a    pút=an,
a    \textbf{go}  \textbf{de}  kéch  Panya.\\
\textsc{3pl}  tell  \textsc{1sg.indp}  \textsc{quot}    if  \textsc{1sg.sbj}  put=\textsc{3sg.obj}
\textsc{1sg.sbj}  \textsc{pot}  \textsc{ipfv}  catch  Spain\\

\glt ‘The told me if I put it [the antenna], I will be receiving Spain.’ [ma0313ni 047]
\z

Other elements that may express future tense notions are the imperfective marker\is{imperfective aspect} \textit{de} (cf. e.g. \REF{ex:key:338}) and the prospective\is{prospective aspect} auxiliary \textit{wánt} (cf. e.g. \REF{ex:key:373}.

\section{Perfect}\label{sec:6.6}

The marker dɔ́n expresses the affirmative perfect, while the synonyms nɛ́a and nɔ́ba express negative perfect. The Pichi perfect is a hybrid category that expresses aspectual and temporal notions simultaneously. The perfect expresses the aspectual notion of completive aspect{\fff} in combination with the temporal notion of relevance to event time. 


The perfect is encountered with dynamic verbs, where it highlights the current relevance of the completed situation \REF{ex:key:392}:



\ea%392
    \label{ex:key:392}
    \gll Di  aráta  \textstylePichiexamplebold{dɔ́n}  \textstylePichiexamplebold{kɔmɔ́t}    ínsay  di  hól.\\
\textsc{def}  rat    \textsc{prf}  come.out  inside  \textsc{def}  hole\\

\glt ‘The rat has come out of the hole [it is outside now].’ [ro05ee 085]
\z

The combination of perfect marking with an inchoative-stative verb usually yields a resultant state interpretation \REF{ex:key:393}. 


\ea%393
    \label{ex:key:393}
    \gll Ɛ,  dán    bɔ́y    \textstylePichiexamplebold{dɔ́n}  \textstylePichiexamplebold{kɔ́t}  ó.\\
\textsc{intj}  that    boy    \textsc{prf}  cut  \textsc{sp}\\

\glt ‘Hey, that guy is badly cut.’ [dj05ce 226]
\z

In combination with stative verbs, perfect marking may convey a sense of total affectation of the referent by the state. In \REF{ex:key:394}, this sense is reinforced through the presence of the degree adverb \textit{bád} ‘extremely’:


\ea%394
    \label{ex:key:394}
    \gll Dán    gál,  e    dɔ́n    lɛ́k=an    bád.\\
That  girl  \textsc{3sg.sbj}  \textsc{prf}    like=\textsc{3sg.obj}  extremely\\

\glt ‘That girl, he really loves her.’ [bo07fn 232]
\z

Perfect marking is asymmetrical in Pichi. The marker dɔ́n ‘prf’ may not appear next to the negator nó ‘neg’. The negative affirmative marker is therefore in complementary distribution with the forms nɛ́a and nɔ́ba, which both function as negative perfect markers. Negative perfect marking often yields the meaning ‘not yet’: 


\ea%395
    \label{ex:key:395}
    \gll E    de  fɔgɛ́t  sé    Rubi    nɔ́ba  chɔ́p.\\
\textsc{3sg.sbj}  \textsc{ipfv}  forget  \textsc{quot}    \textsc{name}  \textsc{neg}.\textsc{prf}  eat\\

\glt ‘He forgets that Rubi has not yet eaten.’ [dj03cd 148]
\z

The negative restriction on \textit{dɔ́n} ‘\textsc{pfv}’ is suspended when it co-occurs with a tense\is{tense} or mood marker. In that case, the ordering rules applying to TMA markers forestall adjacency of the negator and the perfect marker. Examples follow with \textit{bin} ‘\textsc{pst}’ \REF{ex:key:396} and \textit{go} ‘\textsc{pot}’ \REF{ex:key:397}:


\ea%396
    \label{ex:key:396}
    \gll Ɛf  e    bin  kán  listin  wí,    e    \textstylePichiexamplebold{nó}  \textstylePichiexamplebold{bin}
\textstylePichiexamplebold{fɔ} \textstylePichiexamplebold{dɔ́n}  \textstylePichiexamplenumberZchnZchn{dáy} náw    só.\\
if  \textsc{3sg.sbj}  \textsc{pst}  \textsc{pfv}  listen  \textsc{1pl.indp}  \textsc{3sg.sbj}  \textsc{neg}  \textsc{pst}
\textsc{prep}  \textsc{prf}  die  now    like.that\\

\glt ‘If he had listened to us, he would not be dead now.’ [dj05ae 058]
\z


\ea%397
    \label{ex:key:397}
    \gll Mék    yu  nó  kán    a  las     cinco,  dán  tɛ́n  a    nó
\textstylePichiexamplebold{go}  \textstylePichiexamplebold{dɔ́n}  \textstylePichiexamplenumberZchnZchn{fínis}.\\
\textsc{sbjv}    \textsc{2sg}  \textsc{neg}  come  at  the.\textsc{pl}  five    that  time  \textsc{1sg.sbj}  \textsc{neg}
\textsc{pot}  \textsc{prf}  finish\\

\glt ‘Don’t come at five o’clock, (at) that time I won’t have finished yet.’ [he07fn 276]
\z

The clause-final adverbial yét ‘yet’ may reinforce the negative perfect without contributing additional meaning \REF{ex:key:398}. A negated factative{\fff} marked verb in conjunction with yét \REF{ex:key:398} can by itself be functionally very similar to the negative perfect expressed by nɛ́a/nɔ́ba:


\ea%398
    \label{ex:key:398}
    \gll Yu  sísta    e    nɛ́a    máred  yét?\\
\textsc{2sg}  sister  \textsc{3sg.sbj}  \textsc{neg}.\textsc{prf}  marry  yet\\

\glt ‘Your sister isn’t married yet?’ [dj05ce 066]
\z


\ea%399
    \label{ex:key:399}
    \gll E    \textstylePichiexamplebold{nó}  \textstylePichiexamplebold{máred}  \textstylePichiexamplebold{yét}?\\
\textsc{3sg.sbj}  \textsc{neg}  marry  yet\\

\glt ‘She isn’t married yet?’ [dj05ce 064]
\z

The perfect marker \textit{dɔ́n} may be cobined with other TMA markers. Compare the future\is{future perfect tense-aspect} perfect in \REF{ex:key:400} and the past perfect\is{past perfect tense-aspect} in \REF{ex:key:401}: 


\ea%400
    \label{ex:key:400}
    \gll Las    cuatro  wi  go  dɔ́n  dé    dé,    mí    sɛ́f  a    wánt,
a    gɛ́fɔ    gó  na  hós.\\
the\textsc{.pl}  four    \textsc{1pl}  \textsc{pot}  \textsc{prf}  \textsc{be.loc}  there  \textsc{1sg.indp}  \textsc{emp}  \textsc{1sg.sbj}  want
\textsc{1sg.sbj}  have.to  go  \textsc{loc}  house\\

\glt ‘At four o’clock we will already be there, I myself want, I have to go home.’ [ma 03ni 005]
\z


\ea%401
    \label{ex:key:401}
    \gll Di  tín    wé  a    \textstylePichiexamplebold{bin}  \textstylePichiexamplebold{dɔ́n}  fɔ́s  sí  wé  a    bin
dɔ́n    tráy=an  (...)\\
\textsc{def}  thing  \textsc{sub}  \textsc{1sg.sbj}  \textsc{pst}  \textsc{prf}  first  see  \textsc{sub}  \textsc{1sg.sbj}  \textsc{pst}
\textsc{prf}    try=\textsc{3sg.obj}\\

\glt ‘The thing that I had first seen when I had tried it (...)’ [ed03sb 188]
\z

With dynamic verbs, the combination of \textit{dɔ́n} ‘\textsc{prf}’ with the imperfective aspect marker \textit{de} renders a perfect progressive\is{perfect progressive} meaning. The combination of the notion of current relevance and progressivity in the marker sequence \textit{dɔ́n de} \textsc{‘prf} \textsc{ipfv’} renders an emphatic imperfective with dynamic verbs. It signals that the situation designated by the verb is (already) in full course \REF{ex:key:402} or on the brink of unfolding \REF{ex:key:403}. Note that the situation in \REF{ex:key:402} is set in the past, hence the sequence of the three TMA markers \textit{bin} \textit{dɔ́n} \textit{de} in \REF{ex:key:402}:


\ea%402
    \label{ex:key:402}
    \gll Wé  e    bin  dɔ́n  de  gó,  e    tɛ́l  mí    sé  di  tín    wé
e    fít  gí  mi,    e    wánt  lɛ́f    mi    sɔn    ríng.\\
\textsc{sub}  \textsc{3sg.sbj}  \textsc{pst}  \textsc{prf}  \textsc{ipfv}  go  \textsc{3sg.sbj}  tell  \textsc{1sg.indp}  \textsc{quot}  \textsc{def}  thing  \textsc{sub}
\textsc{3sg.sbj}  can  give  \textsc{1sg.indp}  \textsc{3sg.sbj}  want  leave  \textsc{1sg.indp}  some  ring\\

\glt ‘When he was just about to go, he told me that the thing he could give me, 
he wanted to leave me a ring.’ [ed03sb 193]
\z


\ea%403
    \label{ex:key:403}
    \gll Di  bɔ́y  dé    dé    e    \textstylePichiexamplebold{dɔ́n}  \textstylePichiexamplebold{de}  \textstylePichiexamplebold{dáy}.\\
\textsc{def}  boy  \textsc{be.loc}  there  \textsc{3sg.sbj}  \textsc{prf}  \textsc{ipfv}  die\\

\glt ‘The boy is just there in his death throes.’ [ye03cd 075]
\z

This perfect progressive\is{perfect progressive} sense is sometimes additionally reinforced by placing the marker sequence \textit{dɔ́n de} before the completive auxiliary verb \textit{fínis} ‘finish’\is{completive aspect}: 


\ea%404
    \label{ex:key:404}
    \gll Náw    a    dɔ́n    de  fínis  sém      fɔ  wɛ́r    dán    sús,
ɛf  a    bin  nó    a    fɔ  kɛ́r    ɔ́da    sús.\\
now    \textsc{1sg.sbj}  \textsc{prf}    \textsc{ipfv}  finish  be.ashamed  \textsc{prep}  wear  that    shoe
if  \textsc{1sg.sbj}  \textsc{pst}  know  \textsc{1sg.sbj}  \textsc{prep}  carry  other  shoe\\

\glt ‘Now I am completely ashamed to be wearing those shoes, if I had known
I would have brought another (pair of) shoes.’ [ma03hm 021]
\z

Perfect marking plays an important role in narrative discourse. The marker \textit{dɔ́n} appears in backgrounded, scene-setting and out-of-sequence discourse sections. Sentence \REF{ex:key:405} begins with an adverbial time clause. It provides background information to the subsequent main clause that is part of the foregrounded main line of the story:


\ea%405
    \label{ex:key:405}
    \gll Wé  a    dɔ́n  jɔ́ch    dɛ́n,    a    sé    tumɔ́ro    sénwe
a    de  gó  mít  in    mán.\\
\textsc{sub}  \textsc{1sg.sbj}  \textsc{prf}  judge  \textsc{3pl.indp}  \textsc{1sg.sbj}  \textsc{quot}    tomorrow  \textsc{emp}
\textsc{1sg.sbj}  \textsc{ipfv}  go meet  \textsc{3sg.poss}  man\\

\glt ‘When I had judged [scolded] them, I said tomorrow \textsc{[emp]} 
I’m going to meet her husband. [ro05rt 023]\is{perfect tense-aspect}
\z

\section{Modality}\label{sec:6.7}

The modal system of Pichi employs functional elements to express mood, and lexical words to express various types of modality. In my classification of modality into the dynamic, deontic, and epistemic categories, I rely on \citep{Palmer2001}.\is{modality}


Pichi has two overtly marked major mood distinctions. The subjunctive\is{subjunctive mood} mood is employed in the realm of deontic modality. The potential mood serves to express interrelated meanings in the domains of epistemic modality and tense\is{tense}. Two minor moods are the abilitive and obligative moods which are encoded in the preverbal elements \textit{kin} ‘\textsc{abl’} (which otherwise expresses habitual aspect) as well as \textit{mɔs} ‘\textsc{obl}’ and \textit{fɔ} ‘\textsc{prep}’. Aside from that, modal verbs and adverbials encode various types of modality. In the Pichi modal system, a number of TMA markers, rather than a single one, therefore share the semantic space of irrealis modality. Subjunctive–indicative and potential–factual are the most general and most systematically applied mood distinctions. Besides that, the imperfective aspect marker \textit{de}\is{imperfective aspect}, factative TMA, and the past marker \textit{bin} fulfil distinct functions in the modal system of Pichi. 


\subsection{Modal elements}\label{sec:6.7.1}

In Pichi, modality is instantiated in adverbs and particles, clause linkers, TMA markers, and modal auxiliary verbs. An overview of the inventory of modal elements according to the modal categories they express is provided in \tabref{tab:key:6.8}. Elements appearing in the same line co-occur in the corpus, with the exception of the sentential modal elements ó ‘sp’ and sɛ́f ‘emp’, which express assertion and may co-occur freely with other elements in the table.


Conditional modality has been included in the table for the sake of completeness and is covered separately in sections \sectref{sec:10.7.11} and \sectref{sec:10.7.12} on adverbial clauses and relations. Details on the subjunctive mood are provided in sections \sectref{sec:6.7.3.3}, \sectref{sec:10.5.5}, and \sectref{sec:10.7.6}.


%%please move \begin{table} just above \begin{tabular
\begin{sidewaystable}
\caption{Modal categories and elements}
\label{tab:key:6.8}
\footnotesize
\begin{tabularx}{\textwidth}{llQQQQ}
\lsptoprule
&Modal category/element & Verbs \is{modal verbs} & TMA markers & Clause linkers & Sentential \\
\midrule 
\textbf{Dynamic} & Physical ability &  & \textit{kin} {‘}{\textsc{abl’}} &  & \\
&& \textit{fít} {‘can’} &  &  & \\
&& \textit{hébul} {‘be capable’} &  &  & \\
&& \textit{mánech} {‘manage’} &  &  & \\
&Root possibility & \textit{fít} {‘can’} &  &  & \\
&Mental ability & \textit{sabí}{ ‘know’} &  &  & \\
&desire & \textit{wánt} {‘want’} &  &  & \\
&intention & \textit{wánt} {‘want’} &  &  & \\
&& \textit{mín} {‘mean to’} &  &  & \\

\midrule
\textbf{Deontic} &   Obligation & gɛ́fɔ ‘have to’ &  &  & \\
&&  & \textit{fɔ} {‘}{\textsc{prep}}{’} &  & \\
&Strong obligation &  & \textit{mɔs} {‘}{\textsc{obl}}{’} &  & \\
&Necessity & gɛ́fɔ ‘have to’ & \textit{fɔ} {‘}{\textsc{prep}}{’} &  & \\
&& \textit{níd} {‘need to’} &  &  & \\
&Permission & \textit{fít} {‘can’} &  &  & \\
&& \textit{grí} {‘agree; allow}’ &  &  & \\
&& lɛ́f ‘allow’ &  &  & \\
&Directives\is{directives} &  &  & \textit{mék} {‘}{\textsc{sbjv}}{’} & \\

\midrule
\textbf{Epistemic} & Possibility & \textit{fít (bí)} {‘can} {(be)’} &  &  & \\
&& \textit{fíba} {‘seem’} &  &  & \\
&&  & \textit{go} {‘}{\textsc{pot}}{’,}\newline  \textit{go dɔ́n}\textit{} {‘}{\textsc{pot}}{} {\textsc{prf}}{’} &  & \textit{sɔntɛ́n} ‘perhaps’,\newline  \textit{mebi} ‘maybe’\\
&Certainty & gɛ́fɔ ‘have to’ & \textit{dɔ́n} {‘}{\textsc{prf}}{’} &  & \\
&Assertion &  &  &  & \textit{ó} ‘sp’; \textit{sɛ́f} ‘emp’ \\

\midrule
\textbf{Conditional}\is{conditional clauses} &  Real &  &  & \textit{ɛf} ‘if’, \textit{lɛ́k} (\textit{sé}) ‘if’ & \\
&&  &  & \textit{adɔnkɛ́} ‘even if’ & \\
&Potential &  & \textit{go} {‘}{\textsc{pot}}{’,} \textit{de} {‘}{\textsc{ipfv}}{’} & \textit{ɛf} ‘if’ & \\
&Counterfactual &  & \textit{bin} {‘}{\textsc{pst}}{’,} \textit{fɔ} {‘}{\textsc{cond’}} & \textit{ɛf} ‘if’ & \\
\lspbottomrule
\end{tabularx}
\end{sidewaystable}
\subsection{Dynamic modality}

Dynamic modality is concerned with the existence of factors internal to the subject with respect to the completion of the situation denoted by the reference verb (\citealt{Palmer2001}:76ff.). In the following, the dynamic modality categories of ability, desire and intention are covered. These categories are primarily expressed through modal auxiliary verbs.

\subsubsection{Ability}\label{sec:6.7.2.1}

Pichi has a three-way distinction of ability. The modal verb \textit{fít} ‘can’ expresses ability in a general sense, but it does not normally cover mental ability \REF{ex:key:406}: 


\ea%406
    \label{ex:key:406}
    \gll \MakeUppercase{A}   nó  bin  \textstylePichiexamplebold{fít} \textstylePichiexamplebold{tɔ́k},    bikɔs  a    nó  \textstylePichiexamplebold{fít} \textstylePichiexamplebold{tɔ́k},
a    kán    a    de  lúk    yú.\\
\textsc{1sg.sbj}  \textsc{neg}  \textsc{pst}  can  talk    because  \textsc{1sg.sbj}  \textsc{neg}  can  talk
\textsc{1sg.sbj}  come  \textsc{1sg.sbj}  \textsc{ipfv}  look    \textsc{2sg.indp}\\

\glt ‘I couldn’t talk, because I couldn’t talk, I came (and) was just looking at you.’ [ed03sb 165]
\z

The verbs \textit{hébul} ‘be capable’ \REF{ex:key:407} and \textit{mánech} ‘be capable; manage’ \REF{ex:key:408} are usually employed to express capacity rather than ability:


\ea%407
    \label{ex:key:407}
    \gll Yu  mamá  nó  go  \textstylePichiexamplebold{hébul}    \textstylePichiexamplebold{pé}  ɔ́l  dán    wók    
wé  di  mán dɔ́n  dú  fɔ  yú.\\
\textsc{2sg}  mother  \textsc{neg}  \textsc{pot}  be.capable  pay  all  that    work  
\textsc{sub}  \textsc{def}  man \textsc{prf}  do  \textsc{prep}  \textsc{2sg.indp}\\

\glt ‘Your mother won’t be able to pay all that work that the man has done 
for you.’ [ab03ay 021]
\z


\ea%408
    \label{ex:key:408}
    \gll \MakeUppercase{A}   nó  \textstylePichiexamplebold{mánech}  \textstylePichiexamplebold{mít}=an    tidé.\\
\textsc{1sg.sbj}  \textsc{neg}  manage  meet=\textsc{3sg.obj}  today\\

\glt ‘I didn’t manage to meet her today.’ [lo07fn 190]
\z

The modal verb \textit{fít} ‘can’ may also express root possibility. It predicates the existence of general (usually social) circumstances that affect the ability of the person involved to perform the situation denoted by the reference verb. The subject of the following sentence has been put to shame by being caught committing a moral offence: 


\ea%409
    \label{ex:key:409}
    \gll E    nó  \textstylePichiexamplebold{fít}  \textstylePichiexamplebold{dú}=an    mɔ́.\\
\textsc{3sg.sbj}  \textsc{neg}  can  do=\textsc{3sg.obj}  more\\

\glt ‘He can’t do it again [he wouldn’t dare do it again].’ [ro05rt 041]
\z

The verb \textit{sabí} ‘(get to) know how to’ is used to express mental or learned ability \REF{ex:key:410}. Compare the uses of the modal auxiliaries \textit{fít} and \textit{sabí} with the reference verb \textit{tɔ́k} ‘talk’ in \REF{ex:key:406} above and \REF{ex:key:410}:\is{auxiliaries}


\ea%410
    \label{ex:key:410}
    \gll Di  mán    e    nó  \textstylePichiexamplebold{sabí}    \textstylePichiexamplebold{tɔ́k}  Panyá.\\
\textsc{def}  man    \textsc{3sg.sbj}  \textsc{neg}  know  talk  Spanish\\

\glt ‘The man doesn’t know how to speak Spanish.’ [ye03cd 063]
\z

The corpus features a single instance in which the habitual\is{habitual aspect} marker \textit{kin} ‘\textsc{hab}’ is unequivocally used to express physical ability \REF{ex:key:411}. The use of \textit{kin} as a marker of abilitive mood\is{abilitive mood} is marginal and obsolescent. The abilitive \textit{cum} habitual function is, however, still widely attested in Krio. Both functions of \textit{kin} grammaticalised from the English ability modal \textit{can}. The fact that the habitual function alone was retained in Pichi might suggest that continuing contact with English has reinforced the ability function in Krio, while absence of contact with English has led to the erosion of the abilitive sense and expression by the modal verbs \textit{fít} ‘can’ and \textit{hébul} ‘be capable’ alone in Pichi.


\ea%411
    \label{ex:key:411}
    \gll Bifó    a    \textstylePichiexamplebold{kin}  \textstylePichiexamplebold{gráp},  a    de  sí  bíg  bíg  fáya.\\
before  \textsc{1sg.sbj}  \textsc{abl}  get.up  \textsc{1sg.sbj}  \textsc{ipfv}  see  big  \textsc{rep}  fire\\

\glt ‘Before I could get up, I was seeing a huge fire.’ [ab03ay 067]
\z

\subsubsection{Desire and intention}\label{sec:6.7.2.2}

The modal verb \textit{wánt} ‘want’ expresses the often indistinguishable notions of desire and intention \REF{ex:key:412}. The verb \textit{mín} ‘mean’ may also express intention \REF{ex:key:413}. Note the exceptional modal use of the imperfective\is{imperfective aspect} aspect in \REF{ex:key:413}, in a complement clause \is{„complement}introduced by \textit{sé} ‘\textsc{quot}’ where one would usually find a subjunctive\is{subjunctive mood} clause introduced by \textit{mék} ‘\textsc{sbjv}’:\is{auxiliaries}


\ea%412
    \label{ex:key:412}
    \gll \MakeUppercase{A}   \textstylePichiexamplebold{wánt}  \textstylePichiexamplebold{tɔ́k}  dán  smɔ́l  tɔ́k  dé.\\
\textsc{1sg.sbj}  want  talk  that  small  talk  there\\

\glt ‘I want to say that particular small word.’ [dj05ae 037]
\z


\ea%413
    \label{ex:key:413}
    \gll Dɛn  gɛ́fɔ    mín    sé    e    de  hambɔ́g  wí.\\
\textsc{3pl}  have.to  mean  \textsc{quot}    \textsc{3sg.sbj}  \textsc{ipfv}  bother  \textsc{1pl.indp}\\

\glt ‘They must mean for it [the dog] to bother us.’ [ma03hm 002]
\z

\subsection{Deontic Modality} 

Deontic modality is concerned with the existence of factors external to the subject which condition the completion of the situation denoted by the reference verb. The deontic category of obligation is expressed by means of the TMA marker mɔs ‘obl’, obligation and necessity by gɛ́fɔ ‘have to’ or the multifunctional element fɔ ‘prep’ alone. Permission is expressed through the verb fít ‘can’. Aside from that, the expression of deontic modality is characterised by the use of the subjunctive mood. Directives{\fff} as well as the entire range of manipulative-directive meanings covered by the complement-taking verbs listed in section \sectref{sec:10.5.1} induce the use of subjunctive clauses introduced by the modal complementiser mék ‘sbjv’. 

\subsubsection{Subjunctive mood}\label{sec:6.7.3.1}

Subjunctive mood is instantiated in the modal complementiser \textit{mék} ‘\textsc{sbjv}’ and the specific TMA marking properties of the subjunctive clause. Subjunctive mood appears in directive main clauses (cf. \sectref{sec:6.7.3.3}). It is also present in the subordinate clauses of deontic modality inducing main verbs (cf. \sectref{sec:10.5}), i.e. verbs whose meaning contains an element of causation, manipulation, proposal, desire and other affective nuances compatible with deontic modality. Thirdly, subjunctive mood occurs in purpose\is{purpose clauses} and consecutive clauses (cf. \sectref{sec:10.7.6}).\is{subjunctive mood}

\subsubsection{Obligation, necessity, and permission}\label{sec:6.7.3.2}

Obligation denotes the existence of compelling factors in the social world. Both strong and weak obligation are most commonly expressed through the verb gɛ́fɔ ‘have to’ \REF{ex:key:414}. Negative obligation is formed by standard negation of gɛ́fɔ and yields a prohibitive{\fff} meaning \REF{ex:key:415}: 


\ea%414
    \label{ex:key:414}
    \gll Ɛf  yu  gɛ́fɔ    baja    diez  veces  yu  gɛ́fɔ
\textstylePichiexamplebold{calcula}    dán  mɔní.\\
if  \textsc{2sg}  have.to  go.down  ten  time.\textsc{pl}  \textsc{2sg}  have.to  
calculate    that  money\\

\glt ‘If you have to go down ten times, you have to calculate that 
(amount of) money.’ [f103fp 006]
\z


\ea%415
    \label{ex:key:415}
    \gll E    nó  gɛ́fɔ    lúk    yú    na  fés.\\
\textsc{3sg.sbj}  \textsc{neg}  have.to  look    \textsc{2sg.indp}  \textsc{loc}  face\\

\glt ‘He [the child] shouldn’t look you in the face [while responding].’ [au07se 140]
\z

The verb gɛ́fɔ (<gɛ́t fɔ ‘get/have prep) is a lexicalised collocation also attested in Krio (\citealt{FyleJones1980}) and Cameroon Pidgin \citep{Nkengasong2016}. It was probably calqued from English ‘have to’ in the protolanguage, but has probably also been reinforced by Spanish tener que ‘have to’. The verb has the distribution of a monorphemic lexeme in contemporary Pichi. It may therefore be followed by de ‘ipfv’ in complement constructions. 


Alternatively, Pichi employs the two obligative mood markers \textit{fɔ} ‘\textsc{prep}’ \REF{ex:key:416} and \textit{mɔs} ‘\textsc{obl}’ (cf. \REF{ex:key:421} below) in order to express obligation. The marker \textit{fɔ} may express both weak and strong obligation. The function of \textit{fɔ} extends further to uses as a TMA marker to indicate counterfactual mood in the \textsc{then-}clause of conditionals\is{conditional clauses}, cf. the first and second occurrence of \textit{fɔ} in \REF{ex:key:416}:



\ea%416
    \label{ex:key:416}
\gll
Ɛf  dán    pikín  bin  tɔ́k  trú,    dɛn  \textstylePichiexamplebold{fɔ}    \textstylePichiexamplebold{púl}   dán \textstylePichiexamplenumberZchnZchn{  pikín,} 
\textstylePichiexamplenumberZchnZchn{dán} \textstylePichiexamplenumberZchnZchn{pikín} e \textstylePichiexamplenumberZchnZchn{nó} \textstylePichiexamplebold{bin}\textstylePichiexamplenumberZchnZchn{} \textstylePichiexamplebold{fɔ}    \textstylePichiexamplebold{dáy}\textstylePichiexamplenumberZchnZchn{.}\\
if  that    child  \textsc{pst}  talk  true    \textsc{3pl}  \textsc{cond}    remove  that child 
that    child  \textsc{3sg.sbj}  \textsc{neg}  \textsc{pst}  \textsc{cond}    die\\

\glt ‘If that child [girl] had told the truth, the child [foetus] would have been removed, 
(and) that child [girl] wouldn’t have died.’\textstylePichiglossZchn{ [ab03ay 121]}
\z

Impersonalised purposive\is{purpose clauses} constructions like \REF{ex:key:417} are likely to be one point of departure for the occurrence of \textit{fɔ} as a mood marker in finite clauses like \REF{ex:key:418}. The various uses of \textit{fɔ} as a clause linker form part of a web of interrelated functions of this element (cf. \sectref{sec:10.2} for an overview): 


\ea%417
    \label{ex:key:417}
    \gll \textstylePichiexamplebold{Na}  \textstylePichiexamplebold{fɔ}  \textstylePichiexamplebold{gó}  las    seis  y  media.\\
\textsc{foc}  \textsc{prep}  go  the\textsc{.pl}  six  and  half\\

\glt ‘It is in order to go at six thirty.’ [ye07fn 191]
\z


\ea%418
    \label{ex:key:418}
    \gll \'{A}fta    yu  fɔ  pé  dɛ́n.\\
then  \textsc{2sg}  \textsc{prep}  pay  \textsc{3pl.indp}\\

\glt ‘Then you have to pay them.’ [ye03cd 113]
\z

The element \textit{fɔ} ‘\textsc{prep}’ also appears with a directive tint in non-assertive contexts like direct\textit{} \REF{ex:key:419} and indirect \REF{ex:key:420} questions featuring the question word\is{question words} \textit{háw} ‘how’: 


\ea%419
    \label{ex:key:419}
    \gll \textstylePichiexamplebold{Háw}  a    \textstylePichiexamplebold{fɔ} dú,  \textstylePichiexamplebold{háw}    a    \textstylePichiexamplebold{fɔ} dú  wet=an?\\
How    \textsc{1sg.sbj}  \textsc{prep}  do  how    \textsc{1sg.sbj}  \textsc{prep}  do  with=\textsc{3sg.obj}\\

\glt ‘How should I do (it), how should I do [proceed] with him?’ [ab03ay 136]
\z


\ea%420
    \label{ex:key:420}
    \gll Yu  fít  hɛ́lp    mí,    a    nó  sabí    háw    fɔ  dú=an.\\
\textsc{2sg}  can  help    \textsc{1sg.indp}  \textsc{1sg.sbj}  \textsc{neg}  know  how    \textsc{prep}  do=\textsc{3sg.obj}\\

\glt ‘Can you help me, I don’t know how I should do it/how to do it.’ [ro05de 020]
\z

Certain characteristics speak for an analysis of fɔ as a TMA marker when it appears in the preverbal position in finite clauses. Like other TMA markers of Pichi, fɔ is monosyllabic and low-toned. Equally, it is subject to restrictions. Although fɔ ‘prep’ is attested together with bin ‘pst’ in order to express counterfactual conditional modality (cf. \REF{ex:key:416} above), it is not encountered with any other TMA marker – unlike modal verbs. Hence, we have e go gɛ́fɔ pé \{3sg.sbj pot have prep pay\} ‘she’ll have to pay’ but not *e go fɔ pé \{3sg.sbj pot prep pay\}. 


The same characteristics hold for the element \textit{mɔs} ‘\textsc{obl’}, which also expresses obligative mood. However, the use of \textit{mɔs} usually renders a strong obligation sense often coupled with a sense of internal compulsion \REF{ex:key:421}. Generally, speakers do not accept the use of \textit{mɔs} ‘\textsc{obl}’ in syntactic positions which would suggest a verbal status of this element either. For instance, like \textit{fɔ} above, \textit{mɔs} is not attested in conjunction with other TMA markers \REF{ex:key:422}: 



\ea%421
    \label{ex:key:421}
    \gll \MakeUppercase{A}   \textstylePichiexamplebold{mɔs}    \textstylePichiexamplebold{gó}  Alemania  wán    dé.\\
\textsc{1sg.sbj}  \textsc{obl}    go  \textsc{place}    one    day\\

\glt ‘I absolutely have to go to Germany one day.’ [to07fn 197]
\z


\ea[*]{%422
    \label{ex:key:422}
    \gll \MakeUppercase{A}   bin \textstylePichiexamplebold{mɔs}   gó  dé.\\
 \textsc{1sg.sbj}  \textsc{pst}  \textsc{obl}    go  there\\
\glt *I had to go there. [ne 07fn 196]
}
\z

Prohibitive\is{prohibitives} clauses featuring \textit{mɔs} ‘\textsc{obl}’ are formed like regular negative imperatives\is{imperatives} without a \textsc{2sg} personal pronoun \REF{ex:key:423}:


\ea%423
    \label{ex:key:423}
    \gll \textstylePichiexamplebold{Nó}  \textstylePichiexamplebold{mɔs}    \textstylePichiexamplebold{gó}  dán    sáy!  \\
\textsc{neg}  \textsc{obl}    go  that    side  \\

\glt ‘(You) must not go to that place! [ne 07fn 194]\is{obligative mood}
\z

Necessity may be differentiated from obligation by making use of the modal verb \textit{níd} ‘need (to)’ in affirmative \REF{ex:key:424} and negative \REF{ex:key:425} clauses. This modal auxiliary can be employed with same and different subject complement clauses\is{complement clauses} in accordance with the pattern outlined in examples \REF{ex:key:1392}–\REF{ex:key:1394}: \is{auxiliaries}


\ea%424
    \label{ex:key:424}
    \gll \MakeUppercase{A}   \textstylePichiexamplebold{níd}    \textstylePichiexamplebold{fɔ} mék    yu  gó  dé.\\
\textsc{1sg.sbj}  need  \textsc{prep}  \textsc{sbjv}    \textsc{2sg}  go  there\\

\glt ‘I need you to go there.’ [to07fn 200]
\z


\ea%425
    \label{ex:key:425}
    \gll Fɔ  tɔ́k  Píchi  yu \textstylePichiexamplebold{nó}  \textstylePichiexamplebold{níd}    \textstylePichiexamplebold{fɔ} gó  skúl.\\
\textsc{prep}  talk  Pichi  \textsc{2sg}  \textsc{neg}  need  \textsc{prep}  go  school\\

\glt ‘In order to talk Pichi you don’t need to go to school.’ [au07se 267]
\z

Permission is expressed by way of fít ‘can’, a causative/permissive construction involving lɛ́f ‘leave; allow’ (cf. \REF{ex:key:1332}ff. for details) or the main verb grí ‘agree; allow’ and a complement clause {\fff}\REF{ex:key:427} (cf. also \REF{ex:key:1389}). Note the presence of the imperfective marker de in the subjunctive clause in the second example: {\fff}


\ea%426
    \label{ex:key:426}
    \gll A    bɛ́g,    yu  go  ɛskyús  mí    pero  yu  nó  fít
ték=an    sóté    e    gɛ́t  quince  años.\\
\textsc{1sg.sbj}  beg    \textsc{2sg}  \textsc{pot}  excuse  \textsc{1sg.indp}  but    \textsc{2sg}  \textsc{neg}  can
take=\textsc{3sg.obj}  until  \textsc{3sg.sbj}  get  fifteen  years\\

\glt ‘Sorry, you’ll excuse me but you can’t take her along until she is fifteen
years old.’ [ab03ay 150]
\z


\ea%427
    \label{ex:key:427}
    \gll So  na  dán  tín    mék,  e    de  \textstylePichiexamplebold{grí}    \textstylePichiexamplebold{sé}    \textstylePichiexamplebold{mék}  
a    de  gí=an    smɔ́l  tín    ɔ́l  tɛ́n.\\
so  \textsc{foc}  that  thing  make  \textsc{3sg.sbj}  \textsc{ipfv}  agree  \textsc{quot}    \textsc{sbjv}  
\textsc{1sg.sbj}  \textsc{ipfv}  give\textsc{=3sg.obj}  small  thing  all  time\\

\glt ‘So that’s why she allows me to give her a small amount all the time.’ [ma03hm 061]\is{auxiliaries}
\z

\subsubsection{Directives}\label{sec:6.7.3.3}

Directives impose conditions of obligation on the addressee. The central form for expressing this modal category is the modal complementiser and subjunctive marker \textit{mék}. The subjunctive marker may be employed to express directives throughout the entire person-number paradigm, which renders the modal categories traditionally referred to as imperative\is{imperatives} (2\textsuperscript{nd} person directives) \REF{ex:key:428} and jussive\is{jussives} (1\textsuperscript{st} and 3\textsuperscript{rd} person directives) \REF{ex:key:429}–\REF{ex:key:430}. The addition of the sentence final particle\is{sentence particle} \textit{ó} gives directives an admonitive tinct \REF{ex:key:428}:


\ea%428
    \label{ex:key:428}
    \gll Mék    yu  mɛ́n=an      ó!\\
\textsc{sbjv}    \textsc{2sg}  care.for=\textsc{3sg.obj}  \textsc{sp}\\

\glt ‘Make sure to take care of her!’ [ab03ay 082]
\z


\ea%429
    \label{ex:key:429}
    \gll \textstylePichiexamplebold{Mék}    \textstylePichiexamplebold{a} gí    yú    di  cheque, (...)\\
\textsc{sbjv}    \textsc{1sg.sbj}  give    \textsc{2sg.indp}  \textsc{def}  cheque\\

\glt ‘Let me give you the cheque (...)’ [ye03cd 119]
\z

The subjunctive marker also introduces cohortatives\is{cohortatives} (1\textsuperscript{st} person plural invitations) \REF{ex:key:438} and optatives (1\textsuperscript{st}, 2\textsuperscript{nd}, 3\textsuperscript{rd} person wishes):


\ea%430
    \label{ex:key:430}
    \gll tín    fɔ  fɔ́s  tɛ́n    mék    e    dé,    bikɔs  pípul
de    kán    fɔ  kán    sí=an.\\
thing  \textsc{prep}  first time  make  \textsc{3sg.sbj}  \textsc{be.loc}  because  people
\textsc{ipfv}    come  \textsc{prep}  come  see=\textsc{3sg.obj}\\

\glt ‘(The) thing of the past, let it be, because people come to see it.’ [hi03cb 068]
\z

Subjunctive clauses must be employed for all directives except \textsc{2sg} and \textsc{2pl} imperatives. With imperatives, subjunctive clauses are optional. There appears to be no difference in meaning between bare and subjunctive marked imperatives. However, singular imperatives must be expressed by the bare verb without a personal pronoun if subjunctive marking is absent \REF{ex:key:431}. Conversely, \textsc{2pl} imperatives take the corresponding personal pronoun \REF{ex:key:432}:


\ea%431
    \label{ex:key:431}
    \gll \'{U}dat  tíf?    Tɛ́l  mí    di  ném!\\
Who  steal  tell  \textsc{1sg.indp}  \textsc{def}  \textsc{name}\\

\glt ‘Who stole (something)? Tell me the name!’ [fr03cd 049]
\z


\ea%432
    \label{ex:key:432}
    \gll \textstylePichiexamplebold{Una}    \textstylePichiexamplebold{mék}    chénch!\\
\textsc{2pl}    make  change\\

\glt ‘Swap [plural]!’ [ro05rt 025]
\z

Negative imperatives (prohibitives) are formed by placing the negator \textit{nó} before the verb \REF{ex:key:433} or by employing a negative subjunctive clause \REF{ex:key:434}: 


\ea%433
    \label{ex:key:433}
    \gll \textstylePichiexamplebold{Nó} láf!\\
\textsc{neg}  laugh\\

\glt ‘Don’t laugh!’ [ru03wt 022]
\z


\ea%434
    \label{ex:key:434}
    \gll \textstylePichiexamplebold{Mék}    \textstylePichiexamplebold{yu}  \textstylePichiexamplebold{nó}  pút  di  watá  mék    e    fɔdɔ́n
fuera  fɔ  di  glas.\\
\textsc{sbjv}    \textsc{2sg}  \textsc{neg}  put  \textsc{def}  water  sbjv    \textsc{3sg.sbj}  fall
outside  \textsc{prep}  \textsc{def}  glass\\

\glt ‘Don’t put the water (in such a way) that it drops outside 
of the glass.’ [dj05be 167]
\z

All other (i.e. 1\textsuperscript{st} and 3\textsuperscript{rd} person) directives may only be negated by means of a negative subjunctive clause \REF{ex:key:435}:


\ea%435
    \label{ex:key:435}
    \gll Mék    e    fɔdɔ́n  ínsay  di  glás,    \textstylePichiexamplebold{mék} \textstylePichiexamplebold{e}
\textstylePichiexamplebold{nó} \textstylePichiexamplenumberZchnZchn{fɔdɔ́n}  na  grɔ́n!\\
\textsc{sbjv}    \textsc{3sg.sbj}  fall    inside  \textsc{def}  glass  \textsc{sbjv}    \textsc{3sg.sbj}
\textsc{neg}  fall    \textsc{loc}  ground\\

\glt ‘Let it flow into the glass, don’t let it flow onto the floor!’ [dj05be 170]\is{prohibitives}
\z

Sequences of imperatives are frequent in discourse. Here, the final verb must be marked for subjunctive mood, while preceding verbs may optionally remain bare. In these circumstances, the subjunctive additionally functions as a marker of consecutive modality: 


\ea%436
    \label{ex:key:436}
    \gll \textstylePichiexamplebold{Tɔ́n}=an    \textstylePichiexamplebold{tɔ́n}=an    \textstylePichiexamplebold{mék} yu  nó  para!\\
turn=\textsc{3sg.obj}  turn=\textsc{3sg.obj}  \textsc{sbjv}    \textsc{2sg}  \textsc{neg}  stop\\

\glt ‘Stir, stir it, and don’t stop!’ [dj03do 058]
\z

The verb \textit{kán} ‘come’ \REF{ex:key:437} may be employed in a way that parallels the use of the subjunctive marker in syntactic position and function \REF{ex:key:438}. However, this usage is restricted to cohortatives\is{cohortatives}:


\ea%437
    \label{ex:key:437}
    \gll Ɛhɛ́,    kán    wi  sigue!\\
\textsc{intj}    come  \textsc{1pl}  continue\\

\glt ‘Let’s continue!’ [ye05ce 101] 
\z


\ea%438
    \label{ex:key:438}
    \gll \textstylePichiexamplebold{Mék} wi  sí!\\
\textsc{sbjv}    \textsc{1pl}  see\\

\glt ‘Let’s see!’ [ma03ni 002]
\z

The force of imperatives can be attenuated. An example follows in \REF{ex:key:439} of a weakened imperative involving the idiom a bɛ́g ‘please’ and the adverbial smɔ́l ‘a bit’:


\ea%439
    \label{ex:key:439}
    \gll A    bɛ́g, kán    yá    smɔ́l!\\
\textsc{1sg.sbj}  beg  come  here    a.bit\\

\glt ‘Please come here a bit [would you please come here?].’ [ch07fn 233]
\z

Alternatively, a directive may involve one of the politeness markers \textit{dúya} ‘please’ (cf. \REF{ex:key:1651} or \textit{plís} ‘please’ it may be couched in a question featuring the modal verb \textit{fít} ‘can’ \REF{ex:key:440}, or be formed through circumlocution featuring the verb \textit{tráy} ‘try’ \REF{ex:key:441}:


\ea%440
    \label{ex:key:440}
    \gll Yu  \textstylePichiexamplebold{fít}  pás  yá?\\
\textsc{2sg}  can  pass  here\\

\glt ‘Can you pass here?’ [ma03ni 001]\is{subjunctive mood}
\z


\ea%441
    \label{ex:key:441}
    \gll \textbf{Tráy}  reduce  ín!\\
tráy    reduce  \textsc{3sg.indp}\\

\glt ‘Try to reduce it [please reduce it]!’ [ru03wt 043]\is{directives}
\z

\subsection{Epistemic modality}

Epistemic modality serves the expression of a speaker’s commitment to asserting a given situation. The epistemic notions of possibility, certainty and assertion are covered in the following four sections. Part of the expression of epistemic possibility accrues to the potential mood marker \textit{go}, which is also employed to express future tense.

\subsubsection{Potential mood}\label{sec:6.7.4.1}

The central function of the TMA marker \textit{go} ‘\textsc{pot}’ is the expression of potential mood, hence the epistemic notion of possibility. With this analysis, I follow \citet{Essegbey2008}, who analyses a functionally similar\textit{} morpheme of Ewe as an instantiation of the potential mood. From this point of departure, the marker \textit{go} ‘\textsc{pot}’ expresses additional related modal and temporal notions like future tense, conditional, hypothetical, and habitual\is{habitual aspect}. 


The following sentence illustrates the modal use of go ‘pot’. In the example, speaker (ge) explains what prompted her to leave her teenage daughter in Madrid instead of bringing her along with her to Malabo on vacation. Obviously, speaker (ge) is not making a prediction; this is corroborated by the presence of the experiential verb fía ‘fear’. Rather, the verb bɛlɛ́ ‘impregnate’ is marked by go ‘pot’ in order to express an epistemic possibility:  



\ea%442
    \label{ex:key:442}
    \gll A    fía  sé    dɛn  go  bɛlɛ́      mi    pikín  fɔ  mí.\\
\textsc{1sg.sbj}  fear  \textsc{quot}    \textsc{3pl}  \textsc{pot}  impregnate  \textsc{1sg.poss}  child  \textsc{prep}  \textsc{1sg.indp}\\

\glt ‘I feared that my child might be impregnated (on me).’ [ge05be 055]
\z

In this example, the potential mood expresses an epistemic possibility, rather than a prediction, in a similar way: 


\ea%443
    \label{ex:key:443}
    \gll (...)  mék    yu  tɔ́n=an,    porque  bɔtɔ́n  \textstylePichiexamplebold{go}  \textstylePichiexamplebold{rós}.\\
  {} \textsc{sbjv}    \textsc{2sg}  turn=\textsc{3sg.obj}  because  bottom  \textsc{pot}  burn\\

\glt ‘(...) turn it, because the bottom might burn.’ [dj03do 055]
\z

The marker go frequently occurs with the epistemic adverbs sɔntɛ́n ‘perhaps’ and mebi ‘maybe’ in order to indicate a future \REF{ex:key:444} or a present possibility \REF{ex:key:445}: 


\ea%444
    \label{ex:key:444}
    \gll Pero    bambáy     bambáy  sɔntɛ́n  yu  go  sí  di  wán
wé  \textstylePichiexamplebold{go}  máred  yú.  \\
but    gradually  \textsc{rep}    perhaps  \textsc{2sg}  \textsc{pot}  see  \textsc{def}  one  
\textsc{sub}  \textsc{pot}  marry  \textsc{2sg.indp}\\

\glt ‘But very gradually perhaps you will find the one who will marry you.’ [ab03ab 204]
\z


\ea%445
    \label{ex:key:445}
    \gll Porque  mébi  a    go  wánt  fɛ́n    di  ném.\\
because  maybe  \textsc{1sg.sbj}  \textsc{pot}  want  look.for  \textsc{def}  name\\

\glt ‘Maybe I might want to find the name [for this word, you never know].’ [au07se 007]
\z

Since \textit{go} alone can express potential mood and future tense\is{future tense}, the TMA marker sequence \textit{go dɔ́n} ‘\textsc{pot} \textsc{prf}’ can indicate a future perfect \is{future perfect}(cf. \REF{ex:key:400}) or a potential perfect. The latter use of potential mood produces a reading of inferred certainty (cf. also \sectref{sec:6.7.4.3}).


\ea%446
    \label{ex:key:446}
    \gll E    go \textstylePichiexamplebold{dɔ́n}  \textstylePichiexamplebold{drɔ́ngo},      e    go \textstylePichiexamplebold{dɔ́n} \textstylePichiexamplebold{slíp}.\\
\textsc{3sg.sbj}  \textsc{pot}  \textsc{prf}  be.dead.drunk  \textsc{3sg.sbj}  \textsc{pot}  \textsc{prf}  sleep\\

\glt ‘He should be dead drunk, he should already be sleeping.’ [ge07fn 088]
\z

Besides its use as a potential mood marker and future tense marker in predictions (cf. \sectref{sec:6.5.3}), hypothetical statements are among the most common contexts in which \textit{go} ‘\textsc{pot’} occurs. A common form of expressive communication in Pichi involves the use of emphatic speech and figurative language and is set within a potential (or hypothetical) modal frame.


The following discourse excerpt involves two speakers who hypothesise about the potential advantage of having a pair of sunglasses that would allow them to see people naked. The use of the linker \textit{if} ‘if’ signals entry into the realm of potential modality \REF{ex:key:447}(a), which is repeatedly marked by \textit{go} in (a), (c) and (e). Note the presence of other modal elements, such as \textit{fít} ‘be able, possible’ in (a), the imperfective marker \textit{de} instead of \textit{go} in (d), and the use of the factative marked stative verb \textit{wánt} ‘want’ with a potential meaning once this modal frame has been established (f): 



\ea%447
    \label{ex:key:447}
\ea{\label{ex:key:447a}
    \gll \MakeUppercase{A}   fít  sé    \textbf{if}  yu  consigue    gafa    we/
  yu  \textstylePichiexamplebold{go}  \textstylePichiexamplebold{wɔ́k}    na  ród.\\
  \textsc{1sg.sbj}  can  \textsc{quot}    if  \textsc{2sg}  obtain    glasses  \textsc{sub}
\textsc{2sg}  \textsc{pot}  walk  \textsc{loc}  road\\
\glt 
  ‘I can tell you if you obtained glasses which/ you would 
walk on the road.’ [ne07ga 007]
}
\ex{\label{ex:key:447b}
\gll
Eyé.  \\
  \textsc{intj}\\

\glt   ‘Good gracious.’ [ye07ga 008]
}
\ex{\label{ex:key:447c}
\gll
Dán    gafa,  yu  \textbf{go}  \textbf{slíp}    wet=an.\\
  that    glasses  \textsc{2sg}  \textsc{pot}  sleep  with=\textsc{3sg.obj}\\

\glt   ‘Those glasses, you would sleep with them.’ [ne07ga 009]
}
\ex{\label{ex:key:447d}
\gll
\MakeUppercase{A}   \textbf{de}  \textbf{slíp}    wet=an    cuñado.\\
  \textsc{1sg.sbj}  \textsc{ipfv}  sleep  with=\textsc{3sg.obj}  brother-in-law\\

\glt   ‘I would sleep with them brother.’ [ye07ga 010]
}
\ex{\label{ex:key:447e}
\gll
  \MakeUppercase{A}   \textbf{go}  \textbf{púl}=an      \textbf{na} mi    yáy  sé    wétin?\\
  \textsc{1sg.sbj}  \textsc{pot}  remove=\textsc{3sg.obj}  \textsc{loc}  \textsc{1sg.poss}  eye  \textsc{quot}    what\\

\glt   ‘I would remove them [the sunglasses] from my eyes for what?’ [ye07ga 011]
}
\ex{\label{ex:key:447f}
\gll
  \MakeUppercase{A}   \textbf{wánt}  dé    flipado    ɔ́l  áwa,    ɔ́l  áwa.\\
  \textsc{1sg.sbj}  want  \textsc{be.loc}  turned.on  all  hour  all  hour\\

\glt   ‘I would want to be turned on all the time, all the time.’ [ye07ga 012]
}
\z
\z

Potential mood is also systematically exploited to render a habitual\is{habitual aspect} reading in narrative discourse anchored in the past \REF{ex:key:448} and in procedural discourse. Note the presence of the generic phrase \textit{di dé wɛn} ‘(on) the day that’ in \REF{ex:key:448}, which tallies with the non-specific meaning of the habitual sense of \textit{go} in this example: 


\ea%448
    \label{ex:key:448}
    \gll Di  dé  wɛ́n    mi    mamá  go  gɛ́t  sɔn    faya-wúd    wé
dɛn  brók=an    na  fám,    e    go  tɛ́l  dɛ́n,    dɛn  go  gó
tót=an    fɔr=an.\\
\textsc{def}  day  \textsc{sub}    \textsc{1sg.poss}  mother  \textsc{pot}  get  some  fire.\textsc{cpd}{}-wood    \textsc{sub}
\textsc{3pl}  break=\textsc{3sg.obj}  \textsc{loc}  farm  \textsc{3sg.sbj}  \textsc{pot}  tell  \textsc{3pl.indp}  \textsc{3pl}  \textsc{pot}  go
carry=\textsc{3sg.obj}  \textsc{prep=3sg.obj}\\

\glt ‘On those days that my mother would get some fire wood that had been 
broken up at the farm, she would tell them (and) they would go and carry 
it for her.’ [ab03ay 023]
\z

\subsubsection{Possibility}\label{sec:6.7.4.2}

The epistemic notion of possibility may be expressed through the use of the potential mood and the epistemic adverbs sɔntɛ́n ‘perhaps’ (cf. \REF{ex:key:444} above) and mebi ‘maybe’ (cf. \REF{ex:key:445} above). Besides that, possibility can be signalled when the verb fít ‘be able; be possible’ functions as a modal auxiliary verb \REF{ex:key:449} or with an expletive subject and a fuller complement clause {\fff}\REF{ex:key:450}: {\fff}


\ea%449
    \label{ex:key:449}
    \gll E    \textstylePichiexamplebold{fít}  \textstylePichiexamplebold{kán}    tumára.\\
\textsc{3sg.sbj}  can  come  tomorrow\\

\glt ‘He might come tomorrow.’ [dj03do 032]
\z


\ea%450
    \label{ex:key:450}
    \gll E    \textstylePichiexamplebold{fít} \textstylePichiexamplebold{bí} sé    na  paludismo.\\
\textsc{3sg.sbj}  can  \textsc{be}  \textsc{quot}    \textsc{foc}  malaria\\

\glt ‘It might be malaria.’ [ru03wt 058]
\z

Possibility can also be expressed through a construction involving an expletive fíba ‘seem’ \REF{ex:key:451} or the adverb sɔntɛ́n ‘perhaps’ with or without potential mood marking \REF{ex:key:452}:


\ea%451
    \label{ex:key:451}
    \gll E    fíba    sé    Boyé  gɛ́t  mɔní. \\
\textsc{3sg.sbj}  seem  \textsc{quot}    \textsc{name}  get  money\\

\glt ‘It seems that Boyé has money.’ [dj07ae 255]
\z


\ea%452
    \label{ex:key:452}
    \gll (...)  sɔntɛ́n  di  bɔ́y  nó  gɛ́t  páwa,  sɔntɛ́n  di  gál
gɛ́t  sɔn    defecto.\\
{} perhaps  \textsc{def}  boy  \textsc{neg}  get  power  perhaps  \textsc{def}  girl
get  some  defect.\\

\glt ‘(...) the boy might have no power [be impotent], (or) the girl 
 might have a defect.’ [ab03ay 044]
\z

\subsubsection{Certainty}\label{sec:6.7.4.3}

Inferred certainty, the firmest degree of assertion, can be expressed by way of inferral from obligation with gɛ́fɔ ‘have to’ as in \REF{ex:key:453}. The potential mood{\fff} marker go is also employed in this function, in particular in combination with dɔ́n ‘prf’ (cf. \REF{ex:key:446} above)):{\fff}


\ea%453
    \label{ex:key:453}
    \gll Dɛn    bin  gɛ́fɔ    sabí    sé    e    go  kán.\\
\textsc{3pl}    \textsc{pst}  have.to  know  \textsc{quot}    \textsc{3sg.sbj}  \textsc{pot}  come\\

\glt ‘They must have known that she would come.’ [ab03ay 128]
\z


\ea%454
    \label{ex:key:454}
    \gll Iris    gɛ́fɔ    gɛ́t,  a    tínk    sé    diez  años.\\
\textsc{name}  have.to  get  \textsc{1sg.sbj}  think  \textsc{quot}    ten  years\\

\glt ‘Iris should be, I think ten years old.’ [fr03ft 121]
\z

\subsubsection{Assertion}\label{sec:6.7.4.4}

The emphatic and focus particle sɛ́f ‘emp’ (cf. \sectref{sec:7.4.2}) and the sentence particle{\fff} ó (cf. \sectref{sec:12.2.4}) function as general markers of assertion when they signal clausal focus. Other than that, the verb trú ‘be true’ may be employed as an adverbial, oftentimes repeated for additional force, in order to signal assertion:


\ea%455
    \label{ex:key:455}
    \gll Dɛn  bɔ́n    na  Corisco  \textstylePichiexamplebold{trú} \textstylePichiexamplebold{trú}.\\
\textsc{3pl}  be.born  \textsc{loc}  \textsc{place}  true  \textsc{rep}\\

\glt ‘They were really born on [the island of] Corisco.’ [to07fn 201]
\z

Beyond that, constructions involving cognition verbs (e.g. sabí ‘(get to) know’, nó ‘know’, chɛ́k ‘think; check (out)’, tínk ‘think’, mɛ́mba ‘think; remember’, and perception verbs (e.g. sí ‘see’, hía ‘hear’) by themselves also signal different degrees of certainty.{\fff} 

\section{Tense, modality, and aspect in discourse}\label{sec:6.8}

In preceding sections, I have provided some examples on the functions of TMA markers in discourse. In the following, I explore these functions further by looking at extracts of narrative discourse. The two relevant, intimately connected discourse-pragmatic notions are sequencing, i.e. the ordering of events along the time axis \citep{Hopper1982}, and grounding, i.e. the distinction between the narrative main line or foreground from the less salient, narratively subordinate background (e.g. \citealt{HopperThompson1980}; \citealt{Longacre1996}; \citealt{YoussefJames1999}).


The picture that emerges from the analysis of the functions of Pichi TMA markers in narrative discourse with respect to grounding and sequencing is presented in \figref{fig:key:6.3}. The distribution of TMA markers in Pichi narrative discourse suggests the existence of a grounding continuum. \figref{fig:key:6.3} takes this into account by differentiating between a more [+high] and a less salient [-high] foreground, marked by the narrative perfective\is{perfective aspect} marker \textit{kán} ‘\textsc{pfv}’ and the factative marked (hence perfective) dynamic verb, respectively. The feature [+/-sequence] denotes the property of TMA markers to signal successive and discrete events along the narrative time line. Temporal and aspectual characteristics are therefore collapsed in this feature. So [+sequence] typifies consecutive, bounded, and dynamic situations, which may not be reordered without changing the iconic temporal order of the narrative at the same time. 



The feature [+/-deixis] allows differentiation between aspect markers without an explicit temporal reference and markers that encode time-deictic reference to a point outside of the predicate. These reference points are event time for \textit{bin} ‘\textsc{pst}’ and \textit{dɔ́n} ‘\textsc{pfv}’, and a hypothetical contingency for \textit{go} ‘\textsc{pot}’ in habitual discourse. 


\begin{figure}
\caption{Functions of TMA markers in narrative discourse}
\label{fig:key:6.3}

\begin{tabularx}{\textwidth}{|l|X|l|l|}
\hline
{ [+deixis]} & \multicolumn{2}{c|}{ [+foreground]} & [-foreground]\\
\hhline{~--~} & { [+high]} & [-high] & \\
\hhline{~---} & &  & \\
& { \textit{kán} ‘\textsc{pfv}’} & Factative TMA & \textit{go} ‘\textsc{pot}’ (=habitual)\\
& & with dynamic and & \textit{dɔ́n} ‘\textsc{prf}’\\
& & inchoative-stative & \textit{bin} ‘\textsc{pst}’\\
& & verbs & \\
\hline
{ [-deixis]} &  && \\
&  &  & \textit{de} ‘\textsc{ipfv}’\\
&  &  & Factative TMA\is{factative TMA} with (inchoative)-\\
&  &  & stative verbs\\
&  &  & \textit{kin} ‘\textsc{hab}’\\
&  &  &\\
\hhline{~---} & \multicolumn{2}{c|}{ [+sequence]} & [-sequence]\\
\hline
\end{tabularx}
\end{figure}
\subsection{Sequencing and grounding}\label{sec:6.8.1}

The beginning of narratives anchored in the past very often features the past marker \textit{bin} ‘\textsc{pst}’ in the “orientation” section (cf. \citealt{Labov1972}:358) characterised by aspect marking of the imperfective domain (hence imperfective and/or habitual\is{habitual aspect} aspect). In this, the past marker is true to its role as a device for backgrounding situations and contributing a sense of temporal remoteness. For similar observations on cognate forms of \textit{bin}, see Winford (2000:398ff.) for Sranan and \citet[63]{Pollard1989} Jamaican Creole. The marker \textit{bin} ‘\textsc{pst}’ fulfills this dual function in the orientation section \REF{ex:key:456}(b)–(e) of the excerpt of a personal narrative below. The backgrounding function of \textit{bin} ‘\textsc{pst}’ correlates with its default aspectual interpretation. 


Sentences \REF{ex:key:456}(a)–(d) demonstrate that there is a strong tendency to conceive of situations marked by \textit{bin} as unbounded, hence imperfective by default. The free variation between \textit{bin} {{‘}}\textit{\textsc{pst}}{{’}}, the imperfective marker \textit{de} ‘\textsc{ipfv}’, and the marker sequence \textit{bin de} in (b)–(e) with dynamic verbs for the expression of backgrounded, unbounded, and overlapping situations demonstrates the functional similarity of the three marking options:



\ea%456
    \label{ex:key:456}
\ea{\label{ex:key:456a}
    \gll 
Dé,    ɛ́ni    káyn  tín    na  mɔní,  yu  fít
  mék    ɛ́ni    káyn  tín,    yu  go  sí  mɔní.\\
  there  every  kind    thing  \textsc{foc}  money  \textsc{2sg}  can
  make  every  kind    thing  \textsc{2sg}  \textsc{pot}  see  money\\
\glt
  ‘There, everything is money, you can do anything, you will earn 
money.’ [ma03hm 054]
}\ex{\label{ex:key:456b}
\gll
Mi    \textstylePichiexamplebold{bin}  \textstylePichiexamplebold{dé}    dé    a    \textstylePichiexamplebold{bin}  mék    dásɔl,
  dís,  a    \textstylePichiexamplebold{de}  \textstylePichiexamplebold{mék}    fínga  dɛn,    manicura.\\
  \textsc{1sg.indp}  \textsc{pst}  \textsc{be.loc}  there  \textsc{1sg.sbj}  \textsc{pst}  make  only
  this  \textsc{1sg.sbj}  \textsc{ipfv}  make  finger  \textsc{pl}    manicure\\

\glt
  ‘(When) I was there, I only used to do, I used to do fingers, 
manicure.’ [ma03hm 055]
}\ex{\label{ex:key:456c}
\gll
A    \textstylePichiexamplebold{de}  \textstylePichiexamplebold{mék}    tapete    dɛn  fɔ  chía,
  a    bin  gɛ́t  mi    mɔní.\\
  \textsc{1sg.sbj}  \textsc{ipfv}  make  table.cloth  \textsc{pl}  \textsc{prep}  chair
  \textsc{1sg.sbj}  \textsc{pst}  get  \textsc{1sg.poss}  money\\

\glt 
  ‘I used to make table cloths [covers] for chairs, I used to 
get my money.’ [ma03hm 056]
}\ex{\label{ex:key:456d}
\gll
\'{A}fta    mɔ́    a    \textstylePichiexamplebold{bin}  wók    dís  sén
  wók    wé  a    de  dú,  a    \textstylePichiexamplebold{de}  dú=an
  dé    sɛ́f.\\
  then  more  \textsc{1sg.sbj}  \textsc{pst}  work  this  same
  work  \textsc{sub}  \textsc{1sg.sbj}  \textsc{ipfv}  do  \textsc{1sg.sbj}  \textsc{ipfv}  do=\textsc{3sg}.ob
  there  \textsc{emp}\\

\glt 
  ‘Apart from that, I used to work in this very job that I do 
(now), I did it there, too.’ [ma03hm 057]
}\ex{\label{ex:key:456e}
\gll
So  a    bin  de  gɛ́t  mi    mɔní  dé
  pero  yá    al    contrario  nada.\\
  so  \textsc{1sg.sbj}  \textsc{pst}  \textsc{ipfv}  get  \textsc{1sg.poss}  money  there
  but    here    at.the  contrary    nothing\\

\glt   ‘So I used to get my money there but here, on the contrary, nothing.’ [ma03hm 058]
}
\z
\z

In its functions, \textit{bin} ‘\textsc{pst}’ is therefore antipodal to the narrative perfective\is{perfective aspect} marker \textit{kán} ‘\textsc{pfv}’ (cf. \REF{ex:key:459}–\REF{ex:key:461} below). Like the former, the latter also simultaneously encodes a tense (past tense) and an aspectual value (perfective), and thereby plays an important role in the organisation of narrative discourse. However, the marker \textit{kán} ‘\textsc{pfv}’ occurs in the most salient, foregrounded sections of the narrative, while \textit{bin} ‘\textsc{pst}’ appears in backgrounded, supportive, and orienting sections.\is{past tense}\is{imperfective aspect} 


Temporal sequence can also be iconically encoded through the linear ordering of bare dynamic verbs as in the “complicating action” \citep{Labov1972} of the narrative in \REF{ex:key:457} below. The temporal interpretation of factative\is{factative TMA} marked inchoative-stative verbs hinges on grounding. The inchoative-stative bare verb \textit{slíp} ‘lie down’ \REF{ex:key:457}(c) receives an inchoative, dynamic reading as it is foregrounded and forced into sequence in the narrative main line:



\ea
\label{ex:key:457}
\ea{\label{ex:key:457a}
\gll
E    \textstylePichiexamplebold{gó},  e    \textstylePichiexamplebold{wás}    di  klós    dɛn.\\
  \textsc{3sg.sbj}  go  \textsc{3sg.sbj}  wash  \textsc{def}  clothing  \textsc{pl}\\

\glt   ‘She went off, she washed the clothes.’ [ru03wt 033]
}
\ex{\label{ex:key:457b}
\gll
E    wás    dí  klós    dɛn,  e    dráy    dɛ́n,
  nó  na  mi    dráy    dɛ́n.\\
  \textsc{3sg.sbj}  wash  this  clothing  \textsc{pl}  \textsc{3sg.sbj}  dry    \textsc{3pl.indp}
  \textsc{neg}  \textsc{foc}  \textsc{1sg.indp}  dry    \textsc{3pl.indp}\\

\glt   ‘She washed the clothes, she dried them, no I dried them.’ [ru03wt 034] 
}
\ex{\label{ex:key:457c}
\gll
Pero    di  klós    dɛn  \textstylePichiexamplebold{slíp}    na  dɔ́n    ó.\\
  but    \textsc{def}  clothing  \textsc{pl}  lie.down  \textsc{loc}  down  \textsc{sp}\\

\glt   ‘But the clothes came to lie on the ground.’ [ru03wt 035]
}
\ex{\label{ex:key:457d}
\gll
Mɔ́nin  tɛ́n    wé  a    kán    lúk,  a    de  sí  sɔn
  klós    dɛn,    a    nó  de  sí  mi  yón     dɛn.\\
  morning  time    \textsc{sub}  \textsc{1sg.sbj}  come  look  \textsc{1sg.sbj}  \textsc{ipfv}  see  some
  clothing  \textsc{pl}    \textsc{1sg.sbj}  \textsc{neg}  \textsc{ipfv}  see  \textsc{1pl}  own    \textsc{pl}\\

\glt 
  ‘(In the) morning, when I looked, I saw some clothes, (but) I didn’t 
see mine.’ [ru03wt 036]
}
\z
\z

In contrast, backgrounded and out-of-sequence stative and inchoative-stative verbs, whether bare or marked with \textit{bin} ‘\textsc{pst}’, receive a stative reading. Sentence \REF{ex:key:458} below is an orientation section. The stative copula \textit{dé} \textsc{‘be.loc’} has a stative reading in the sentence. The same holds true for the inchoative-stative verb \textit{sidɔ́n} ‘sit (down)’. It co-occurs with the past marker \is{past tense}\textit{bin} ‘\textsc{pst}’, which once more not only signals the presence of backgrounded information. The imperfective, unbounded reading of \textit{bin} also resolves the potential ambiguity between an inchoative and a stative interpretation of \textit{sidɔ́n} in favour of the latter:


\ea%458
    \label{ex:key:458}
    \gll Mí    \textstylePichiexamplebold{bin}  \textstylePichiexamplebold{dé}    na  bích    wé  a    \textstylePichiexamplebold{bin}  \textstylePichiexamplebold{sidɔ́n}  wet
mi    papá,  mi    bin  dé    na  bích    mɔ́nin  tɛ́n 
a    gó  latrin  a    gó  kaká  (...)\\
\textsc{1sg.indp}  \textsc{pst}  \textsc{be.loc}  \textsc{loc}  beach  \textsc{sub}  \textsc{1sg.sbj}  \textsc{pst}  sit.down  with
\textsc{1sg.poss}  father  \textsc{1sg.indp}  \textsc{pst}  \textsc{be.loc}  \textsc{loc}  beach  morning  time
\textsc{1sg.sbj}  go  latrine  \textsc{1sg.sbj}  go  defecate\\

\glt ‘I [\textsc{emp}] was at the beach while I was sitting with my father, I [\textsc{emp}] was 
at the beach in the morning, I went to the latrine, I went to shit (…)’ [ed03sb 171]\is{past tense}
\z

Both (inchoative-)stative and dynamic verbs can also be explicitly marked for [+sequence] by the narrative perfective marker \textit{kán} ‘\textsc{pfv}’. The boundary-activating function of \textit{kán} propels verbs marked by \textit{kán} into the temporally sequenced narrative main line irrespective of their lexical aspect. With (inchoative-)stative verbs, this invariably induces an inchoative reading. With dynamic verbs, both boundaries of the situation are activitated. These two aspect readings – bounded for dynamic verbs and inchoative for stative verbs – make \textit{kán} ‘\textsc{pfv}’ a typical perfective marker (cf. \citealt{Sasse1991b}:11–14), even if its use is specialised to narrative discourse in Pichi. 


The orientation section in \REF{ex:key:459}(a)–(b) is followed by a complicating action section in (c), which contains the first foregrounded situation, the inchoative-stative verb \textit{sabí} ‘(get to) know’. The verb is marked by \textit{kán} ‘\textsc{pfv}’ and receives an inchoative reading: 



\ea%459
    \label{ex:key:459}
\ea{\label{ex:key:459a}
\gll
Bueno,  mi    mamá,  mi    gran-má    wet  
  mi    mamá,  nɔ́,  dɛn  kɔmɔ́t    na  wán  pueblo
  wé  in    ném    na  {Basakato dé la Sagrada Familia}\\
  good  \textsc{1sg.poss}  mother  \textsc{1sg.poss}  grand-ma  with
  \textsc{1sg.poss}  mother  \textsc{intj}  \textsc{3pl}  hail.from  \textsc{loc}  one  village
  \textsc{sub}  \textsc{3sg.poss}  \textsc{name}  \textsc{foc}  \textsc{place}\\
\glt
  ‘Well, my mother, my grandmother and my mother, right, they 
hail from a village whose name is Basakato dé la Sagrada Familia.’ [fr03ft 042]
}
\ex{\label{ex:key:459b}
\gll
Sɔn    tɛ́n    dɛn  wi  kin  de  gó  dé    sɛ́f  fɔ  gó,
  bueno,  fɔ  gó  visít    nɔ́,  fɔ  pás  vacaciones  dɛn.\\
  some  time    \textsc{pl}  \textsc{1pl}  \textsc{hab}  \textsc{ipfv}  go  there  \textsc{emp}  \textsc{prep}  go
  good  \textsc{prep}  go  visit    \textsc{intj}  \textsc{prep}  pass  holiday.\textsc{pl}    \textsc{pl}\\

\glt 
  ‘Sometimes we even used to go there in order to, well, in order 
to go visit, in order to spend our holidays.’ [fr03ft 043]
}
\ex{\label{ex:key:459c}
\gll
Na  dé    a    \textstylePichiexamplebold{kán}  \textstylePichiexamplebold{sabí}    mi    mamá
  in    papá  in    fámbul.\\
  \textsc{foc}  there  \textsc{1sg.sbj}  \textsc{pfv}  know  \textsc{1sg.poss}  mother
  \textsc{3sg.poss}  father  \textsc{3sg.poss}  family\\

\glt   ‘That’s where I got to know my mother’s father’s family.’ [fr03ft 044]
}
\z
\z

The following extract illustrates the importance that \textit{kán} ‘\textsc{pfv}’ has for organising the events of a paragraph with respect to narrative saliency. The verbs in \REF{ex:key:460}(a)–(d) are marked for perfective aspect due to the novel information they contain. Meanwhile, \REF{ex:key:460}(e) reiterates information already contained in \REF{ex:key:460}(c) and (d), therefore dispenses with perfective marking and is characterised by the presence of stative, narratively downshifted verbs: 


\ea%460
    \label{ex:key:460}
\ea{\label{ex:key:460a}
\gll
A    \textstylePichiexamplebold{kán}  \textstylePichiexamplebold{recupera}    smɔ́l.\\
  \textsc{1sg.sbj}  \textsc{pfv}  recover    small\\

\glt   ‘(Then) I recovered a bit.’ [ab03ay 096]
}
\ex{\label{ex:key:460b}
\gll
A    \textstylePichiexamplebold{kán}  \textstylePichiexamplebold{kɔmɔ́t}  na  dán  hós    wé  a    bin  dé.\\
  \textsc{1sg.sbj}  \textsc{pfv}  go.out  \textsc{loc}  that  house  \textsc{sub}  \textsc{1sg.sbj}  \textsc{pst}  \textsc{be.loc}\\

\glt   ‘Then I left that house where I was.’ [ab03ay 097]
}
\ex{\label{ex:key:460c}
\gll
A    \textstylePichiexamplebold{kán}  \textstylePichiexamplebold{gó}  na  mi    ɔnkúl  in    papá  
  in    lét  brɔ́da.\\
  \textsc{1sg.sbj}  \textsc{pfv}  go  \textsc{loc}  \textsc{1sg.poss}  uncle  \textsc{3sg.poss}  father
  \textsc{3sg.poss}  late  brother\\

\glt   ‘Then I went to my uncle’s father’s late brother.’ [ab03ay 098]
}
\ex{\label{ex:key:460d}
\gll
Mi    lét  papá  in    brɔ́da,  a    \textstylePichiexamplebold{kán}  \textstylePichiexamplebold{dé}
  na  in    hós.\\
  \textsc{1sg.poss}  late  father  \textsc{3sg.poss}  brother  \textsc{1sg.sbj}  \textsc{pfv}  \textsc{be.loc}
  \textsc{loc}  \textsc{3sg.poss}  house\\

\glt   ‘My late father’s brother, I came to stay at his house.’ [ab03ay 099]
}
\ex{\label{ex:key:460e}
\gll
Na  dé    a    \textstylePichiexamplebold{dé}   wán  hía  a    nó  \textstylePichiexamplebold{fít}
  dú  nó  nátin.\\
  \textsc{foc}  there  \textsc{1sg.sbj}  \textsc{be.loc}  one  year  \textsc{1sg.sbj}  \textsc{neg}  can
  do  \textsc{neg}  nothing\\

\glt   ‘It is there that I was for one year, I couldn’t do anything.’ [ab03ay 100]
}
\z
\z

The use of kán ‘pfv’ in \REF{ex:key:461}(b) points to the role of the perfective marker in additionally highlighting narratively salient, [+high] foreground information. At the same time, less salient [-high] foreground occurs in the unmarked form of the verb (i.e. the two occurrences of sɛ́n ‘send’ in \REF{ex:key:461}(b)), which incidentally coincides with a backgrounding passive{\fff} construction, another downshifting device (i.e. dɛn sɛ́n mí (...) ‘I was sent (...)’). The introduction of information considered more relevant, and with it the resumption of the main line, then once more features the perfective marker kán ‘pfv’ with the verb lɔs ‘lose’: 


\ea%461
    \label{ex:key:461}
\ea{\label{ex:key:461a}
\gll
E    kán  gó  na  hós    e    kán  lɛ́f    mí
  sɔn    dirección  fɔ  Chicago,  a \textstylePichiexamplebold{kán}  \textstylePichiexamplebold{ráyt}.\\
  \textsc{3sg.sbj}  \textsc{pfv}  go  \textsc{loc}  house  \textsc{3sg.sbj}  \textsc{pfv}  leave  \textsc{1sg.indp}
  some  address    \textsc{prep}  \textsc{place}  \textsc{1sg.sbj}  \textsc{pfv}  write\\

\glt 
  ‘He went home (and) he left me an address in Chicago (and) 
I wrote to him.’ [ed03sb 206]
}
\ex{\label{ex:key:461b}
\gll
E    de  ánsa    mi,    a    sɛ́n    mɔní,
  dɛn  sɛ́n    mí    sɔn    portamonedas  bɔt  e    kán  lɔs.\\
  \textsc{3sg.sbj}  \textsc{ipfv}  answer  \textsc{1sg.indp}  \textsc{1sg.sbj}  send  money
  \textsc{3pl}  send  \textsc{1sg.indp}  some  wallet      but  \textsc{3sg.sbj}  \textsc{pfv}  lose\\

\glt 
  ‘He used to reply to me, I sent money (and) I was sent a wallet 
but it got lost.’ [ed03sb 207]\is{perfective aspect}
}
\z
\z

Like the imperfective marker \textit{de} ‘\textsc{ipfv}’, the habitual marker \textit{kin} ‘\textsc{hab}’ marks [-sequence] situations that furnish the background frame for the narrative main line. One may find entire paragraphs marked for habitual aspect in order to provide orientation. Next to the habitual marker \textit{kin}, the potential marker \textit{go} also fulfils an important role in expressing habituality with respect to routine procedures. This is shown in the following extract that relates the effect zombification has on its victims. Consider the prolific use of \textit{go} ‘\textsc{pot}’ to signal (potential) habituality set in a hypothetical frame: 


\ea%462
    \label{ex:key:462}
\ea{\label{ex:key:462a}
\gll
Porque  if  yu  mék,  yu  sí  dán  polvo  e    de
  pút=an    ínsay,  yu  \textbf{kán}    yu  \textbf{dríng},  dɛn  \textstylePichiexamplebold{go}  \textstylePichiexamplebold{gó}
  na  hós.\\
  because  if  \textsc{2sg}  make  \textsc{2sg}  see  that  powder  \textsc{3sg.sbj}  \textsc{ipfv}
  put=\textsc{3sg.obj}  inside  \textsc{2sg}  come  \textsc{2sg}  drink  \textsc{3pl}  \textsc{pot}  go
\textsc{loc}  house\\
\glt 
  ‘Because if you make, you see that powder (as) he’s putting it inside, (after)
you’ve come and drunk (it) they go back home.’ [ed03sb 099]
}
\ex{\label{ex:key:462b}
\gll
Lɛk  háw  dɛn  wánt  kɛ́r    yú    na  hospital  yu  dɔ́n  dáy.\\
  like  how  \textsc{3pl}  want  carry  \textsc{2sg.indp}  \textsc{loc}  hospital  \textsc{2sg}  \textsc{prf}  die\\

\glt 
  ‘Just when they want to bring you to hospital, you’re already dead.’
[ed03sb 100]
}
\ex{\label{ex:key:462c}
\gll
Lɛk  háw  dɛn  \textstylePichiexamplebold{go}  \textstylePichiexamplebold{pút}  yú    na  tébul  yu  dɔ́n  de  rɔ́tin,
  fɔ  mék    dɛn  gó  bɛ́r    yú    kwík.\\
  like  how  \textsc{3pl}  \textsc{pot}  put  \textsc{2sg.indp}  \textsc{loc}  table  \textsc{2sg}  \textsc{prf}  \textsc{ipfv}  rot
  \textsc{prep}  \textsc{sbjv}    \textsc{3pl}  go  bury  \textsc{2sg.indp}  quickly\\

\glt 
  ‘As soon as they put you on the table, you’re already about to rot, 
in order for them to bury you quickly.’ [ed03sb 101]
}
\ex{\label{ex:key:462d}
\gll
Ɛf  dɛn  go  gó  bɛ́r    yú,    dɛ́n    sénwe  go  gó  
  na  dán  bɛ́rin.\\
  if  \textsc{3pl}  \textsc{pot}  go  bury  \textsc{2sg.indp}  \textsc{3pl.indp}  \textsc{emp}    \textsc{pot}  go  
  \textsc{loc}  that  burial\\

\glt   If they go to bury you, they themselves will go to that burial.’ [ed03sb 102]
}
\ex{\label{ex:key:462e}
\gll
Na  nɛ́t    a  las     doce  dɛn  go  kán    dɛn  púl    yú
  yu  nɔ́ba  dáy.\\
  \textsc{loc}  night  at  the.\textsc{pl}   twelve  \textsc{3pl}  \textsc{pot}  come  \textsc{3pl}  remove  \textsc{2sg.indp}
  \textsc{2sg}  \textsc{neg}.\textsc{prf}  die\\

\glt 
  ‘In the night, at twelve o’clock they’ll come and remove you (and) 
you haven’t died.’ [ed03sb 103]
}
\ex{\label{ex:key:462f}
\gll
Dɛn  \textstylePichiexamplebold{go}  \textstylePichiexamplebold{rɛdí}    yú    dɛn  \textstylePichiexamplebold{go}  \textstylePichiexamplebold{mék}    lɛk  háw  dɛn  de
  mék    fɔ  wích,  dɛn  \textbf{ték}  yú    dɛn  \textbf{pút}  yú
  na  avión  dɛn  sɛ́n    yú    fɔ  ɔ́da    kɔ́ntri
  yu  \textbf{gó}  wók    mɔní.\\
  \textsc{3pl}  \textsc{pot}  prepare  \textsc{2sg.indp}  \textsc{3pl}  \textsc{pot}  make  like  how  \textsc{3pl}  \textsc{ipfv}
  make  \textsc{prep}  sorcery  \textsc{3pl}  take  \textsc{2sg.indp}  \textsc{3pl}  put  \textsc{2sg.indp}
  \textsc{loc}  plane  \textsc{3pl}  send  \textsc{2sg.indp}  \textsc{prep}  other  country
  \textsc{2sg}  go  work  money\\

\glt 
  ‘They’ll prepare you the way it’s done with sorcery, they’ll take you, 
put you into a plane and send you to another country (and) you’ll go
earn money (for them).’ [ed03sb 104]\is{potential mood}
}
\z
\z

Foregrounded sections of sequential action conceived of as particularly tightly-knit may feature clause chaining (cf. \sectref{sec:11.4}). In chained clauses, tense, aspect and mood marking is overtly expressed with the first initial verb(s) in order to provide orientation and grounding. Subsequent clauses remain bare and occur one after the other without an intonation break or intervening clause linkers. Chained predicates invariably feature resumptive personal pronouns; the subject\is{subjects} is repeated with each verb in the series. Verbs that participate in clause chaining are always dynamic, and are hence part of the foregrounded narrative main line. Sequences of chained clauses can be found in \REF{ex:key:462}(a) (\textit{yu} \textit{kán} \textit{yu} \textit{dríng}), (e) (\textit{dɛn} \textit{púl} \textit{yú}), and (f) (beginning with \textit{dɛn} \textit{ték} \textit{yú} until the end of the paragraph).


After a brief interruption by a listener comes a transition to habitual marking via \textit{kin} ‘\textsc{hab}’ in \REF{ex:key:463} below. Extracts \REF{ex:key:462}–\REF{ex:key:463} lay bare the difference between habitual discourse centred on \textit{go} ‘\textsc{pot}’ and \textit{kin} ‘\textsc{hab}’, respectively. The expression of habituality with \textit{go} rests on the prior establishment of a hypothetical contingency. Hence, paragraph \REF{ex:key:462} is interlaced with elements characteristic of irrealis\is{irrealis modality} modality. The extract begins in \REF{ex:key:462}(a) with a conditional clause\is{conditional clauses} serving as the referential frame for the \textit{go}{}-marked discourse up to (f); another conditional clause follows in (d), and the habitual, generic use of \textit{go} coincides with the impersonalised, non-referential use of the \textsc{2sg} personal pronoun \textit{yu}.



In contrast herewith, habitual discourse centred on kin in \REF{ex:key:463} is introduced by the phrase e kán bí sé ‘3sg.sbj pfv be quot’ = ‘it came to pass that’, a conventionalised opening formula employed in personal accounts and other types of factual narrative. The subjectively high truth value of \REF{ex:key:462} is underlined by the closure in (g) a dɔ́n sí, yɛ́s ‘1sg.sbj prf see yes’ = ‘I have seen (this before), yes’.



\ea%463
    \label{ex:key:463}
\ea{\label{ex:key:463a}
\gll
Dɛn  \textstylePichiexamplebold{go}  púl    dán  mán,  a    sé
  e    kán  bí  sé    dɛn/  pípul  dɛn
  kɛ́r=an,    dɛn  lɛ́f    di  cadáver  dɛn  rɔ́n.\\
  \textsc{3pl}  \textsc{pot}  remove  that  man    \textsc{1sg.sbj}  \textsc{quot}
  \textsc{3sg.sbj}  \textsc{pfv}  \textsc{be}  \textsc{quot}    \textsc{3pl}    people  \textsc{pl}
  carry=\textsc{3sg.obj}  \textsc{3pl}  leave  \textsc{def}  corpse  \textsc{3pl}  run\\

\glt 
  ‘They’ll remove that man, I say, it came to pass that they/
people carried him, they left the corpse and run away.’  
[ed03sb 107]\is{modality}  
}
\ex{\label{ex:key:463b}
\gll
A    tínk    sɔn    fámbul  dɛn  wé  dɛn  \textstylePichiexamplebold{kin}  \textstylePichiexamplebold{sí} sé
  dí  mi    fámbul  dé    lɛk  háw    e    dáy
  e    nó  kɔrɛ́t.\\
  \textsc{1sg.sbj}  think  some  family  \textsc{pl}  \textsc{sub}  \textsc{3pl}  \textsc{hab}  see  \textsc{quot}
  this  \textsc{1sg.poss}  family  there  like  how    \textsc{3sg.sbj}  die
  \textsc{3sg.sbj}  \textsc{neg}  be.correct\\

\glt 
  ‘I think some families, when they see that this my family member 
there, how he died that’s not correct.’ [ed03sb 108]  
}
\ex{\label{ex:key:463c}
\gll
Dɛn  \textstylePichiexamplebold{kin}  \textstylePichiexamplebold{gó}  na  bɛrin-grɔ́n    wet    gɔ́n.\\
  \textsc{3pl}  \textsc{hab}  go  \textsc{loc}  burial.\textsc{cpd}{}-ground  with    gun\\

\glt   ‘They go to the cemetery with a gun.’ [ed03sb 109]  
}
\ex{\label{ex:key:463d}
\gll
A    hía    sé    Bata    dɛn  \textstylePichiexamplebold{kin}  \textstylePichiexamplebold{sút}   yú.\\
  \textsc{1sg.sbj}  hear    \textsc{quot}    \textsc{place}  \textsc{pl}  \textsc{hab}  shoot  \textsc{2sg.indp}\\

\glt   ‘I heard that the mainlanders (even) shoot you.’ [kw03sb 110]
}
\ex{\label{ex:key:463e}
\gll
Dɛn  \textstylePichiexamplebold{kin}  \textstylePichiexamplebold{sút}.\\
  \textsc{3pl}  \textsc{hab}  shoot\\

\glt   ‘They shoot (you).’ [ed03sb 111]
}
\ex{\label{ex:key:463f}
\gll
Wé  dɛn  sút    di  pɔ́sin,  di  pɔ́sin  \textstylePichiexamplebold{kin}    \textstylePichiexamplebold{sék}.\\
  \textsc{sub}  \textsc{3pl}  shoot  \textsc{def}  person  \textsc{def}  person  \textsc{hab}    shake\\

\glt   ‘When they’ve shot the person, the person shakes.’ [ed03sb 112]
}
\ex{\label{ex:key:463g}
\gll
A    dɔ́n  sí,  yɛ́s.\\
  \textsc{1sg.sbj}  \textsc{prf}  see  yes\\

\glt   ‘I have experienced (this), yes.’ [ed03sb 113]\is{habitual aspect}
}
\z
\z

The perfect tense-aspect marker \textit{dɔ́n} ‘\textsc{pfv}’ is employed with [-sequence] situations that digress from the linear narrative main line. The use of this marker prepares terrain for foregrounded and bounded action, a role reserved for functionally equivalent forms in many languages (cf. \citealt{Anderson1982}; Li, \citealt{ThompsonThompson1982}; \citealt{ThompsonThompson1982}; \citealt{Slobin1994}). The perfect marker may therefore play an important role in signalling the anteriority and causality of a situation immediately relevant to the situations of the narrative main line. Consider \REF{ex:key:464}, which is an excerpt of a narrative about a woman who wants to divorce her husband and is obliged by tradition to pay back the dowry. In this excerpt, the perfect aspect lends itself to use in an “embedded abstract” \citep{Labov1972}, which often occurs in a well-formed Pichi narrative. Through this technique, a speaker steps out of the story line, condenses and adds on to previous foreground material in a series of perfect marked verbs as in (a–c).


Note that the speaker employs some features characteristic of Nigerian (Pidgin) English, since she lived in Nigeria for some time (i.e. (dé) yɔ́ng ‘be young’, dé frɛ́sh ‘be fresh’, sɛventín ‘seventeen’, etín ‘eighteen’, twɛ́nti ‘twenty’, and yíɛs ‘years’):



\ea%464
    \label{ex:key:464}
\ea{\label{ex:key:464a}
\gll
Yu  yɔ́ng,    yu  jɔ́s/  sɔntɛ́n  yu  gɛ́t  sɛventín,    etín    yíɛs
  ɔ  twɛ́nti,  yu  dé    yɔ́ng  yu  dé    frɛ́sh,  yu  dɔ́n  kɔmɔ́t,
  yu  \textstylePichiexamplebold{dɔ́n}  \textstylePichiexamplebold{bɔ́n}      fó  pikín,  yu  \textbf{dɔ́n}  \textbf{bɔ́n}      fáyf,
  yu  dɔ́n  bɔ́n      tɛ́n.\\
  \textsc{2sg}  be.young    \textsc{2sg}  just  perhaps  \textsc{2sg}  get  seventeen  eighteen  years
  or  twenty  \textsc{2sg}  \textsc{be.loc}  young  \textsc{2sg}  \textsc{be.loc}  fresh  \textsc{2sg}  \textsc{prf}  go.out
  \textsc{2sg}  \textsc{prf}  give.birth  four  child  \textsc{2sg}  \textsc{prf}  give.birth  five
  \textsc{2sg}  \textsc{prf}  give.birth  ten\\

\glt 
  ‘You’re young, you just/ perhaps you’re seventeen, eighteen years old or twenty, 
you’re young, you’re fresh, you’ve left [the parental home], you’ve given birth 
to four children, you’ve given birth to five, you’ve given birth to ten.’ [hi03cb 187]
}
\ex{\label{ex:key:464b}
\gll
Náw    wé  yu  \textstylePichiexamplebold{dɔ́n}  \textstylePichiexamplebold{de}  \textstylePichiexamplebold{gó}  yu  \textstylePichiexamplebold{dɔ́n}/\\
  now    \textsc{sub}  \textsc{2sg}  \textsc{prf}  \textsc{ipfv}  go  \textsc{2sg}  \textsc{prf}  \\

\glt   ‘Now that you’re about to leave [the man], you’ve/ [hi03cb 188]
}
\ex{\label{ex:key:464c}
\gll
Dɛn  tɛ́l  yú    sé    mék  yu  bák    dán  mɔní  wé
  yu  \textstylePichiexamplebold{dɔ́n},    dán  mán    \textstylePichiexamplebold{dɔ́n}  \textstylePichiexamplebold{pé}  fɔ  yu  héd.\\
  \textsc{3pl}  tell  \textsc{2sg.indp}  \textsc{quot}    \textsc{sbjv}  \textsc{2sg}  return  that  money  \textsc{sub}
  \textsc{2sg}  \textsc{prf}    that  man    \textsc{prf}  pay  \textsc{prep}  \textsc{2sg}  head\\

\glt 
  ‘They tell you to return that money that you have, that man 
has paid for you.’ [hi03cb 189]
}
\z
\z

The completive aspect\is{completive aspect} involving the auxiliary \textit{fínis} ‘finish’ may fulfil a discourse function similar to that of the perfect. The use of the completive aspect in signalling precedence of a situation in relation to reference time in ground-preparing, digressive sequences is illustrated in \REF{ex:key:465}, where it appears together with \textit{dɔ́n} ‘\textsc{prf}’:\is{„auxiliaries“}


\ea%465
    \label{ex:key:465}
    \gll Kip,  dɛn  \textstylePichiexamplebold{dɔ́n}  \textstylePichiexamplebold{fínis}  remata    ín    dé,    Boyé  dɔ́n  kán
e    púl    wí    torí    torí.\\
\textsc{ideo}  \textsc{3pl}  \textsc{prf}  finish  finish.off    \textsc{3sg.indp}  there  \textsc{name}  \textsc{prf}  come
\textsc{3sg.sbj}  remove  \textsc{1pl.indp}  story   \textsc{rep}\\

\glt ‘(When) they had finished him off there [by hitting him with blunt objects] 
(and) Boyé had come, he told us the story.’ [dj05ce 101]\is{aspect}
\z

\section{Comparison}\label{sec:6.9}

Pichi employs particles and verbs for expressing comparative, superlative, and equative degree. Sentence \REF{ex:key:466} exemplifies one of the most common ways of expressing comparative degree. It features the comparee \textit{di tín} ‘the thing’, the parameter verb \textit{bɔkú} ‘be much’, the comparative particle \textit{mɔ́}, the standard marker \textit{pás} ‘(sur)pass’, and the standard \textit{di watá} ‘the water’. As can be seen, the expression of comparison involves a participant-introducing\index{} comparative SVC, in which the V2 \textit{pás} ‘(sur)pass’ functions as the standard marker: 


\ea%466
    \label{ex:key:466}
    \gll Pero    ɛf  di  tín    kán    \textstylePichiexamplebold{bɔkú}  \textbf{mɔ́}    \textbf{pás}\textstylePichiexamplebold{} di  watá,
e    go  lɛ́f    wán  pasta,  (...)\\
but    if  \textsc{def}  thing  \textsc{pfv}    be.much  more  pass    \textsc{def}  water
\textsc{3sg.sbj}  \textsc{pot}  remain  one  paste\\

\glt ‘But if the thing has become more than the water, a paste
will remain (...)’ [dj03do 059]
\z

Pichi exhibits a rich variety of constructions for comparison. They include the cross-linguistic types of “Exceed-1” and “Exceed-2” comparatives \citep{Stassen1985}. The “Exceed-1” comparative involves a comparative SVC featuring the V2 \textit{pás} ‘(sur)pass’. We also find a mixture of a Particle and Exceed comparatives (cf. \sectref{sec:6.9.1}). Equatives, which express equality of degree between a comparee and a standard, may appear in a construction involving a particle, or alternatively, one involving the verb \textit{rích} ‘arrive; equal’.

\tabref{tab:key:6.9} provides an overview of Pichi constructions employed for comparison as well as “similatives” (cf \sectref{sec:6.9.3}). For illustration, it contains elicited variations of the same sentence. The more common constructions are found under the heading “primary”, while the column “secondary” features less common ones. Glosses for the Pichi words in the table are: e ‘3sg.sbj’, fɔ ‘prep’, kin ‘hab’, lɛ́k ‘like’, lɔ́n ‘be long; tall’, mán ‘man; person’, mí ‘1sg.indp’, mɔ́ ‘more’, ɔ́l ‘all’, pás ‘(sur)pass’, rích ‘arrive; equal’, sɛ́ns ‘intelligence’ and wáka ‘walk’:

%%please move \begin{table} just above \begin{tabular
\begin{table}
\caption{Comparison}
\small
\label{tab:key:6.9}
\resizebox{\linewidth}{!}{
\begin{tabularx}{\textwidth}{llQll}
\lsptoprule
 & \multicolumn{2}{c}{Primary} & \multicolumn{2}{c}{Secondary}\\
Type & Subtype & Example & Subtype & Example\\
\midrule
Comparative & (1) Particle + Exceed-1 & {\itshape e mɔ́ lɔ́n pás mí;}

\itshape e lɔ́n mɔ́ pás mí & Exceed-2 & \itshape e pás mí fɔ sɛ́ns\\
& (2) Exceed-1 SVC & \itshape e lɔ́n pás mí &  & \\

\tablevspace
Superlative & (1) Particle + Exceed-1 & \itshape e mɔ́ lɔ́n pás ɔ́l mán & Exceed-2 & \itshape e pás ɔ́l mán fɔ sɛ́ns\\
& (2) Exceed-1 SVC & \itshape e lɔ́n pás ɔ́l mán &  & \\

\tablevspace
Equative & Particle & \itshape e lɔ́n lɛk mí & Equal & \itshape e rích mí fɔ sɛ́ns\\

\tablevspace
Similative & Particle & \itshape e kin wáka lɛk mí & — & \\
\lspbottomrule
\end{tabularx}
}
\end{table}
In general, relative comparison featuring an explicit standard is less common than absolute comparatives and superlatives, in which the standard must be recovered from discourse context. Speakers often employ the rich inventory of inherently graded verbs, adverbs, particles, phrasal expressions and suprasegmentals for the expression of gradation. 

\subsection{Comparatives}\label{sec:6.9.1}

A participant-introducing\index{} SVC featuring the verb \textit{pás} ‘(sur)pass’ is employed to express comparative degree in an “Exceed-1” comparative \citep{Stassen1985}. The following example features the property item \textit{bíg} ‘be big’ as the parameter verb: 


\ea%467
    \label{ex:key:467}
    \gll Dán    gɛ́l,  a    tɛ́l  yú    sé    e    chapea
lɛk  wán    sáy    wé  e    \textstylePichiexamplebold{bíg}    \textstylePichiexamplebold{pás}    dí  wán.\\
that    girl  \textsc{1sg.sbj}  tell  \textsc{2sg.indp}  \textsc{quot}    \textsc{3sg.sbj}  weed
like  one    side    \textsc{sub}  \textsc{3sg.sbj}  be.big  pass    this  one\\

\glt ‘That girl, I tell you that she weeded like a place that was bigger
than this.’ [ed03sb 060]
\z

In contexts other than comparison, the verb \textit{pás} occurs as a lexical verb with the meanings ‘(sur)pass; pass by; move along’ as in the following three examples: 


\ea%468
    \label{ex:key:468}
    \gll Porque  a    bin  \textbf{pás}  na  Camerún    fɔ́s.\\
because  \textsc{1sg.sbj}  \textsc{pst}  pass  \textsc{loc}  \textsc{place}    first\\

\glt ‘Because I passed through Cameroon first.’ [fr03ft 098]
\z


\ea%469
    \label{ex:key:469}
    \gll Tú  dé  wé  e    pás  bihɛ́n,  a    sí  mi    mamá.\\
two  day  \textsc{sub}  \textsc{3sg.sbj}  pass  behind  \textsc{1sg.sbj}  see  \textsc{1sg.poss}  mother\\

\glt ‘Two days ago, I saw my mother.’ [ye05ce 044]
\z


\ea%470
    \label{ex:key:470}
    \gll Yu  sí  di  stík    e    de  \textbf{pás}  ɔntɔ́p  watá?\\
\textsc{2sg}  see  \textsc{def}  tree    \textsc{3sg.sbj}  \textsc{ipfv}  pass  on    water\\

\glt ‘Do you see the stick passing by on the water?’ [ro05de 002]
\z

An SVC can express comparison \REF{ex:key:467} on its own. However, the adverb of degree \textit{mɔ́} ‘more’ is equally often employed in addition to \textit{pás} to form a “mixed comparative” \citep{Stassen1985}. The adverb \textit{mɔ́} ‘more’ functions as an intensifier, albeit highly conventionalised in its use, rather than being an indispensable element of the comparative construction. It exhibits word order\is{word order} flexibility and may occur after \REF{ex:key:471} or before \REF{ex:key:472} the parameter verb:


\ea%471
    \label{ex:key:471}
    \gll Dán  wán  wé  e    \textstylePichiexamplebold{lɔ́n}    \textstylePichiexamplebold{mɔ́},    na  ín
de  salút  dán  ɔ́da    tú  húman  dɛn.\\
that  one  \textsc{sub}  \textsc{3sg.sbj}  be.long  more  \textsc{foc}  \textsc{3sg.indp}
\textsc{ipfv}  greet  that  other  two  woman  \textsc{pl}\\

\glt ‘The one who is taller, it’s her that’s greeting the other two
women.’ [dj07re 039]
\z


\ea%472
    \label{ex:key:472}
    \gll Náw    náw    mí    de  chɛ́k  sé    Libreville  wet    yá,
yá    \textstylePichiexamplebold{mɔ́}    \textstylePichiexamplebold{día}      pás  dé.\\
now    \textsc{rep}    \textsc{1sg.indp}  \textsc{ipfv}  check  \textsc{quot}    \textsc{place}    with    here
here    more  be.expensive  pass  there\\

\glt ‘Right now, I [\textsc{emp}] think that Libreville and here, here is more expensive
than there.’ [ma03hm 052]
\z

I assume that preverbal \textit{mɔ́} ‘more’ is being reinforced by the Spanish comparative construction featuring the adverb \textit{más} ‘more’. The comparative constructions of both languages exhibit an identical linear structure. Compare \REF{ex:key:473} in colloquial Spanish with \REF{ex:key:472} above:


\ea%473
    \label{ex:key:473}
    \gll Aquí  es     \textbf{más}    \textbf{caro}      \textstylePichiexamplebold{que}    allá.\\
here    is    more  expensive  than  there\\

\glt ‘Here [it] is more expensive than there.’
\z

In the absolute comparative in \REF{ex:key:474} below, \textit{mɔ́} ‘more’ occurs as a prenominal modifier to the Spanish noun \textit{énfasis} ‘emphasis’. The categorial flexibility of \textit{mɔ́} ‘more’ is exploited by insertion in a Spanish adjective position in a code-mixed collocation. This Pichi-Spanish verb-noun combination is creatively used to render the meaning ‘be emphatic’:


\ea%474
    \label{ex:key:474}
    \gll Mék    e    gɛ́t  mɔ́    énfasis.\\
\textsc{sbjv}    \textsc{3sg.sbj}  get  more  emphasis\\

\glt ‘Let it be more emphatic [than usual].’ [dj05ce 126]
\z

The corpus also contains an example in which \textit{mɔ́} ‘more’ is employed both in pre- and post-verbal position in order to signal an emphatic absolute comparative:


\ea%475
    \label{ex:key:475}
    \gll E    púl    mɔ́    plɛ́nte    mɔ́.\\
\textsc{3sg.sbj}  remove  more  be.plenty  more\\

\glt ‘He removed much more.’ [au07fn 109]
\z

However, unmixed “Exceed” comparatives are particularly common when the parameter is dynamic, not a property item, and hence semantically neutral as to gradation. The use of \textit{mɔ́} ‘more’ with such verbs automatically results in a quantity gradation, and \textit{mɔ́} can only occur after the parameter in order to modifiy the predicate in its entirety \REF{ex:key:476}:


\ea%476
    \label{ex:key:476}
    \gll Porque  ɔ́da    sáy  fít  dé    wé,  a    go  wók    só,
a    go  \textstylePichiexamplebold{wín}   \textstylePichiexamplebold{mɔ́}    \textstylePichiexamplebold{pás}   dé.\\
because  other  side  can  \textsc{be.loc}  \textsc{sub}  \textsc{1sg.sbj}  \textsc{pot}  work  like.that
\textsc{1sg.sbj}  \textsc{pot}  earn    more  pass    there\\

\glt ‘Because there could be another place where, (if) I worked like this,
I might earn more than there.’ [dj07ae 495]
\z

When a verb is to be graded as to some defined quantity or some kind of quality, \textit{mɔ́} ‘more’ is usually omitted. Instead, a degree modifier or an object that specifies the quality or quantity may intervene between the parameter and \textit{pás} ‘(sur)pass’. Compare the adverbial modifier \textit{fáyn} ‘fine’ in \REF{ex:key:477} and the object \textit{Bubɛ} ‘Bube’ in \REF{ex:key:478}:


\ea%477
    \label{ex:key:477}
    \gll Dí  wán  dɔ́n  de  tɔ́k,  dí  wán  de  \textstylePichiexamplebold{tɔ́k}  \textstylePichiexamplebold{fáyn}  
\textstylePichiexamplebold{pás}   in    sísta\\
this  one  \textsc{prf}  \textsc{ipfv}  talk  this  one  \textsc{ipfv}  talk  fine  
pass    \textsc{3sg.poss}  sister\\

\glt ‘This one talks, this one talks [the Bube language] 
better than her sister.’ [ab03ab 010]
\z


\ea%478
    \label{ex:key:478}
    \gll Lage    de  \textstylePichiexamplebold{tɔ́k}  \textstylePichiexamplebold{Bubɛ}  \textstylePichiexamplebold{pás}  mí.\\
\textsc{name}  \textsc{ipfv}  talk  Bube  pass  \textsc{1sg.indp}\\

\glt ‘Lage talks Bube (better) than me.’ [fr03ab 012]
\z

When the parameter is a motion verb,\is{motion verbs} the “Exceed” comparative may acquire quite a literal meaning as in \REF{ex:key:479}. The example below also shows that the standard can be modified further by way of a relative clause\is{relative clauses}. Such a relative clause with a locative\is{locative clauses} head noun may be employed in contexts where the parameter is non-gradable and the standard is an entire clause \REF{ex:key:480}:\is{locative clauses}


\ea%479
    \label{ex:key:479}
    \gll A    de  gó  fawe  \textstylePichiexamplebold{pás}  \textstylePichiexamplebold{di}  \textstylePichiexamplebold{sáy}  wé  Paquita  sidɔ́n.\\
\textsc{1sg.sbj}  \textsc{ipfv}  go  far    pass  \textsc{def}  side  \textsc{sub}  Paquita  stay\\

\glt ‘I’m going farther than the place where Paquita lives.’ [ro05ee 082]
\z


\ea%480
    \label{ex:key:480}
    \gll A    báy  \textstylePichiexamplebold{pás}  \textstylePichiexamplebold{di}  \textstylePichiexamplebold{sáy}  wé  di  mɔní  rích.\\
\textsc{1sg.sbj}  buy  pass  \textsc{def}  side  \textsc{sub}  \textsc{def}  money  arrive\\

\glt ‘I bought more than the money was sufficient for.’ [rofn05 001]
\z

The collocation \textit{lɛk háw} ‘the way (that); as soon as’ may also introduce the standard of complex comparatives like \REF{ex:key:481}, in which the standard is an entire adverbial clause. Note the presence of the standard marker \textit{pás} ‘(sur)pass’: 


\ea%481
    \label{ex:key:481}
    \gll Na  lɛk  sé  yu  wánt  tɛ́l  wán  pɔ́sin  sé  yu  dú  sɔn
tín    pás    lɛk  háw    yu  bin  gɛ́fɔ    dú=an.\\
\textsc{foc}  like  \textsc{quot}  \textsc{2sg}  want  tell  one  person  \textsc{quot}  \textsc{2sg}  do  some
thing  pass    like  how    \textsc{2sg}  \textsc{pst}  have.to  do=\textsc{3sg.obj}\\

\glt ‘It’s as if you want to tell a person that you’ve done something more 
than what you should have done.’ [au07ec 049]
\z

The standard clause in \REF{ex:key:482} is also introduced by \textit{lɛk háw} ‘the way (that); as soon as’. The sentence features the locative\is{locative nouns} noun \textit{pantáp} ‘on; in addition to’ as a standard marker instead of \textit{pás}. The use of \textit{pantáp} in this way is only attested in such complex comparatives: 


\ea%482
    \label{ex:key:482}
    \gll Bɔt  yu  nó  fít  tɔ́k  sé    a    chɔ́p  trí    spún
\textstylePichiexamplebold{pantáp}  \textstylePichiexamplebold{lɛk}  \textstylePichiexamplebold{háw}   a    kin  chɔ́p.\\
but  \textsc{2sg}  \textsc{neg}  can  talk  \textsc{quot}    \textsc{1sg.sbj}  eat    three  spoon
on    like  how    \textsc{1sg.sbj}  \textsc{hab}  eat\\

\glt ‘But you can’t say that you have eaten three spoons more than you
usually eat. [au07ec 045]
\z

A second way of forming comparatives is rare. In “Exceed-2” comparatives \citep{Stassen1985} the parameter is expressed as a PP, hence a nominal. The marker of comparison, the verb \textit{pás} ‘(sur)pass’, is the only verb of the clause and is employed as an inchoative-stative verb. 


For these reasons, the construction is more likely to appear with quality-denoting nouns like sɛ́ns ‘intelligence’ in \REF{ex:key:483} than with property-denoting verbs. Compare \REF{ex:key:484}, where the property gɛ́t sɛ́ns ‘have brain’ = ‘be intelligent’ is graded in an “Exceed-1” comparative:



\ea%483
    \label{ex:key:483}
    \gll di  pikín  pás    yú    fɔ  sɛ́ns.\\
\textsc{def}  child  pass    \textsc{2sg.indp}  \textsc{prep}  brain\\

\glt ‘The child is more intelligent than you.’ [ro05de 038]
\z


\ea%484
    \label{ex:key:484}
    \gll E    gɛ́t  sɛ́ns    pás    yú.\\
\textsc{3sg.sbj}  get   brain  pass    \textsc{2sg.indp}\\

\glt ‘He is more intelligent than you.’ [eb07fn 234]
\z

In a second, equally rare variant of the “Exceed-2” comparative, the property is expressed as a possessed noun of the comparee \REF{ex:key:485}: 


\ea%485
    \label{ex:key:485}
    \gll In    sɛ́ns    pás  yu  yón.\\
\textsc{3sg.poss}  brain  pass  \textsc{2sg}  own\\

\glt ‘His intelligence surpasses yours.’ [ro05de 040]
\z

Relative comparatives are rivalled in their frequency by absolute comparatives in which the standard of comparison is absent and logically implied. In absolute comparatives, the use of \textit{mɔ́} ‘more’ as a degree adverbial \REF{ex:key:486} is the most common option. 


\ea%486
    \label{ex:key:486}
    \gll Dí  wán    na  di    hós    wé  \textstylePichiexamplebold{fáyn}    \textstylePichiexamplebold{mɔ́}.\\
\textsc{def}  one    \textsc{foc}  \textsc{def}    house  \textsc{sub}  be.fine    more\\

\glt ‘This is the house that’s more beautiful.’ [nn05fn 011]
\z

In contrast, an SVC with a sentence-final, ‘stranded\is{stranding}’ \textit{pás} as in \REF{ex:key:487} is not accepted by the majority of speakers who were tested: 


\ea%487
    \label{ex:key:487}
    \gll ?Dí  wán  na  di  bɔ́y  wé  \textstylePichiexamplebold{fáyn}  \textstylePichiexamplebold{pás}.\\
 \textsc{def}  one  \textsc{foc}  \textsc{def}  boy  \textsc{sub}  be.fine  pass\\

\glt ?This is the boy who is more handsome. [to07fn 235]
\z

A sentence-final \textit{pás} is all the same common where it occurs in a clause as the only verb (rather than the V2 of an SVC) with the meaning ‘surpass an acceptable limit’ \REF{ex:key:488}: 


\ea%488
    \label{ex:key:488}
    \gll E    dɔ́n  de  \textstylePichiexamplebold{pás}.\\
\textsc{3sg.sbj}  \textsc{prf}  \textsc{ipfv}  pass\\

\glt ‘It’s become too much now.’ [ro05rr 011]
\z

I should point out that in spite of its apparent categorial flexibility, \textit{mɔ́} ‘more’ may not be used as a lexical verb meaning ‘surpass’ like \textit{pás} ‘(sur)pass’ in Pichi, unlike the verb \textit{moro} ‘surpass’ in Sranan Tongo (cf. \citealt{BlankerDubbeldam2010}:139).\is{comparative degree}

\subsection{Superlatives}

Superlatives are formed by the same formal means as comparatives. The reference of the standard NP is extended to englobe the entire set of possible referents by means of a standard NP featuring ɔ́l ‘all’ or ɛ́ni ‘every’ and the relevant group of referents. The standard NP often consists of the generic noun{\fff}s pɔ́sin ‘person’, mán ‘man; person’, húman ‘woman’, and pípul ‘people’ if the comparee is human:


\ea%489
    \label{ex:key:489}
    \gll Boyé  \textstylePichiexamplebold{stáwt}    \textstylePichiexamplebold{pás}  \textstylePichiexamplebold{ɔ́l}  \textstylePichiexamplebold{mán}  na  di  hós.\\
\textsc{name}  be.corpulent  pass  all  man    \textsc{loc}  \textsc{def}  house\\

\glt ‘Boyé is more corpulent than every person in the house.’ [ro05de 060]
\z

However, the most common way of rendering a superlative relation is by means of an absolute superlative without explicit mention of a standard \textsc{NP}. Such constructions are no different from absolute comparatives, and the difference in meaning between the two constructions is inferred from context. 


In the following absolute superlative, the Spanish adjective \textit{difícil} ‘difficult’ is followed by \textit{mɔ́} ‘more’ with a superlative meaning. This sentence was uttered after the speaker had taken us on a tour through a new house and explained the hassles involved in building it:



\ea%490
    \label{ex:key:490}
    \gll Di  tín    wé  bin  dé    \textstylePichiexamplebold{difícil}  \textstylePichiexamplebold{mɔ́}    na  dí  hós,    fɔ  pút  nivel.\\
\textsc{def}  thing  \textsc{sub}  \textsc{pst}  \textsc{be.loc}  difficult  more  \textsc{loc}  this  house  \textsc{prep}  put  level\\

\glt ‘The thing that was most difficult [of all the construction work] in this house, (was) 


\glt to level (the ground).’ [ye07fn 065]
\z

Aside from constructions like \REF{ex:key:490}, which involve an implicit standard, the data abounds with absolute superlatives where the standard is even more vague. Such “superlatives” form part of the inventory of intensifying and emphatic devices of the language. They involve lexicalised phrases like \textit{pás mák} ‘pass (the) limit’ or \textit{nó smɔ́l} ‘\textsc{neg} small’ = ‘not in the least’:


\ea%491
    \label{ex:key:491}
    \gll Di  smɔ́l  wán  dɔ́n  de  tɔ́k  \textstylePichiexamplebold{pás}  \textstylePichiexamplebold{mák}.\\
\textsc{def}  small  one  \textsc{prf}  \textsc{ipfv}  talk  pass  mark\\

\glt ‘The small one already talks unbelievably well.’ [lo07fn185]
\z


\ea%492
    \label{ex:key:492}
    \gll E    nó  fúl      \textstylePichiexamplebold{nó} \textstylePichiexamplebold{smɔ́l}.\\
\textsc{3sg.sbj}  \textsc{neg}  be.foolish  \textsc{neg}  be.small\\

\glt ‘She’s not in the least foolish.’ [ro05ee 135]
\z

Superlative degree may also be signalled by the multifunctional word \textit{óva} ‘over; excessively’ when used as a verb \REF{ex:key:493} and an adverbial \REF{ex:key:494}. 


\ea%493
    \label{ex:key:493}
    \gll Di  chɔ́p  \textstylePichiexamplebold{óva}.\\
\textsc{def}  food    be.excessive\\

\glt ‘The food is too much.’ [au07ec 042]
\z


\ea%494
    \label{ex:key:494}
    \gll Wɛ́n    dɛn  dɔ́n  dríng  óva,    nɔ́?\\
\textsc{sub}    \textsc{3pl}  \textsc{prf}  drink  over    \textsc{intj}\\

\glt ‘When they’ve drunk excessively, right?’ [ma03hm 069]
\z

\textit{\'{O}va} may also appear as the first component of a compound verb which expresses an excessive degree of the situation denoted by the verb (cf. \sectref{sec:4.4.3} for more details):


\ea%495
    \label{ex:key:495}
    \gll Di  hós    \textstylePichiexamplebold{ova-dɔtí}.\\
\textsc{def}  house  over.\textsc{cpd}{}-be.dirty\\

\glt ‘The house is excessively dirty.’ [au07ec 027]
\z

Emphatic absolute superlatives may also involve the use of degree adverbs like \textit{bád} ‘extremely’ \REF{ex:key:496}, \textit{tú (mɔ́ch)} ‘too much’\textit{} \REF{ex:key:497}, or \textit{sóté} ‘until; extremely:


\ea%496
    \label{ex:key:496}
    \gll Dán    húman  \textstylePichiexamplebold{lɔ́n}    \textstylePichiexamplebold{bád.}\\
that    woman  be.long  bad\\

\glt ‘This woman is excessively long.’ [li07pe 064]
\z


\ea%497
    \label{ex:key:497}
    \gll Di  chɔ́p  e    \textstylePichiexamplebold{tú}  \textstylePichiexamplebold{bɔkú}.\\
\textsc{def}  food    \textsc{3sg.sbj}  too  be.much\\

\glt ‘The food is too much.’ [dj05ae 125]
\z

Beyond that, Pichi features a number of inherently comparative and superlative words. Like the degree expressions óva ‘over’ covered above, these words are multifunctional and may be employed as adverbs or verbs alike. The words bɛ́ta ‘be very good’, wos ‘be very bad’, tú mɔ́ch ‘be very/too much’, as well as bɔkú ‘be (very) much’ alone may signal an exceptionally high degree of a quality or quantity: 


\ea%498
    \label{ex:key:498}
    \gll E    \textstylePichiexamplebold{wós}.\\
\textsc{3sg.sbj}  be.very.bad\\

\glt ‘It’s very bad.’ \textsc{or} ‘It’s worse.’ [ra07fn 036]
\z


\ea%499
    \label{ex:key:499}
    \gll Di  prɔ́blɛm  dɛn  \textstylePichiexamplebold{dɔ́n}  \textstylePichiexamplebold{tú}  \textstylePichiexamplebold{mɔ́ch}  (...)\\
\textsc{def}  problem  3\textsc{pl}  \textsc{prf}  too  be.much\\

\glt ‘The problems became too much (…).’ [ma03ni 029]
\z


\ea%500
    \label{ex:key:500}
    \gll Di  chɔ́p  \textstylePichiexamplebold{bɔkú},  di  chɔ́p  e    \textstylePichiexamplebold{tú}  \textstylePichiexamplebold{bɔkú}.\\
\textsc{def}  food    be.much  \textsc{def}  food    \textsc{3sg.sbj}  too  be.much\\
\glt ‘The food is very (or too) much, the food is too much.’
\z

These inherently superlative words may combine with \textit{mɔ́} ‘more’ for additional intensity and emphasis\is{emphasis} as in the following examples. Note the characteristic syntactic flexibility of \textit{mɔ́} in these sentences: 


\ea%501
    \label{ex:key:501}
    \gll E    \textstylePichiexamplebold{mɔ́}    \textstylePichiexamplebold{wós}.\\
\textsc{3sg.sbj}  more  be.very.bad\\

\glt ‘It’s much worse.’ [ra07fn 035]
\z


\ea%502
    \label{ex:key:502}
    \gll Panyá,  na  ín    \textstylePichiexamplebold{wós}      \textstylePichiexamplebold{mɔ́}.\\
Spain  \textsc{foc}  \textsc{3sg.indp}  be.very.bad  more\\

\glt ‘As for Spain, that’s really bad [as a place to live in].’ [ra07fn 040]
\z


\ea%503
    \label{ex:key:503}
    \gll E    bɛ́ta      mɔ́.\\
\textsc{3sg.sbj}  be.very good  more\\

\glt ‘It’s much better.’ [ge07fn 038]
\z


\ea%504
    \label{ex:key:504}
    \gll E    mɔ́    bɛ́ta.\\
\textsc{3sg.sbj}  more  be.very.good\\

\glt ‘It’s much better.’ [ge07fn 039]
\z

Nuances of superlative degree may also be signalled through the use of emphatic suprasegmental features such as extra-high pitch, pitch range expansion, or vowel lengthening, as well as through other emphatic devices, like ideophones\is{ideophones} and reduplication\is{reduplication}.\is{superlative degree}

\subsection{Equatives}\label{sec:6.9.3}

Equative constructions are formed in two ways. The most frequent one involves the preposition \textit{lɛ(kɛ)} ‘like’ as the standard marker. The preposition is inserted between the parameter and the standard. This construction assigns the same degree of a property to both the comparee and the standard: 


\ea%505
    \label{ex:key:505}
    \gll Nó  chɔ́p  nó  dé    wé  e    \textstylePichiexamplebold{swít}    \textstylePichiexamplebold{lɛk} kokó.\\
\textsc{neg}  food    \textsc{neg}  \textsc{be.loc}  \textsc{sub}  \textsc{3sg.sbj}  be.tasty  like  cocoa.yam\\

\glt ‘There’s no food that’s as tasty as cocoa yam.’ [ro05ee 141]
\z


\ea%506
    \label{ex:key:506}
    \gll E    nó  \textstylePichiexamplebold{fáyn}  \textstylePichiexamplebold{lɛk}  mí.\\
\textsc{3sg.sbj}  \textsc{neg}  fine    like  \textsc{1sg.indp}\\

\glt ‘He isn’t as handsome as me.’ [ye07fn 135]
\z

Take note of the lexicalised equative construction bɔkú lɛ́k nyɔ́ní ‘be many like ants’ in \REF{ex:key:507}: 


\ea%507
    \label{ex:key:507}
    \gll Yu  fít  tɔ́k  sé    ‘mi    brɔ́da  dɛn  \textstylePichiexamplebold{bɔkú} \textstylePichiexamplebold{lɛk}  \textstylePichiexamplebold{nyɔ́ní}’.\\
\textsc{2sg}  can  talk  \textsc{quot}    \textsc{1sg.poss}  brother  \textsc{3pl}  be.much  like  ant\\

\glt ‘You can say “my siblings are many just like ants”.’ [ro05ee 034]
\z

In constructions featuring an entire equative clause as the standard, the collocation lɛk háw ‘like how’ = ‘the way that’ is used instead of lɛ́k \REF{ex:key:508}–\REF{ex:key:509}. The second example below features a code-mixed equative construction featuring the Spanish element tan ‘as; so’. In unmixed sentences, Pichi does not employ an additional parameter marker like tan before the parameter verb:


\ea%508
\label{ex:key:508}
\gll  (...)  mék    yu  nó  para    sóté    mék    e    \textstylePichiexamplebold{tík}    \textstylePichiexamplebold{lɛk}  \textstylePichiexamplebold{háw}
e    bin  dé    só.\\
{}  \textsc{sbjv}    \textsc{2sg}  \textsc{neg}  stop    until  \textsc{sbjv}    \textsc{3sg.sbj}  be.thick  like  how  
\textsc{3sg.sbj}  \textsc{pst}  \textsc{be.loc}  like.that \\
\glt ‘(...) don’t stop until it’s (as) thick as it was.’ [dj03do 058]
\z


\ea%509
    \label{ex:key:509}
    \gll Mí    nóto  tan  dɛ́bul  lɛk  háw    yu  de  chɛ́k  mí.\\
\textsc{1sg.indp}  \textsc{neg}.\textsc{foc}  as  devil  like  how    \textsc{2sg}  \textsc{ipfv}  think  \textsc{1sg.indp}\\

\glt ‘I’m not as much of a devil as you think I am.’ [ye07fn 002]
\z

Pichi speakers employ a second, albeit marginal equative construction, in which the verb \textit{rích} ‘arrive’ is the only verb. At the same time, the parameter appears as a nominal constituent in a \textit{fɔ}{}-prepositional phrase. Like the verb \textit{pás} ‘(sur)pass’ in \REF{ex:key:483} above, the verb \textit{rích} is employed as an inchoative-stative verb in these instances: 


\ea%510
    \label{ex:key:510}
    \gll E    nó  \textstylePichiexamplebold{rích}    mí    \textstylePichiexamplebold{fɔ}  \textstylePichiexamplebold{fáyn}.\\
\textsc{3sg.sbj}  \textsc{neg}  arrive  \textsc{1sg.indp}  \textsc{prep}  fine\\

\glt ‘He doesn’t equal me in beauty.’ [ye07fn 134]\is{equative degree}
\z

Other than that, verb \textit{rích} is employed as an allative motion verb\is{motion verbs} ‘reach; arrive (at)’. In addition to its literal sense, \textit{rích} also occurs with the meaning ‘equal; be sufficient’ \REF{ex:key:511}. \textit{Rích} may also be found as a minor verb in the V2 position of a motion-direction SVC \REF{ex:key:512}: 


\ea%511
    \label{ex:key:511}
    \gll E    dɔ́n  \textbf{rích}.\\
\textsc{3sg.sbj}  \textsc{prf}  arrive\\

\glt ‘It’s enough.’ \textsc{Or} ‘S/he has arrived.’ [dj07ae 356]
\z


\ea%512
    \label{ex:key:512}
    \gll A    wánt  \textbf{fláy}  \textbf{rích}    na  tɔ́n    náw  náw.\\
\textsc{1sg.sbj}  want  fly  arrive  \textsc{loc}  town  now  \textsc{rep}\\

\glt ‘I want to hurry to town right now.’ [dj07ae 362]\is{comparative constructions}
\z

