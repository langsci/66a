\chapter{Segmental phonology}

The phonological system of Pichi features a phoneme inventory of twenty-two consonants and seven vowels. There is a good deal of free and allophonic variation in the use of these phonemes. Phonological processes include nasalisation, the use of clitics and the appearance of a linking /r/ during cliticisation, as well as the reduction of consonant clusters by deletion and insertion. In general, however, Pichi speakers tend to fully articulate consonants and vowels. The majority of Pichi words consist of one or two syllables. There are no phonemic long vowels but words may feature clusters of up to three consonants. The segmental system of Pichi interacts in various ways with the suprasegmental system (cf. chapter 3).

\section{Consonants}\label{sec:2.1}

The maximal inventory of twenty-two consonant phonemes in Pichi is presented in IPA symbols in \tabref{tab:key:2.1}. Details on the status and distribution of these phonemes are discussed in sections \sectref{sec:2.2} and \sectref{sec:2.6.2.1}.

%%please move \begin{table} just above \begin{tabular
\begin{table}
\caption{Consonant and approximant phonemes}
\label{tab:key:2.1}
\fittable{
\small
\begin{tabular}{lllllllllllllllll}
\lsptoprule

& 
 \multicolumn{2}{p{15mm}}{Bi-labial} &
 \multicolumn{2}{p{15mm}}{Labio-dental} & 
 \multicolumn{2}{p{15mm}}{(Post-)\newline alveolar} & 
 \multicolumn{2}{c}{Palatal} & 
 \multicolumn{2}{c}{Velar} & 
 \multicolumn{2}{p{10mm}}{Labio-velar} & 
 \multicolumn{2}{c}{Uvular} & 
 \multicolumn{2}{c}{Glottal}\\
\midrule 
Stop & p & b &  &  & t & d &  &  & k & g & kp & gb &  &  &  & \\
Affricate &  &  &  &  & tʃ & dʒ &  &  &  &  &  &  &  &  &  & \\
Fricative &  &  & f & v & s &  &  &  &  &  &  &  &  & ʁ &  & h\\
Nasal &  & m &  &  &  & n &  & ɲ &  & ŋ &  &  &  &  &  & \\
Liquid &  &  &  &  &  & l &  &  &  &  &  &  &  &  &  & \\
Approximant &  &  &  &  &  &  &  & j &  & w &  &  &  &  &  & \\
\lspbottomrule
\end{tabular}
}
\end{table}
The following (near-)mininal pairs establish the phonemic status of the segments contained in \tabref{tab:key:2.1}:

%%please move \begin{table} just above \begin{tabular
\begin{table}
\caption{Consonant phoneme minimal pairs}
\label{tab:key:2.2}

\begin{tabularx}{\textwidth}{Qlll@{\qquad\qquad} lll}
\lsptoprule
/p/  /b/ & \itshape plánt & [plánt] & ‘plant’ & \itshape blánt & [blánt] & ‘reside’\\
/t/  /d/ & \itshape tɛ́n & [tɛ́n] & ‘time’ & \itshape dɛ́n & [dɛ́n] & ‘\textsc{3pl.indp}’\\
/k/  /g/ & \itshape kɔ́n & [kɔ́n] & ‘corn’ & \itshape gɔ́n & [gɔ́n] & ‘gun’\\
/tʃ/  /dʒ/ & \itshape chɔ́ch & [tʃɔ́tʃ] & ‘church’ & \itshape jɔ́ch & [dʒɔ́tʃ] & ‘(to) judge’\\
/f/  /p/ & \itshape fát & [fát] & ‘fat’ & \itshape pát & [pát] & ‘part’\\
/v/  /b/ & \itshape greví & [grèví] & ‘gravy’ & \itshape bebí & [bèbí] & ‘baby’\\
/s/  /t/ & \itshape sɔn & [sɔ̀n] & ‘some’ & \itshape tɔ́n & [tɔ́n] & ‘town’\\
/r/  /l/ & \itshape rɔ́n & [rɔ́n] & ‘run’ & \itshape lɔ́n & [lɔ́n] & ‘be long’\\
/h/  ø & \itshape hól & [hól/ & ‘hole’ & \itshape ól & [ól] & ‘be old’\\
/m/  /n/ & \itshape motó & [motó] & ‘car’ & \itshape nóto & [nótò] & ‘\textsc{neg}.\textsc{foc}’\\
/ŋ/  /n/ & \itshape tɔ́n & [tɔ́n] & ‘town’ & \itshape tɔ́ng & [tɔ́ŋ] & ‘tongue’\\
/ɲ/  /y/ & \itshape nyú & [ɲú] & ‘be new’ & \itshape yú & [jú] & ‘\textsc{2sg.indp}’\\
/j/  /w/ & \itshape yés & [jés] & ‘ear’ & \itshape wés & [wés] & ‘buttocks’\\
/kp/ /gb/ & \itshape kpu & [kpù] & ‘\textsc{ideo}’ & \itshape gbin & [gbìn] & ‘\textsc{ideo}’\\
\lspbottomrule
\end{tabularx}
\end{table}
\section{Consonant allophony and alternation}\label{sec:2.2}

/\textbf{b}/ and /\textbf{v}/:

The voiced labio-dental plosive /v/ is a phoneme in its right in a small number of words, where it does not alternate with /b/, e.g. greví [grèví] ‘gravy’ and gív=an [gívàn] ‘give him/her/it’. In a second group of words, /v/ is in free variation with /b/, e.g vájin [bádʒìn{\textasciitilde}vádʒìn] ‘virgin’, ívin [íbìn{\textasciitilde}ívìn] ‘evening’, óva [óbà{\textasciitilde}óvà] ‘over; be excessive’, sɛven [sɛ́bèn{\textasciitilde}sɛ́vèn] ‘seven’, and ríva [ríbà{\textasciitilde}rívà] ‘river’. Free variation is also encountered in the Spanish-derived lexicon of most speakers, as in abuela [abwɛla{\textasciitilde}aßwela{\textasciitilde}avwɛla] ‘grandmother’. 


In a third group of words, we only find /b/, which therefore does not alternate with /v/. Hence, we find fíba [fíbà] ‘resemble’, líba [líbà] ‘liver’, súb [súb] ‘shove’, híb [híb[F05D?] ‘throw’, bába [bábà] ‘cut hair’, and dɛ́bul [dɛ́bùl] ‘devil’. The orthographic representation chosen for words of the second group, in which we find free alternation between [b] and [v], is <v>. Alternating words are given with both variants in the Pichi–English vocabulary section.


/\textbf{tʃ}/ and /\textbf{dʒ}/:

This voiceless postalveolar affricate tends to be unstable with many speakers and optionally alternates with the voicless palatal plosive [c] and sometimes with the voiceless postalveolar fricative [ʃ], particularly in word-final position. Hence we find \textit{tɔ́ch} [tɔ́tʃ{\textasciitilde}tɔ́c{\textasciitilde}tɔ́ʃ] ‘touch’. A small number of speakers, all of which belong to Group 1 (cf. \sectref{sec:1.3}) exhibit an allophonic variation between /tʃ/ and /dʒ/ in some words, with the latter allophone appearing in word-final position before the clitic \textit{=an} ‘\textsc{3sg.obj}’, i.e. \textit{jɔ́ch=an} [dʒɔ́dʒàn] ‘judge him/her/it’. 


The vast majority of speakers, however, and Group 1 speakers in particular, use word-final /tʃ/ in every environment including ones which are not prone to devoicing, i.e. \textit{chénch=an} [tʃéntʃàn] ‘change him/her/it’. I have accounted for the fact that most speakers exhibit no such variation by opting for <ch> in the orthography even though word-final /tʃ/ may be an allophone of /dʒ/ for a minority of speakers in words like \textit{jɔ́ch} ‘judge’ (but not in others, e.g. \textit{kéch} ‘catch’).


/\textbf{s}/:

The voiced alveolar fricative [z] is attested as a free variant of the voicless alveolar fricative between two vowels in word-medial position, e.g. \textit{ísi} [ízì{\textasciitilde}ísì] ‘be easy’ and \textit{lési} [lézì{\textasciitilde}lésì] ‘be lazy’. I take [z] to be a non-phonemic variant of /s/ in these words. 


Furthermore, most Group 1 speakers (cf. \sectref{sec:1.3}) apply an opposition between /s/ and /ʃ/ (rendered by the grapheme <sh>), which produces minimal pairs like \textit{só} [só] ‘sew’ and \textit{shó} [ʃó] ‘show’. For Group 2 speakers, this opposition is, however, neutralised in favour of /s/, and they employ the voiceless alveolar fricative [s] in any position in which Group 1 speakers may use the voiceless postalveolar fricative [ʃ]. Group 2 speakers therefore produce homonyms like \textit{só} [só] ‘sew’ and \textit{só} [só] ‘show’.



Additionally, Group 2 speakers usually insert a palatal glide /j/ between /s/ and either of the mid vowels /e/ and /ɔ/ where Group 1 speakers only employ /ʃ/{\fff}. This inter-group variation applies to the following words in the data: kwɛ́sɔn [kwɛ́sjɔ̀n{\textasciitilde}kwɛ́sʃɔ̀n] ‘question’, nésɔn [nésjɔ̀n{\textasciitilde}néʃɔ̀n] ‘nation(ality)’, séb [sjéb{\textasciitilde}ʃéb] ‘share’, sék [sjék{\textasciitilde}ʃék] ‘shake’, sém [sjém{\textasciitilde}ʃém] ‘shame’, sɔ́t [sjɔ́t{\textasciitilde}ʃɔ́t] ‘be short; shirt’, sén [sjén{\textasciitilde}sén] ‘same’, and sɔ́p [ʃɔ́p] ‘shop’. Although the insertion of /j/ is optional, it is very common with the words listed. The insertion of /j/ is, however, not generalised to two other words in the corpus featuring a sequence of the phonemes /sé/. Hence, we find sé [sé] ‘quot’ and fɔséka [fɔ̀sékà] ‘due to’. 



The orthography does not represent the segment /j/ in words to which insertion applies. The words that exhibit this alternation are listed in the preceding paragraph and are additionally identified in the Pichi–English vocabulary. 


/\textbf{n}/ and /\textbf{m}/: 

The realisation of the alveolar nasal /n/ and the bilabial nasal /m/ is conditioned by a number of factors, which are covered in (\sectref{sec:2.5.2}). 

/\textbf{ny}/ and /\textbf{ɲ}/:

A prothetic /n/ is optional (and present in at least half of the occurrences recorded) in a specific group of words with an underlying word-initial /j/. The relevant words are\textit{ yandá} [jàndá{\textasciitilde}njàndá] ‘yonder’, \textit{yún} [jún{\textasciitilde}njún] ‘be young’ and \textit{yús} [jús{\textasciitilde}njús] ‘use’. In this group of words, I therefore analyse the combination of these segments as a cluster consisting of the alveolar nasal /n/ and the palatal approximant /j/. \is{insertion of segments}


In a second, equally small group of words, I posit the phoneme /ɲ/, compare the minimal pair \textit{nyú} [ɲú] ‘be new’ vs. \textit{yú} [jú] \textsc{‘2sg.indp’}. The other words that do not alternate in my data and therefore appear to feature a word-initial /ɲ/ rather than the cluster /nj/ are \textit{nyangá} [ɲàŋgá] ‘put on airs’, \textit{nyankwé} [ɲànkwé] ‘(the) nyankwé (dance)’, \textit{nyɔ́ní} [ɲɔ́ní] ‘ant’, and \textit{nyús} [ɲús] ‘news’. The phoneme /ɲ/ is also found in a word-medial, syllable onset position in two words in the corpus, namely in the place name \textit{Panyá} [pàɲá] ‘Spain’ and in the ideophone \textit{ményéményé} [méɲéméɲé] ‘whine; nag in a childlike fashion’.



A third group of words with a word-initial /j/ does not usually exhibit nasal prothesis at all, e.g. yɛ́s [jɛ́s] ‘yes’, yét [jét] ‘yet’, yɛ́stadé [jɛ́stàdé] ‘yesterday’, and yáy [jáj] ‘eye’. In the orthography, I only render an initial /n/ with the second group of words, i.e. words that feature the phoneme /ɲ/. Words with an optional prothetic /n/ are listed above and given with their alternate forms in the Pichi–English vocabulary.


/\textbf{j}/:

This voiced palatal approximant is a phoneme in its own right in words like yú [jú] ‘2sg.indp’, yá [já] ‘here’, yɛ́s [jɛ́s] ‘yes’ and yét [jét] ‘yet’. Besides that, some words with a word-initial /j/ optionally appear with a prothetic /n/ (cf. on /n/ below). The segment /j/ is also optionally inserted between /s/ and one of the mid-vowels /e/ and /ɔ/ in another group of words (cf. on /ʃ/ below). {\fff}


Further, /j/ is optionally inserted between either of the velar consonants /g/ and /k/ and the front vowels /a/ and /ɛ/. However, this process only applies to a few relevant words of English origin with which it occurs in the majority of instances. The corpus contains the following words to which this applies: gádin [gádìn{\textasciitilde}gjádìn], gál [gál{\textasciitilde}gjál] ‘girl’, gɛ́l [gɛ́l{\textasciitilde}gjɛ́l] ‘girl’, káp [káp{\textasciitilde}kjáp] ‘cap’, kápinta [kápìntà{\textasciitilde}kjápìntà] ‘carpenter’, and kɛ́r [kɛ́r{\textasciitilde}kjɛ́r] ‘carry’. In contrast, a /j/ is not normally inserted in other words of English origin like gɛ́t [gɛ́t] ‘get’, kán [kán{\textasciitilde}kám] ‘come’, and káyn [kájn] ‘kind’, as well as a group of words of non-English origin with an L.H pitch pattern, amongst them garí [gàrí] ‘garí’, kaká [kàká] ‘defecate’, kasára [kàsárà] ‘cassava’, and kandá [kàndá] ‘skin’. 



The orthography does not render the epenthetic /j/ in words that feature it. All relevant words are listed above and are identified in the Pichi–English vocabulary section. 


/\textbf{r}/:

The symbol /r/ varies in pronounciation between that of a voiced uvular fricative [ʁ] and a velar fricative [ɣ]. Some speakers use an alveolar tap [ɾ] instead of these two segments, and I have also occasionally heard an uvular trill [ʀ]. We therefore find variants like the following: máred [máʁèd{\textasciitilde}máɣèd{\textasciitilde}máɾèd] ‘marry’, dríng [dʁíng{\textasciitilde}dɣíng{\textasciitilde}dɾíng] ‘drink’, kɛ́r [kɛ́ʁ{\textasciitilde}kɛ́ɣ{\textasciitilde}kɛ́ɾ] ‘carry’, and rɛ́s [ʁɛ́s{\textasciitilde}ɣɛ́s{\textasciitilde}ɾɛ́s] ‘rice’. The orthography represents this segment as <r> and as [r] for phonemic and phonetic transcriptions.

/\textbf{h}/:

This voiced glotal fricative is phonemic in a small group of words which is delineated by minimal pairs like hól [hól] ‘hole; hold’ vs. ól [ól] ‘be old’. The group contains words like hát [hát] ‘hurt; heart’, hála [hálà] ‘shout’, hós [hós] ‘house’, and héd [héd] ‘head’. The group also includes two words with a word-medial /h/, namely bihɛ́n [bìhɛ́n] ‘behind’ and wahála [wàhálà] ‘trouble’. 


With a second and larger group, /h/ may be inserted at the beginning of the vowel-initial word. Such a prothetic /h/, although optional, occurs more often than not with most words in this group. Hence we find variants like ánsa [ánsà{\textasciitilde}hánsà] ‘respond’, áks [áks{\textasciitilde}háks] ‘ask’, ópin [ópìn{\textasciitilde}hópìn] ‘open’, and évi [évì{\textasciitilde}ébì{\textasciitilde}hévì{\textasciitilde}hébì] ‘be heavy’. In some instances, it is however impossible to determine whether a word-initial /h/ is prothetic or part of the segmental structure of a word, because the data contains no recorded instance without an initial /h/. Some of the words to which this applies are húman ‘woman’, hɛ́lp ‘help’, hébul ‘be able’, hía ‘year’, hásis ‘ashes’, and hós ‘house’. I have chosen to render these words with an initial <h>.{\fff}



A third group of vowel-initial words is not attested with a prothetic /h/, e.g. \textit{óva} [óvà] ‘be excessive; over’; \textit{ónli} [ónlì] ‘only’, \textit{áfta} [áftà] ‘then’, and \textit{éch} [étʃ] ‘age’. In the orthography, the segment /h/ is only represented with words that always appear with a word- or syllable-initial /h/. 


/\textbf{gb}/ and /\textbf{kp}/:

These two voiced and voiceless labiovelar plosives are marginally phonemic and only occur in a handful of ideophones\is{ideophones}, e.g.\textit{ nák gbin} ‘hit \textsc{ideo}’ = ‘hit hard and unexpectedly’, \textit{sút kpu} ‘shoot \textsc{ideo’} = ‘shoot followed by the sound of a dull impact on the body’.

\section{Vowels}\label{sec:2.3}

The following seven vowel phonemes are found in Pichi. Vowel length is not distinctive. Consonant allophony and alternation are discussed below:

%%please move \begin{table} just above \begin{tabular
\begin{table}
\caption{Vowel phonemes}
\label{tab:key:2.3}

% \begin{tabularx}{\textwidth}{lXXXXXX}
%  & \multicolumn{3}{c}{Front} & \multicolumn{2}{c}{Central} & Back\\
% \lsptoprule
% Close & i &  &  &  &  & u\\
% Close-mid &  & \biberror{[F065?]} &  &  &  & o\\
% Open-mid &  &  & ɛ &  &  & ɔ\\
% Open &  &  &  & a &  & \\
% \lspbottomrule
% \end{tabularx}
\begin{tikzpicture}
 \aeiouEO
\end{tikzpicture}

\end{table}
The following (near-)minimal pairs establish the phonemic status of the segments contained in \tabref{tab:key:2.3}:

%%please move \begin{table} just above \begin{tabular
\begin{table}
\caption{Vowel phoneme minimal pairs}
\label{tab:key:2.4}

\begin{tabularx}{.66\textwidth}{XXX}
\lsptoprule
\itshape mín & [mín] & ‘mean’\\
\itshape mún & [mún] & ‘moon’\\
\itshape mɛ́n & [mɛ́n] & ‘heal’\\
\itshape mán & [mán] & ‘man’\\
\itshape yés & [jés] & ‘ear’\\
\itshape yɛ́s & [jɛ́s] & ‘yes’\\
\itshape ɔ́l & [ɔ́l] & ‘all’\\
\itshape ól & [ól] & ‘be old’\\
\itshape kɔ́l & [kɔ́l] & ‘call’\\
\itshape kól & [kól] & ‘be cold’\\
\lspbottomrule
\end{tabularx}
\end{table}
\section{Vowel allophony and alternation}\label{sec:2.4}

Pichi shows some lexically determined vowel alternation. Hence we find alternate forms like kɛ́r{\textasciitilde}kɛ́ri{\textasciitilde}kári ‘carry; take’, lɛ́k{\textasciitilde}láyk ‘(to) like’, gɛ́l{\textasciitilde}gál ‘girl’, unu{\textasciitilde}una ‘2pl’, wɔ́nt{\textasciitilde}wánt ‘want’. Other than that, there is some variation in the use of mid-vowels, with a tendency towards the reduction of phonemic contrasts. Furthermore, Pichi has vowel-vowel combinations, as well as sequences consisting of an approximant and a vowel. There are no phonemic long vowels in Pichi. The properties of sequences of non-identical vowels are covered in \sectref{sec:2.6.2.2}.

/\textbf{e}/ and /\textbf{ɛ}/: 

Minimal pairs such as yɛ́s [jɛ́s] ‘yes’ vs. yés [jés] ‘ear’ establish the phonemic status of the unrounded close-mid front vowel /e/ and the unrounded open-mid front vowel /ɛ/. However, many speakers collapse the phonemic contrast between /e/ and /ɛ/ by raising /ɛ/ towards /e/. The opposite direction is far less common. Hence, variants like the following ones are attested: lɛ́k [lɛ́k{\textasciitilde}lék] ‘like’, chɛ́k [tʃɛ́k{\textasciitilde}tʃék] ‘check’, kɛ́r [kɛ́r{\textasciitilde}kér] ‘carry’, and nɛ́k [nɛ́k{\textasciitilde}nék] ‘neck’. The use of either variant of a content word also often conditions the vowel quality of preceding or following function words (cf. \sectref{sec:2.5.3}).

/\textbf{o}/ and /\textbf{ɔ}/: 

The phonemic status of the rounded close-mid back vowel /o/ and the rounded open-mid back vowel /ɔ/ is evident in minimal pairs like \textit{kól} [kól] ‘be cold’ vs. \textit{kɔ́l} [kɔ́l] ‘call’ and \textit{fɔ} [\textit{\textup{fɔ̀}}] ‘\textsc{prep}’ vs. \textit{fó} [fó] ‘four’. Nonetheless, many speakers also neutralise this phonemic contrast by raising /ɔ/ towards /o/. With content words, this neutralisation is less common than the /e{\textasciitilde}ɛ/ alternation. However, it is almost generalised with Group 1 speakers (cf. \sectref{sec:1.3}) in words with grammatical functions, such as the associative preposition \textit{fɔ} [\textit{\textup{fɔ̀}}{\textasciitilde}fò] ‘\textsc{prep}’, the comparative adverb \textit{mɔ́} [mɔ́{\textasciitilde}mó] ‘more’, the negator \textit{nó} [nó{\textasciitilde}nɔ́] ‘\textsc{neg}’, the coordinator \textit{ɔ} [ɔ̀{\textasciitilde}ò] ‘or’, the TMA marker \textit{nɔ́ba} [nɔ́bà{\textasciitilde}nóbà] ‘\textsc{neg.prf}’. The negative focus marker \textit{cum} negative identity copula \textit{nóto} ‘\textsc{neg}.\textsc{foc}’ is however routinely pronounced [nótò].

\section{Phonological processes}\label{sec:2.5}

Phonological processes include lenition and fortition, nasalisation, vowel assimilation, deletion and insertion, as well as cliticisation.

\subsection{Lenition and fortition}\label{sec:2.5.1}

Lenition, the weakening of segments, may affect stops in intervocalic position as in \textit{bigín} [bìɣín] ‘begin’. Strengthening, or fortition, affects voiced obstruents, which are generally devoiced in word-final position. Devoicing therefore produces the following word-final variant of segments. The details regarding lenition and fortition outside of these specific contexts require further investigation:


\ea%24
    \label{ex:key:24}
    \gll   Bi\textstylePichiexamplebold{g}.dé      [bì\textstylePichiexamplebold{g}dé]      →    E    bíg.      [è  bí\textstylePichiexamplebold{k}]\\
big.day                  \textsc{3sg.sbj}  be.big\\

\glt ‘Festivity’                ‘It’s big.’
\z


\ea%25
    \label{ex:key:25}
    \gll   Híb=an!      [hí\textstylePichiexamplebold{b}àn]      →    Híb!          [hí\textstylePichiexamplebold{p}]\\
throw=\textsc{3sg.obj}                throw\\


\glt ‘Throw it!’                ‘Throw!’\\
\z

\ea%26
    \label{ex:key:26}
    \gll   Bad-hát      [bàdhát]      →    E    bád.      [è  bá\textstylePichiexamplebold{t}]\\
bad.\textsc{cpd}{}-heart                \textsc{3sg.sbj}  be.bad\\

\glt ‘be mean’                ‘It’s bad.’
\z

\subsection{Nasals and nasal place assimilation}\label{sec:2.5.2}

A number of processes involve nasals and nasalisation. These apply in diverse ways to different groups of words. We have seen that /n/ prothesis or prenasalisation is optional with a group of words featuring an initial /j/ (cf. \sectref{sec:2.2}). Secondly, the following group of verbs with a word-final /i/ and an H.L pitch configuration is optionally (and very frequently) subjected to word-final nasalisation (realised as /n/ or nasalisation of the final /i/): \textit{grídi} [grídì{\textasciitilde}grídìn] ‘be greedy’, \textit{hángri} [hángrì{\textasciitilde}hángrìn] ‘be hungry’, \textit{hɔ́nti} [hɔ́ntì{\textasciitilde}hɔ́ntìn] ‘hunt’, \textit{hɔ́ri} [hɔ́rì{\textasciitilde}hɔ́rìn] ‘hurry’, \textit{ísi} [ísì{\textasciitilde}ísìn] ‘be easy’, \textit{lési} [lésì{\textasciitilde}lésìn] ‘be lazy’, \textit{lɔ́ki} [lɔ́kì{\textasciitilde}lɔ́kìn] ‘be lucky’, \textit{sɔ́ri} [sɔ́rì{\textasciitilde}sɔ́rìn] ‘be sorry’, \textit{wɔ́ri} [wɔ́rì{\textasciitilde}wɔ́rìn] ‘worry’, and \textit{tɔ́sti} [tɔ́stì{\textasciitilde}tɔ́stìn] ‘be thirsty’. This group of words may be contrasted with a second group that also features a word-final /i/, but exclusively occurs with a word-final nasal. In this latter group, we find words such as \textit{físin} [físìn] ‘(to) fish’, \textit{ívin} [ívìn] ‘evening’, \textit{mɔ́nin} [mɔ́nìn] ‘morning’, and \textit{pikín} [pìkín] ‘child’. \is{insertion of segments}


A third group of words features a word-final /i/, but is not attested with a final /n/. This group includes words with an L.H pitch configuration, such as \textit{rɛdí} [rɛ̀dí] ‘be ready’, \textit{greví} [grèví] ‘gravy’, and \textit{dɔtí} [dɔ̀tí] ‘be dirty’. It also contains monosyllabic words like \textit{mí} [mí] ‘\textsc{1sg.indp}’, \textit{sí} [sí] ‘see’, and \textit{grí} [grí] ‘agree’. 



A fourth group involves function words that are subjected to nasal place assimilation. The relevant words are the personal pronouns =an ‘3sg.obj’, dɛn ‘3pl’, and dɛ́n ‘3pl.indp’, the preposition frɔn ‘from’, the locative noun{\fff} bɔtɔ́n ‘under(side)’, the TMA marker and verb kán ‘pfv; come’, the determiner sɔn ‘some; a’, and the pronominal sén ‘same’. In these words, the final nasal is conditioned by the place of articulation of the following segment:



\ea%27
    \label{ex:key:27}
    \gll   Dɛn    bɔkú.            [dɛ̀\textstylePichiexamplebold{m}  \textstylePichiexamplebold{b}ɔ̀\textstylePichiexamplebold{\textmd{kú}}]\\
\textsc{3pl}    be.much\\

\glt ‘They’re many.’
\z


\ea%28
    \label{ex:key:28}
    \gll   Dɛn    gó  dé.            [dɛ̀\textstylePichiexamplebold{ŋ}    \textstylePichiexamplebold{g}ó  dé]\\
\textsc{3pl}    go  there\\
\glt ‘They went there.’
\z


\ea%29
    \label{ex:key:29}
    \gll   Pút=an    dé!            [pútà\textstylePichiexamplebold{n}  \textstylePichiexamplebold{d}é]\\
put=\textsc{3sg.obj}  there\\

\glt ‘Put it there!’
\z

Anticipatory nasalisation of a vowel preceding the nasal consonant of these function words is also commonplace \REF{ex:key:30}. The word-final nasal of these words may be deleted altogether, in which case a nasal trace is left behind with the preceding vowel \REF{ex:key:31}:


\ea%30
    \label{ex:key:30}
    \gll   Dɛn    kán    gí    yú.      [d\textstylePichiexamplebold{ɛ̀ŋ}    k\textstylePichiexamplebold{ã́ŋ}    gí  jú]\\
\textsc{3pl}    \textsc{pfv}    give    \textsc{2sg.indp}\\

\glt ‘(Then) they gave (it) to you.’
\z


\ea%31
    \label{ex:key:31}
    \gll   Háw    dɛn  de  kɔ́l=an?        [háw  dɛ̀n  dè  kɔ́l  \textstylePichiexamplebold{ã̀}]\\
how    \textsc{3pl}  \textsc{ipfv}  call=\textsc{3sg.obj}\\

\glt ‘How is it called?’
\z

Before a pause, hence when there is no assimilatory pressure from following segments, the word-final nasal in these function words may either be realised as [n] or [m], as in \REF{ex:key:32} and \REF{ex:key:33}, respectively. The analysis of a subcorpus revealed that two thirds of prepausal instances of the word-final nasal were realised as [n], with the remaining third being realised as [m]. Instances of prepausal \textit{kán} necessarily involve the content word ‘come’ rather than the homonymous preverbal aspect marker \textit{kán} ‘\textsc{pfv}’. The Pichi equivalent of the content word ‘come’ is more often pronounced as [kám] than [kán] \REF{ex:key:34}:


\ea%32
    \label{ex:key:32}
    \gll   A    sabí=an.            [à  sàbíà\textstylePichiexamplebold{n}]\\
\textsc{1sg.sbj}  know=\textsc{3sg.obj}\\

\glt ‘I know her.’
\z


\ea%33
    \label{ex:key:33}
    \gll   A    gɛ́t  sɔn    dɛn.        [à  gɛ́t  sɔ̀n  dɛ̀m]\\
\textsc{1sg.sbj}  get  some  \textsc{pl}\\

\glt ‘I have some of them.’
\z


\ea%34
    \label{ex:key:34}
    \gll   Kán!                  [ká\textstylePichiexamplebold{m}]\\
come\\

\glt Come!
\z

The orthographic choice of <n> for for the word-final nasal with these grammatical words reflects these tendencies. Nevertheleless, the content word ‘come’ is also written as \textit{kán} in order to preserve the orthographic unity of the etymologically related aspect marker and content word. 

\subsection{Vowel assimilation}\label{sec:2.5.3}

Pichi features a tongue root vowel harmony targeting mid-vowels. The distinction between the [+high] vowel /e/ and the [-high] vowel /ɛ/, and between [+high] /o/ and [-high] /ɔ/ is collapsed in stem vowels. Enclitics and adjoining function words harmonise with the stem. Hence we find → ‘I [pfv] drink (alcohol)’, and dɛn de kéch dɛ́n → [den de kéch dén] ‘they [ipfv] catch them’. Compare \REF{ex:key:35} and \REF{ex:key:36}. Note that in \REF{ex:key:35}, the speaker also collapses the phonemic contrast between /e/ and /ɛ/ in mék /mék/ ‘make’ (cf. \sectref{sec:2.4}):


\ea%35
    \label{ex:key:35}
    \gll   Dɛ́n  dé  mék=an    só.          [dɛ̀n  dɛ̀ mɛ́kàn só]\\
\textsc{3pl}  \textsc{ipfv}  make=\textsc{3sg.obj}  like.that\\

\glt ‘They do it like that.’
\z


\ea%36
    \label{ex:key:36}
    \gll   Dɛ́n  de  kéch  dɛ́n    dé.        [dèn   dè kéch dén dé]\\
\textsc{3pl}  \textsc{ipfv}  catch  \textsc{3pl.indp}  there\\

\glt ‘They habitually catch them there.’
\z


\ea%37
    \label{ex:key:37}
    \gll   E    dɔ́n    drɔ́ngo.          [è d\textbf{ó}n dr\textbf{ó}ngò]\\
\textsc{3sg.sbj}  \textsc{pfv}    be.dead.drunk\\

\glt ‘He is dead drunk.’
\z

These harmonic processes are reflective of a general tendency of function words to be phonologically assimilated to adjoining words. 

\subsection{Insertion and deletion}\label{sec:2.5.4}

We have seen that the insertion of consonants affects various types of words (cf. \sectref{sec:2.5.2} and the entries /h/, /s/, /j/, and /n/ in \sectref{sec:2.6.2.1}). Deletion is less frequent. In general, vowels and consonants of content words tend to be fully articulated (except cf. \REF{ex:key:39}–\REF{ex:key:40}). Nevertheless, high-frequency (function) words tend to be phonologically reduced or fused with adjoining words to a greater degree than other words. One function word, the TMA marker nɛ́a ‘neg.prf’, is not pronounced as the fuller variant [nɛ́và{\textasciitilde}nɛ́bà] in natural speech in the corpus. The virtually complete sound change of this TMA marker is reflected in the orthographic choice of nɛ́a \REF{ex:key:38}. 


This contrasts with the pronunciation of the functionally equivalent word \textit{nɔ́ba} [nɔ́bà{\textasciitilde}nɔ́à] ‘\textsc{neg}.\textsc{prf}’ which occurs equally often in the reduced and full variants. Note that segment deletion may have repercussions for the use of tone (cf. \sectref{sec:3.2.2}): 



\ea%38
    \label{ex:key:38}
    \gll   Dɛn    nɛ́a    rích    dé.        [dɛ̀n    nɛ́à    rích    dé]\\
\textsc{3pl}    \textsc{neg}.\textsc{prf}  arrive  there\\

\glt ‘They haven’t arrived there yet.’\is{deletion of segments“ r}
\z

Pichi speakers exhibit a systematic tendency to break up onset consonant clusters \is{consonant clusters}in which the first segment is the fricative /s/ and the second a liquid or nasal. Both insertion and deletion are employed to achieve this end. The biconsonantal clusters /sl/, /sn/, and /sm/ are very often broken up by insertion of the vowels /i/ or /u/. Thus we have \textit{slíp} [slíp{\textasciitilde}sìlíp] ‘lie down’, \textit{smɔ́l} [smɔ́l{\textasciitilde}sìmɔ́l{\textasciitilde}sùmɔ́l] ‘be small’, and \textit{snék} [snék{\textasciitilde}sìnék] ‘snake’. Biconsonantal sequences of /sk/ and /sp/ are not reduced – hence \textit{skín} [skín] ‘body’ and \textit{spún} [spún] ‘spoon’. 


Optional reduction can be observed with onset clusters involving a sequence of the fricative /s/, a stop, and a fricative or approximant, namely the biconsonantal cluster /st/ and the triconsonantal clusters /str/, /skr/, and /skw/. The possibility of reduction is, however, lexically restricted to specific words in the corpus. Therefore *[tímà] is, for example, rejected for \textit{stíma} [stímà] ‘ship’. The pronunciation of the initial /s/ is optional in the following words, with either variant being equally common: \textit{skrách} [skrátʃ{\textasciitilde}krátʃ] ‘scratch’,\textit{ skwís} [skwís{\textasciitilde}kwís] ‘squeeze’, \textit{stík} [stík{\textasciitilde}tík] ‘tree’, \textit{stón} [stón{\textasciitilde}tón] ‘stone’, \textit{strít} [strít{\textasciitilde}trít] ‘street’, and \textit{strɔ́n} [strɔ́n{\textasciitilde}trɔ́n] ‘be strong’. Next to the words listed above, four additional words occur with an initial /s/ only once in the corpus, namely \textit{tínap} [stínàp{\textasciitilde}tínàp] and its variant \textit{tánap} [stánàp{\textasciitilde}tánàp] ‘stand (up)’, \textit{pínch} [spíntʃ{\textasciitilde}píntʃ] ‘pinch’, and \textit{trímbul} [strímbùl{\textasciitilde}trímbùl] ‘tremble’. Most speakers do not, however, feel comfortable with the /s/-initial alternants of these words. I therefore assume that these alternants are the result of spontaneous back-formation. Words to which optional /s/ deletion applies are given with their alternate forms in the Pichi–English vocabulary list\is{deletion of segments“ r}.



The tendency to avoid clustering also frequently leads to the insertion of an epenthetic vowel into coda consonant clusters featuring liquid-stop sequences. Hence, with the three possible coda clusters /lp/, /lt/, and /lk/ (cf. \tabref{tab:key:2.8}), insertion produces free variants like hɛ́lp [hɛ́lp{\textasciitilde}hɛ́lɛ̀p] ‘help’, bɛ́lt [bɛ́lt{\textasciitilde}bɛ́lɛ̀t], and milk [mílk{\textasciitilde}mílìk] ‘milk’. In addition, Pichi speakers manifest a marked tendency to avoid the clustering of consonants across word boundaries. This leads to the deletion of word-final consonants as in \REF{ex:key:39} and \REF{ex:key:40} below. 



\ea%39
    \label{ex:key:39}
    \gll   A    de  sí    bíg  bíg  fáya.      [à  dè  sí  \textstylePichiexamplebold{bí}  \textstylePichiexamplebold{bí}  fájà]\\
\textsc{1sg.sbj}  \textsc{ipfv}  see    big  \textsc{rep}  fire\\

\glt ‘I was seeing a huge fire.’\index{}
\z


\ea%40
    \label{ex:key:40}
    \gll   If  yu  hól    wán    motó  (…).      [ìf  jù  \textstylePichiexamplebold{hó}  wã\'{}   mòtó]\\
if  \textsc{2sg}  hold    one    car\\

\glt ‘If you temporarily have a car (…).’
\z

The deletion of word-final consonants and the reduction of word-initial clusters is indicative of a general tendency towards CV syllable structures where this is possible. Other processes in which insertion is relevant are covered in \sectref{sec:2.2} and \sectref{sec:2.6.3} and \sectref{sec:3.3}. The latter section also treats the insertion of a linking /r/.\is{insertion of segments}

\section{Phonotactics}\label{sec:2.6}

The distribution of some consonants and vowels has already been touched upon in \sectref{sec:2.2} and \sectref{sec:2.4}. The following sections provide details on the ordering principles of Pichi phonemes. Pichi also exhibits an instance of tone-conditioned suppletive allomorphy, a phenomenon relating to suprasegmental phonotactics covered after the basics of the tone system have been described (cf. \sectref{sec:3.2.5}).

\subsection{The word}\label{sec:2.6.1}

The vast majority of Pichi words are mono- and bisyllabic. In addition, most words carry a single H tone over their only, penultimate, or final syllable (cf. \sectref{sec:3.1.3}). The presence of a single H tone per word and knowledge of the possible tonal configurations therefore provides a means of metrically delineating the prosodic word in very much the same way as the position of stress does in intonation-only languages.

\subsection{The syllable}\label{sec:2.6.2}

The syllable template in Pichi is (C)(C)(C)(V)V(C)(C). A vowel consititutes the syllable nucleus. There are a few single-vowel roots, all of which are function words, e.g. \textit{a} ‘\textsc{1sg.sbj}’, \textit{e} ‘\textsc{3sg.sbj}’, or \textit{ó} ‘\textsc{sp}’. There are no phonemic long vowels in Pichi, adjacent vowels are invariably heterosyllabic. 


Pichi has many words with initial biconsonantal clusters. Some word-initial clusters consisting of three consonants also exist. But both bi- and triconsonantal word-initial onsets tend to be broken up by deletion and insertion (cf. \sectref{sec:2.5.4}). Word-final consonant clusters contain up to two segments and involve nasals, liquids and approximants as the penultimate segment, or the fricative /s/ as the final segment of the coda. In connected speech, a word-final consonant, whether as the final consonant of a clustered coda or the only consonant of a coda, is often deleted. 


\subsubsection{Distribution of consonants}\label{sec:2.6.2.1}
\tabref{tab:key:2.5} presents the distribution of the twenty-two Pichi consonants in syllables (syllable-initial in the onset and syllable-final in the coda) and words (initial, medial, and final). The following abbreviations apply: IO = word initial-onset; MO = word-medial onset; MC = word-medial coda; FC = word-final coda.

%%please move \begin{table} just above \begin{tabular
\begin{table}
\caption{Distribution of consonant phonemes}
\label{tab:key:2.5}

\begin{tabularx}{\textwidth}{rXXXXXXXXXXXXXXXXXXXXXX}
\lsptoprule
 & p & b & t & d & k & g & tʃ & dʒ & f & v & s & r & h & m & n & ɲ & ŋ & l & w & j & kp & gb\\
\midrule
\MakeUppercase{io} & + & + & + & + & + & + & + & + & + & + & + & + & + & + & + & + & {}- & + & + & + & + & +\\
\MakeUppercase{mo} & + & + & + & + & + & + & + & + & + & + & + & + & + & + & + & + & {}- & + & + & + & {}- & {}-\\
\MakeUppercase{mc} & + & {}- & {}- & {}- & + & {}- & {}- & {}- & + & {}- & + & + & {}- & + & + & {}- & + & + & + & + & {}- & {}-\\
\MakeUppercase{fc} & + & + & + & + & + & + & + & {}- & + & {}- & + & + & {}- & + & + & {}- & + & + & + & + & {}- & {}-\\
\lspbottomrule
\end{tabularx}
\end{table}

\tabref{tab:key:2.5} allows the conclusion that all twenty-two consonant phonemes save /ŋ/ occur as word-initial onsets. All consonants except /ŋ/, /kp/, and /gb/ occur as word-medial onsets as well. The latter two phonemes are only attested as word-initial onsets in ideophones. Eleven consonants appear in word-medial codas out of which two consonants appear as word-medial onsets in only two words each, namely /ɲ/ (Panyá ‘Spain; Spanish’ and ményéményé ‘whine; nag in a childlike fashion’) and /h/ (bihɛ́n ‘behind’ and wahála ‘trouble’). Sixteen consonants occur in word-final codas. Examples for the distribution of consonants follow in \tabref{tab:key:2.6}:

%%please move \begin{table} just above \begin{tabular
\begin{sidewaystable}
\caption{Examples for consonant distribution}
\label{tab:key:2.6}

\begin{tabularx}{\textwidth}{l lX lX lX lX}
 & \MakeUppercase{io} &  & MO &  & \MakeUppercase{mc} &  & \MakeUppercase{fc} & \\
\lsptoprule
/p/ & \itshape pépa & ‘paper’ & \itshape kapú & ‘fight’ & \itshape baptáys & ‘baptise’ & \itshape tép & ‘tape’\\
/b/ & \itshape bɛ́t & ‘bite’ & \itshape líba & ‘liver’ & \itshape {}--- & {}--- & \itshape híb & ‘throw’\\
/t/ & \itshape tɔ́ch & ‘touch’ & \itshape nóto & ‘\textsc{neg}.\textsc{foc}’ & \itshape {}--- & {}--- & \itshape pút & ‘put’\\
/d/ & \itshape dásɔl & ‘only’ & \itshape ɔ́da & ‘other’ & \itshape {}--- & {}--- & \itshape blɔ́d & ‘blood’\\
/k/ & \itshape kúk & ‘cook’ & \itshape bɔkú & ‘much’ & \itshape dɔ́kta & ‘doctor’ & \itshape lúk & ‘look’\\
/g/ & \itshape gɔ́d & ‘God’ & \itshape bigín & ‘begin’ & \itshape {}--- & {}--- & \itshape bɛ́g & ‘ask for’\\
/tʃ/ & \itshape chɔ́p & ‘eat’ & \itshape máchis & ‘matches’ & \itshape {}--- & {}--- & \itshape wách & ‘watch’\\
/dʒ/ & \itshape júmp & ‘jump’ & \itshape vájin & ‘virgin’ & \itshape {}--- & {}--- & \itshape {}--- & \textit{{}---}\\
/f/ & \itshape fút & ‘foot, leg’ & \itshape fufú & ‘fufu’ & \itshape áfta & ‘then’ & \itshape lɛ́f & ‘leave’\\
/v/ & \itshape visít & ‘visit’ & \itshape greví & ‘gravy’ & \itshape {}--- & {}--- & \itshape \textup{{}---} & \textit{{}---}\\
/s/ & \itshape sté & ‘stay’ & \itshape pɔ́sin & ‘person’ & \itshape lístin & ‘listen’ & \itshape nɛ́ks & ‘next’\\
/r/ & \itshape rɔ́b & ‘rub’ & \itshape torí & ‘story’ & \itshape malérya & ‘malaria’ & \itshape bɛ́r & ‘bury’\\
/h/ & \itshape héd & ‘head’ & \itshape bihɛ́n & ‘behind’ & \itshape {}--- & {}--- & \itshape {}--- & {}---\\
/m/ & \itshape mék & ‘make’ & \itshape mamá & ‘mother’ & \itshape hambɔ́g & ‘bother’ & \itshape ném & ‘name’\\
/n/ & \itshape nák & ‘hit’ & \itshape fínis & ‘finish’ & \itshape wínda & ‘window’ & \itshape bin & ‘\textsc{pst’}\\
/ɲ/ & \itshape nyɔ́ní & ‘ant’ & \itshape Panyá & ‘Spain’ & \itshape {}--- & {}--- & \itshape {}--- & {}---\\
/ŋ/ & \itshape {}--- & {}--- & \itshape {}--- & {}--- & \itshape bangá & ‘palmtree’ & \itshape líng & ‘lean’\\
/l/ & \itshape lét & ‘be late’ & \itshape pála & ‘parlour’ & \itshape sólya & ‘soldier’ & \itshape púl & ‘remove’\\
/w/ & \itshape wín & ‘defeat’ & \itshape áwa & ‘hour’ & \itshape páwda & ‘powder’ & \itshape háw & ‘how’\\
/j/ & \itshape yá & ‘here’ & \itshape fáya & ‘fire’ & \itshape dráyva & ‘driver’ & \itshape yáy & ‘eye’\\
/kp/ & \itshape kpu & ‘\textsc{ideo}’ & \itshape {}--- & \textit{{}---} & \itshape {}--- & {}--- & \itshape {}--- & \textit{{}---}\\
/gb/ & \itshape gbin & ‘\textsc{ideo}’ & \itshape {}--- & \textit{{}---} & \itshape {}--- & {}--- & \itshape {}--- & \textit{{}---}\\
\lspbottomrule
\end{tabularx}
\end{sidewaystable}
Only roots are taken into account in the table above, not phonological words. In compounds, all consonants that may appear in word-final position in roots may additionally do so in word-medial coda position at the morpheme boundary. Compare the opaque compound \textit{big-dé} ‘big.\textsc{cpd}{}-day’ = ‘festivity’, the reduplicative compound \textit{tɔch-tɔ́ch} ‘touch repeatedly’, and the lexicalised reduplication and ideophone \textit{gbogbogbo} ‘in haste’.


More than one consonant may appear in syllable onsets and codas. \tabref{tab:key:2.7} lists the possible permutations of consonant clusters in syllable onsets, \tabref{tab:key:2.8} lists consonant combinations in the coda. \tabref{tab:key:2.7} shows that up to three consonants may cluster in onsets. Clusters of three consonants may be broken up by deletion and insertion (cf. \sectref{sec:2.5.4}). The sequences /gj/, /kj/, and /sj/ may be said to arise through phonological processes alone (cf. also \sectref{sec:2.2}). The sequences /gj/ and /kj/ surface through optional /j/ epenthesis in words like gál [gál{\textasciitilde}gjál] ‘girl’ and kɛ́r [kɛ́r{\textasciitilde}kjɛ́r] ‘carry’, while the sequence /sj/ appears in variants like sɔ́p [sɔ́p{\textasciitilde}sjɔ́p] ‘shop’ (cf. also \sectref{sec:2.2}).


%%please move \begin{table} just above \begin{tabular
\begin{table}
\caption{Onset consonant clusters}
\label{tab:key:2.7}

\begin{tabularx}{\textwidth}{lX lX}
\lsptoprule
Structure & Composition & Example & Translation\\
\midrule
CCV & Stop + fricative & \itshape pré & ‘pray’\\
&  & \itshape brók & ‘break’\\
&  & \itshape trén & ‘train’\\
&  & \itshape drím & ‘dream’\\
&  & \itshape krés & ‘be crazy’\\
&  & \itshape grí & ‘agree’\\
& Stop + liquid & \itshape plé & ‘play’\\
&  & \itshape bló & ‘relax’\\
&  & \itshape glás & ‘glass’\\
&  & \itshape klás & ‘class’\\
& Stop + approximant & \itshape pyɔ́ & ‘be pure’\\
&  & \itshape bwɛ́l & ‘boil’\\
&  & \itshape ɛskyús & ‘excuse (me)’\\
&  & \itshape tyúsde & ‘Tuesday’\\
&  & \textstyleTablePichiZchn{gál} \textstyleTableEnglishZchn{[gjál]} & ‘girl’\\
&  & kɛ́r [kjɛ́r] & ‘carry; take’\\
&  & \itshape kwáta & ‘quarter’\\
& Fricative + stop & \itshape spɛ́tikul & ‘glasses’\\
&  & \itshape stón & ‘stone’\\
&  & \itshape skúl & ‘school’\\
& Fricative + nasal & \itshape smɔ́l & ‘small’\\
&  & \itshape snék & ‘snake’\\
& Fricative + liquid & \itshape sló & ‘be slow’\\
& Fricative + approximant & \itshape kɔnfyús & ‘confuse’\\
&  & \itshape fwífwífwí & ‘sound of wind blowing’\\
&  & \textstyleTablePichiZchn{séb} \textstyleTableEnglishZchn{[sjéb]} & ‘divide; share’\\
&  & \itshape swɛ́t & ‘(to) sweat’\\
& Fricative + fricative & \itshape fráy & ‘fry’\\
& Affricate + approximant & \itshape jwɛ́n & ‘join’\\
& Nasal + approximant & \itshape nyús & ‘news’\\
CCCV & Fricative + stop + fricative & \itshape strét & ‘be straight’\\
&  & \itshape skrách & ‘scratch’\\
& Fricative + stop + approximant & \itshape spwɛ́l & ‘spoil; spend’\\
&  & \itshape styú & ‘stew’\\
&  & \itshape skwís & ‘squeeze’\\
\lspbottomrule
\end{tabularx}
\end{table}
Coda clusters are limited to maximally two consonants. Coda clusters always involve nasals or continuants, and liquid-stop sequences may also be broken up by epenthetic vowels (e.g. hɛ́lp [hɛ́lɛ̀p] ‘help’). Possible cluster permutations in the coda are listed in \tabref{tab:key:2.8}: {\fff}

%%please move \begin{table} just above \begin{tabular
\begin{table}
\caption{Coda consonant clusters}
\label{tab:key:2.8}

\begin{tabularx}{\textwidth}{lX lX}
\lsptoprule
Structure & Composition & Example & Translation\\
\midrule
VCC & Stop + fricative & \itshape ɛ́ks & ‘egg’\\
& Nasal + stop & \itshape lámp & ‘lamp’\\
&  & \itshape pént & ‘paint’\\
&  & \itshape kɔ́nk & ‘snail’\\
& Nasal + affricate & \itshape chénch & ‘change’\\
& Nasal + fricative & \itshape sɛ́ns & ‘brain’\\
& Liquid + stop & \itshape hɛ́lp & ‘help’\\
&  & \itshape bɛ́lt & ‘belt’\\
&  & \itshape milk & ‘milk’\\
& Liquid + affricate & \itshape bɛ́lch & ‘belch’\\
& Approximant + stop & \itshape wáyp & ‘wipe’\\
&  & \itshape dráyv & ‘drive’\\
&  & \itshape táyt & ‘be tight’\\
&  & \itshape háyd & ‘hide’\\
&  & \itshape láyk & ‘like’\\
&  & \itshape stáwt & ‘be corpulent’\\
&  & \itshape práwd & ‘be boastful’\\
& Approximant + fricative & \itshape láyf & ‘life’\\
&  & \itshape náys & ‘be nice’\\
& Aproximant + nasal & \itshape fáyn & ‘be fine’\\
&  & \itshape ráwn & ‘surround’\\
& Approximant + liquid & \itshape stáyl & ‘manner’\is{consonant clusters}\\
\lspbottomrule
\end{tabularx}
\end{table}
\subsubsection{Distribution of vowels and approximants}\label{sec:2.6.2.2}

All Pichi vowels may occur in the word-initial position. In general, however, vowels only appear in word-initial position in a small number of words. The majority of Pichi words, and content words in particular, either have a consonant, an approximant or a prothetic /h/, sometimes a prothetic /y/ or /w/, in the onset of their initial syllable. 


Most words that do have an initial vowel are function words: personal pronouns (e.g. a ‘1sg.sbj’, e ‘3sg.sbj’, una ‘2pl’, and ín ‘3sg.indp), question word{\fff}s (e.g. údat ‘who’ and all words featuring the clitic question particle ús= ‘q’), clause linkers (e.g. adɔnkɛ́ ‘even if’, ɛf ‘if’, and áfta ‘then’), locative nouns{\fff} (e.g. ínsay ‘inside’ and ɔntɔ́p ‘(on)top’), quantifiers (e.g. ɔ́da ‘other’, ɛ́ni ‘every’), and interjections (e.g. ékié ‘good gracious’, áy ‘expression of pain’). Some content words also feature a word-initial vowel (e.g. aráta ‘rat’, éch ‘age(-grade)’, ívin ‘evening’, and ɛ́nta ‘enter’). In contrast, vowels in word-final position are very common and we find them throughout all word classes (e.g. mí ‘1sg.sbj’, butú ‘stoop over’, sóté ‘until’, nó ‘know’, bɛlɛ́ ‘belly’, fɔ ‘prep’, and sísta ‘sister’). There are certain restrictions on sequences of vowels. Not only are there no phonemic strings of two identical vowels (i.e. long vowels) in Pichi; vowel-vowel sequences are heterosyllabic. In such cases of vowel hiatus, the immediately adjacent nuclei bear polar tones, e.g. bi.ó [L.H] ‘behold’, klí.a [H.L] ‘clear’ vs. *fɔ=an [L.L] ‘for him/her’). This tonotactic restriction triggers a tone-conditioned suppletive allomorphy of two forms instantiating 3sg object case, a typologically interesting phenomenon not attested in other Afro-Caribbean English-lexifier Creoles (cf. \sectref{sec:3.2.5}). There are also only certain types of admissable vowel combinations, provided in \tabref{tab:key:2.9}:


%%please move \begin{table} just above \begin{tabular
\begin{table}
\caption{Vowel sequences}
\label{tab:key:2.9}

\begin{tabularx}{\textwidth}{XXXXXX}
\lsptoprule
 & i & u & o & ɛ & a\\
 \midrule
i &  &  & ìó & íɛ̀ & íà\\
\lspbottomrule
\end{tabularx}
\end{table}
Sequences involving an approximant and a vowel are presented in \tabref{tab:key:2.10}. Pichi features both falling and rising sequences. In the former, the vowel comes first (e.g. /ɔj/), while in rising sequences, the vowel follows the approximant (e.g. [wi]). The logically possible sequences *[ji] and *[ɔw] are not attested in the corpus:

%%please move \begin{table} just above \begin{tabular
\begin{table}
\caption{Sequences involving an approximant and a vowel}
\label{tab:key:2.10}

\begin{tabularx}{\textwidth}{XXXXXXXXXX}
\lsptoprule
 & j & w & i & u & e & o & ɔ & ɛ & a\\
\midrule 
j &  &  & {}--- & ju & je & jo & jɔ & jɛ & ja\\
w &  &  & wi & wu & we & wo & wɔ & wɛ & wa\\
ɔ & ɔj & {}--- &  &  &  &  &  &  & \\
a & aj & aw &  &  &  &  &  &  & \\
\lspbottomrule
\end{tabularx}
\end{table}
A comparison of \tabref{tab:key:2.9} and \tabref{tab:key:2.10} shows that opening sequences are realised as vowel-vowel sequences, while closing sequences are realised as vowel-approximant strings. The circumstances surrounding cliticisation \is{cliticisation}speak to the validity of differentiating between vowel-vowel and vowel-approximant sequences. Due to a restriction imposed by tonal phonotactics, \textit{=an} may not encliticise to a vowel-terminal host if the final vowel of the host carries a low tone (cf. \sectref{sec:3.2.5}). Monosyllabic verbs featuring an approximant as the final segment may, however, take the object pronoun \textit{=an}. Compare the verb \textit{báy} ‘buy’ in \REF{ex:key:41}: 

\ea%41
    \label{ex:key:41}
    \gll   Yu  wánt  \textstylePichiexamplebold{báy}=an    na  puerto  (...)\\
\textsc{2sg}  want  buy=\textsc{3sg.obj}  \textsc{loc}  port\\
\glt  ‘(If) you want to buy it at the port (...).’ 
\z

If the word-final approximant /j/ in \textit{báy} [báj] ‘buy’ were an [i], i.e. a vowel, and a tone-bearing segment in its own right, it should be low-toned in accordance with Pichi tonal phonotactics (since it is preceded by a high-toned vowel [á]). A low-toned final vowel would, in turn, block the encliticisation of \textit{=an} as it does with other verbs with a final low tone. This is, however, not the case, since the sequence [áj] is monomorphemic and bears a single high tone. There is thus no restriction on the encliticisation of \textit{=an}. The same principle applies to other verbs with a final approximant, e.g. \textit{aláw=an} ‘allow=\textsc{3sg.obj}’ = ‘allow her/him’.


The distribution of approximants in the syllable may be read from the tables given in \sectref{sec:2.6.2.1}. Some observations are in order here on variation in strings of approximants and vowels. The verb drɛ́b ‘drive’ features the variants [drɛ́b{\textasciitilde}drájb]. However this free alternation is not encountered with other words to which it could potentially apply. Hence on the one hand, we find bɛ́t [bɛ́t] and fɛ́t [fɛ́t] ‘fight’. On the other hand, words like bráyt [brájt] ‘be bright’, táyt [tájt] ‘be tight’, and wáyp [wájp] ‘wipe’ do not have less complex variants with a monosegmental [ɛ] instead of the bisegmental [aj]. 



The series [ɔj] is found in two groups of words. The first group consists of only two words in the corpus. A second group of words exhibits a free alternation between the strings [ɔj] and [wɛ] with a preference for the latter sequence. A third group of words invariably features [wɛ] and is not attested with the [ɔj] variant:

\eabox{\label{ex:key:42}
\begin{tabularx}{\textwidth}{llll}
Group 1 & \itshape bɔ́y & [bɔ́j] & ‘boy’\\
 & \itshape ɔ́yl & [ɔ́jl] & ‘oil’\\
Group 2 & \itshape spwɛ́l & [spwɛ́l{\textasciitilde}spɔ́jl] & ‘spoil; spend’\\
 & \itshape bwɛ́l & [bwɛ́l{\textasciitilde}bɔ́jl] & ‘boil’\\
 & \itshape jwɛ́n & [dʒwɛ́n{\textasciitilde}dʒɔ́jn] & ‘join’\\
Group 3 & \itshape swɛ́la & [swɛ́là] & ‘swallow’\\
 & \itshape kwɛ́nch & [kwɛ́ntʃ] & ‘die off’\\
 & \itshape kwɛ́sɔn & [kwɛ́sjɔ̀n] & ‘question’\\
 & \itshape wɛ́l & [wɛ́l] & ‘be well’\\
\end{tabularx}
}
Note that group 1 contrasts with group 2 in that [ɔj] in group 1 is either word-final (i.e. bɔ́y) or word-initial and the nucleus of a syllable without an onset (i.e. ɔ́yl). In turn, words in group 3 are either bisyllabic (i.e. kwɛ́sɔn and swɛ́la) and feature a consonant cluster{\fff} in the coda (i.e. kwɛ́nch) or begin with the alternating feature (i.e. wɛ́l). Hence the characteristic environment for the [wɛ́{\textasciitilde}ɔ́j] alternation is a monosyllabic word with a heavy syllable, a single consonant in the coda, and an onset featuring a stop (or a stop component like the affricate [dʒ].{\fff} 

\subsection{Cliticisation}\label{sec:2.6.3}

Pichi has at least two clitics which participate in forming phonological words. The proclitic question particle \textit{ús=} ‘\textsc{q}’ attaches to mostly generic nouns in order to form basic question words. The enclitic object pronoun \textit{=an} ‘\textsc{3sg.obj}’ attaches to verbs, prepositions, locative nouns, and in double-object constructions to other object pronouns (i.e. “the hosts”). 


Cliticisation in Pichi is characterised by segmental reduction, the loss of morphosyntactic independence, and inseparability from the host. Two elements can be considered full clitics by these criteria: The object pronoun \textit{=an} ‘\textsc{3sg.obj}’ and the question particle \textit{ús=} ‘\textsc{q}’. Other elements are clitic-like to a lesser degree: Dependent person pronouns may be said to be enclitic to the following element of the predicate, the pluraliser\is{pluraliser} \textit{dɛn} ‘\textsc{pl}’ to the preceding noun.



The question element \textit{ús=} ‘\textsc{q}’ is proclitic to generic nouns in question words\is{question words}. These question words form single prosodic words, and the proclitic is phonologically adapted to the host; hence \textit{ús=tín} [útín] ‘what’ and \textit{ús=káyn} [úkájn] ‘which’.



The object pronoun =an ‘\textsc{3sg.obj}’ is enclitic to the preceding verb, preposition, or locative noun\index{} with which it forms a single phonological word. The pronoun \textit{=an} ‘\textsc{3sg.obj}’ may also encliticise to a preceding H-toned object pronoun in double-object constructions (cf. \sectref{sec:9.3.4}). The pronoun undergoes a higher than usual degree of segmental reduction, hence we find the variants [=àn{\textasciitilde}ã\`{} {\textasciitilde}à]. Under certain conditions, the enclisis of \textit{=an} triggers a tone-conditioned suppletive allomorphy, a (tonal) phonotactic phenomenon described in \sectref{sec:3.2.5}.


