\chapter{Pragmatic elements and routines}

The pragmatically oriented elements covered in this chapter form part of a range of performative and expressive devices which extend into the realm of gesture and body posture. In this chapter, sections are dedicated to ideophones, interjections, and two other elements that have much in common with interjections but defy rigid demarcation: The sentence particle \textit{ó} ‘\textsc{sp}’ as well as ‘suck teeth’, a phonetic realisation with important pragmatic functions. 


Like interjections, ideophones represent an expressive dimension of communicative interaction. Also common to both micro word class\is{word classes}es is that many of their members do not enter into grammatical constructions with other word classes, and are fít into discourse by adjunction. Equally, many interjections and ideophones manifest some degree of sound symbolism, hence the relation between form and meaning is not entirely arbitrary. However, while interjections can occur as utterances on their own, ideophones (unless they are interjectional) occur as parts of utterances. 



The two final sections cover two important manifestations of the socio-pragmatic domain of politeness, namely the address system and greeting routines. 


\section{Ideophones}\label{sec:12.1}

Ideophones are words with expressive and imaginistic semantics and particular structural characteristics (e.g. \citealt{Westermann1930}:187–189; \citealt{Doke1935}:118–119; \citealt{Dingemanse2017}). Ideophones are found in different syntactic categories in Pichi, but they may be seen to constitute a coherent semantic class. Compare the ideophone \textit{gbogbogbo} in \REF{ex:key:1613}, which expresses haste or precipitous hurry: 


\ea%1613
    \label{ex:key:1613}
    \gll Tokobé  dɔ́n  wɛ́r    klós    gbogbogbo.\\
\textsc{name}  \textsc{prf}  wear  clothing  \textsc{ideo}\\

\glt ‘Tokobé had put on (her) clothes in haste.’ [ab03ab 111]
\z

The importance of iteration with ideophones reflects the sound symbolic dimension of this word class. Compare the ideophonic verbs (a) and ideophonic nouns (b) in \REF{ex:key:1614}:

\eabox{\label{ex:key:1614}
\begin{tabularx}{\textwidth}{rlX}
a. & \itshape kata.kátá & ‘be (hyper-)active; hectic’\\
 & \itshape ményé.ményé & ‘nag in a childlike fashion’\\
b. & \itshape pɔtɔ.pɔ́tɔ́ & ‘mushy substance; mud’\\
 & \itshape wuru.wúrú & ‘disgrace; confusion’\\
\end{tabularx}
}
Like other ideophones, the words listed above have a rather unusual segmental structure: they involve bisyllabic simplex forms which feature a single vowel type (e.g. /a/ in \textit{kata}{}-) and two “similar” consonants (e.g. /w/ and /r/ in \textit{wuru}{}-). Other ideophones feature\is{reduplication} the phonemes /gb/ and /kp/, which are only attested with this word class (i.e. \textit{gbin}, \textit{gbogbogbo}, and \textit{kpu}) or otherwise rare clusters like /fw/ in \textit{fwífwífwí}\textit{ \textup{‘sound of the wind blowing’}}. 


Most ideophones in the corpus involve some form of iteration (cf. \sectref{sec:4.5}). Iteration may involve reduplication (e.g. \textit{katakátá} ‘be hectic’) or be syntactic and involve repetition (e.g. \textit{fwífwífwí} ‘sound of the wind blowing’). Most ideophones are preferably used as adverbs and therefore occur in the postverbal, adverbial position in order to modify the verb with respect to manner. A few ideophones preferably function as verbs or nouns, and one is used as an interjection (\textit{kɔ́ngkɔ́ngkɔ́ng} ‘seek permission to enter’). All ideophones that involve a form of iteration are listed in \REF{ex:key:1615}:


\eabox{\label{ex:key:1615}
\begin{tabularx}{\textwidth}{llXX}
Word class attested & Example & Translation\\
Verb & \itshape kakara & ‘be restless’\\
 & \itshape katakátá & ‘be active, hectic’\\
 & \itshape ményéményé & ‘whine; nag in a childlike fashion’\\
Verb \& adjective & \itshape \textstyleTablePichiZchn{wɔwɔ́} & ‘be ugly; in disorder’\\
Noun & \itshape pɔtɔpɔ́tɔ́ & ‘mushy substance; mud’\\
 & \itshape wuruwúrú & ‘deceit’\\
Adverb & \itshape fwífwífwí & ‘sound of wind blowing’\\
 & \itshape gbin & ‘sound of a hard, sudden blow’\\
 & \itshape gbogbogbo & ‘in haste’\\
 & \itshape kamúkamú & ‘sight of the buttocks moving’\\
 & \itshape kutuku & ‘sound of heart beating’\\
 & \itshape wéwé; wówó & ‘sound of crying and wailing’\\
Interjection & \itshape kɔ́ngkɔ́ngkɔ́ng & ‘seek permission to enter’\is{word classes}\\
\end{tabularx}
}
Ideophones that involve reduplication feature a suprasegmental structure of the type that we find with bisyllabic iterative reduplications like \textit{hala-hála} ‘\textsc{red.cpd-}shout’ = ‘repeated shouting’ in \REF{ex:key:61} above: Two phonetic L tones over the first two syllables of the reduplicant are followed by a succession of two phonetic H tones over the base (with the last H tone resulting from raising of the lexical L to H, which is phonologised and does not vary). One such ideophone is the property item \textit{katakátá} ‘be (hyper-)active; hectic’, which appears in the prenominal modifier position in the following sentence. Another ideophone belonging to this group is the noun \textit{pɔtɔpɔ́tɔ́} ‘mud’, also below:


\ea%1616
    \label{ex:key:1616}
    \gll Na  wán    \textstylePichiexamplebold{katakátá}  mán.\\
\textsc{foc}  one    hectic    man\\

\glt ‘He’s a hectic man.’ [tr07fn 229]
\z


\ea%1617
    \label{ex:key:1617}
    \gll Dán  sáy  gɛ́t  bɔkú  pɔtɔpɔ́tɔ́.\\
that  side  get  much  mud  \\

\glt ‘[Mind you] that place is very muddy.’ [ne07fn 230]
\z

The most commonly used ideophonic verb (and a generally quite frequent verb) is \textit{wɔwɔ́} ‘be ugly; messy; in disorder’. This verb also belongs to the group of ideophones with a tone configuration that suggests the operation of reduplication rather than repetition.\is{reduplication} 

A second group of ideophones involves repetition. Some words of this group may be encountered as simplex forms, (i.e. \textit{kutuku} ‘sound of the heart beating’, \textit{kakara} ‘be restless’) and may optionally be repeated in order to express meanings associated with repetition, such as emphasis\is{emphasis} or duration. Iterations of such ideophones therefore do not involve lexicalisation proper, even if there is a strong tendency for them to be repeated in discourse. 


Hence, the ideophonic verb \textit{kakara} ‘be restless’ is employed as a dynamic verb in \REF{ex:key:1624}, preceded by the imperfective marker \textit{de} ‘\textsc{ipfv}’ and repeated for emphasis. The comma after the first \textit{kakara} signals the presence of a short pause, which indicates that this ideophone can also stand alone as simplex form: 



\ea%1618
    \label{ex:key:1618}
    \gll \'{I}n    \textstylePichiexamplebold{de}  \textstylePichiexamplebold{kakara,}    \textstylePichiexamplebold{kakara}  \textstylePichiexamplebold{kakara}.\\
\textsc{3sg.indp}  \textsc{ipfv}  be.restless  \textsc{rep}    \textsc{rep}\\

\glt ‘He [\textsc{emp}] was all restless.’ [ab03ab 047]
\z

The ideophone \textit{kutuku} ‘sound of the heart beating’ may also optionally be repeated for emphasis, as in the following sentence: 


\ea%1619
    \label{ex:key:1619}
    \gll Na  só    in    hát    \textstylePichiexamplebold{mék}    \textstylePichiexamplebold{kutuku}  \textstylePichiexamplebold{kutuku}  \textstylePichiexamplebold{kutuku}.\\
\textsc{foc}  like.that  \textsc{3sg.poss}  heart  make  \textsc{ideo}    \textsc{ideo}    \textsc{ideo}\\

\glt ‘That’s how his heart was going “kutuku kutuku kutuku”.’ [ab03ab 070]
\z

Other ideophones that formally involve repetition are not usually encountered as simplex forms. Therefore, the ideophone \textit{gbogbogbo} which expresses haste or precipitous hurry has no attested simplex form *\textit{gbò}. The ideophone only occurs as a triplicated iteration, as in this example: 


\ea%1620
    \label{ex:key:1620}
    \gll Tokobé  dɔ́n  wɛ́r    klós    gbogbogbo.\\
\textsc{name}  \textsc{prf}  wear  clothing  \textsc{ideo}\\

\glt ‘Tokobé had put on (her) clothes in haste.’ [ab03ab 111]
\z

Likewise, the ideophone \textit{fwífwífwí} ‘sound of the wind blowing’ is only used as a triplicated lexicalised repetition. In the example below, this ideophone modifies the preceding clause headed by the Spanish-derived verb \textit{sopla} ‘(to) fan; (to) blow’:


\ea%1621
    \label{ex:key:1621}
    \gll Na  só    a    de  wáyp=an,  a    de  sopla  ín    \textstylePichiexamplebold{fwífwífwí.}\\
\textsc{foc}  like.that  \textsc{1sg.sbj}  \textsc{ipfv}  wipe=\textsc{3sg.obj}  \textsc{1sg.sbj}  \textsc{ipfv}  blow  \textsc{3sg.indp}  \textsc{ideo}\\

\glt ‘I was wiping him like that, I was fanning him.’ [ab03ab 068]\is{repetition}
\z

Both groups of ideophones, i.e. those involving lexicalised reduplication and those involving repetition that is lexicalised in varying degrees, can be contrasted with ideophones like \textit{gbin} ‘sound of a hard; sudden blow’ in \REF{ex:key:1622}. This ideophone is not encountered with any form of iteration in the corpus:

\ea%1622
    \label{ex:key:1622}
    \gll E    \textstylePichiexamplebold{gí}  mí    \textstylePichiexamplebold{gbin}.\\
\textsc{3sg.sbj}  give  \textsc{1sg.indp}  \textsc{ideo}\\

\glt ‘He hit me hard and suddenly.’ [ne07fn 008]
\z

Some other combinations of verbs and ideophonic manner adverbs that are not encountered with iteration in the data are: \textit{nák kìp} ‘hit=\textsc{3sg.obj} \textsc{ideo’} = ‘hit and produce a dull thud’, \textit{mék nɔ́ys} \textit{tík} ‘make noise \textsc{ideo}’ = ‘make a cracking noise’.


A look back at the examples in this section show that iteration (whether it involves reduplication or repetition) with most ideophones also evokes the same type of “disaggregation” of the relevant situation that we find with iterated non-ideophones. This may explain why ideophones like \textit{gbin}, \textit{kip}, and \textit{tík} are not iterated. These ideophones denote sudden and inherently terminative situations, which are not normally associated with the typically cyclic, repetitive, disaggregated events depicted by iterated ideophones. 



The following sentence is particularly illustrative of the notion of a series of often quick motion events\index{} that is attached to iterated ideophones. The ideophone \textit{kamúkamú} depicts the countermovement of a pair of buttocks as their owner strides along: 



\ea%1623
    \label{ex:key:1623}
    \gll Yu  sí  lɛk  háw    in    bata    dɛn  de  sék    \textstylePichiexamplebold{kamúkamú}?\\
\textsc{2sg}  see  like  how    \textsc{3sg.poss}  buttocks  \textsc{pl}  \textsc{ipfv}  shake  \textsc{ideo}\\

\glt ‘Do you see her buttocks moving to-and-fro (as she walks along)?’ [ye07fn 231]\is{ideophones}
\z

Ideophones are not very prominent in the corpus and tend to be employed more by older, Group 2 (cf. \sectref{sec:1.3}) speakers. All ideophones encountered in the data are listed in \tabref{tab:key:12.1}. Many of the ideophones listed below and in particular those listed under “manner adverb” and “verb” in particular appear to be multicategorial. It is highly likely that they may be used in the syntactic positions of other word classes as well. On the other hand, an ideophonic noun like \textit{pɔtɔpɔ́tɔ́} ‘mud’ and the verb/adjective \textit{wɔwɔ́} seem to be firmly entrenched as members of their wordclasse(s). The list also features an ideophonic interjection. 


Many of the ideophonic manner adverbs given in the table only occur once in the corpus. It is therefore difficult to ascertain how widespread the use of these ideophones is, and whether some of them are sound symbolic ad hoc creations, whether they are carried over from other languages used by the speaker, or whether they form part of the lexicon of Pichi (e.g. \textit{bwa, fwífwíf\'{w}i} and \textit{wówó/wéwé}).


%%please move \begin{table} just above \begin{tabular
\begin{table}
\caption{Ideophones}
\label{tab:key:12.1}

\begin{tabularx}{\textwidth}{llQ}
\lsptoprule

Word class attested & Example & Translation\\
\midrule 
Verb & \itshape kakara & ‘be restless’\\
& \itshape katakátá & ‘be active; hectic’\\
& \itshape ményéményé & ‘whine; nag in a childlike fashion’\\

\tablevspace
Verb \& adjective & \itshape \textstyleTablePichiZchn{wɔwɔ́} & ‘be ugly; in disorder’\\

\tablevspace
Noun & \itshape pɔtɔpɔ́tɔ́ & ‘mushy substance; mud’\\
& \itshape wuruwúrú & ‘deceit’\\

\tablevspace
Manner adverb & \itshape bwa & ‘sound of water gushing’\\
& \itshape bya & ‘sound of coughing’\\
& \itshape fwífwífwí & ‘sound of wind blowing’\\
& \itshape gbin & ‘sound of a hard, sudden blow’\\
& \itshape gbogbogbo & ‘in haste’\\
& \itshape kamúkamú & ‘sight of buttocks moving’\\
& \itshape kip & ‘sound of a dull thud’\\
& \itshape kutuku & ‘sound of heart beating’\\
& \itshape kwáráng & ‘sound of round and hard object(s) falling into a receptacle’\\
& \itshape kpu & ‘sound of impact on a soft matter’\\
& \itshape príng & ‘sound of ringing’\\
& \itshape súkútúpampa & ‘in a cheap and mean fashion’\\
& \itshape tík & ‘cracking sound’\\
& \itshape wéwé; wówó & ‘sound of crying and wailing’\\

\tablevspace
Interjection & \itshape kɔ́ngkɔ́ngkɔ́ng & ‘seek permission to enter’\\
\lspbottomrule
\end{tabularx}
\end{table}
Ideophones differ from other word classes in three respects: most of the ideophones listed above belong to minor tone classes; about half of the ideophones listed above represent cases of lexicalised full or partial duplication and triplication (cf. also \sectref{sec:4.5.3}); three ideophones feature the phonemes /gb/ and /kp/, which are only attested with this word class (i.e. \textit{gbin}, \textit{gbogbogbo}, and \textit{kpu}), while others exhibit “unusual” phoneme combinations. For example, the word-initial cluster /fw/ is not attested in any other word than the ideophone \textit{fwífwífwí}. Equally, many of the ideophones listed feature otherwise rare CV syllable structures (e.g. \textit{súkútúpampa}, \textit{kutuku}, \textit{wéwé}). Further, at least one ideophone, namely \textit{bwa}, may be pronounced with a breathy voice. 


Ideophonic verbs are found in the syntactic positions available to any other property item of the language. Hence, the ideophone \textit{kakarakakara} ‘be restless’ is employed as a dynamic verb in \REF{ex:key:1624}, and preceded by the imperfective marker \textit{de} ‘\textsc{ipfv}’. Note the repetition of the ideophone for emphasis\is{emphasis}: 



\ea%1624
    \label{ex:key:1624}
    \gll \'{I}n    \textstylePichiexamplebold{de}  \textstylePichiexamplebold{kakara},    kakara  kakara.\\
\textsc{3sg.indp}  \textsc{ipfv}  be.restless  \textsc{rep}    \textsc{rep}\\

\glt ‘He [\textsc{emp}] was all restless.’ [ab03ab 047]
\z

Like other property items, ideophonic verbs also appear in the prenominal modifier position. Compare \textit{katakátá} ‘be (hyper-)active; hectic’ in the following sentence: 


\ea%1625
    \label{ex:key:1625}
    \gll Na  wán    \textstylePichiexamplebold{katakátá}  mán.\\
\textsc{foc}  one    hectic    man\\

\glt ‘He’s a hectic man.’ [tr07fn 229]
\z

The most commonly used ideophonic (and generally quite frequent) verb is \textit{wɔwɔ́} ‘be ugly; messy; in disorder’. This verb, too, is attested as a stative verb \REF{ex:key:1626}, and in a prenominal position as an attributive modifier \REF{ex:key:1627}. Some speakers also employ \textit{wɔwɔ́} as an adjective, i.e. a complement to the locative-existential copula \textit{dé} \textsc{‘be.loc’} \REF{ex:key:1628}. Another indication of the ideophonic nature of \textit{wɔwɔ́} besides its segmental structure is that it is often pronounced with reduced speed and overarticulation, and is accompanied by a facial expression suggestive of disapproval:


\ea%1626
    \label{ex:key:1626}
    \gll Di  sáy  wɔwɔ́  ɛ́n.\\
\textsc{def}  side  be.ugly  \textsc{sp}\\

\glt ‘The place is messy, you know.’ [ma03ni 014]
\z


\ea%1627
    \label{ex:key:1627}
    \gll Na  Afrika  e    gɛ́t  wɔwɔ́  wɔwɔ́  tín    dɛn  (…).\\
\textsc{loc}  \textsc{place}  \textsc{3sg.sbj}  get  ugly    \textsc{rep}    thing  \textsc{pl}  \\

\glt ‘In Africa there are really messy things [happening], (…).’ [ed03sb 187]
\z


\ea%1628
    \label{ex:key:1628}
    \gll Dís  chɔ́p  \textstylePichiexamplebold{dé}    \textstylePichiexamplebold{wɔwɔ́}.\\
this  food    \textsc{be.loc}  ugly\\

\glt ‘This food is a mess.’ [dj05ae 181]
\z

Ideophonic nouns appear in the same syntactic position as other nouns. In the following sentence, \textit{pɔtɔpɔ́tɔ́} ‘mud’ is the head of an object NP featuring the quantifier \textit{bɔkú} ‘much’:


\ea%1629
    \label{ex:key:1629}
    \gll Dán  sáy  gɛ́t  bɔkú  pɔtɔpɔ́tɔ́    ó.\\
that  side  get  much  mud      \textsc{sp}\\

\glt ‘[Mind you] that place is very muddy.’ [ne07fn 230]
\z

Ideophonic adverbs usually modify verbs in the clause-final position. Sentence \REF{ex:key:1630} illustrates the depictive power of an ideophone like \textit{kwáráng} when used to express the sensory experience connected to playing the African board game Oware. Example \REF{ex:key:1631} presents the ideophone \textit{fwí} ‘sound of the wind blowing’, which modifies the preceding Spanish-derived verb \textit{sopla} ‘(to) fan; (to) blow’: 


\ea%1630
    \label{ex:key:1630}
    \gll Dɛn  de  plé=an    \textstylePichiexamplebold{kwáráng}.\\
\textsc{3pl}  \textsc{ipfv}  play=\textsc{3sg.obj}  \textsc{ideo}\\

\glt ‘It is played with this hollow sound (of the seeds 


\glt falling into the pits of the wooden Oware board).’ [ro07fn 519]
\z


\ea%1631
    \label{ex:key:1631}
    \gll Na  só    a    de  wáyp=an,  a    de  sopla  ín    \textstylePichiexamplebold{fwífwífwí}\\
\textsc{foc}  like.that  \textsc{1sg.sbj}  \textsc{ipfv}  wipe=\textsc{3sg.obj}  \textsc{1sg.sbj}  \textsc{ipfv}  blow  \textsc{3sg.indp}  \textsc{ideo}\\

\glt ‘I was wiping him, I was fanning him just like that.’ [ab03ab 068]
\z

In the following sentence, speaker (ro) uses the ideophone \textit{súkútúpampa} in order to depict the supposedly cheap and mean manner in which sex workers in Malabo offer themselves for sale: 


\ea%1632
    \label{ex:key:1632}
    \gll Dɛn  de  sɛ́l  dɛn  skín    súkútúpampa\\
\textsc{3pl}  \textsc{ipfv}  sell  \textsc{3pl}  body  \textsc{ideo}\\

\glt ‘They barter their bodies away.’ [ro05fn 240]
\z

Ideophonic manner adverbs sometimes occur in what appears to be a nominal postion as in the following two sentences. Actually, the ideophones do not enter syntactic constructions in these examples either. Instead, the preceding generic verb \textit{mék} ‘make’ and \textit{gí} ‘give’ may be said to function as a kind of quotative index followed by a syntactically independent utterance consisting of the ideophonic adverb: 


\ea%1633
    \label{ex:key:1633}
    \gll Na  só    in    hát    \textstylePichiexamplebold{mék}    \textstylePichiexamplebold{kutuku}  \textstylePichiexamplebold{kutuku}  \textstylePichiexamplebold{kutuku}.\\
\textsc{foc}  like.that  \textsc{3sg.poss}  heart  make  \textsc{ideo}    \textsc{rep}    \textsc{rep}\\

\glt ‘That’s how his heart was going “kutuku kutuku kutuku”.’ [ab03ab 070]
\z


\ea%1634
    \label{ex:key:1634}
    \gll E    \textstylePichiexamplebold{gí}  mí    \textstylePichiexamplebold{gbin}.\\
\textsc{3sg.sbj}  give  \textsc{1sg.indp}  \textsc{ideo}\\

\glt ‘He gave (it) to me “gbin”.’ [ne07fn 008]\is{ideophones}
\z

Some other combinations of verbs and ideophonic manner adverbs encountered in the data are: \textit{nák kìp} ‘hit=\textsc{3sg.obj} \textsc{ideo’} = ‘hit and produce a dull thud’, \textit{mék nɔ́ys} \textit{tík} ‘make noise \textsc{ideo}’ = ‘make a cracking noise’, \textit{kráy wówó wówó} ‘cry \textsc{ideo} \textsc{rep’} = ‘cry bitterly’.

\section{Interjections}\label{sec:12.2}

In the following, I employ the term “interjection” liberally as a cover term for individual words, phrases, and clauses that index physical and discursive entities \citep{Kockelman2003}, cognitive and emotional states \citep{Ameka1992}, and social relations. Interjections are pragmatically oriented elements that appear at the beginning or end of an utterance or constitute utterances onto themselves.


In \REF{ex:key:1635}, the initial interjection ɛ ‘intj’ (cf. \tabref{tab:key:12.4}) functions as an attention-getter and by doing so, indexes the following utterance. The sentence-final element ɛ́n functions as a channel checker, i.e. ‘have you heard what I’ve just said?’ and thereby solicits a preferably affirmative response. The example also shows that interjections are set off from the rest of the utterance by a prosodic break (indicated by the comma). This indicates that they function as co-text rather than forming an integral part of the clause:



\ea%1635
    \label{ex:key:1635}
    \gll ɛ́,  dí  mán    gɛ́t  líba,    ɛ́n.\\
\textsc{intj}  this  man    get  liver  \textsc{intj}\\

\glt ‘Hey, this man has guts, you know.’ [dj05ce 290]
\z

Following Ameka (1992a; 1992b), I classify Pichi interjections along three functions: Expressive, conative, and phatic. Many interjections are “primary” \citep{Ameka1992a} and constitute a micro word class\is{word classes} of mostly monosyllabic “small words” which do not occur in contexts other than those described here. Some primary interjections are also phonologically deviant. For example, interjections constitute the only word class in which vowel length may be distinctive (i.e. \textit{a} ‘\textsc{1sg.sbj}’ vs. \textit{aa} ‘expression of insight’). Other interjections are “secondary” and also employed as members of other word classes, and they may enter into grammatical constructions with other constituents. 


In the following, I cover the most commonly used interjections. Some interjections are cross-classified and may therefore be members of more than one of the three functional types (e.g. \textit{mamá} ‘mother’ which is employed as an expressive and a phatic interjection). 


\subsection{Expressive}

Expressive interjections reflect the emotional and cognitive state of the speaker, but they also serve a communicative purpose by drawing the attention of potential listeners to the mental state of the utterer. Consider the following expressive interjections: 

%%please move \begin{table} just above \begin{tabular
\begin{table}
\caption{Expressive interjections}
\label{tab:key:12.2}

\begin{tabularx}{\textwidth}{Xlll}
\lsptoprule
 & Interjection & Gloss & Function\\
\midrule
 Primary & \itshape cháy/chɛ́ & ‘\textsc{intj}’ & Exasperation\\
& \itshape áy & ‘\textsc{intj}’ & Extreme sensation\\
& \itshape ékié & ‘\textsc{intj}’ & Counterexpectation\\
& \itshape ‘chíp’ & ‘suck teeth’ & Irritation, fatigue\\

\tablevspace
Secondary & \itshape papá gɔ́d & ‘father God’ & Exasperation, self-pity\\
& \itshape nawá (ó) & ‘oh my’ & Exasperation, (self) pity \\
& \itshape mamá & ‘mother’ & Surprise, shock\\
& \itshape chico & ‘boy’ & Surprise, admiration\\
& \itshape dios (mío) & ‘my God’ & Surprise, irritation\\
& \itshape señor(mío) & ‘my Lord’ & Surprise, irritation\\
& \itshape bió bió & ‘behold’ & Pleasant surprise\\
& \itshape mierda & ‘shit’ & Annoyance, anger\\
\lspbottomrule
\end{tabularx}
\end{table}
An exemplary primary interjection with an expressive meaning is chɛ́ or cháy, which conveys the feeling of exasperation in the face of a difficult task. In \REF{ex:key:1636}, chɛ́ is the reaction of (dj) to a particularly ungrammatical sentence that I (ko) submit to him for a grammaticality judgement:


\ea%1636
    \label{ex:key:1636}
\ea{
    \gll Na  di  púl    di  motó  fɔ  di  mecánico/\\
  \textsc{foc}  \textsc{def}  remove  \textsc{def}  car    \textsc{prep}  \textsc{def}  mechanic\\

\glt   ‘[Can you say:] “It’s the removal of the car from the mechanic”/ [ko0502e2 045]
}\ex{\gll
Chɛ́! \\
  \textsc{intj}\\

\glt   ‘Phew [now this is too much]!’ [dj05be 045]
}\z\z

An extreme physical sensation is expressed by the primary interjection \textit{áy}. As indicated by the two following examples, the sensation may be pain or pleasure; in particular the pleasure of good food or sexual delight. There are therefore overlaps in meaning with the (near-)identical interjection \textit{ay} in Spanish:


\ea%1637
    \label{ex:key:1637}
    \gll Dɛn  fít  nák  yú    yu  fít  tɔ́k  sé
‘\textstylePichiexamplebold{áy}  a    fíl  hát  ó!’\\
\textsc{3pl}  can  hit  \textsc{2sg.indp}  \textsc{2sg}  can  talk  \textsc{quot}  
\textsc{intj}  \textsc{1sg.sbj}  feel  hurt  \textsc{sp}\\

\glt ‘You could be hit (and) you would say “ouch I feel pain.”’ [dj05ae 083]
\z


\ea%1638
    \label{ex:key:1638}
    \gll \textstylePichiexamplebold{\'{A}y},  di  tín    swít    ó.\\
\textsc{intj}  \textsc{def}  thing  be.tasty  \textsc{sp}\\

\glt ‘Wow, this tastes/feels good.’
\z

The interjection \textit{ékié} is an established loan\is{loan words} from Fang. It expresses counterexpectation, amazement, and (often playful) indignation. In the corpus, \textit{ékié} is mainly used by female speakers. Sentence \REF{ex:key:1639} is a humorous comment by speaker (ge) addressed to her female friend. The latter has just said that she finds a white European acquaintance of hers attractive. Speaker (ge) teases her friend by pretending to be outraged and calls her \textit{busca-blanco} ‘look.for-white.male’ = female sex worker specialised on white men’: 

\ea%1639
    \label{ex:key:1639}
    \gll \textstylePichiexamplebold{\'{E}kié},  busca-blanco.\\
\textsc{intj}    look.for.\textsc{cpd}{}-white.male\\

\glt ‘Good gracious, (you’re a) prostitute.’ [ge07fn 077]
\z

Pichi also features expressive “secondary interjections” (\citealt{Bloomfield1935}:176; \citealt{Ameka1992a}) which function as members of other word class\is{word classes}es besides their use as deictic-pragmatic elements. One group of secondary interjections stems from religious terminology. The lexicalised collocation \textit{papá gɔ́d} is a Pichi term for ‘God’ \REF{ex:key:1640}. As an interjection,\textit{ papá gɔ́d} is used to implore the help of God during prayer and inner speech \REF{ex:key:1641}, or to express self-pity and exasperation \REF{ex:key:1642}. Note that \textit{papá gɔ́d} in \REF{ex:key:1641} is preceded by the conative interjection \textit{oó}, which introduces an emphatic vocative\is{vocatives} (cf. \tabref{tab:key:12.4}):


\ea%1640
    \label{ex:key:1640}
    \gll \textstylePichiexamplebold{Papa}  \textstylePichiexamplebold{gɔ́d}  go  mék    mék  e    chénch,  mék  e    chénch  fásin.\\
father  God  \textsc{pot}  make  \textsc{sbjv}  \textsc{3sg.sbj}  change  \textsc{sbjv}  \textsc{3sg.sbj}  change  manner\\
\glt ‘God will make him change, change (his) habits.’ [dj07ae 160]
\z


\ea%1641
    \label{ex:key:1641}
    \gll Di  tín    de  gó  mí    bád,    \textstylePichiexamplebold{ó  papá  gɔ́d},  
mék    mí    so.\\
\textsc{def}  thing  \textsc{ipfv}  go  \textsc{1sg.indp}  bad    \textsc{intj}  father  God  
make  \textsc{1sg.indp}  like.that\\
\glt ‘The matter is going bad for me, oh God do this for me.’ [dj07ae 161]
\z

\ea%1642
    \label{ex:key:1642}
    \gll Sé    \textstylePichiexamplebold{papá  gɔ́d}  ús=káyn  trɔ́bul  dís?\\
\textsc{quot}    father  God  \textsc{q}=kind  trouble  this\\

\glt ‘(I said) God, what (kind of) trouble (is) this?’ [ab03ab 082]
\z

A number of expressive secondary interjections in the corpus are Spanish-derived and used in similar ways in peninsular Spanish. The interjections \textit{Señor mío} ‘good Lord’ and \textit{Dios mío} ‘my God’ express sentiments like surprise, irritation, and frustration \REF{ex:key:1643}:


\ea%1643
    \label{ex:key:1643}
    \gll Señor  mío,  tɛ́l  mí,    mi    mán
e    de  kɔmɔ́t  wet    yú?\\
Lord  my    tell  \textsc{1sg.indp}  \textsc{1sg.poss}  man
\textsc{3sg.sbj}  \textsc{ipfv}  go.out  with    \textsc{2sg.indp}\\

\glt ‘Good Lord tell me, is my husband going out with you?’ [ro05rt 009]
\z

A second group of expressive secondary interjections includes kinship\is{kinship terminology} terms and other human-denoting nouns. These nouns are intermediary in their function. On the one hand, these nouns resonate with a strong emotive component when used as interjections. However, by the very nature of their meaning as kinship\is{kinship terminology} terms and terms of address, they also index the social relation which they stand for and thereby convey a phatic message to interlocutors. 


The Spanish noun \textit{chico} ‘boy’ is one of the most frequently employed secondary interjections and covers a large range of expressive meanings. It conveys real, playful, or mock surprise \REF{ex:key:1644}(a)–( (b), shock and amazement (c), awe and admiration (d):



\ea%1644
    \label{ex:key:1644}
\ea{
    \gll \textstylePichiexamplebold{Chico},  yu  nó  bríng  mí    glás?\\
  boy    \textsc{2sg}  \textsc{neg}  bring  \textsc{1sg.indp}  glass\\

\glt   ‘What, you haven’t brought (along) a glass for me?’ [fr03cd 079]
}\ex{\gll
Chico,  dí  mán    e    tú  ópin    in    sɛ́f.\\
  boy    this  man    \textsc{3sg.sbj}  too  be.open  \textsc{3sg.poss}  self\\

\glt   ‘Oh boy, this man boasts too much.’ [ye07je 131]
}\ex{\gll
Chico,  yu  de  mít    ɛ́ni    káyn  colór  dé.\\
  boy    \textsc{2sg}  \textsc{ipfv}  meet  every  kind    colour  there\\

\glt   ‘Man, you find any kind of (skin) colour there [in Cuba].’ [ed03sp 030]
}\ex{\gll
\textstylePichiexamplebold{Chico},  Jibril  trɔ́n      ó!\\
  boy    \textsc{name}  be.strong  \textsc{sp}\\

\glt   ‘Wow, Jibril is really great.’ [ye05ce 023]
}\z\z

The following excerpt renders reported discourse of a conversation, in which speaker (ro) is taking her husband to task for cheating on her. The husband tells (ro) that he and his lover would meet up in a car. An incredulous (ro) repeats what her husand has just told her in \REF{ex:key:1645}(a), and then cries out \textit{mamá} ‘mother’ in shock (b). Her mental state at that moment is reflected by (c). The kinship term \textit{papá} ‘father’ is employed as an expressive interjection in a similar way to \textit{mamá} (cf. \REF{ex:key:1649} below):


\ea%1645
    \label{ex:key:1645}
\ea{
    \gll \'{I}nsay  di  motó,  na  dé    unu  de  slíp    unu  sɛ́f?\\
  inside  \textsc{def}  car    \textsc{foc}  there  \textsc{2pl}  \textsc{ipfv}  sleep  \textsc{2pl}  self\\
\glt   \textit{‘}In the car, that’s where you sleep with each other?’ [ro05rt 020]
}\ex{\gll
\textstylePichiexamplebold{Mamá}.\\
  mother\\
\glt   ‘Good gracious.’ [ro05rt 021]
}\ex{\gll
\MakeUppercase{A}   krés.\\
  \textsc{1sg.sbj}  be.crazy\\
\glt   ‘I went mad.’ [ro05rt 022]
}\z\z

The interjection \textit{mamá} therefore expresses the emotional stress that speaker (ro) was experiencing at that moment. But beyond that, \textit{mamá} is also a meta-comment on the amorality of the husband’s act, a performative element embedded in reported discourse, directed at us, the listeners of the narrative. This type of “rhetorical underlining” \citep[39]{Longacre1996}, in which the narrator steps out of the narrative and addresses her audience is a significant element of Pichi narrative technique. The use of \textit{mamá} in this way sheds light on the communicative dimension of expressive interjections in Pichi.


\is{kinship terminology}The interjection \textit{bió bió} ‘(lo and) behold’ expresses surprise. By doing so, this interjection also has a strong phatic component to its meaning: 



\ea%1646
    \label{ex:key:1646}
    \gll \textstylePichiexamplebold{Bió}    \textstylePichiexamplebold{bió},  dɛn  dɔ́n  kán.\\
behold  \textsc{rep}  \textsc{3pl}  \textsc{prf}  come\\

\glt ‘Lo and behold, they’ve (finally) come.’ [pa05fn 456]
\z

The Spanish noun \textit{mierda} ‘shit’ is used as an expressive interjection for anger and annoyance. The Pichi equivalent \textit{kaká} ‘faeces’ is not used in this way. However, the Pichi compound \textit{kaka-rás} \{faeces.\textsc{cpd}{}-arse\} ‘shitty arse’ is used as an insult: 


\ea%1647
    \label{ex:key:1647}
    \gll \textstylePichiexamplebold{Mierda}  \textstylePichiexamplebold{mierda},  ús=sáy  e    pás?\\
shit    \textsc{rep}    \textsc{q}=side  \textsc{3sg.sbj}  pass\\

\glt ‘Shit, shit, which way did she go?’ [ro05rt 002]
\z

\subsection{Phatic}\label{sec:12.2.2}

Phatic interjections and phrases are embedded in the verbal interaction between interlocutors. These elements are interactional and are aimed at constructively maintaining the communicative situation. \tabref{tab:key:12.3} lists the phatic interjections encountered in the corpus. The functions of the phatic elements and agreement markers yɛ́(s) ‘yes’ and nó/nɔ́ ‘no’ are covered in detail in \sectref{sec:7.3.3}:

%%please move \begin{table} just above \begin{tabular
\begin{table}
\caption{Phatic interjections}
\label{tab:key:12.3}

\begin{tabularx}{\textwidth}{Xlll}
\lsptoprule
 & Interjection & Gloss & Function\\
\midrule
 Primary & \itshape aa & ‘\textsc{intj}’ & Insight\\
& \itshape o.k & ‘okay’ & Insight\\
& \itshape é & ‘\textsc{intj}’ & Dismay, empathy\\
& \itshape \textup{‘chip’} & ‘suck teeth’ & (Solicit) empathy\\

\tablevspace
Secondary & \itshape papá (gɔ́d) & ‘father (God)’ & Express/solicit empathy\\
& \itshape mamá & ‘mother’ & Express/solicit empathy\\
& \itshape dúya & ‘please’ & Solicit favour\\
& \itshape plís & ‘please’ & Solicit favour\\
& \itshape ɛskyús & ‘excuse (me)’ & Present excuses\\
& \itshape kúsɛ́ (ó) & ‘good job’ & Encouragement for work\\
& \itshape yɛ́(s) & ‘yes’ & Agreement, appreciation\\
& \itshape nɔ́/nó & ‘no’ & Disagreement, doubt\\
\lspbottomrule
\end{tabularx}
\end{table}
The phatic interjection \textit{aa} expresses sudden insight into a proposition or real-world fact. In this, its meaning is similar to \textit{o.k.} \REF{ex:key:1648}(b):


\ea%1648
    \label{ex:key:1648}
\ea{
    \gll A    dé    wet    Paquita.\\
  \textsc{1sg.sbj}  \textsc{be.loc}  with    \textsc{name}\\

\glt   ‘I’m with Paquita.’ [ko03cb 075]
}\ex{\gll
\textstylePichiexamplebold{Aa}  \textstylePichiexamplebold{o.k}.\\
  \textsc{intj}  \textsc{intj}\\
\glt   ‘Alright.’ [hi03cb 076]
}\z\z

The interjection \textit{e} is usually uttered with a compressed voice and an extra-high tone. It is also usually lengthened to up to three beats. It is best translated as ‘good gracious’ and expresses dismay and empathy with a deplorable event or fact. In \REF{ex:key:1649}, the expressive meaning of \textit{e} is underlined by the presence of the interjection \textit{papá} ‘father’:


\ea%1649
    \label{ex:key:1649}
\ea{
    \gll  E    gɛ́t  bɛlɛ́,    wé  e    wɔ́nt  púl    di  bɛlɛ́.\\
  \textsc{3sg.sbj}  get  belly  \textsc{sub}  \textsc{3sg.sbj}  want  remove  \textsc{def}  belly\\

\glt   ‘She was pregnant and wanted to abort the pregnancy.’ [ko03cb 099]
}\ex{\gll
\textstylePichiexamplebold{\'{E}  papá}!\\
  \textsc{intj}  father\\

\glt   ‘Good gracious!’ [bo03cb 100]
}\z\z

The kinship\is{kinship terminology} terms \textit{papá} ‘father’ and \textit{mamá} ‘mother’ are also employed as phatic interjections in appealing for consideration, empathy, and compassion by evoking the nature of the kinship relation that holds between a parent and a child, a provider and a dependent. Consider \REF{ex:key:1650}, where (ye) relates how Rubi appeals to the person represented by =\textit{a}\textit{n} ‘\textsc{3sg.obj}’ to leave him his fair share of the remaining food:


\ea%1650
    \label{ex:key:1650}
\ea{
    \gll
E    de  fɔgɛ́t  sé    Rubi    nɔ́ba  chɔ́p.\\
  \textsc{3sg.sbj}  \textsc{ipfv}  forget  \textsc{quot}    \textsc{name}  \textsc{neg}.\textsc{prf}  eat\\
\glt   ‘He forgot that Rubi hadn’t yet eaten.’ [dj03cd 148]
}\ex{\gll
E    tɛ́l=an    sé    ‘papá  mí    nɛ́va    
  chɔ́p  mí    sénwe.’\\
  \textsc{3sg.sbj}  tell=\textsc{3sg.obj}  \textsc{quot}    father  \textsc{1sg.indp}  \textsc{neg}.\textsc{prf}  
  chop  \textsc{1sg.indp}  \textsc{emp}\\

\glt   ‘(So) he [Rubi] told him ‘please, I also haven’t eaten myself.’ [ye03cd 149]
}\z\z

The interjection \textit{dúya} ‘please’ and the less frequent \textit{plís} ‘please’ play an important role as politeness markers. Both interjections are used in polite imperatives\is{imperatives} like the following one: 


\ea%1651
    \label{ex:key:1651}
    \gll Pút=an    mɔ́    \textstylePichiexamplebold{dúya}!\\
put=3\textsc{sg.obj}  more  please\\

\glt ‘Put [play] it again, please!’ [au07se 095]
\z

\subsection{Conative}\label{sec:12.2.3}

Conative interjections solicit a verbal or kinetic response from listeners. By their imperative nature, they are used in calling and responding, seeking approval and confirmation, constraining and restraining the interlocutor. \tabref{tab:key:12.4} lists common conative interjections: 

%%please move \begin{table} just above \begin{tabular
\begin{table}
\caption{Conative interjections}
\label{tab:key:12.4}

\begin{tabularx}{\textwidth}{Xlll}
\lsptoprule
 & Interjection & Gloss & Function\\
 \midrule 
Primary & \itshape ó & ‘\textsc{sp}’ & Vocative, warning\\
& \itshape yée & ‘\textsc{intj}’ & Response to call\\
& \itshape ónli & ‘\textsc{intj}’ & Response to call\\
& \itshape yɛ́s & ‘yes’ & Response to call\\
& \itshape oó & ‘\textsc{intj}’ & Emphatic vocative\\
& \itshape ɛ́ & ‘\textsc{intj}’ & Attention getter\\
& \itshape ɛ́n? & ‘\textsc{intj}’ & Channel check\\
& \itshape hɛ́ & ‘\textsc{intj}’ & Rebuke\\
& \itshape aa & ‘\textsc{intj}’ & Impatience, repudiation\\
& \itshape ‘chip’ & ‘suck teeth’ & Remonstrative\\

\tablevspace
Secondary & \itshape nɔ́? & ‘\textsc{intj}’ & Channel check\\
& \itshape (yu de) hía? & ‘(do you) hear?’ & Channel check\\
& \itshape nó tɔ́k (ɛ́n)! & ‘don’t talk!’ & Solicit approbation\\
& \itshape nó láf (ɛ́n)! & ‘don’t laugh’ & Solicit approbation\\
& \itshape a tɛ́l yú & ‘I tell you’ & Emphasise veracity\\
& \itshape kɔ́ngkɔ́ngkɔ́ng & ‘\textsc{ideo}’ & Seek permission to enter\\
& \itshape dí bɔ́y/ dí gɛ́l & ‘hey you (\textsc{m/f)}’ & Vocative (for \textsc{m}/\textsc{f})\\
\lspbottomrule
\end{tabularx}
\end{table}
One of the numerous functions of the sentence-final particle \textit{ó} is its use as a vocative marker in combination with a personal name (cf. \sectref{sec:12.2.3} for more). An emphatic, imploring vocative is formed by preposing the interjection \textit{oó} to the name or term of address of the person called (cf. \REF{ex:key:1641} above). 


The appropriate way of responding to the call of a social superior is by calling out the term of address of the caller \REF{ex:key:1652}(b). If the caller is a peer, the person called may also simply respond with yɛ́s ‘yes’



\ea%1652
    \label{ex:key:1652}
\ea{
    \gll Pancho!\\
  \textsc{name}\\
}\ex{\gll
\textstylePichiexamplebold{Mamá}!\\
  mother\\
\glt   ‘(Yes) mum.’
}\z\z

Alternatively, a person can respond with a response call involving the vowels /e/ and /o/ with different degrees of lengthening and in slightly varying pitch configurations over the lengthened vowel. A response call may simply follow the call or additionally feature the caller’s name \REF{ex:key:1653}(b):


\ea%1653
    \label{ex:key:1653}
\ea{
    \gll Pancho!\\
  \textsc{name}\\
}\ex{\gll
\textstylePichiexamplebold{Yéé}    Paquita!\\
  \textsc{intj}    \textsc{name}\is{vocatives}\\
}\z\z

The interjection hɛ́ is employed as a remonstrative when a grown-up or social superior scolds a child or a socially inferior. It is used shortly before, or in the very moment a person commits a transgression in order to warn and rebuke them: 


\ea%1654
    \label{ex:key:1654}
    \gll Di  pikín  dɔ́n  gɛ́t  sɛ́ven  hía,    e    go  wánt  gó  wáka,
‘hɛ́,  nó  kɔmɔ́t  na  hós!’\\
\textsc{def}  child  \textsc{prf}  get  seven  year    \textsc{3sg.sbj}  \textsc{pot}  want  go  walk
\textsc{intj}  \textsc{neg}  go.out  \textsc{loc}  house\\
\glt ‘(When) the child is seven years old, she will want to roam the streets, 
(then you say to her) “don’t you dare leave the house!”’ [ab03ay 115]
\z

The interjection \textit{aa} (homonymous with the phatic \textit{aa} in \REF{ex:key:1648} above)) expresses negligence. In that sense, it may communicate to an interlocutor not to worry or bother about a situation. In the appropriate context, negligence may shade off into impatience and serve to express irritation with a person’s insisting or nagging stance. In the latter case, \textit{àá} is often pronounced with a rising contour and supported by ‘suck teeth’ \REF{ex:key:1656}. 


The uses of this interjection point towards an area of transition between phatic interjections aimed at constructively maintaining a communicative situation and conative interjections with their imperative nature: 



\ea%1655
    \label{ex:key:1655}
    \gll Aa,  lɛ́f=an    dé!\\
\textsc{intj}  leave=\textsc{3sg.obj}  there\\

\glt ‘Just leave it there [don’t bother to pick it up]!’ 
\z


\ea%1656
    \label{ex:key:1656}
    \gll ‘chip’  \textstylePichiexamplebold{aa}  apaga    eso!\\
\textsc{skt}    \textsc{intj}  extinguish  this\\

\glt ‘Switch this off! [you’re getting on our nerves 


\glt with that noise]’ [dj05be 116]
\z

The interjections and phrases ɛ́n, nɔ́, hía ‘hear’, and yu de hía? ‘2sg ipfv hear’ = ‘do you hear?’ are employed as channel checking devices in seeking feedback or approval from discourse participants. Thus, they always bear the boundary tone associated with question intonation. Compare ɛ́n, which occurs in sentence-final position often after a pause in order to increase dramatic effect, as well as nɔ́ (cf. \REF{ex:key:655} further above): 


\ea%1657
    \label{ex:key:1657}
    \gll Di  trú  comedor    dé    fɔ  soja,    Manolete,
Corrobes,  ɛ́n.\\
\textsc{def}  true  dining.room  \textsc{be.loc}  \textsc{prep}  soy    \textsc{name}
\textsc{name}    \textsc{intj}\\
\glt ‘The true dining room had soy (oil), Manolete (oil), 
Corrobes (oil), you hear?’ [ab03ab 031]
\z

A sentence-final \textit{hía} ‘hear’ may require explicit approval, so it is used in addressing a listener of equal or inferior social rank. The undertone of authority is stronger with the question phrase \textit{yu (de) hía?} in \REF{ex:key:1659}(a), which always requires explicit approval, usually with the idiom \textit{a hía} ‘I have heard’ \REF{ex:key:1659}(b): 


\ea%1658
    \label{ex:key:1658}
    \gll Nó  obstante,  a    bɛ́g  gó  sí  dɔ́kta  fɔ́s,  hía?\\
nonetheless    \textsc{1sg.sbj}  beg  go  see  doctor  first  hear\\

\glt ‘Nonetheless, please go see a doctor first, you hear?’ [fr03wt 053]
\z


\ea%1659
    \label{ex:key:1659}
\ea{
    \gll Yu  \textstylePichiexamplebold{de}  \textstylePichiexamplebold{hía}?\\
  \textsc{2sg}  \textsc{ipfv}  hear\\
\glt   ‘Do you hear?’ [fr03wt 056]
}\ex{\gll
A    \textstylePichiexamplebold{hía}.\\
  \textsc{1sg.sbj}  hear\\
\glt   ‘I’ve heard.’ [ru03wt 057]
}\z\z

Other phrasal interjections are employed to seek attention, approbation and confirmation. The phrases nó tɔ́k ‘don’t talk’, which appears together with nɔ́ and ɛ́n in \REF{ex:key:655} further above and nó láf ‘neg laugh’ = ‘I’m not kidding’ \REF{ex:key:1660}(b) underline the credibility of the speaker’s proposition or story. A similar function is fulfilled by a tɛ́l yú ‘I tell you’ in \REF{ex:key:1661}:


\ea%1660
    \label{ex:key:1660}
\ea{
    \gll Djunais  tɔ́k  sé    ‘nɔ́  Rubi,  dí  gɛ́l  lɛ́k  yú,
  dí  gɛ́l  lɛ́k  yú,    náw  bigín  mék=an    só.’\\
  \textsc{name}  talk  \textsc{quot}    \textsc{intj}  \textsc{name}  this  girl  like  \textsc{2sg.indp}
  this  girl  like \textsc{2sg.indp}  now  begin  make=\textsc{3sg.obj}  like.that\\

\glt 
  ‘(So) Djunais said “really Rubi, this girl likes you, this girl likes you,
now go about it like this with her.”’ [ru03wt 021] 
}\ex{\gll
\textstylePichiexamplebold{Nó}  \textstylePichiexamplebold{láf}!\\
  \textsc{neg}  laugh\\
\glt   ‘I’m not kidding.’ [ru03wt 022]
}\z\z


\ea%1661
    \label{ex:key:1661}
    \gll A    tɛ́l  yú,    dɛn  nó  lɛ́k  pɔ́sin.\\
\textsc{1sg.sbj}  tell  \textsc{2sg.indp}  \textsc{3pl}  \textsc{neg}  like  person\\

\glt ‘I tell you, they don’t like people.’ [ma03hm 010]
\z

The interjections dí bɔ́y ‘this boy’ and dí gál/gɛ́l ‘this girl’ are used as vocatives[FFF9?] for calling social equals or inferiors whose names are unknown. These interjections of address are rather informal but not necessarily impolite. They are frequently heard on the streets of Malabo, where they are employed particularly by the youth: 


\ea%1662
    \label{ex:key:1662}
    \gll Dí  bɔ́y, ús=sáy  dɛn  de  sɛ́l  brɛ́d    na  yá?\\
this  boy  \textsc{q=}side  \textsc{3pl}  \textsc{ipfv}  sell  bread  \textsc{loc}  here\\

\glt ‘Hey you, where is bread sold around here?’ [nn07fn 241]\is{interjections}
\z

\subsection{Suck teeth}\label{sec:12.2.4}

The term “suck teeth”, or alternatively “kiss teeth”, (transcribed as “chip” and glossed as \textsc{skt}) are terms employed for a phonetic realisation whose various meanings are determined through pragmatic context. “Suck teeth” is produced by closing the mouth and creating a suction in the oral cavity. The lips are then opened while keeping the teeth closed. The influx of air through the teeth in order to fill the vacuum of the oral cavity produces a release sound followed by a short hiss. “Suck teeth” can be modulated in sound and meaning by manipulating the amount of suction and pursing the lips in varying degrees while the air rushes through. 


“Suck teeth” is employed as a signifier of “negative affect” \citep{Figueroa2005} throughout large parts of Africa and in African-descended communities of the Americas (\citealt{RickfordRickford1976}). The large range of functions and meanings of “suck teeth” in Pichi corresponds to those recorded for the entire cultural area.



“Suck teeth” is employed as an expressive interjection to convey negatively loaded sentiments ranging from annoyance, irritation, and frustration to exasperation, fatigue, and weariness. In \REF{ex:key:1663}, speaker (ed) remembers the hard times he went through as a student in Cuba when the Equatoguinean government stopped paying him his living allowance. “Suck teeth” underlines his feeling of exasperation as he delves into his memories: 



\ea%1663
    \label{ex:key:1663}
\ea{
    \gll Sɔ́fa    dán  mɔní  bin  de  dú  mí    fɔ  ús=tín  ‘\textstylePichiexamplebold{chip}’?\\
  suffer  that  money  \textsc{pst}  \textsc{ipfv}  do  \textsc{1sg.indp}  \textsc{prep}  \textsc{q}=thing  \textsc{skt}\\
\glt   ‘(The) suffering that [lack of] money caused me for what \textsc{[skt]}?’ [ed03sp 099]
}\ex{\gll
Tɛ́l  yú,    a    sɔ́fa    dé.\\
  tell  \textsc{2sg.indp}  \textsc{1sg.sbj}  suffer  there\\
\glt   ‘(I) tell you, I suffered there.’ [ed03sp 100]
}\z\z

In \REF{ex:key:1664}, the negative affect associated with “suck teeth” is downgraded to signal a frustrated effort and an ensuing change of heart. After her grandchild has fallen sick in the night, speaker (ab) is at a loss about the appropriate treatment. “Suck teeth” expresses her indecision: 


\ea%1664
    \label{ex:key:1664}
    \gll A    wánt  ték    solwatá    mék    a    gí=an,
a    sé    ‘\textstylePichiexamplebold{chip}’,  nɔ́.\\
\textsc{1sg.sbj}  want  take    saltwater  \textsc{sbjv}    \textsc{1sg.sbj}  give=\textsc{3sg.obj}
\textsc{1sg.sbj}  \textsc{quot}    \textsc{skt}    \textsc{intj}\\

\glt ‘I wanted to take salt-water and give it to him (and) I said (to myself) 
\textsc{[skt]} no.’ [ab03ab 094]
\z

“Suck teeth” is rarely used to exclusively render inner speech. Rather, there is a smooth transition from expressive to phatic meanings. Sentence \REF{ex:key:1665} is coloured by reproach. Speaker (ye) employs “suck teeth” – albeit with a humorous undertone – to indicate his irritation with the fact that he has not been invited to go eat at Marathon (a restaurant), while his interlocutors had:


\ea%1665
    \label{ex:key:1665}
    \gll Náw    só,    e    falta,  una  dɔ́n  gó  na  Marathon,
mí    nó  gó  na  Marathon  ‘\textstylePichiexamplebold{chip}’.\\
now    like.that  \textsc{3sg.sbj}  lack    \textsc{2pl}  \textsc{prf}  go  \textsc{loc}  \textsc{place}
\textsc{1sg.indp}  \textsc{neg}  go  \textsc{loc}  \textsc{place}    \textsc{skt}\\

\glt ‘Right now it remains, you [\textsc{emp}] have gone to Marathon, while 
I haven’t gone to Marathon \textsc{[skt]}.’ [ye05ce 303]
\z

In sentence \REF{ex:key:1666}, speaker (ma) recollects the circumstances of the separation from her husband. “Suck teeth” not only expresses the negative feelings that she recalls. The interjection also communicates to the interlocutor that (ma) attaches a negative moral judgment to the fact that her husband gɛ́t ɔ́da húman ‘got (himself) another woman’:


\ea%1666
    \label{ex:key:1666}
    \gll E    gɛ́t  ɔ́da    húman  ‘chip’,  bɔkú  problema,
dán,    mí    bin  dɔ́n  sté,    wi  bin  gɛ́t  bɔkú  
problema  dé    áfta/\\
\textsc{3sg.sbj}  get  other  woman  \textsc{skt}    much  problem
that    \textsc{1sg.indp}  \textsc{pst}  \textsc{prf}  be.long  \textsc{1pl}  \textsc{pst}  get  much
problem    there  then\\
\glt ‘He got another woman \textsc{[skt]}, many problems, that, I [\textsc{emp}] stayed 
(and) we had many problems at that time, then/’ [ma03ni 031]
\z

“Suck teeth” in sentence \REF{ex:key:1667} below combines expressive and phatic meanings in a similar way as in \REF{ex:key:1666} above. Speaker (ed) relates that he had not intended to marry a woman from his place of origin, \textit{Pagalú}, the island of Annobón, until his mother arranged a marriage for him. “Suck teeth” aptly summarises the negligent disinterest that speaker (ed) states to have had for women from \textit{Pagalú}: 


\ea%1667
    \label{ex:key:1667}
\ea{
    \gll  Sí  fɔ́s  tɛ́n  a    bin  dé    hía,    a    nó
  bin  de  bísin  fɔ  Pagalú  gɛ́l  dɛn.\\
  see  first  time  \textsc{1sg.sbj}  \textsc{pst}  \textsc{be.loc}  here    \textsc{1sg.sbj}  \textsc{neg}
  \textsc{pst}  \textsc{ipfv}  be.busy  \textsc{prep}  \textsc{place}  girl  \textsc{pl}\\
\glt   ‘See formerly, I was here, I didn’t bother about Annobonese girls.’ [ed03sp 005]
}\ex{\gll
‘\textstylePichiexamplebold{chip}’  a    nó  bin  bísin.\\
  \textsc{skt}    \textsc{1sg.sbj}  \textsc{neg}  \textsc{pst}  be.busy\\
\glt   ‘\textsc{[skt]} I couldn’t care less.’ [ed03sp 006]
}\z\z

Feigned disinterest and playful insubordination colour the use of “suck teeth” by female Pichi speakers in flirtatious male–female interaction. A simple “suck teeth” by Beatrice in \REF{ex:key:1668}(b) is sufficient to ward off the humorous advances of speaker (ye) in \REF{ex:key:1668}(a). The remonstrative character of “suck teeth” in (b) points towards an area of transition from expressive and phatic to conative functions of “suck teeth”:


\ea%1668
    \label{ex:key:1668}
\ea{
    \gll Beatrice,  wétin  mék    yu  dɔ́n  frɛ́s    só?\\
  \textsc{name}  what  make  \textsc{2sg}  \textsc{prf}  be.fresh  like.that\\

\glt   ‘Beatrice, how come you’re so fresh recently?’
}\ex{\gll
‘\textstylePichiexamplebold{chip}’.\\
  \textsc{skt}\\
\glt    \textsc{[skt]}
}\z\z

The conative function of “suck teeth” is brought to its conclusion in \REF{ex:key:1669}(a), where\textit{ chip} accompanies an insult. The pungency of the insult, and by extension the gesture of “suck teeth” itself, is commented by (dj) in \REF{ex:key:1669}(b):


\ea%1669
    \label{ex:key:1669}
\ea{
    \gll ‘\textstylePichiexamplebold{chip}’  aa  múf,    kɔmɔ́t  yá!\\
  \textsc{skt}    \textsc{intj}  move  go.away  here\\
\glt   ‘\textsc{[skt]} move, get lost.’ [dj07ae 367]
}\ex{\gll
Yu  sí  dán  pɔ́sin  lɛk  wán  dɔ́g.\\
  \textsc{2sg}  see  that  person  like  one  dog\\
\glt   ‘You take that person for a dog.’ [dj07ae 368]
}\z\z

\subsection{The particle \textit{ó} ‘\textsc{sp’}}\label{sec:12.2.5}

The sentence-final particle \textit{ó} plays an important pragmatic role. It is employed for degree modification\is{degree modification} (cf. e.g. \REF{ex:key:899}), may signal clausal focus \is{focus}(cf. \sectref{sec:7.4.2}), is used as a vocative marker, and provides a means of modifying a sentence with various expressive and emphatic\is{emphasis} meanings . The function of the particle also extends into the domain of modality. I analyse the element \textit{ó} as a “sentence particle”, because it is never set off by a prosodic break from the utterance it modifies. Instead, \textit{ó} forms a prosodic unit with the preceding utterance. One indication for this is that \textit{ó} normally forms a syllable with the final consonant of the preceding word, i.e. \textit{e bád ó} [é bá tó] ‘\textsc{3sg.sbj} be.bad \textsc{sp’} = ‘it’s really bad’.


The particle \textit{ó} serves as a vocative marker in combination with personal names in order to call people from a distance \REF{ex:key:1670} or get their attention during conversation \REF{ex:key:1671}. Presumably, it is this function of alerting which lies at the heart of the other uses that follow: 



\ea%1670
    \label{ex:key:1670}
    \gll Concha  ó,  Maura  ó,  una  kán,    a    bɛ́g!\\
\textsc{name}  \textsc{sp}  \textsc{name}  \textsc{sp}  \textsc{2pl}  come  \textsc{1sg.sbj}  beg\\

\glt ‘Concha! Maura! Come over, please!’ [he07fn 612]
\z


\ea%1671
    \label{ex:key:1671}
    \gll Lindo  ó,  Charly  ó,  una  de  sí,  a    bin  tɛ́l  dí  gál  sé
mék  e    nó  hambɔ́g  mí    \textstylePichiexamplebold{ó},  a    go  hát=an.\\
\textsc{name}  \textsc{sp}  \textsc{name}  \textsc{sp}  \textsc{2pl}  \textsc{ipfv}  see  \textsc{1sg.sbj}  \textsc{pst}  tell  this  girl  \textsc{quot}
\textsc{sbjv}  \textsc{3sg.sbj}  \textsc{neg}  bother  \textsc{1sg.indp}  \textsc{sp}  \textsc{1sg.sbj}  \textsc{pot}  hurt=\textsc{3sg.obj}\\

\glt ‘Lindo, Charly, you see, I told this girl not to bother me (lest) I might 
hurt her!’ [ye05ce 079]\is{vocatives}
\z

Urgency, advise, and warning colour the sentences in which this particle is used. The following, successively spoken sentences \REF{ex:key:1672}(a)–(c) are characterised by an air of urgency and warning as speaker (ab) relates a near-death experience:


\ea%1672
    \label{ex:key:1672}
\ea{
    \gll A    dɔ́n  tɛ́l  mi    sísta    sé    ‘na  di  pikín  dát  ó.’\\
  \textsc{1sg.sbj}  \textsc{prf}  tell  \textsc{1sg.poss}  sister  \textsc{quot}    \textsc{foc}  \textsc{def}  child  that  \textsc{sp}\\
\glt   ‘I had already told my sister “mind you, this is the [my] child”.’ [ab03ay 081]
}\ex{\gll
Mék    yu  mɛ́n=an      ó!’\\
  \textsc{sbjv}    \textsc{2sg}  care.for=\textsc{3sg.obj}  \textsc{sp}\\
\glt   ‘Be sure to take good care of her [because I’m going to die].’ [ab03ay 082]
}\ex{\gll
A    dɔ́n  de  gó  \textstylePichiexamplebold{ó},  a    dɔ́n  de  gó  \textstylePichiexamplebold{ó}.\\
  \textsc{1sg.sbj}  \textsc{prf}  \textsc{ipfv}  go  \textsc{sp}  \textsc{1sg.sbj}  \textsc{prf}  \textsc{ipfv}  go  \textsc{sp}\\
\glt   ‘I’m going [dying], I’m going.’ [ab03ay 083]
}\z\z

Further gradations of the meanings of \textit{ó} are found in the following sentence. In the example, \textit{ó} assumes the function of a modal particle, a marker of assertion, which signals commitment by the speaker to the truth of the proposition:


\ea%1673
    \label{ex:key:1673}
    \gll Yɛ́s,  a    sabí    dé    yɛ́s,  bɔt  a    nɔ́ba  ɛ́nta    ínsay  ó.\\
yes  \textsc{1sg.sbj}  know  there  yes  but  \textsc{1sg.sbj}  \textsc{neg}.\textsc{prf}  enter  inside  \textsc{sp}\\

\glt ‘Yes, I know that place, yes. But mind you, I haven’t entered the place 


\glt before.’ [ma03hm 016]
\z

\section{Terms of address}\label{sec:12.3}

Often, the African and European given names of individuals are only known to relatives and close friends. Peers tend to address each other by nicknames which may be conferred on an individual during interaction with family members, friends, the neighbourhood, and the wider community. Nicknames may also change in the passage of time as new events come to mark a person’s daily life. 


I list three nicknames in \REF{ex:key:1674} that are used by peers most of the time in addressing their bearers. The bearer of the first name is female, the second and third names are borne by men. As can be seen, peer nicknames tend to be characterised by an air of informality:


\eabox{\label{ex:key:1674}
\begin{tabularx}{.9\textwidth}{ll}
    Nickname &     Origin\\
\textstyleTablePichiZchn{Lage}  &      ‘Líneas aéreas de Guinea Ecuatorial’\\
\textstyleTablePichiZchn{Boyé Loco} &      ‘Crazy Boyé’\\
\textstyleTablePichiZchn{Johnson} &‘Johnson Power Systems’\\
\end{tabularx}
}
\textit{Lage} was born aboard a flight from Madrid to Malabo, operated by the now defunct National Airline of Equatorial Guinea, in Spanish \textit{Líneas Aéreas de Guinea Ecuatorial} (abbrev. LAGE). Her birth back then was the talk of the town and the name stuck for a life time. \textit{Boyé Loco}’s name is composed of his Bube given name \textit{Boyé} and the Spanish adjective \textit{loco} ‘crazy’ due to his reputation as a charismatic \textit{bon} \textit{vivant}. The byname \textit{Johnson} originates in the brand name ‘Johnson Power Systems’. Due to the unreliablity of power supply in Malabo, generators produced by ‘Johnson’ are ubiquitous in Malabo. The nickname is a humorous allusion to the bearer’s supposed sexual prowess. 

\figref{fig:key:12.1} presents the degree of formality from informal (the –pole) to very formal (the +pole) attached to the terms of address covered in the following \citep[209]{Mühleisen2005}. The corresponding kinship \is{kinship terminology}terms \is{kinship terminology}can be taken from \figref{fig:key:12.2} further below.

\begin{figure}
\caption{Degree of formality of terms of address}
\label{fig:key:12.1}
\small
\begin{tabularx}{\textwidth}{lQQQl}
\hfill --& \multicolumn{3}{c}{Degree of formality} &\hfill +\\
\midrule 
Nicknames & {Same generation kinship terms + FN} & Kinship terms for 1 \& 2 generations older + FN & {\textit{don/doña} \textstyleTableEnglishZchn{+ FN}} \newline
		    {\textit{señor/a} \textstyleTableEnglishZchn{+ FN}}\newline
		    {\itshape sa}\newline
		    {\itshape má}
 & \textit{señor/a}\textstyleTableEnglishZchn{ + LN}\\
\end{tabularx}
\end{figure}
Spanish honorifics are employed for the most formal degrees of relationships between interactants. Without doubt, this circumstance is intimately tied to the status of Spanish itself as a language of dominance, distance, and social asymmetry. The address terms \textit{señor} (male) and \textit{señora} (female) are in use with first names (\MakeUppercase{fn}) or last names (\MakeUppercase{ln}). The latter option follows Spanish usage (i.e \textit{Señora Belobe Toichoa} ‘Ms Belobe Toichoa’) and is commonly employed in symmetrical or asymmetrical relations in institutional or work contexts in the formal sector of the economy. 


The former option, \textit{señor} or\textit{ señora} with an FN (i.e.\textit{ Señora} \textit{Maura}; \textit{Señor Javier}), is not common in Spanish. In Pichi, it is a means of respectfully addressing an already familiar, social superior in less formal situations than the ones appropriate for \textit{señor/a} and LN. The use of \textit{señor}\textit{\textup{/}}\textit{a} and FN parallels that of the Spanish honorifcs \textit{don} (male) and \textit{doña} (female) followed by FN, for elderly and respected members of the communty, i.e \textit{Don Samuel} and \textit{Doña Cristina}. The combination \textit{don}/\textit{doña} and FN is, however, current in Spanish.



Two Pichi address terms of a high degree of formality are, also in use, namely \textit{sá} ‘sir’ and \textit{má} ‘madam; mother’. These two terms are used as address terms and sentence-final address tags when interacting with an elder of higher social rank, usually without an FN or LN. For example \textit{sá} ‘sir’ can be found in the respectful speech of a well-behaved child or youngster when replying to an enquiry by an elder. Compare the following answers by a child to a \textsc{when} enquiry by a female elder \REF{ex:key:1675} and a yes–no question by a male elder who is not a family relative \REF{ex:key:1676}: \is{answers}



\ea%1675
    \label{ex:key:1675}
    \gll Yɛ́stadé    má.\\
yesterday  madam\\

\glt ‘Yesterday, madam.’ [ra07se 150]
\z


\ea%1676
    \label{ex:key:1676}
    \gll Yɛ́s  sá.\\
yes  sir\\

\glt ‘Yes, sir.’ [au07se 153]
\z

Kinship-based terms of address are situated in the middle range of formality and may be used in addressing familiar persons or strangers. The dimension of age naturally relates to the degree of formality in so far as senior members of society are more likely to be addressed by one of the more formal terms of address in \figref{fig:key:12.1}. Under normal circumstances, the use of an FN presupposes the use of a kinship term if the addressee is older than oneself (i.e. \textit{Mamí} \textit{Rose} ‘mother Rose’ = ‘respectful address term for Rose, who is of my mother’s generation’). The use of a first name alone for an older person is highly inappropriate. For people of the same age group, and young people in particular, kinship terms are, however, not required as terms of address. Social equals may refer to each other by their first names or their nicknames alone.


 \figref{fig:key:12.2} provides the kinship-based address terms referred to in \figref{fig:key:12.2} arranged along the dimension of age: 


\begin{figure}
\caption{Kinship-derived terms of address}
\label{fig:key:12.2}
\small
\begin{tabularx}{\textwidth}{llllll}

\multicolumn{2}{c}{{}-} & \multicolumn{2}{c}{Age} & \multicolumn{2}{c}{+}\\
\multicolumn{2}{c}{Same generation} & \multicolumn{2}{c}{1 generation older} & \multicolumn{2}{c}{2 generations older}\\
\midrule 
\itshape cuñada & ‘sister-in-law’ & \itshape mamá, mamí & ‘mother’ & \itshape gran-má & ‘grandmother’\\
\itshape cuñado & ‘brother-in-law’ & \itshape mɔmí & ‘mother’ & \itshape gran-pá & ‘grandfather’\\
\itshape sísta & ‘sister’ & \itshape papá, papí & ‘father’ & \itshape abuela & ‘grandmother’\\
\itshape brɔ́da & ‘brother’ & \itshape antí & ‘aunt’ & \itshape abuelo & ‘grandfather’\\
\itshape brá & ‘brother’ & \itshape ɔnkúl & ‘uncle’ &  & \\
&  & \itshape tía & ‘aunt’ &  & \\
&  & \itshape tío & ‘uncle’ &  & \\
\end{tabularx}
\end{figure}
As a general principle, any of the address terms listed may be combined with an FN. In practice, an FN hardly ever follows the same generation terms \textit{cuñado}\textit{\textup{/}}\textit{a} ‘brother/sister-in-law’, \textit{sísta ‘}sister’ or \textit{brɔ́da} ‘brother’. At the same time, the use of an FN with a kinship term for an addressee one or two generations older tends to be avoided as well unless there is a high degree of familiarity and/or an actual kinship relation between the interlocutors. Compare the following combination of address term and FN:


\ea%1677
    \label{ex:key:1677}
    \gll \textstylePichiexamplebold{Tía}    \textstylePichiexamplebold{Tokó},  ús=sáy  yu  de  gó?\\
aunt    \textsc{name}  \textsc{q}=side  \textsc{2sg}  \textsc{ipfv}  go\\

\glt ‘Auntie Tokó, where are you going?’ [ye07fn 213]
\z

I should point out that Spanish kinship terms form an integral part of the address system of Pichi. The Spanish terms \textit{cuñado}\textit{\textup{/}}\textit{a} ‘brother/sister-in-law’ – with \textit{cuñado} invariably being pronounced as [kùnjáò] – have been appropriated and changed in their meaning. In Pichi, these two kinship terms function as markers of aknowledgment and solidarity amongst peers. They are therefore used to address any person of the same generation, whether related or not. In this function, \textit{cuñado}\textit{\textup{/}}\textit{a} are far more common than the equivalent \textit{sísta} ‘sister’ and \textit{brɔ́da} or \textit{bra} ‘brother’: 


\ea%1678
    \label{ex:key:1678}
    \gll Cuñado,      mí    gɛ́fɔ    gó  fɛ́n    dán  mi    
prima  ó,  Cristina.\\
brother-in-law  \textsc{1sg.indp}  have.to  go  look.for  that  \textsc{1sg.poss}  
cousin  \textsc{sp}  \textsc{name}\\
\glt ‘Brother(-in-law), I [\textsc{emp}] really have to go look for that my (female) cousin, Cristina.’ [ge07ga 045]
\z

In the same vein, the Spanish kinship terms \textit{tía} ‘aunt’ and \textit{tío} ‘uncle’ are equally common as \textit{=antí} ‘aunt’ and \textit{ɔnkúl} ‘uncle’ as terms of address. The same holds for the Spanish-derived terms \textit{abuela} ‘grandmother’ and \textit{abuelo} ‘grandfather’ as opposed to \textit{gran-má} ‘grandmother’ and \textit{gran-pá} ‘grandfather’. However, the Pichi words \textit{antí} ‘aunt’ and \textit{ɔnkúl} ‘uncle’ are more often used to denote the kinship relation as such \REF{ex:key:1679}: 


\ea%1679
    \label{ex:key:1679}
    \gll E    mít    mi    \textstylePichiexamplebold{antí}.\\
\textsc{3sg.sbj}  meet  \textsc{1sg.poss}  aunt\\

\glt ‘He met my aunt.’ [fr03ft 084]
\z

Conversely, the Spanish words \textit{abuela} ‘grandmother’ and \textit{abuelo} ‘grandfather’ are more common as terms of address and at least as common as \textit{gran-má} ‘grandmother’ and \textit{gran-pá} ‘grandfather’ in denoting the kinship relation as such:


\ea%1680
    \label{ex:key:1680}
    \gll \textstylePichiexamplebold{Abuela},    Guinea      fít=an?\\
grandmother  Equatorial.Guinea  fit=\textsc{3sg.obj}\\

\glt ‘Grandmother, (so) Equatorial Guinea is good for him?’ [fr03ab 171]\is{kinship terminology}
\z

Since first names are not normally used to refer to social superiors, including next of kin, a kinship term will normally be used to refer to a common kin. In \REF{ex:key:1681}, speaker (ro) is conversing with her nephew. She refers to her own husband as \textit{yu ɔnkúl} ‘your uncle’: 


\ea%1681
    \label{ex:key:1681}
    \gll Yu  ɔnkúl  nó  gɛ́t  nó  hambɔ́g  fɔ  chɔ́p.\\
\textsc{2sg}  uncle  \textsc{neg}  get  \textsc{neg}  bother  \textsc{prep}  food\\

\glt ‘Your uncle [my husband] is not picky about food.’ [ro05rt 058]
\z

\section{Greetings and other routines}\label{sec:12.4}

A general greeting routine is normally initiated by addressing an individual with the phrase in \REF{ex:key:1682} and a group of people by \REF{ex:key:1683}. These phrases may be reformulated at will to enquire after the health of partners, children, or other relatives \REF{ex:key:1684}. A general observation is that conventional Spanish greeting routines are widely used together with Pichi routines (i.e. \textit{buenos días} ‘good morning’):


\ea%1682
    \label{ex:key:1682}
    \gll Háw    fɔ  yú?\\
how    \textsc{prep}  \textsc{2sg.indp}\\

\glt ‘How are you?’ [ye07je 063]
\z


\ea%1683
    \label{ex:key:1683}
    \gll Háw    fɔ  una?\\
how    \textsc{prep}  \textsc{2pl}\\

\glt ‘How are you [\textsc{pl}]?’ [ye07je 064]
\z


\ea%1684
    \label{ex:key:1684}
    \gll Háw    fɔ  yu  mamá?\\
how    \textsc{prep}  \textsc{2sg}  mother\\

\glt ‘How is your mother?’ [ne07fn 215]
\z

The enquiry is usually replied to by one of the phrases in \REF{ex:key:1685}–\REF{ex:key:1687}:


\ea%1685
    \label{ex:key:1685}
    \gll A    dé.\\
\textsc{1sg.sbj}  \textsc{be.loc}\\

\glt ‘I’m (fine).’ [ye07je 065]
\z


\ea%1686
    \label{ex:key:1686}
    \gll Dɛn  dé    fáyn.\\
\textsc{3pl}  \textsc{be.loc}  fine\\

\glt ‘They’re fine.’ [ye07je 067]
\z


\ea%1687
    \label{ex:key:1687}
    \gll A    wɛ́l.\\
\textsc{1sg.sbj}  be.well\\

\glt ‘I’m well.’ [li07fn 011]
\z

The most wide-spread greeting formula amongst the youth or peers and in relaxed and informal social settings is featured in \REF{ex:key:1688}. This greetings involves the element \textit{fá}, not found in any other context in Pichi, but almost certainly derived from the English word \textit{fashion.} It is also attested in Krio, Nigerian Pidgin and Cameroon Pidgin. Notably, it is also found in Maroon Spirit Language (Jamaica) \citep[50]{Bilby1983}, as well as in Sranan and the other creoles of Suriname in the almost identical form \textit{o} \textit{fa} ‘which fashion; how’ \citep[50]{Wilner1994}. In Pichi, a common reply to the idiom is \REF{ex:key:1689}:


\ea%1688
    \label{ex:key:1688}
    \gll Háw    fá?\\
how    fashion\\

\glt ‘What’s up?’ [be07fn 174]
\z


\ea%1689
    \label{ex:key:1689}
    \gll Chico,  wi    de  pús=an.\\
boy    \textsc{1sg.sbj}  \textsc{ipfv}  push=\textsc{3sg.obj}\\

\glt ‘Man, we’re pushing it [we’re managing].’ [ch07fn 214]
\z

Longer exchanges of greetings are usually initiated by employing the property item \textit{gúd} ‘be good’ together with the noun that denotes the period of the day in which the greeting takes place. The resulting collocations constitute greeting formulas by themselves but are very often followed by one of the general greeting formulas in \REF{ex:key:1682}–\REF{ex:key:1684} above. The collocation \textit{(gúd) mɔ́nin} ‘good morning’ or a simple \textit{mɔ́nin} ‘morning’ is used from sunrise to noon \REF{ex:key:1690}:


\ea%1690
    \label{ex:key:1690}
    \gll ‘\textstylePichiexamplebold{Gúd  mɔ́nin}’  na  sóté    las    doce.\\
good  morning  \textsc{foc}  until  the.\textsc{pl}  twelve\\

\glt ‘Good morning is until twelve o’clock.’ [ye07je 015]
\z

\textit{Gúd ívin} is used from noon to sunset \REF{ex:key:1691}. The collocation \textit{gúd áftanun} ‘good afternoon’ is sometimes used by Group 1 (cf. \sectref{sec:1.3}) speakers instead of \textit{gúd ívin}, but it is virtually absent from the speech of Group 2 speakers:


\ea%1691
    \label{ex:key:1691}
    \gll Frɔn    las    doce,  sóté    e    gó  las    seis,
na  ‘\textstylePichiexamplebold{gúd}    \textstylePichiexamplebold{ívin}’.\\
from  the.\textsc{pl}  twelve  until  \textsc{3sg.sbj}  go  the.\textsc{pl}  six  
\textsc{foc}  good  evening\\
\glt ‘From twelve to six o’clock, its “good evening”.’ [ye07je 011]
\z

The collocation gúd náyt ‘good night’ is used after night has fallen. The presence of the otherwise rare variant náyt ‘night’ in the greeting instead of nɛ́t ‘night’ is indicative of the formulaic, lexicalised character of the collocation. 


Also note the apposition of the \textsc{2pl} pronoun \textit{una} when a greeting is directed to more than one person. The use of \textit{yu} ‘\textsc{2sg}’ in the same position as \textit{una} in greetings directed at an individual is not attested. Responses to greetings usually involve the repetition of the corresponding phrase by the interlocutor:



\ea%1692
    \label{ex:key:1692}
    \gll Una    \textstylePichiexamplebold{gúd}    \textstylePichiexamplebold{náyt}.\\
\textsc{2pl}    good  night\\

\glt ‘Good night to you [\textsc{pl}].’ [ye07je 045]
\z

Other greetings are issued on specific occasions rather than periods of the day. On the occasion of imminent travel, the most common way of bidding farewell is by saying wáka fáyn ‘walk fine’. Upon arrival, the traveler is greeted by wɛ́lkɔm ‘welcome’. 

The greeting formula kúsɛ́ (< Yoruba kuṣẹ,  (cf. \citealt{Abraham1958})) is said as a token of encouragement and empathy towards one or more people engaged in physically strenuous work (e.g. a group of construction workers working on the road). Kúsɛ́ is also used to congratulate a person for their good work:


\ea%1693
    \label{ex:key:1693}
    \gll Una  kúsɛ́  ó!\\
\textsc{2pl}  good.job  \textsc{sp}\\

\glt ‘(We) encourage you [\textsc{pl}] in your good work!’ [ye07je 028]
\z

Gratitude is expressed by means of tɛ́nki ‘thank you’ \REF{ex:key:1694}(a). Reply options are provided in (b) and (c). Note that fɔ nátinin (b) and na nátin(c) are calques[FFF9?] from Spanish dé nada ‘of nothing’ = ‘you’re welcome’: 


\ea%1694
    \label{ex:key:1694}
\ea{
    \gll  Tɛ́nki.\\
  thanks\\
\glt   ‘Thank you.’ [ye07fn 096]
}\ex{\gll
Nɔ́,  \textstylePichiexamplebold{fɔ}  \textstylePichiexamplebold{nátin}.\\
  \textsc{intj}  \textsc{prep}  nothing\\
\glt   ‘No, not at all.’ [hi07fn 097]
}\ex{\gll
Lɛ́f,    na  nátin.\\
  leave  \textsc{foc}  nothing\\
\glt   ‘Don’t mention it, it’s nothing.’ [ye07fn 503]
}\z\z

