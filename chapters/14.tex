\chapter{Typological summary of Pichi}

Pichi is an Afro-Caribbean \ili{English}-lexifier creole language spoken on the island of Bioko, Equatorial Guinea. With somewhere between 100–150,000 speakers, Pichi is one of the most widely spoken languages of the country. Pichi is an offshoot of 19\textsuperscript{th} century \ili{Krio} (Sierra Leone) and shares many characteristics with its sister languages Krio, \ili{Aku} (Gambia), and Nigerian, Cameroonian, and Ghanaian Pidgin. However, insulation from English and intense contact with \ili{Spanish}, the colonial and official language of Equatorial Guinea, have given Pichi a character distinct from the other West African English-lexifier creoles and pidgins. 


Pichi has a nominative-accusative alignment, SV(O) word order and adjective-noun order, prenominal determiners, and prepositions. Pichi has a seven-vowel system and twenty-two consonant phonemes, including two labio-velar plosives. The language has a two-tone system with tonal minimal pairs, morphological tone for the marking of pronominal case distinctions, and numerous tonal processes. The morphological structure of Pichi is largely isolating. However, there is some inflectional and derivational morphology in which affixation and tone are put to use. Pichi is characterised by a weak verb-adjective distinction. 



The categories of tense, modality, and aspect are primarily expressed through preverbal particles. Pichi is an aspect-prominent language in which aspect, rather than tense, plays a dominant role in expressing temporal relations. Besides that, the modal system includes an indicative-subjunctive opposition. The copula system employs various suppletive forms and is differentiated along the semantic criterion of time-stability.



Pichi verbs fall into three lexical aspect classes: dynamic, inchoative-stative, and stative. Content questions are formed by way of a mixed question-word system which involves transparent and opaque question elements. Clause linkage is characterised by a large variety of strategies and forms, in which a subordinator, a quotative marker, and two modal complementisers stand out as multifunctional elements with overlapping functions. The language also features various types of multiverb constructions. These include secondary predication, clause chaining, and serial verb constructions. Amongst the latter figure instrumental serial verb constructions involving the verb \textit{ték} ‘take’ as well as comparative constructions featuring the verb \textit{pás} ‘(sur)pass’. 



Many of the typological characteristics summarised above align Pichi closely with the Atlantic-Congo languages spoken in the West African littoral zone and beyond. At the same time, characteristics like the prenominal position of adjectives and determiners show a typological overlap with English. There are also numerous structural and lexical parallels with the Afro-Caribbean English-lexifier creoles of the (Circum-)Caribbean, such as, for example, \ili{Jamaican}, Creolese (Guyana), and the creole languages of Suriname.

